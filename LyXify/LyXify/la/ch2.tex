
\section{行列式的定义以及完全展开}

\section{行列式的性质}

\section{行列式的典型计算法}

\section{Cramer法则}

\section{线性代数中出现的内容}

\subsection*{一些简单的行列式}
\begin{enumerate}
\item 问题的提出: 在求解二元一次方程组的时候, 注意到可以使用合适的记号表示整个方程问题. 见AA例2.1.1.
\item 二阶行列式
\begin{itemize}
\item 定义: $\begin{vmatrix}a_{11} & a_{12}\\
a_{21} & a_{22}
\end{vmatrix}=a_{11}a_{22}-a_{21}a_{12}$应用: 求解方程组$\begin{cases}
a_{11}x_{1}+a_{12}x_{2} & =b_{1}\\
a_{12}x_{1}+a_{22}x_{2} & =b_{2}
\end{cases}$的解为(前提是$\begin{vmatrix}a_{11} & a_{12}\\
a_{21} & a_{22}
\end{vmatrix}$不是0).
\end{itemize}
\[
x_{1}=\frac{\begin{vmatrix}\boldsymbol{b}_{\boldsymbol{1}} & a_{12}\\
\boldsymbol{b}_{\boldsymbol{2}} & a_{22}
\end{vmatrix}}{\begin{vmatrix}a_{11} & a_{12}\\
a_{21} & a_{22}
\end{vmatrix}},x_{2}=\frac{\begin{vmatrix}a_{11} & \boldsymbol{b}_{\boldsymbol{1}}\\
a_{21} & \boldsymbol{b}_{\boldsymbol{2}}
\end{vmatrix}}{\begin{vmatrix}a_{11} & a_{12}\\
a_{21} & a_{22}
\end{vmatrix}}.
\]

\begin{itemize}
\item 几何意义: 描述一个空间经历线性变换之后, (面积/体积/更高维度的度量)相对于原来改变的倍数.(EOLA)
\end{itemize}
\item 三阶行列式
\begin{itemize}
\item 问题的提出: 我们猜测这个内容可以推广到更高阶解方程的问题上. 因此我们尝试解三解方程. 见AA例2.1.3.
\item 定义及其由来: 注意到: $\begin{vmatrix}a_{11} & a_{12} & a_{13}\\
a_{21} & a_{22} & a_{23}\\
a_{31} & a_{32} & a_{33}
\end{vmatrix}=a_{11}\begin{vmatrix}a_{22} & a_{23}\\
a_{32} & a_{33}
\end{vmatrix}-a_{12}\begin{vmatrix}a_{21} & a_{23}\\
a_{31} & a_{33}
\end{vmatrix}+a_{13}\begin{vmatrix}a_{21} & a_{22}\\
a_{31} & a_{32}
\end{vmatrix}$, 继续展开就有:
\[
\begin{vmatrix}a_{11} & a_{12} & a_{13}\\
a_{21} & a_{22} & a_{23}\\
a_{31} & a_{32} & a_{33}
\end{vmatrix}=a_{11}a_{12}a_{13}+a_{12}a_{23}a_{31}+a_{13}a_{21}a_{32}-a_{13}a_{22}a_{31}-a_{12}a_{21}a_{33}-a_{11}a_{23}a_{32}.
\]
\item 应用: 解三元一次方程组$\begin{cases}
a_{11}x_{1}+a_{12}x_{2}+a_{13}x_{3} & =b_{1}\\
a_{12}x_{1}+a_{22}x_{2}+a_{23}x_{3} & =b_{2}\\
a_{13}x_{1}+a_{23}x_{2}+a_{33}x_{3} & =b_{3}
\end{cases}$的解为(前提是$\begin{vmatrix}a_{11} & a_{12} & a_{13}\\
a_{21} & a_{22} & a_{23}\\
a_{31} & a_{32} & a_{33}
\end{vmatrix}$不是0).
\end{itemize}
\[
x_{1}=\frac{\begin{vmatrix}\boldsymbol{b_{1}} & a_{12} & a_{13}\\
\boldsymbol{b_{2}} & a_{22} & a_{23}\\
\boldsymbol{b_{3}} & a_{32} & a_{33}
\end{vmatrix}}{\begin{vmatrix}a_{11} & a_{12} & a_{13}\\
a_{21} & a_{22} & a_{23}\\
a_{31} & a_{32} & a_{33}
\end{vmatrix}},x_{2}=\frac{\begin{vmatrix}a_{11} & \boldsymbol{b_{1}} & a_{13}\\
a_{21} & \boldsymbol{b_{2}} & a_{23}\\
a_{31} & \boldsymbol{b_{3}} & a_{33}
\end{vmatrix}}{\begin{vmatrix}a_{11} & a_{12} & a_{13}\\
a_{21} & a_{22} & a_{23}\\
a_{31} & a_{32} & a_{33}
\end{vmatrix}},x_{3}=\frac{\begin{vmatrix}a_{11} & a_{12} & \boldsymbol{b_{1}}\\
a_{21} & a_{22} & \boldsymbol{b_{2}}\\
a_{31} & a_{32} & \boldsymbol{b_{3}}
\end{vmatrix}}{\begin{vmatrix}a_{11} & a_{12} & a_{13}\\
a_{21} & a_{22} & a_{23}\\
a_{31} & a_{32} & a_{33}
\end{vmatrix}}.
\]

\item $n$阶行列式
\begin{itemize}
\item 定义及其由来: 定义一阶方阵的行列式$A_{1}=(a_{11})$为$a_{11},$如果$n-1$阶行列式已经定义好, 那么$n$阶行列式的定义为:

\[
\sum_{j=1}^{n}(-1)^{1+j}a_{1j}M_{1j}
\]
 其中$M_{1j}$是划去$A$的第$1$行第$j$列的新得到的行列式, 
\item 常见矩阵的行列式值
\begin{itemize}
\item 上三角矩阵: $|A|=a_{11}a_{22}\cdots a_{nn}$, 其中$n$是上三角矩阵的阶数.
\end{itemize}
\item 下一步的要求: 需要找到$n$阶行列式的完全展开.
\end{itemize}
\end{enumerate}

\subsection*{全排列和对换}
\begin{enumerate}
\item 问题的提出: 在上面套娃展开的时候, 我们注意到所有的项数都有出现, 并且前面的符号只能为+1或者-1. 为了探究这个规律是不是成立以及何时前面是+1,
-1, 引入如下的操作.
\item 奇排列和偶排列. 设$i_{1},i_{2},\cdots,i_{n}$是n个不同自然数的排列, 如果$k<j$的时候$i_{k}<i_{j}$,
那么称$i_{k},i_{j}$构成一个\textbf{正序}. 否则称$i_{k},i_{j}$构成一个\textbf{逆序}.
称一个排列的逆序数总数为这个排列的\textbf{逆序数}. 记为$\tau(i_{1},i_{2},\cdots,i_{n})$,
如果是奇数(偶数), 称这个排列是奇(偶)排列.
\item 性质. 一个排列中的任意两个元素调换位置, 排列改变奇偶性. 
\item 行列式的完全展开: 
\end{enumerate}
\[
|A|=\sum_{(i_{1}i_{2}\cdots i_{n})}(-1)^{\tau(i_{1},i_{2},\cdots,i_{n})}a_{1i_{1}}a_{1i_{2}}\cdots a_{1i_{n}}
\]

\begin{itemize}
\item 下一步操作: 需要寻找计算行列式的简单的方法, 因为现有的方法过于复杂.
\end{itemize}

\subsection*{行列式的性质}

单个行列式的一些性质.
\begin{itemize}
\item 转置值不变: $|A|=|A^{T}|$.(AA定理2.2.2)
\item 对换要变号: 任意对换行列式的两行(两列), 行列式要变号. (AA命题2.2.4)
\item 线性性I: : 若某\textbf{行}(列)存在公因式, 可以提到外面来. (AA命题2.2.5)
\item 线性性II: 如果行列式某行某个元素都是两个元素的和, 这个行列式可以拆分成两个行列式的和. (AA命题2.2.6)
\item 某一行(列)的$k$倍加到另一行(列)去, 行列式值不变.(AA命题2.2.8)
\end{itemize}
``划去''的一般规律.
\begin{itemize}
\item 动机: 由于总是划去第一行有时候不是很方便, 因此希望可以每一次不在第一行划去, 或者挑选多行划去.
\item 余子式和代数余子式: 见AA定理2.2.10.
\begin{itemize}
\item 可以自然推导出\textbf{Cramer法则}, 并且引入\textbf{Kronecker记号}, 见AA引理2.4.1. 
\end{itemize}
\item Laplace展开: 形式化叙述见AA定义2.2.13. 先选定$m$行(列)不动, 然后在列(行)上面枚举所有可能的选择$m$列(行)的情况(一共要枚举${n \choose m}$种情形),
并且根据所选两个排列的和的奇偶性判定前面符号, 最后相加.
\end{itemize}
一些特殊的例子:

\[
\begin{vmatrix}A & C\\
\mathbf{0} & B
\end{vmatrix}=\begin{vmatrix}A & \mathbf{0}\\
D & B
\end{vmatrix}=|A|||B|(\text{暗示了矩阵乘法)}
\]

\[
\begin{vmatrix}0 & C\\
-I_{n} & B
\end{vmatrix}=|C|
\]

Vandermonde行列式.
\[
\begin{vmatrix}1 & 1 & \cdots & 1\\
x_{1} & x_{2} & \cdots & x_{n}\\
x_{1}^{2} & x_{2}^{2} & \cdots & x_{n}^{2}\\
\cdots & \cdots & \cdots & \cdots\\
x_{1}^{n-1} & x_{2}^{n-1} & \cdots & x_{n}^{n-1}
\end{vmatrix}=\prod_{1\leq j\leq i\leq n}(x_{i}-x_{j}).
\]

(与算法竞赛的联系) 这个公式保证了单位根构造多项式是可行的, 因此在FFT中间遇见过很多次.

\section{常见习题及其思想}

求行列式的值: 

求下列行列式的值: 
\begin{problem}
\label{prob:poly1}$D_{5}=\begin{vmatrix}b & c\\
a & b & c\\
 & a & b & c\\
 &  & a & b & c\\
 &  &  & a & b
\end{vmatrix}$
\end{problem}

\begin{sol*}
考虑递推关系. 不妨先按照第一行第一列展开, 有

$\begin{aligned}D_{5}= & \begin{aligned}b\begin{vmatrix}b & c\\
a & b & c\\
 & a & b & c\\
 &  & a & b
\end{vmatrix}-c\begin{vmatrix}a & c\\
 & b & c\\
 & a & b & c\\
 &  & a & b
\end{vmatrix}\end{aligned}
\\
= & bD_{4}-c\left(a\begin{vmatrix}b & c\\
a & b & c\\
 & a & b
\end{vmatrix}\right)\\
= & bD_{4}-acD_{3}
\end{aligned}
$ 

然后就可以按照这样的方法得到递推的解答. 不难发现, $D_{n}=bD_{n-1}+acD_{n-2}$是对于一般的$n$成立的. 
\end{sol*}
%
\begin{problem}
$D=\begin{vmatrix}a_{1}+x & a_{2} & a_{3} & a_{4}\\
-x & x\\
 & -x & x\\
 &  & -x & x
\end{vmatrix}$
\end{problem}

\begin{sol*}
不妨把后面三列的所有的元素加到第一列上面去, 这样就可以得到类似$\begin{vmatrix}x+\sum_{i=1}^{4}a_{i}\\
 & x\\
 & -x & x\\
 &  & -x & x
\end{vmatrix}$的矩阵. 展开之后就可以处理剩下的式子$\begin{vmatrix}x\\
-x & x\\
 & -x & x
\end{vmatrix}$, 就是比较轻松的. 根据\ref{prob:poly1}得到的答案是$x^{3}\left(x+\sum_{i=1}^{4}a_{i}\right)$. 

这个问题给我们另一个启发就是, 如果行与行之间的元素是相似或者可以相互抵消的, 可以把它们加到同一行上面去.
\end{sol*}
%
\begin{problem}
$D=\begin{vmatrix}a_{1} & 1 & 1 & 1\\
1 & a_{2}\\
1 &  & a_{3}\\
1 &  &  & a_{4}
\end{vmatrix}$
\end{problem}

\begin{sol*}
不妨提取出$a_{2},a_{3},a_{4}$, 并且最后可以让上面的1来与之相减. 具体的, 有

\[
D=a_{2}a_{3}a_{4}\begin{vmatrix}a_{1} & 1 & 1 & 1\\
1/a_{2} & 1\\
1/a_{3} &  & 1\\
1/a_{4} &  &  & 1
\end{vmatrix}=a_{2}a_{3}a_{4}\begin{vmatrix}a_{1}-\sum_{i=2}^{4}1/a_{i} & 0 & 0 & 0\\
1/a_{2} & 1\\
1/a_{3} &  & 1\\
1/a_{4} &  &  & 1
\end{vmatrix}=a_{2}a_{3}a_{4}\left(a_{1}-\sum_{i=2}^{4}1/a_{i}\right)
\]
\end{sol*}
\begin{problem}
\label{prob:poly-jordan-form}$D=\begin{vmatrix}x & 0 & 0 & \cdots & 0 & a_{0}\\
-1 & x & 0 & \cdots & 0 & a_{1}\\
0 & -1 & x & \cdots & 0 & a_{2}\\
 &  & \vdots & \ddots & \vdots & \vdots\\
0 & 0 & 0 & \cdots & x & a_{n-2}\\
0 & 0 & 0 & \cdots & -1 & x+a_{n-1}
\end{vmatrix}$
\end{problem}

\begin{sol*}
这个矩阵巧妙的把多项式和矩阵联系在了一起. 事实上, 他表示的是$x^{n}+a_{n-1}x^{n-1}+\cdots+a_{0}.$
事实上, 这个想法在后面证明多项式与矩阵的联系(Jordan标准型一节, 线性代数不讲)打下了坚实的基础. 下面我们来证明这件事情. 

\begin{align*}
D_{n-1}= & \begin{vmatrix}x & 0 & 0 & \cdots & 0 & a_{0}\\
-1 & x & 0 & \cdots & 0 & a_{1}\\
0 & -1 & x & \cdots & 0 & a_{2}\\
 &  & \vdots & \ddots & \vdots & \vdots\\
0 & 0 & 0 & \cdots & x & a_{n-2}\\
0 & 0 & 0 & \cdots & -1 & x+a_{n-1}
\end{vmatrix}\\
= & x\begin{vmatrix}x & 0 & \cdots & 0 & a_{1}\\
-1 & x & \cdots & 0 & a_{2}\\
 & \vdots & \ddots & \vdots & \vdots\\
0 & 0 & \cdots & x & a_{n-2}\\
0 & 0 & \cdots & -1 & x+a_{n-1}
\end{vmatrix}+\begin{vmatrix}0 & 0 & \cdots & 0 & a_{0}\\
-1 & x & \cdots & 0 & a_{2}\\
 & \vdots & \ddots & \vdots & \vdots\\
0 & 0 & \cdots & x & a_{n-2}\\
0 & 0 & \cdots & -1 & x+a_{n-1}
\end{vmatrix}\\
= & xD_{n-2}+\left(\begin{vmatrix}0 & \cdots & 0 & a_{0}\\
\vdots & \ddots & \vdots & \vdots\\
0 & \cdots & x & a_{n-2}\\
0 & \cdots & -1 & x+a_{n-1}
\end{vmatrix}\right)=xD_{n-2}.
\end{align*}

根据递推关系就可以得到这个结论. 这样的方法在\ref{prob:poly1}中也曾经用到过. 

与之同样有趣的是,$D=\begin{vmatrix}x & 0 & 0 & \cdots & 0 & a_{0}\\
-1 & x & 0 & \cdots & 0 & a_{1}\\
0 & -1 & x & \cdots & 0 & a_{2}\\
 &  & \vdots & \ddots & \vdots & \vdots\\
0 & 0 & 0 & \cdots & x & a_{n-1}\\
0 & 0 & 0 & \cdots & -1 & a_{n}
\end{vmatrix}$一个$n+1$阶的行列式表示的也是一个多项式$\sum_{i=0}^{n}a_{i}x^{i},$这次它的系数和指标是相对应的. 
\end{sol*}

