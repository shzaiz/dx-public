%% LyX 2.3.7 created this file.  For more info, see http://www.lyx.org/.
%% Do not edit unless you really know what you are doing.
\documentclass{ctexart}
\usepackage{amsmath}
\usepackage{amsthm}
\usepackage{fontspec}
\usepackage{geometry}
\geometry{verbose,tmargin=3cm,bmargin=3cm,lmargin=2cm,rmargin=2cm,headheight=2cm,headsep=2cm,footskip=2cm}
\usepackage{fancyhdr}
\pagestyle{fancy}
\usepackage{algorithm2e}
\usepackage{graphicx}
\usepackage[unicode=true]
 {hyperref}

\makeatletter

%%%%%%%%%%%%%%%%%%%%%%%%%%%%%% LyX specific LaTeX commands.
\providecommand\textquotedblplain{%
  \bgroup\addfontfeatures{Mapping=}\char34\egroup}

%%%%%%%%%%%%%%%%%%%%%%%%%%%%%% Textclass specific LaTeX commands.
\theoremstyle{definition}
 \newtheorem{example}{\protect\examplename}
\newenvironment{lyxcode}
	{\par\begin{list}{}{
		\setlength{\rightmargin}{\leftmargin}
		\setlength{\listparindent}{0pt}% needed for AMS classes
		\raggedright
		\setlength{\itemsep}{0pt}
		\setlength{\parsep}{0pt}
		\normalfont\ttfamily}%
	 \item[]}
	{\end{list}}
\theoremstyle{definition}
\newtheorem{defn}{\protect\definitionname}

\makeatother

\usepackage{babel}
\providecommand{\definitionname}{Definition}
\providecommand{\examplename}{Example}

\begin{document}
\title{Python程序设计}
\maketitle

\section{为什么要学习计算机编程}

计算机是一个笨重的工具, 在时间或空间限制下可以有效地执行许多重复的任务. 到现在为止, 计算机还不能分析问题并提出解决方案. 另一方面,
人类在分析和解决问题方面非常出色, 但很容易对重复的任务感到厌倦. 

人类通过利用他们的分析和解决问题的技能, 可以为可计算的问题提出算法(有限的指令, 与有限的输入一起工作, 产生输出) . 然后,
计算机可以遵循这些指令并产生一个答案. 

你可以将你生活中的各种多余的任务自动化. 一点点的Python可以给你的生活带来神奇的效果. 甚至提高你的生产力. 

\section{程序语言的语法(Syntax)和语义(Schematics)}

我们在学习英语的时候很重要的一部分是语法: 也就是什么样的语言是可以被接受(acceptable)的. 比如下面这个英文句子是没有语法错误的: 
\begin{verbatim}
Fuhai Zhu said that this test is only a small test, so don't panic.
\end{verbatim}
但是这句话不同人有不同的理解方式, 这就是这句话的语义. 
\begin{itemize}
\item 我们可以推断朱富海老师在安抚准备参加小测试学生们的情绪
\item 但是南京大学数学系的同学知道这是一个有名的梗: 上学期教高等代数的朱富海老师把线上期中考叫做``小测验'', 并在同学询问考试范围的时候微笑的答道:``从小学学的都考.''
\end{itemize}
\begin{quote}
简而言之(不严谨) , 现在我们有一个由一堆字符串和推导规则组成的形式系统, 语法决定了这个形式系统能生存什么样的字符串, 而至于这些字符串有什么样的含义则是语义的范畴.
语法类似材料, 语义类似与材料组成的各种建筑物, 我们可以通过语法研究语义层面的推导, 同时也可以从语义层面捕获语法中内涵的结构,
其实语法和语义是相互区别又紧密联系, 即从范畴论的角度看语法和语义是伴随的(其实不同的人做数学证明可以有不同的风格: 偏语法和偏语义,
不过大部分数学家更喜欢语义风格的证明, 可能因为更直观, 更容易被人脑接受\textasciitilde\textasciitilde ) 

作者: 知乎用户 链接: https://www.zhihu.com/question/31347357/answer/892133941
来源: 知乎 著作权归作者所有. 商业转载请联系作者获得授权, 非商业转载请注明出处. 
\end{quote}

\section{Python代码执行可视化: 一个网站}

\subsection{Pythontutor: 代码执行可视化}

这是\href{https://pythontutor.com/visualize.html\#mode=display}{Python代码执行可视化}的机器,
我们可以用一个小程序来测试之.
\begin{verbatim}
for i in range(10):
    print(i**2)
\end{verbatim}
点击Visualize Execution就可以了, 你可以点击Next来继续模拟执行下一步.

\includegraphics[width=1\columnwidth,height=0.6\columnwidth,keepaspectratio,bb = 0 0 200 100, draft, type=eps]{tex/FIg/vis-py.png}

\subsection{需要了解的内容}

什么是Global Frame, Object? 暂时先不用管. 不过你确实可以看到点击Next的时候Print output一栏一步一步的模拟了你的代码. 

\section{Python程序的语法和语义}

可以知道, 代码按照行数执行, 一次执行一行, 每一次执行计算机内部结构的状态(右侧的面板). 下面我们化繁为简, 来看一看一个系统(数学意义上)能够完成任何人类完成的操作需要的最小可能的操作是什么. 

\subsection{能够写出任何程序的最小指令集}

像数学的公理体系那样, 我们自然希望得到一个最小的指令集合, 并且我们可以用来写出任何的程序. 我们不妨从日常生活中找一点灵感吧.
\begin{example}
(等红绿灯) 观察红绿灯, \textbf{如果}是绿灯, 那就通过这个路口; \textbf{否则}继续等待. (遵纪守法的好公民)

(做作业) 明确今天的作业范围, 从\textbf{第一题}开始写, 写完题目\textbf{或者}一题目没有思路之后做\textbf{下一道}题,
\textbf{直到}做完所有的问题.

(排序成绩单) 获得班上同学的所有\textbf{成绩单}, 拿一张新的白纸打好\textbf{表格}, \textbf{每一次}从成绩单中选取最大的分数,
把那一行\textbf{抄写到}新的白纸上. 之后把原来那张纸上的内容划去. \textbf{一直重复下去}, \textbf{直到}原来的成绩单上没有任何可以被划去的内容.
\end{example}
我们需要找一些东西来具象化我们脑子中的``红绿灯的状态'', ``现在在做作业的题目编号'', 这些内容, 因此我们就希望把这些抽象出来.
因此我们有了变量的概念, 也就是值存在的空间. 

把上面的三个内容转化成伪代码(不唯一)就是:
\begin{verbatim}
------ GO THROUGH CONJUNCTION ------
if traffic light's color is green:
    go pass by
else
    wait

------ DO HOMEWORK ------
range = [a..b]
working on problem = a
while working on problem <= b:
    finished this problem or can't work out
    working on problem = working on problem+1

------ SORT EXAM SCORE ------
list = get the source table
result = empty(for now)
while list is not empty
    k= get the max element of list
    write k to the next line of result
show result
\end{verbatim}
其实要是能够构造出任何程序的原材料并不复杂. 无非就是变量的赋值, 判断, 跳转, 终止. 也就是, 如果你能声称有一套系统可以自动化的解决这四个内容,
那么这个系统就具有机械化地做任何人类做的事情. 换句话说, 你可以用这个工具创造整个世界. 

其实最早这个想法是在计算机诞生之前人们孜孜以求的问题. Alan Turing 在1936年就提出了这样的设想. 他就是由只一条(无限长)的纸带和一根笔(可以改纸带的内容,
并且查看纸带的内容并据此做判断), 并且有一个程序(墙上的表格), 指示下一步要往哪转移. 只要能够移动读写头, 写纸带的某一个格子,
读纸带的某一个格子, 跳转, 以及终止, 这个机器就和我们人类的计算能力等价. 

\includegraphics[width=0.6\columnwidth,height=0.6\linewidth,keepaspectratio,bb = 0 0 200 100, draft, type=eps]{tex/FIg/trm.png}
\begin{example}
运行上面图片的程序, 左右按照我们的左右进行(规定A\textbf{B}C右移一格是AB\textbf{C}).

(1) 现在机器的状态是A(头部的字母), 看到的是1(放大镜的字母)

(2) 于是把当前的格子改为1, 纸带向右移动一格, 然后停机.

\textbf{假设}当前纸带的放大镜看到的是0, 再运行一次:

状态:A 纸带状态: 0 1 1 1 \textbf{0} 1 1 0 0

(1)现在机器的状态是A(头部的字母), 看到的是0(放大镜的字母), 执行第一行第一列的指令$1(\text{改\text{为1)}}\rightarrow(\text{向\text{右移动一格)}}B(\text{状\text{态改为B)} . }$

状态:B 纸带状态: 0 1 1 1 1 \textbf{1} 1 0 0

(2)现在机器的状态是B(头部的字母), 看到的是1(放大镜的字母), 执行第二行第二列的指令$1(\text{改\text{为1)}}\rightarrow(\text{向\text{右移动一格)}}B(\text{状\text{态改为B)} . }$

状态:B 纸带状态: 0 1 1 1 1 1 \textbf{1} 0 0

(3)现在机器的状态是B(头部的字母), 看到的是1(放大镜的字母), 执行第二行第二列的指令$1(\text{改\text{为1)}}\rightarrow(\text{向\text{右移动一格)}}B(\text{状\text{态改为B)} . }$

状态:B 纸带状态: 0 1 1 1 1 1 1 \textbf{0 }0

(4)现在机器的状态是B(头部的字母), 看到的是0(放大镜的字母), 执行第二行第一列的指令$0(\text{改\text{为0)}}\rightarrow(\text{向\text{右移动一格)}}C(\text{状\text{态改为C)} . }$

状态:C 纸带状态: 0 1 1 1 1 1 1 0 \textbf{0}

(5)现在机器的状态是C(头部的字母), 看到的是0(放大镜的字母), 执行第三行第一列的指令$1(\text{改\text{为1)}}\leftarrow(\text{向\text{左移动一格)}}C(\text{状\text{态改为C)} . }$

状态:C 纸带状态: 0 1 1 1 1 1 1 \textbf{0} 1

(6)现在机器的状态是C(头部的字母), 看到的是0(放大镜的字母), 执行第三行第一列的指令$1(\text{改\text{为1)}}\leftarrow(\text{向\text{左移动一格)}}C(\text{状\text{态改为C)} . }$

状态:C 纸带状态: 0 1 1 1 1 1 \textbf{1} 1 1

(7)现在机器的状态是C(头部的字母), 看到的是1(放大镜的字母), 执行第三行第二列的指令$1(\text{改\text{为1)}}\leftarrow(\text{向\text{左移动一格)}}A(\text{状\text{态改为A)} . }$

状态:A 纸带状态: 0 1 1 1 1 \textbf{1} 1 1 1

(8)现在机器的状态是A(头部的字母), 看到的是1(放大镜的字母), 执行第一行第二列的指令$1(\text{改\text{为1)}}\rightarrow(\text{向\text{右边移动一格)}}\dagger(\text{停\text{机)}}$

状态:A 纸带状态: 0 1 1 1 1 1 \textbf{1} 1 1
\end{example}
下面我们来看一看为什么说可以用Python创造整个世界.

\subsection{在这之前: 寻求网络资源}

请认真阅读并实践(无论是在脑子还是在交互器里面)文档https://docs.python.org/zh-cn/3/tutorial/introduction.html的内容.
可以让你了解更多易于理解的东西. 如果文章中有描述C和Pascal的句子, 忽略它就可以. 

\subsection{Python中的整个世界}

下面的内容其实不用单独记忆, 只要明确有哪些语句, 这些语句造成的效果是什么就行了. 如果看到有任何的问题, 可以去搜一搜词典. 

下面会有两个术语(term), 分别是表达式(expression)和过程(procedure), 表达式可以暂且认为是形如x+12,
{[}2{]}{*}3这样的可以进行计算的内容, 过程就是一系列执行的过程, 不一定要能得到值. 我们用一个例子感受一下.
\begin{lyxcode}
def~InsertionSort(A):~~~~~~~~~~~~~~~~~~~~~~~~~~~~

~~~~for~j~in~range(1,~len(A)):~~~~~~~~~~~~~~~~~~~\#Proc

~~~~~~~~key~=~A{[}j{]}~~~~\#A{[}j{]},key~are~expr~\#Proc~~~\#|

~~~~~~~~i~=~j~-~1~~~~~\#j-1,i~are~an~expr~\#v~~~~~~\#|

~~~~~~~~while~(i~>=0)~and~(A{[}i{]}~>~key):~~~~~~~~~~\#|~~~

~~~~~~~~\#~~~~<-expr->~~~~<-{}-{}-{}-expr-{}-{}-{}-{}->~~~~~~~~~\#|~~~

~~~~~~~~\#~~~~<-{}-{}-{}-{}-{}-{}-{}-{}-{}-{}-expr-{}-{}-{}-{}-{}-{}-{}-{}-{}-{}->~~~~~~~~\#|~~~

~~~~~~~~~~~~A{[}i+1{]}~=~A{[}i{]}~\#A{[}i{]}is~expr~~~\#Proc~~~\#|

~~~~~~~~~~~~i~=~i~-~1~~~~~\#i-1~is~expr~~~\#v~~~~~~\#|

~~~~~~~~A{[}i+1{]}~=~~~~key~~~~~~~~~~~~~~~~~~~~~~~~~~\#v

~~~~~~~<-expr->~~<-expr->
\end{lyxcode}
其实上面的Proc表示过程, 然后右边的是一个字符画, 表示$\downarrow$, <-expr->其实表示的意思是$\underbrace{\texttt{example}}_{\text{expr}}$这一段是表达式. 

重要的是, 把示例代码放到上面提到的可视化网站里面看一看就会很清楚, 很多概念都是不用记忆的. 

\subsubsection{变量的定义与赋值}
\begin{defn}
(变量的赋值) 变量名=变量的值
\end{defn}
语义: 

下面我们给出注解:
\begin{itemize}
\item 在不加修饰的情况下, 变量的名称只在当前的缩进块内有效
\item 命名是用来指代对象的. 这就是为什么有时候可视化工具里面Frames后面有一个箭头指着Objects.
\item 如果用一个变量=另一个变量, 大多数情况是现计算出来右手边表达式的值之后给左手边的变量. 有时候一些文章里面写作$lhs\leftarrow rhs$.
\end{itemize}
\begin{lyxcode}
b=114514

a=b+1~\#执行完本句之后a=114515

b~=~b+1~\#~执行完本句之后b=114515,~a=114515不变

a~=~b+1~\#~执行完本句之后a=114515,~b=114516
\end{lyxcode}
\begin{itemize}
\item \textbf{在Python中}, 变量的值的类型可以是任意的. 因为Python声明变量的时候没有说明类型.
\end{itemize}
\begin{lyxcode}
a=''Fuhai~Zhu~teached~Advanced~Algebra''~\#~a现在是字符串

a=1~\#~a现在是整数

a=None~\#~None是一个关键字,~表示什么都没有.
\end{lyxcode}
\begin{itemize}
\item 如果没有定义就使用了一个变量, 通常就会有如下的报错:
\end{itemize}
\begin{lyxcode}
print(a+1)

~~~~~~\textasciicircum{}

Traceback~(most~recent~call~last):~~~File~\textquotedbl <stdin>\textquotedbl ,~line~1,~in~<module>~

NameError:~name~'n'~is~not~defined

(命名错误:~名称'n'没有定义)
\end{lyxcode}
什么是Traceback? stdin又是什么? 后面可能会注意到. 

可能经常会常用的变量类型: 数字、字符串、列表. 这时候可以参看官方文档https://docs.python.org/zh-cn/3/tutorial/introduction.html来继续.

\subsubsection{控制语句: 判断与循环}
\begin{defn}
(条件判断) 可以使用if语句进行条件判断, 一般的, 有如下的形式:

\noindent\begin{minipage}[t]{1\columnwidth}%
\begin{lyxcode}
if~表达式1:

~~~~过程1

elif~表达式2:~\#~可以有零个或者多个elif,~但是else后面不能有elif

~~~~过程2

else:

~~~~过程r
\end{lyxcode}
%
\end{minipage}
\end{defn}
语义: 它通过逐个计算\textbf{表达式}, 直到发现一个\textbf{表达式}为真, 并且执行使\textbf{表达式}为\textbf{真}的这个\textbf{过程}(完成后\textbf{不}执行或计算if语句的其他部分的判断\textbf{表达式})
. 如果所有表达式都为false, 如果存在else下方语句块的过程.

下面我们同样给出注记和例子.
\begin{itemize}
\item 什么是真? 什么是假? 我们会在后面探讨. 首先可以认为非0数字和True是真, 0和False和None是假. 
\end{itemize}
\begin{lyxcode}
if~\textquotedbl AK\textquotedbl :

~~~~print(\textquotedbl AK\textquotedbl )~\#~会输出AK,~这是怎么判断的?(后续会回答)
\end{lyxcode}
\begin{itemize}
\item 可以用逻辑运算符 and(且) or(或) not(非) 进行逻辑表达, 比如
\end{itemize}
\begin{lyxcode}
zgw~=~0

kertz~=~1

ak~=~1

cmo~=~1

if~kertz~and~ak~and~cmo~:

~~~~print(``Zixuan~Yuan~got~full~mark~in~CMO'')

elif~zgw~and~ak~and~cmo:

~~~~print(``zgw~got~full~mark~in~CMO'')

else:

~~~~print(``zgw~is~such~a~noob'')

\#~会输出Kertz~got~full~mark~in~CMO,~由于已经找到了一个表达式的值为真的

\#~表达式,~所以执行完print(``Zixuan~Yuan~got~full~mark~in~CMO'')之后就

\#~会跳转到这个语句块的尾部了.~不会执行print(``zgw~is~such~a~noob'').

\#~(为自己菜爆的数学基础做了一个掩盖(大雾))
\end{lyxcode}
\begin{itemize}
\item 如果结构不完整, 或者在else之后还有elif, 那么就会出发形如这样的错误:
\end{itemize}
\begin{lyxcode}
例子1.py-{}-{}-{}-{}-{}-{}-{}-{}-{}-{}-{}-{}-

if~True:

print(\textquotedbl Err\textquotedbl )

-{}-{}-{}-

File~\textquotedbl main.py\textquotedbl ,~line~3

~~~~print(\textquotedbl Err\textquotedbl )

~~~~\textasciicircum ~IndentationError:~expected~an~indented~block

(缩进错误:~我预期有一个带着缩进的语句块,~但是没有)

例子2.py-{}-{}-{}-{}-{}-{}-{}-{}-{}-{}-{}-{}-{}-{}-

if~False:

~~~~print(1)

else:

~~~~print(2)

elif~True:

~~~~print(3)

-{}-{}-

~~File~\textquotedbl main.py\textquotedbl ,~line~5

~~~~elif~True:

~~~~\textasciicircum ~SyntaxError:~invalid~syntax

(语法错误:~无效的语法)
\end{lyxcode}
\begin{defn}
(while循环) 可以使用if语句进行条件判断, 一般的, 有如下的形式:

\noindent\begin{minipage}[t]{1\columnwidth}%
\begin{lyxcode}
while~表达式:

~~~~过程1

else:~\#~可以有,~也可以没有

~~~~过程2
\end{lyxcode}
%
\end{minipage}
\end{defn}
语义: 这样反复测试表达式, 如果为真, 则执行\textbf{过程1};如果表达式为假(这可能是第一次测试) , 则执行else子句的\textbf{过程2}(如果存在的话)
, 然后循环终止. 在\textbf{过程1}中执行的\textbf{break}语句会终止循环, 且不执行else子句的\textbf{过程2}.
在\textbf{过程1}中执行的continue语句跳过\textbf{过程1}的continue语句之后的其余部分, 然后立刻回到测试表达式语句. 

有了循环, 我们就可以解读这个东西:
\begin{lyxcode}
def~InsertionSort(A):

~~~~j=1

~~~~while(j<len(A)):

~~~~~~~~key~=~A{[}j{]}

~~~~~~~~i~=~j~-~1

~~~~~~~~while~(i~>=0)~and~(A{[}i{]}~>~key):

~~~~~~~~~~~~A{[}i+1{]}~=~A{[}i{]}

~~~~~~~~~~~~i~=~i~-~1

~~~~~~~~A{[}i+1{]}~=~key

~~~~~~~~j=j+1

~~~~return~A

InsertionSort({[}1,1,4,5,1,4{]})
\end{lyxcode}
这个做的事情就和排序成绩类似. 

\subsubsection{程序的终止}
\begin{defn}
Python程序的终止可能包含有如下的情况:

(1) 执行到了最后一条语句, 且没有下一条语句可以执行;

(2) 程序有没有被处理的异常;

(3) 通过语句exit(0)退出. 
\end{defn}
因此, 我们就得到了最小的可以(理论上)执行任何与人类计算能力等价的模型

这些内容看上去十分的平凡, 但是通过一些过程的复合, 我们就能看到更多的魔力.

\subsection{函数: 整合相似过程}

我们可以把相似的过程写在一起, 为了简洁和可维护.

下面, 可以阅读https://docs.python.org/zh-cn/3/tutorial/controlflow.html\#defining-functions
的4.7, 4.8.1-4.8.6节的内容, 把所有代码是怎么执行的放在pythontutor里面模拟着看一遍. 文字可以不用看,
但是代码一定要执行一遍. 

\subsubsection{递归(Recursion)过程和栈帧(Stack Frame)}

观察下面的代码, 可能难以想象是怎么执行的:
\begin{lyxcode}
def~fib(n):

~~~~if(n==1):

~~~~~~~~~return~1

~~~~if(n==2):

~~~~~~~~~return~1

~~~~else:

~~~~~~~~~return~fib(n-1)~+~\textbackslash{}

~~~~~~~~~~~~~~~~fib(n-2)

fib(5)
\end{lyxcode}
像这样用自己调用自己的函数调用通常叫做递归(recursion). 一个关于递归的有趣定义是:
\begin{quote}
递归的定义: 如果你没有理解什么是递归, 那么参见递归. 
\end{quote}
事实上, 我们可以把它放在pythontutor里面执行一下, 发现如下的规则:
\begin{itemize}
\item 原来的程序就像是一张纸, 上面标注着当前执行到的行数;
\item 每次函数调用的时候, 就会在一张新的纸片上抄下来调用的内容, 并且代换传进来的参数;
\item 把这个内容放在原来纸片上面, 然后从第一行开始执行;
\item 执行完的纸片扔掉.
\end{itemize}
看上去就像是:
\begin{itemize}
\item 你在晚自习上看课外书(执行原来的函数) 
\item 老师来了, 让你写作业(函数调用) 
\item 你把作业叠放在课外书上, 开始做作业 (执行函数)
\item 做完作业之后你把作业扔了继续看课外书(回到原来的函数)
\end{itemize}
像羽毛球球桶那样, 只能从一个方向插入, 弹出的内容的东西叫做``\textbf{栈(stack)}'', 由于这些内容通常都是一些数据,
由此我们用术语\textbf{数据结构}(data structure)来描述. 能被取出来的那个元素是\textbf{栈顶(top
of the stack)}, 在这个可视化工具里面用蓝色标示出来了. 

Traceback就是出错之后, Python顺着栈一层一层找的结果. Trace是跟踪, back是返回, 意思可能就是说堆栈的\textbf{回溯(traceback)}.
\end{document}
