%% LyX 2.3.7 created this file.  For more info, see http://www.lyx.org/.
%% Do not edit unless you really know what you are doing.
\documentclass[a4paper,UTF8]{ctexart}
\usepackage[T1]{fontenc}
\pagestyle{headings}
\usepackage{xcolor}
\usepackage{amsmath}
\usepackage{amsthm}
\usepackage{stackrel}
\usepackage{esint}
\PassOptionsToPackage{normalem}{ulem}
\usepackage{ulem}
\usepackage[unicode=true,
 bookmarks=true,bookmarksnumbered=true,bookmarksopen=true,bookmarksopenlevel=3,
 breaklinks=true,pdfborder={0 0 0},pdfborderstyle={},backref=page,colorlinks=true]
 {hyperref}
\hypersetup{pdftitle={大学物理C1回顾(2023春)},
 pdfauthor={张桄玮}}

\makeatletter

%%%%%%%%%%%%%%%%%%%%%%%%%%%%%% LyX specific LaTeX commands.
\pdfpageheight\paperheight
\pdfpagewidth\paperwidth

\newcommand{\lyxmathsym}[1]{\ifmmode\begingroup\def\b@ld{bold}
  \text{\ifx\math@version\b@ld\bfseries\fi#1}\endgroup\else#1\fi}


%%%%%%%%%%%%%%%%%%%%%%%%%%%%%% Textclass specific LaTeX commands.
\theoremstyle{definition}
\newtheorem{example}{\protect\examplename}

%%%%%%%%%%%%%%%%%%%%%%%%%%%%%% User specified LaTeX commands.
% 如果没有这一句命令,XeTeX会出错,原因参见
% http://bbs.ctex.org/viewthread.php?tid=60547
\DeclareRobustCommand\nobreakspace{\leavevmode\nobreak\ }
\usepackage{tikz}
\usepackage[pdf]{pstricks}
\usepackage{circuitikz}

\makeatother

\providecommand{\examplename}{例}

\begin{document}
\title{大学物理C1: 复习与回顾}
\author{AUGPath}

\maketitle
\global\long\def\dd{\mathrm{d}}%

\global\long\def\bs#1{\boldsymbol{#1}}%

\global\long\def\vec#1{\overrightarrow{#1}}%
\global\long\def\w{\omega}%
\global\long\def\bt{\beta}%
\global\long\def\red#1{{\color{red}#1}}%
\global\long\def\eqs{\epsilon}%
\global\long\def\ef{\frac{1}{4\pi\eqs_{0}}}%

\newcommand{\incfig}[1]{\begin{center}\includegraphics[width=.4\textwidth]{figs/#1.eps}\end{center}}
\newcommand{\incfigw}[1]{\begin{center}\includegraphics[width=.8\textwidth]{figs/#1.eps}\end{center}}


\section*{写在前面}

本文试图通过一些例子完成本学期大学物理知识体系的构建. 正如Warrior和众多大学教师一直建议的那样, 学习整理的过程永远要比结果更重要,
更值得享受. 
\begin{quote}
Yeats有一句名言:“教育不是注满一桶水, 而是点燃一把火. ”他说得既对也错. 你必 须“给桶注一点水”, 这本书当然可以帮助你完成这部分的教育.
毕竟, 当你去 Google 面试 时, 他们会问你一个关于如何使用信号量的技巧问题, 确切地知道信号量是什么感觉真好,  对吧?

但是, Yeats的更主要的观点显然是正确的:教育的真正要义是让你对某些事情感兴趣, 这不仅仅是搞明白课上学习的东西, 得个好成绩.
更重要的是课后可以独立了解这个主题的更多东西. 正如我们的父亲(作者的父亲 Vedat Arpaci)曾说过的, “在课堂以外学习.
”

我们编写本书的目的是激发你对操作系统的兴趣. 这就可以让你能自行了解有关操作系统的更多知识, 进而 可以与你的教授讨论, 甚至参与该领域相关的那些令人兴奋的研究.
这是一个伟大的 领域: 这里有充满了激动人心和精妙的想法, 它们以深刻而重要的方式塑造了计算历史. 虽然我们知道这 团火不会为你们所有人点燃,
但我们希望这能对许多人, 甚至是少数人有所帮助. 因为一旦这团火 被点燃, 那你就真正有能力做出伟大的事情. 因此, 教育过程的真正意义在于:一往无前,
学习更 多的新的和引人入胜, 美妙绝伦的知识, 通过学习不断成熟, 最重要的是, 找到能为你点燃那团火的人.  – 《Operating
Systems: Three Easy Pieces》
\end{quote}

\part{核心内容回顾}

\section{运动学}
\begin{description}
\item [{一.}] 运动与坐标系的选取. 
\begin{description}
\item [{1.}] 矢量与微积分的观点
\item [{2.}] 描述运动的三个物理量
\begin{description}
\item [{(1)}] 微分关系

\begin{align*}
\mathbf{r} & =x_{i}\mathbf{i}+y_{j}\mathbf{j}+z_{k}\mathbf{k}\\
\mathbf{v} & =v_{i}\mathbf{i}+v_{j}\mathbf{j}+v_{k}\mathbf{k}\\
\mathbf{a} & =a_{i}\mathbf{i}+a_{j}\mathbf{j}+a_{k}\mathbf{k}
\end{align*}
以及有微分关系$\mathbf{v}=\frac{\dd\bs r}{\dd\bs t},\bs a=\frac{\dd\bs v}{\dd\bs t}$, 
\begin{description}
\item [{(a)推论:}] 求$\bs{v,t}$对于$\bs x$的关系: 使用关系$\frac{\dd\bs r}{\dd\bs x}\frac{\dd\bs x}{\dd\bs t}.$
\end{description}
\item [{(2)}] 积分关系
\begin{align*}
\dd\bs r & =\bs v\dd t,\vec r=\vec{r_{0}}+\int_{0}^{t}\vec v\dd t\\
\dd\bs v & =\bs a\dd t,\vec v=\vec{a_{0}}+\int_{0}^{t}\vec a\dd t
\end{align*}
\end{description}
\item [{3.}] 坐标系的选取: 
\begin{description}
\item [{(1)}] 直角坐标
\item [{(2)}] 自然坐标
\item [{(3)}] 极坐标
\end{description}
\item [{4.}] 圆周运动
\begin{description}
\item [{(1)}] 描述圆周运动的物理量: 角速度$\omega$, 角加速度$\beta$
\begin{description}
\item [{(a)}] 之间关系: $\omega=\frac{\dd\theta}{\dd t},\beta=\frac{\dd\omega}{\dd t};\theta=\theta_{0}+\int_{0}^{t}\omega\dd t,\omega=\omega_{0}+\int_{0}^{t}\beta\dd t.$
\end{description}
\item [{(2)}] 自然坐标系中的加速度: $\vec a=\vec{a_{n}}+\vec{a_{t}}.$
\begin{description}
\item [{(a)}] $a_{n}:$法向分量, 指向圆心$a_{n}=\frac{v^{2}}{R}=\omega^{2}R$
\item [{(b)}] $a_{t}:$切向分量, 沿轨迹方向 $a_{t}=\frac{\dd v}{\dd t}=R\frac{\dd\omega}{\dd t}=R\beta.$
\end{description}
\end{description}
\end{description}
\item [{二.}] Newton运动定律及其延伸
\begin{description}
\item [{1.}] Newton第二定律: $\vec F=\frac{\dd(m\vec v)}{\dd t},\vec F=m\vec a;$
\item [{2.}] 动量与冲量
\begin{description}
\item [{定义.}] $\vec F\dd t=\dd(m\vec v),\dd\vec I=\dd\vec P,\vec I=\Delta\vec P$
\item [{质点动量定理.}] $\int_{t_{1}}^{t_{2}}\vec F\dd t=m\vec{v_{2}}-m\vec{v_{1}}$
\item [{质点系动量定理.}] $\int_{t_{1}}^{t_{2}}\left(\sum_{i}\vec{F_{i_{\text{外}}}}\right)\dd t=\sum_{i}m_{i}\vec{v_{i}}_{2}-\sum_{i}m_{i}\vec{v_{i}}_{1}$.
\item [{动量守恒定律.}] 当$\sum_{i}\vec{F_{i_{\text{外}}}}=0,\sum_{i}m_{i}v_{i}$守恒. 
\end{description}
\item [{3.}] \textcolor{teal}{\uwave{角动量}}
\begin{description}
\item [{定义.}] $\vec L=\vec r\times m\vec v$. 其中$r$是从原点当前点的位置矢量. 
\item [{角动量守恒定律:}] 有心力(受力指向圆心, 外力矩为0)
\end{description}
\item [{4.}] 功: $A=\int_{\vec{r_{1}}}^{\vec{r_{2}}}\vec F\cdot\dd\vec r$,
可以分坐标系求和. 
\item [{5.}] 保守力场中$P$点的势能: $E_{p}(P)=\int_{(P)}^{(\text{零势能点)}}\vec{F_{\text{保守}}}\cdot\dd\vec r$.
\item [{6.}] 质点的动能定理: $A_{\text{外}}+A_{\text{内}}=\Delta\left(\sum_{i}E_{k_{i}}\right)=\sum_{i}\frac{1}{2}m_{i}v_{i_{2}}^{2}-\sum_{i}\frac{1}{2}m_{i}v_{i_{1}}^{2}.$
\item [{7.}] 功能原理: $A_{\text{外}}+A_{\text{非保守内}}=\Delta\left(\sum_{i}E_{i}\right),E=E_{k}+E_{P}$. 
\begin{description}
\item [{机械能守恒定律:}] 当只有保守力做功的时候, $\sum_{i}E_{i}$守恒.
\end{description}
\end{description}
\item [{三.}] 刚体的定轴转动
\begin{description}
\item [{1.}] 运动的描述
\begin{description}
\item [{描述:}] 圆周运动
\item [{刚体定轴转动定律:}] $M=J\beta,J=\sum_{i}m_{i}r_{i}^{2}=\int_{M}r^{2}\dd m$.
\end{description}
\item [{2.}] \textcolor{teal}{\uwave{刚体定轴转动的动能定理}}
\begin{description}
\item [{描述:}] $A=\Delta E_{k},A=\int_{\theta_{1}}^{\theta_{2}}M\dd\theta,E_{k}=\frac{1}{2}J\omega^{2}$
\item [{机械能守恒定律:}] 当只有保守力做功, $\sum_{i}E_{i}$守恒. 写作
\[
E=E_{k}+E_{p}=\left(\frac{1}{2}mv^{2}+\red{\frac{1}{2}J\omega^{2}}\right)+\left(mgh+\red{mgh_{\text{重心}}}\right)
\]
\end{description}
\item [{3.}] \textcolor{teal}{\uwave{角动量}}: $L=J\omega$, 方向垂直于转动平面. 
\begin{description}
\item [{角动量守恒.}] 当合外力矩为0, $L=J\omega$守恒
\end{description}
\end{description}
\item [{四.}] 相对论
\begin{description}
\item [{1.}] 基本理论
\begin{description}
\item [{(a)}] 相对性原理
\item [{(b)}] 光速不变性
\end{description}
\item [{2.}] 时间/地点相同的变换
\begin{description}
\item [{变换因子}] 
\[
\gamma=\frac{1}{\sqrt{1-\frac{u^{2}}{c^{2}}}}\geq1
\]
\end{description}
\item [{3.}] \textcolor{teal}{\uwave{Lorentz变换}}: 相对于$x-x'$方向的时空间隔变换:
\begin{align*}
\Delta x=\gamma(\Delta x'+u\Delta t') & ,\Delta x'=\gamma(\Delta x-u\Delta t)\\
\Delta t=\gamma\left(\Delta t'+\frac{u\Delta x'}{c^{2}}\right) & ,\Delta t'=\gamma\left(\Delta t'-\frac{u\Delta x'}{c^{2}}\right)
\end{align*}
\item [{4.}] \textcolor{teal}{\uwave{相对论的动力学关系}}
\begin{description}
\item [{(a)}] 运动的质量. $m=\gamma m_{0}$
\item [{(b)}] 相对论动量. $p=\gamma m_{0}u$
\item [{(c)}] 相对论动能. $E_{k}=E-E_{0}$
\item [{(d)}] 动能, 静止能量: $E=\gamma m_{0}c^{2},E_{0}=m_{0}c^{2}$.
\end{description}
\end{description}
\end{description}
%

\section{电磁学}
\begin{description}
\item [{一.}] 真空中的静电场
\begin{description}
\item [{1.}] 点电荷模型
\begin{description}
\item [{a.}] 库伦定律: $\vec f=\frac{1}{4\pi\epsilon_{0}}\frac{q_{1}q_{2}}{r^{2}}\vec{e_{r}}$
\item [{b.}] 点电荷形成的电场: $\vec E=\ef\frac{q}{r^{2}}\vec{e_{r}}$
\end{description}
\item [{2.}] 电场叠加原理: 
\[
\vec E=\sum_{i}\vec{E_{i}}=\sum_{i}\ef\frac{q_{i}}{r_{i}^{2}}\vec{e_{r_{i}}}=\int_{Q}\dd\vec E=\int_{Q}\ef\frac{\dd q}{r^{2}}\vec{e_{r}}.
\]

\begin{description}
\item [{a.}] 模型电场
\begin{description}
\item [{长直导线.}] $\vec E=\frac{1}{2\pi\eqs_{0}}\frac{\lambda}{r}\vec{e_{i}}$.
\item [{大平面.}] $\vec E=\frac{\sigma}{2\eqs_{0}}\vec{e_{n}}.$
\end{description}
\item [{b.}] Gauss定理$\oint_{S}\vec E\cdot\dd\vec S=\frac{1}{\eqs_{0}}\sum_{i}q_{i\text{内}},\phi_{c}=\int_{S}\vec E\cdot\dd\vec S.$
\begin{description}
\item [{应用:}] \textcolor{teal}{\uwave{用Gauss定理求对称性电场分布: }}
\begin{description}
\item [{球体:}] $E4\pi r^{2}=q_{in}/\eqs_{0}\implies E=\frac{q_{in}}{\eqs_{0}4\pi r^{2}}$
\item [{圆柱体:}] $E2\pi rh=q_{in}/\eqs_{0}\implies E=\frac{q_{in}}{\eqs_{0}2\pi rh}$
\item [{平面:}] $E2S_{\text{底}}=q_{in}/\eqs_{0}\implies E=\frac{q_{in}}{\eqs_{0}2S_{\text{底}}}$
\end{description}
\end{description}
\end{description}
\item [{3.}] 电势能与电场力做的功
\begin{description}
\item [{a.}] 电场力做的功$W_{a}-W_{b}=q_{0}(V_{a}-V_{b})=q_{0}\int_{(a)}^{(b)}\vec E\cdot\dd\vec r$
\item [{b.}] 电势 $V_{P}(P)=\int_{(P)}^{(\text{零势点)}}\vec E\cdot\dd\vec r$
\item [{c.}] 点电荷的电势 $V_{\text{点电荷}}=\ef\frac{q}{r}$
\item [{d.}] 电势的叠加 
\[
V_{\text{点电荷系}}=\sum_{i}V_{i}=\sum_{i}\ef\frac{q_{i}}{r_{i}}=\int_{Q}\dd V=\int_{Q}\ef\frac{\dd q}{r}.
\]
\end{description}
\end{description}
\item [{二.}] 导体和电介质
\begin{description}
\item [{1.}] 导体的静电平衡条件
\begin{description}
\item [{最终形态:}] $E_{\text{内}}=0,E_{\text{外表面}}=\frac{\sigma}{\eqs_{0}}.$导体是等势体,
其内无净电荷导体存在时静电场的分布
\end{description}
\item [{2.}] 电介质
\begin{description}
\item [{(a)}] 场强分布: $\vec E=\vec{E_{0}}/\eqs_{r}$
\item [{(b)}] 电偶极矩: $\vec p=q\vec l$
\item [{(c)}] 极化强度: $\vec P=n\vec p$
\end{description}
\item [{3.}] $\vec D$的Gauss定理$\oint_{S}\vec E\cdot\dd\vec S=\frac{1}{\eqs_{0}}\sum\left(q_{0\text{内}}+q_{\text{内}}^{'}\right),$意味着
\[
\oint_{S}\vec D\cdot\dd\vec S=\sum q_{0\text{内}},D=\eqs_{0}\eqs_{e}\vec E.
\]

\begin{description}
\item [{应用:}] \textcolor{teal}{\uwave{用Gauss定理求对称性电场分布: }}
\begin{description}
\item [{球体:}] $D4\pi r^{2}=q_{in}\implies E=\frac{q_{in}}{\eqs_{0}\eqs_{r}4\pi r^{2}}$
\item [{圆柱体:}] $D2\pi rh=q_{in}\implies E=\frac{q_{in}}{\eqs_{0}\eqs_{r}2\pi rh}$
\item [{平面:}] $D2S_{\text{底}}=q_{in}\implies E=\frac{q_{in}}{\eqs_{0}\eqs_{r}2S_{\text{底}}}$
\end{description}
\end{description}
\item [{4.}] 电容 $C=Q/U$
\begin{description}
\item [{三类电容器的电容}]~
\end{description}
\item [{5.}] 电场能量: 
\[
W_{e}=\frac{QU}{2}=\frac{CU^{2}}{2}=\frac{Q^{2}}{2C}=\red{\int_{\Omega}\frac{1}{2}DE\dd V}.
\]
\end{description}
\item [{三.}] 磁场
\global\long\def\mf#1{\frac{\mu_{0}I}{#1}}%

\begin{description}
\item [{1.}] 磁场的Biot-Savart定律: 
\[
\vec B=\int_{I}\dd\vec B=\int_{I}\mf{4\pi}\frac{\dd l\times\vec{e_{r}}}{r^{2}}.
\]
\item [{2.}] 磁场的叠加: $B=\sum_{i}\vec{B_{i}.}$
\begin{description}
\item [{模型磁场:}]~
\begin{description}
\item [{直线}] $B=\mf{4\pi r}(\cos\theta_{2}-\cos\theta_{1})$
\item [{圆弧}] $B=\mf{2r}\frac{\phi}{2\pi}$, 方向与电流方向成右手螺旋的关系. 
\end{description}
\end{description}
\item [{3.}] 磁场的Gauss定理和环路定理
\begin{description}
\item [{Gauss定理}] $\oint_{S}\vec B\cdot\dd\vec S=0.$(无源)
\item [{环路定理}] $\oint_{L}\vec B\cdot\dd\vec l=\mu_{0}\sum I_{\text{内}}$
\item [{应用:}] 用环路定理求对称性磁场的分布
\begin{description}
\item [{无限长直线.}] $B=\mf{2\pi r}$
\item [{螺线管.}] $B=n\mu_{o}I$
\item [{螺线环.}] $B=\frac{N\mu_{0}I}{2\pi r}$.
\end{description}
\end{description}
\end{description}
\item [{四.}] 磁力
\begin{description}
\item [{1.}] Lorentz力 $\vec{F_{m}}=q\vec v\times\vec B$
\begin{description}
\item [{a.}] 螺旋运动, $R=\frac{mv_{\perp}}{qB},h=\frac{2\pi mv_{//}}{qB}$
\end{description}
\item [{2.}] Ampere力 $\vec F=\int_{L}\dd\vec F=\int_{L}I\dd\vec l\times\vec B,\dd\vec F=nS\dd lq\vec v\times\vec B=I\dd\vec l\times\vec B$
\item [{3.}] 磁矩: $p_{m}=IS\vec{e_{n}}$
\item [{4.}] 磁力矩: $\vec M=\vec{p_{m}}\times\vec B$.
\end{description}
\item [{五.}] 电磁感应
\begin{description}
\item [{1.}] 电磁感应定律: $\eqs=-\frac{\dd\phi_{m}}{\dd t},\phi_{m}=\int_{S}\vec B\cdot\dd\vec S$
\item [{2.}] 动生电动势: $\eqs=\int_{L}(\vec v\times\vec B)\cdot\dd\vec l$,
其中非静电力是$\vec f=\vec v\times\vec B.$
\item [{3.}] 感生电动势: $\eqs=-\frac{\dd\phi_{m}}{\dd t}=\int_{S}\frac{\partial\vec B}{\partial t}\dd\vec S$
\begin{description}
\item [{特例.}] 非静电力$E_{i}$圆柱形均匀磁场的变化激发$E_{i}2\pi r=\frac{\dd B}{\dd t}S$
\end{description}
\end{description}
\item [{六.}] 位移电流与传导电流
\begin{description}
\item [{1.}] 位移电流与传导电流均有磁效应
\[
\oint_{L}\vec B\cdot\dd\vec l=\mu_{0}\sum(I_{0}+I_{d})_{\text{内}}.
\]
\item [{2.}] 位移电流源自变化的电场$I_{d}=\frac{\dd\phi_{D}}{\dd t}=\eqs_{0}\frac{\dd E}{\dd t}S$
\begin{description}
\item [{特例:}] 圆柱形均匀电场变化激发磁场$B2\pi r=\mu_{0}I_{d\text{内}}.$
\end{description}
\end{description}
\end{description}
%

\part{常见问题类型}

\global\long\def\qst{\underline{~~~~~~~~~}}%


\section{运动学中的常见问题}

\subsection*{单个变量的替代, 积分求解}
\begin{description}
\item [{1.}] 选取: 选取任意的状态, 将要得到的表达为一个关于任意的量. 
\item [{2.}] 计算: 积分的上下限应当选取从初态到末态. 
\end{description}
\begin{example}
匀加速直线运动的加速度为$a$, 试着推导$a-x$之间的关系. 

答案: $v=\sqrt{2ax}$

分析: $a=\frac{\dd v}{\dd x}\frac{\dd x}{\dd t}=\frac{\dd v}{\dd x}v,$于是$a\dd x=v\dd v.$两边同时积分可以得到. 
\end{example}
%
\begin{example}
在离水面高为$h$的岸边, 有人用绳拉船靠岸, 船在离岸$x$m处, 当人以$v_{0}$m/s恒定的速率收绳时, 试求船的速度、
加速度的大小. 

答案: $v=\frac{v(tv-L)}{\sqrt{(L-tv)^{2}-h^{2}}},a=-\frac{h^{2}v^{2}}{\left((L-tv)^{2}-h^{2}\right)^{3/2}}$

分析: 注意到关系式$x(t)=\sqrt{(L-tv)^{2}-h^{2}}$, 求两次导数即可.
\end{example}
%
\begin{example}
(2-2)在一个加速度为$a$向上运动的电梯里, 悬挂一个劲度系数$k$的弹簧(质量不计), 其下悬挂质量为$M$的物块, 物块相
对电梯静止, 当电梯加速度突然变为零, 电梯内的观察 者看到物块的最大速度为$\qst$. 

\begin{circuitikz} \tikzstyle{every node}=[font=\LARGE] \draw [, line width=0.5pt ] (2.25,11.75) rectangle (5.5,7.5); \draw [line width=0.5pt](4,11.75) to[L ] (4,10.25); \draw [, line width=0.5pt ] (3.75,10.25) rectangle (4.25,9.75); \draw [ line width=0.5pt, -Stealth] (6,10) -- (6,11.5); \node [font=\LARGE] at (6.5,10.5) {}; \draw [ line width=0.5pt, -Stealth] (6,10) -- (6,11.5); \draw [ line width=0.5pt, -Stealth] (6,10) -- (6,11.5); \end{circuitikz}

答案: $a\sqrt{m/k}$

分析: 设\textbf{弹簧的伸长量为}$x_{0}$, 选择向上为正方向, 于是起始时刻受力分析: $F_{\text{弹}}-mg=ma$,
意味着$k\Delta x-mg=ma\implies\Delta x=\frac{m(a+g)}{k}.$根据机械能守恒, $\frac{1}{2}k(\Delta x)^{2}=\frac{1}{2}mv^{2}+mg\Delta x$.
因此$v_{max}=\sqrt{\frac{m}{k}}a.$
\end{example}
%
\begin{example}
(2-9)光滑的水平桌面上放置一固定的圆环带, 半径为$R$, 一 物体贴着环带的内侧运动, 如图所示, 物体与环带间的滑动 摩擦系数为$\mu$,
设物体在某一时刻经A点时的速率为 $v_{0}$, 求 此后$t$ 时间物体的速率以及从$A$点开始所经过的路程.

\begin{circuitikz} \tikzstyle{every node}=[font=\LARGE] \draw (5.5,8) circle (2.75cm); \draw (5.5,8) circle (2.75cm); \draw (5.5,8) circle (2.5cm); \draw (7.25,10) rectangle (7.5,9.75); \draw [ -Stealth] (5.5,8) -- (8,8); \draw [ -Stealth] (5.5,8) -- (8,8); \end{circuitikz}

答案: $s=\frac{R}{\mu_{k}}\ln\frac{R+v_{0}\mu_{k}t}{R}$

分析: 对于任意的一个时刻, 沿着切向分析, 有$F_{t}=f=\mu F_{N},$沿着法向分析, 有$F_{n}=\frac{mv^{2}}{R}.$z直到$a_{t}=\frac{\mu v^{2}}{R}.$展开,
有$\frac{\dd v}{\dd t}=\frac{mv^{2}}{R},$得到方程式$\frac{\dd v}{v^{2}}=\frac{v^{2}}{R}\dd t$,
两边同时对初始状态和末了状态积分, 有
\[
\int_{v_{0}}^{v_{\text{末}}}\frac{\dd v}{v^{2}}=\int_{0}^{t}\frac{v^{2}}{R}\dd t
\]
解得
\[
v_{\text{末}}=\frac{1}{\frac{\mu}{R}t+\frac{1}{v_{0}}}.
\]
然后把$v$展开为$\dd s/\dd t$, 对两边同时做定积分就可以求出路程$s=\frac{R}{\mu_{k}}\ln\frac{R+v_{0}\mu_{k}t}{R}.$
\end{example}
%
\begin{example}
(2-15)一个初速度$v_{0}$从地面上竖直向上抛出质量为$m$的小球, 小球除了受重力外, 还受到一个大小为$\alpha mv^{2}$的粘滞阻力($\alpha$是常数,
$v$是小球的运动速率). 求小球回到地面的时候的速率. 

答案: $v_{1}=\frac{v_{0}\sqrt{g}}{\sqrt{\alpha v_{0}^{2}+g}}.$

分析: 如果一心想着一下子求出来, 就会陷入困境. 我们把这个问题转化为求上升和下降两段路程的问题, 最后让他们相等即可. 

以向上为正方向, 上升过程中, 有$-ma_{1}=-mg-f.$因为$a=\frac{\dd v}{\dd x}\frac{\dd x}{\dd t}=v\frac{\dd v}{\dd x},$原式变为
\[
\frac{v}{g+\alpha v^{2}}\dd v=\dd x
\]
两边同时积分, 有$\int_{v_{0}}^{0}\frac{v}{g+\alpha v^{2}}\dd v=\int_{0}^{h}\dd x$,
解得
\[
h_{1}=-\frac{1}{2a}\left(\ln\frac{g}{g+\alpha v_{0}^{2}}\right).
\]

另外一方面, 考虑下来的时候, 进行受力分析, $-mg+f=-ma_{2}.$ 同理, 有
\[
h_{2}=-\frac{1}{2a}\left(\ln\frac{g-\alpha v_{m}}{g}\right).
\]
因为$h_{1}=h_{2},$解答得$v_{m}=\frac{v_{0}\sqrt{g}}{\sqrt{\alpha v_{0}^{2}+g}}.$
\end{example}
%
\begin{example}
(3-13)有一运送砂子的皮带以恒定的速率$v$水平运动, 砂子经一静止 的漏斗垂直落到皮带上, 忽略机件各部位的摩擦及皮带另一端的其
它影响, 试问:(1) 若每秒有质量为$M'=\dd M/\dd t$的砂子落到皮带上,  要维持皮带以恒定速率$v$运动, 需要多大的功率?
(2)若$M'=100$kg/s, $v=0.5$m/s, 水平牵引力多大? 所需功率多大? 

答案: $P=v^{2}M',50$N,$25$W.

分析: 对于任意时刻分析, 设有$\dd m$这样大的沙子落入传送并且获得速度, 那么$\dd I=\dd m\cdot v.$根据$F=\frac{\dd I}{\dd t}=v\frac{\dd m}{\dd t},$那么$P=Fv=v^{2}\frac{\dd m}{\dd t}.$第二问带入数据即可.
\end{example}
%
\begin{example}
(4-14) 质量$m$的匀质链条, 置于桌面, 链条与桌面的摩擦系数为$\mu$,  下垂端的长度为$a$, 在重力作用下,
由静止开始下落, 求链条完全 滑离桌面时重力、摩擦力的功. 

\begin{circuitikz} \tikzstyle{every node}=[font=\small] \draw (2.5,9.25) rectangle (7,9); \draw (3.25,9) to[short] (3.25,7); \draw (5.5,9) to[short] (5.5,7); \draw (4.75,9.5) ellipse (0.5cm and 0.25cm) ; \draw (5.25,9.5) ellipse (0.5cm and 0.25cm) ; \draw (4.25,9.5) ellipse (0.5cm and 0.25cm); \draw (6,9.5) ellipse (0.5cm and 0.25cm) ; \draw (6.5,9.5) ellipse (0.5cm and 0.25cm) ; \draw (7,9.5) ellipse (0.5cm and 0.25cm); \draw (7.25,9.25) ellipse (0.25cm and 0.5cm); \draw (7.25,8.75) ellipse (0.25cm and 0.5cm); \draw (7.25,8.25) ellipse (0.25cm and 0.5cm) ; \end{circuitikz}

答案: $-\frac{\mu mg}{2l}(l-a)^{2}$. 

分析($\frown$): 假设任意时刻, 掉下去的部分是$x$, 还没有掉下去的部分是$l-x$. 考虑之后很小的一段位移,
$\dd A_{\text{重力}}=\frac{x}{l}m\cdot\frac{\dd x}{2}.$ 这时候对于$0\to x$积分,
有$\int A_{\text{重力}}=\int_{0}^{a}\frac{x}{l}m\cdot\frac{\dd x}{2}=-\frac{\mu mg}{2l}(l-a)^{2}.$
\end{example}

\subsection*{刚体的定轴转动综合问题}
\begin{description}
\item [{1.}] 刚体与Newton第二定律的对应关系: $J\to M,\beta\to a,\text{合外力矩}\to F,\omega\to v$
\begin{description}
\item [{刚体的定轴转动定律.}] $M=J\beta$
\item [{刚体的定轴转动的动能.}] $E_{k}=\frac{1}{2}J\omega^{2}$
\begin{description}
\item [{守恒条件:}] 没有非保守外力做功
\end{description}
\item [{刚体的定轴转动角动量.}] $L=J\omega$
\begin{description}
\item [{守恒条件:}] 有心力或合外力矩为0. 
\end{description}
\end{description}
\item [{2.}] 刚体的能量
\begin{description}
\item [{力矩:}] $\vec M=\vec F\times\vec R$, 其中$\vec R$是从转动原点指向现在的点的向量.
\item [{刚体的定轴转动的动能.}] $E_{k}=\frac{1}{2}J\omega^{2}$
\begin{description}
\item [{动能定理:}] 外力矩的功=刚体定轴转动动能的增量, 即$A_{\text{外}}=\frac{1}{2}J\omega_{2}^{2}-\frac{1}{2}J\omega_{1}^{2}$
\end{description}
\item [{刚体的重力势能:}] $E_{P}=mgh_{c}$, 其中$h_{c}$是重心的高度. 
\item [{机械能:}] 在重力场中, 包含定轴转动的系统, 机械能包含\textbf{刚体的重力势能}和\textbf{定轴转动功能}.
\end{description}
\item [{3.}] 刚体的动量
\begin{description}
\item [{角动量}] $\vec L=J\vec{\omega},$方向沿$z$轴. 
\item [{角动量定理}] 外力矩的冲量=角动量的增量, 即$\int_{t_{1}}^{t_{2}}M_{\text{外}z}\dd t=\Delta L_{z}=J\omega_{2}-J\omega_{1}.$
\item [{角动量守恒}] 定轴转动的刚体外力矩为0, 刚体关于定轴的角动量守恒
\begin{description}
\item [{(1)}] 不受外力;
\item [{(2)}] 外力作用穿过轴线;
\item [{(3)}] 外力与作用轴平行.
\end{description}
\end{description}
\item [{4.}] 圆周运动切向与法向的关系
\end{description}
下面是一些关于滑轮与连接体的问题. 
\begin{example}
(5-3)求$A$, $B$两定滑轮的角加速度$\bt_{A},\bt_{B}$, 已知定滑轮半径$R$, 拉力$F=Mg$与重物的质量相同. 

\incfig{prob53}

答案: $\bt_{A}>\bt_{B}$

分析: 对于$A$图, 对物体进行受力分析, $Mg-T=ma_{t}$. 对于滑轮分析, $TR=J\beta.$但是对于右图,
$MgR=J\beta.$
\end{example}
%
\begin{example}
(5-9)如图, 滑块 $A$, 重物 $B$ 和滑轮$C$ 的 质量分别$m_{A},m_{B},m_{C}.$ 滑轮半径为$R$.$A$与
桌面之间,滑轮与轴承间均无摩擦, 绳质 量可不计, 绳与滑轮间无相对滑动.求 滑块 $A$的加速度及滑轮两边绳中的张力. (提示:
$J_{C}=m_{c}R^{2}/2$)

\begin{circuitikz} \tikzstyle{every node}=[font=\LARGE]  \draw [](2,20.5) to[short] (5.75,20.5); \draw [](2,20.5) to[short] (2,19.25); \draw (1.75,20.5) circle (0.25cm); \draw [short] (1.5,20.5) -- (1.5,19.25); \draw [short] (1.75,20.75) -- (4,20.75); \draw (1.25,19.25) rectangle (1.75,18.75); \draw (4,21) rectangle (4.5,20.5); \node [font=\LARGE] at (1.5,19) {A}; \node [font=\LARGE] at (4.25,20.75) {B}; \draw (4,21) rectangle (4.5,20.5); \end{circuitikz}

答案: $a=\frac{m_{B}g}{m_{A}+m_{B}+m_{C}/2},T_{A}=m_{A}a,T_{B}=m_{B}(g-a).$

分析: 选取重物下落后的方向为正方向. 假设水平段中绳子的受力是$T_{A},$竖直段的受力为$T_{B}.$加速度为$a$.
\textcolor{orange}{\uwave{(注意这里一根绳子上的力是不等的, 但是绳子会传达加速度相等的消息)}}
对物体$A$: $m_{A}a=T_{A}\quad(1)$; 对物体$B:m_{B}a=mg-T_{B}\quad(2)$.
由于滑轮与两端绳子接触, 因此依据Newton第三定律, 合外力矩为$(T_{B}-T_{A})R,$因此根据刚体的定轴转动定律,
就有$(T_{A}-T_{B})R=J\beta\quad(3).$ 又因为$\bt$与$a$的关系, 有$a=r\bt\quad(4)$.联立上述的式子,
解得到答案. 

解答上述方程的提示: 首先方程的切入口是(4), 然后分理处变量$a$即可. 
\end{example}
%
\begin{example}
(5-13) 质量分别为 $m$ 和 $2m$, 半径分别为 $r$ 和$2r$的两个均匀圆盘,同 轴地粘在一起, 可以绕通过盘心且垂直盘面的水平光滑固定轴转
动, 对转轴的转动惯量为 $9mr^{2}/2$, 大小圆盘边缘都绕有绳子, 绳 子下端都挂一质量为 $m$ 的重物, 如图.
求盘的角加速度的大小.

\begin{circuitikz}[scale=0.5]  \draw (4.5,15.5) circle (2.5cm); \draw (4.5,15.5) circle (1.25cm); \draw [short] (2,15.5) -- (2,11.5); \draw [short] (5.75,15.75) -- (5.75,11.5); \draw (1.75,11.5) rectangle (2.5,10.75); \draw (5.5,11.5) rectangle (6.25,10.75); \node at (2,11) {$m$}; \node  at (5.75,11) {$m$}; \draw (4.5,15.5) to[short] (5.75,15.5); \node  at (5,15.5) {$r$}; \node  at (4,14.5) {$2r$}; \draw (4.5,15.5) to[short] (4.5,13); \node at (4,16) {$m$}; \node at (5.25,17) {$2m$}; \draw (4.5,15.5) to[short] (4.5,13); \end{circuitikz}

答案: $\frac{2g}{19r}.$

分析: 可以对每一个圆盘分别分析, 把沿重物下坠的方向记为正方向, 设两段绳子的拉力分别为$T_{1},T_{2}.$对应的加速度分别为$a_{1},a_{2}$,于是有:
\begin{align*}
mg-T_{1} & =ma_{1}\\
T_{2}-mg & =ma_{2}\\
-T_{2}r+T_{1}2r & =J\beta\\
a_{1} & =\beta2r\\
a_{2} & =\beta r
\end{align*}
联立求解即可. 

另外一种方法是使用整体法. 由于$M_{\text{总}}=J\beta,$那么有
\[
mg\cdot2r-mgr=\left(m(2r)^{2}+mr^{2}+\frac{9}{2}mr^{2}\right)\bt.
\]

同样可以得到答案. 
\end{example}
%
\begin{example}
(5-10) 用力 $F$ 将一粗糙平面紧压在轮上, 平面与轮 间滑动摩擦系数 $\mu$,轮的初角速度 $\omega_{0}$,问:
转过多 少角度及经过多长时间轮即停止转动? 轮半径 $R$,  质量$m$分布均匀;忽略轴质量且力 $F$ 均匀. 

\begin{circuitikz}[scale=0.5] \tikzstyle{every node}=[font=\LARGE] \draw (7.25,16) ellipse (1.25cm and 2.25cm) ; \draw (7.25,16.25) rectangle (11.75,16); \draw [ fill={rgb,255:red,0; green,0; blue,0} ] (7.25,16.25) rectangle (11.75,16); \draw (3,19.25) rectangle (5.5,13); \draw [ fill={rgb,255:red,0; green,0; blue,0} ] (3,19.25) rectangle (3.25,13); \node [font=\LARGE] at (4,18) {$\mu$}; \draw [ fill={rgb,255:red,0; green,0; blue,0} ] (3,19.25) rectangle (3.25,13); \end{circuitikz}

答案: $\Delta\theta=\frac{3mR\omega_{0}^{2}}{8\mu F},\Delta t=\frac{3mR\omega_{0}}{4\mu F}$.

分析: 这个问题我们希望使用刚体定轴转动的动能定理($\frac{1}{2}J\omega_{2}^{2}-\frac{1}{2}J\omega_{1}^{2}=A_{\text{外}}\stackrel{\text{恒定外力矩}}{\implies}M\theta$)和动量定理($J\omega_{2}-J\omega_{1}=p_{\text{外}}\stackrel{\text{恒定外力矩}}{\implies}Mt$)求解,
但是发现关键的问题在于求解外力矩. 

选取若干个同心小圆环. 因为小圆环上面的$r$都是一样的, 因此量就好算了很多. 注意到$\dd F=\frac{2\pi r}{\pi r^{2}}\dd rF,\dd f=\mu\dd F=\mu\frac{r}{R}F,\dd M=\dd f\dd r=2\mu\left(\frac{r}{R}\right)^{2}F\dd r.$从$0\sim R$积分,
有$M=\frac{2\mu RF}{3}.$ 于是可以用动能定理$M\Delta\theta=J\omega_{0}^{2}/2,$以及角动量定理$M\Delta t=J\omega_{0}$求出答案. 
\end{example}
%
\begin{example}
(5-14) 质量为 $m$, 长为 $L$的匀质木棒可绕 $O$ 轴自 由转动,开始木棒铅直悬挂, 现在有一只质量为 $m$
的小猴以水平速度 抓住棒的一端 (如图),  求:(1) 小猴与棒开始摆动的角速度;(2) 小猴与 棒摆到最大高度时, 棒与铅直方向的夹角.
(此时的转动惯量是$1/3mL^{2}$).

\begin{circuitikz} \tikzstyle{every node}=[font=\footnotesize] \draw [short] (5.25,18.75) -- (6.25,18.75); \draw [short] (5.25,18.75) -- (5.5,18.25); \draw [short] (5.5,18.25) -- (6,18.25); \draw [short] (6,18.25) -- (6.25,18.75); \draw [short] (5.75,18.25) -- (5.75,15.5); \draw [short] (5.75,18.25) -- (5.75,15.75); \draw [dashed] (5.75,18.25) -- (4,16); \draw (7,15.5) ellipse (0.75cm and 0.25cm) ; \draw [-Stealth, dashed] (8,16) -- (6.25,16); \node [font=\footnotesize] at (6,16.75) {$L$}; \node [font=\footnotesize] at (6,16.25) {$m$}; \node [font=\footnotesize] at (6,18) {$O$}; \node [font=\footnotesize] at (7,15.5) {$m$}; \node [font=\footnotesize] at (6.5,15) {Monkey}; \node [font=\footnotesize] at (5.5,17.75) {$\theta$}; \draw [-Stealth, dashed] (8,16) -- (6.25,16); \end{circuitikz}

答案: $\omega=3v_{0}/4L,\theta=\cos^{-1}\left(1-\frac{v_{0}^{2}}{4gL}\right).$

分析: 对于(1), 可以使用角动量守恒
\[
mvL=\left(mL^{2}+\frac{1}{3}mL^{2}\right)\omega
\]
就可以求得. 对于(2), 考虑机械能守恒, 就有
\[
\frac{1}{2}\left(\frac{1}{3}mR^{2}+mR^{2}\right)\omega^{2}=\left(\frac{1}{2}mgL(1-\cos\theta)\right)+mgL(1-\cos\theta).
\]
注意刚体是要看重心移动的距离. 
\end{example}
%
\begin{example}
(5-15) 质量$m$, 、长 $l$ 的匀质杆, 以 $O$ 点为轴, 静止 从与竖直方向成 角处自由下摆, 到竖直位置
时与光滑桌面上的静止物块$m$(视为质点) 发生弹 性碰撞, 求(1) 棒开始转动时的角加速度$\bt$; (2) 棒 转到竖直位置碰撞前的角速度$\omega_{1}$.
及棒中央点 $C$的速率$v_{c_{1}}$; (3) 碰撞后杆的角速度$\omega_{2}$和物块的线速率$v_{2}.$

答案: $\bt=3g\sin\theta_{0}/2l;\omega_{1}=\sqrt{\frac{3g(1-\cos\theta_{0})}{l}};v_{c_{1}}=\frac{1}{2}\sqrt{3gl(1-\cos\theta_{0})};v_{2}=\frac{1}{2}\sqrt{3gl(1-\cos\theta_{0})};\omega_{2}=-\frac{1}{2}\sqrt{\frac{3g}{l}(1-\cos\theta_{0})}.$

分析: (1) 一开始, 根据转动定律, 有$\frac{1}{2}mg\sin\theta_{0}=\left(\frac{1}{3}ml^{2}\right)\beta\implies\beta=3g\sin\theta_{0}/2l.$
对于(2), 根据转动机械能守恒, 有$mg\frac{l}{2}(1-\cos\theta_{0})=\frac{1}{2}J\omega_{1}^{2}.$由此可解答.
(3) 弹性碰撞中棒与物块对转轴角动量守恒, 动能守恒. 于是列式子:
\begin{align*}
J & =\frac{1}{3}ml^{2} & \qquad(1)\\
J\omega_{1} & =J\omega_{2}+mv_{2}l & \qquad(2)\\
\frac{1}{2}J\omega_{1}^{2} & =\frac{1}{2}J\omega_{2}^{2}+\frac{1}{2}mv_{2}^{2} & \qquad(3)
\end{align*}

对于(3)的计算建议
\global\long\def\blue#1{{\color{blue}#1}}%
\global\long\def\green#1{{\color{teal}#1}}%
: 可以对于(3)两端乘上$J,$然后让(2)平方, 有$J^{2}\omega_{1}^{2}=\blue{J^{2}\omega_{2}^{2}+mJv_{2}^{2}}\quad(3)$,
以及$J^{2}\omega_{1}^{2}=\green{J^{2}\omega_{2}^{2}+m^{2}v_{2}^{2}l^{2}+2J\omega_{2}mv_{2}l}\quad(4).$把(3)带入(4)中的左侧,
有$\blue{J^{2}\omega_{2}^{2}+mJv_{2}^{2}}=\green{J^{2}\omega_{2}^{2}+m^{2}v_{2}^{2}l^{2}+2J\omega_{2}mv_{2}l}\quad(5)$.
化简(5), 就有$Jv_{2}=mv_{2}l^{2}+2J\omega_{2}l\quad(6)$. 知道$\frac{2Jl\green{\omega_{2}}}{(J-ml^{2})}=\blue{v_{2}}\quad(7),$也就是得到了$v_{2}$与$\omega_{2}$的关系.
带入(2), 有值为$\omega_{2}=-\frac{1}{2}\omega_{1},\omega_{1}=\frac{1}{2}\omega_{1}.$
\end{example}

\subsection*{狭义相对论}

\begin{description}
\item [{1.}] 不同的参考系: 假设参考系$S'$相对与$S$相对于$x$轴正方向运动
\begin{description}
\item [{事件:}] 发生的事情
\item [{距离差:}] 以某一个参考系的度量测量的距离
\item [{时间差:}] 以某一个参考系的度量测量的时间
\end{description}
\item [{2.}] Lorentz速度变换: 在参考系$S$中, 有$\Delta x,\Delta t$, 在参考系$S'$中,
有$\Delta x',\Delta t'$, 相对运动速率为$u.$
\begin{description}
\item [{参考系$S\to S'$.}] $\Delta x\red '=\gamma\left(\red -u\Delta t+\green{\Delta x}\right)\qquad\qquad\Delta t\red '=\gamma\left(\red -\frac{u\Delta x}{c^{2}}+\green{\Delta t}\right)$
\item [{参考系$S'\to S$.$\Delta x=\gamma\left(u\Delta t\red '+\green{\Delta x'}\right)\qquad\qquad\Delta t=\gamma\left(\frac{u\Delta x\red '}{c^{2}}+\green{\Delta t'}\right)$}]~
\item [{理解.}] 只是在原来的内容上面做了一点修正. 修正的变量需要依赖另一个和相对运动的速度. 注意正负号和变换因子. 
\end{description}
\item [{3.}] 相对论体系下的动量和能量
\begin{description}
\item [{相对论质量.}] $m=\gamma m_{0}$(速度越大的时候加速越难)
\item [{相对论动量.}] $\vec p=m\vec v=\gamma m_{0}\vec v.$ 这个表达式保证在Lorentz变换的条件下,
动量守恒在所有的惯性系中成立. 
\item [{相对论动能.}] $E_{k}=mc^{2}-m_{0}c^{2}=(\gamma-1)m_{0}c^{2}.$
\item [{相对论能量.}] $E_{\text{总}}=mc^{2},E_{\text{静}}=m_{0}c^{2},E_{\text{动}}=E_{\text{总}}-E_{\text{静}}$.
\item [{相对论动量能量关系.}] $E^{2}=p^{2}c^{2}+m_{0}^{2}c^{4}.$
\item [{相互作用的守恒关系.}] 质能守恒, 动量守恒. 
\end{description}
\end{description}
\begin{example}
(6-3) 北京某日19时一工厂断电, 同日19时0分0.00026 s天津某人放了一响 礼炮, 两者的直线距离120km, 一艘飞船相对地面以0.8$c$从北京到天津飞
行, 问飞船上的航天员观察, 哪件事情先发生?

答案: $\Delta t=-0.0006$s, 天津事件先发生. 

分析: 这里面有两个时间. 事件1: 工厂断电; 事件2:天津放炮. $\Delta x=0.00026\text{s},\Delta s=120\text{km}$.
根据变换公式, 有$\Delta t'=\gamma\left(\Delta t-\frac{u\Delta x}{c^{2}}\right).$
首先求出$\gamma$, 也就是$\gamma=\frac{1}{\sqrt{1-\left(\frac{u}{c}\right)^{2}}}=5/3.$
把数据带入科学计算器, 就有答案$\Delta t'=-0.191977$s. 
\end{example}
%
\begin{example}
解答下列相对论的小问题:
\begin{enumerate}
\item (6-10)火箭静止质量100t, 当它以第二宇宙速度$v=11.2\times10^{3}$m/s飞 行时, 其质量增加$\qst$.
\item (6-12) $\alpha$粒子在加速器中被加速, 当其质量为静止质量的6倍其 动能是静止能量的$\qst$倍. 
\item (6-14) (a) 一粒子的相对论动量等于2倍非相对论动量, 求粒子速度. (b) 一粒子的动能等于静能, 求粒子速度. 
\item (6-18) 一个高能质子进入磁场$B$发生偏转, 已知偏转半 径$R$. 求质子的动量$p$和能量$E_{k}.$
\end{enumerate}
答案: (1) $0.7\times10^{-4}$kg; $5m_{0}c^{2}$; $v=\frac{\sqrt{3}}{2}c,v=\frac{\sqrt{3}}{2}c$;$p=RBq,E=\sqrt{p^{2}c^{2}+E_{0}^{2}}$.
\\
分析: 对于(1), 
\[
\Delta m=E_{k}/c^{2}=\frac{\frac{1}{2}mv^{2}}{c^{2}}.
\]
 因为速度远小于光速. 对于(2), 得到$m=\gamma m_{0}\to\gamma=6.$意味着$E_{k}=(\gamma-1)m_{0}c^{2}=5m_{0}c^{2}.$
对于(3). 我们有$p=\gamma m_{0}v=2p_{0}=2m_{0}v\to\gamma=2.$以及$E_{k}=\gamma m_{0}c^{2}-m_{0}c^{2}=m_{0}c^{2}\to\gamma=2.$于是可以求出$v$.
对于(4), 我们已经知道$R=\frac{p}{qB}.$得到$p=RBq,$再根据公式求出最后的值即可. 
\end{example}
%
\begin{example}
(6-17) 在$S$系, 静止质量$m_{0}$的两个粒子, 其中一个静止, 另一个 速度大小为0.8$c$, 沿一条直线作对心碰撞后合成为一个粒子,
 求合成粒子的静止质量$M_{0}$. 

答案: 2.31$m_{0}$. 

分析: 考虑质能守恒得到$m_{0}c^{2}+\gamma_{1}m_{0}c^{2}=Mc^{2}=\gamma_{2}M_{0}c^{2}$,
然后使用直线上的动量守恒得到$\gamma_{1}m_{0}v_{1}=Mv_{2}=\gamma_{2}M_{0}v_{2}$就可以算出答案. 
\end{example}

\section{电场中的常见问题}

\subsection*{叠加法}
\begin{description}
\item [{1.}] 运用叠加法解题的一般步骤
\begin{description}
\item [{选取恰当的微元.}] 找到一小段微元, 使得这个微元上面某一个量是恒定的值. 
\item [{给出微小一段上的量.}] 如果是标量, 那么就仅有一个数值; 如果是一个矢量, 那么需要考虑是否应当分$x$轴和$y$轴进行分析. 
\item [{积分运算.}] 从初始状态积分值末了状态, 进行积分运算. \textcolor{purple}{\uwave{注意对称性分析}}.
\end{description}
\item [{2.}] 静电场的场强求解
\begin{description}
\item [{模型.}]~
\begin{description}
\item [{点电荷:}] $\vec E=\ef\frac{Q}{r^{2}}\vec{e_{r}}$
\item [{无限大直导线:}] $\vec E=\frac{\lambda}{2\pi\eqs_{0}r}\vec{e_{r_{\perp}}}$
\item [{无限大平面:}] $\vec E=\frac{\sigma}{2\eqs_{0}}\vec{e_{n}}$
\end{description}
\item [{积分变元的变换:}] $\lambda\stackrel[=\dd x\times]{\times\text{小面}=}{\leftarrow--\rightarrow}\sigma\stackrel[=\dd x\times]{\times\text{小面}=}{\leftarrow--\rightarrow}v$. 
\end{description}
\item [{3.}] 静电场的电势求解
\begin{description}
\item [{模型.}]~
\begin{description}
\item [{点电荷.}] $V=\ef\frac{Q}{r}$
\item [{均匀带电球面球心.}] $V=\ef\frac{Q}{R}.$
\end{description}
\end{description}
\end{description}
\begin{example}
(7(1)-4) 电量$Q$均匀分布在长为$2L$的细棒上, 在细棒的延长 线上距细棒中心$O$距离为$x$的$P$点处放一带电量为$q(q>0)\text{的点}\text{电荷,求带电细棒对该点电荷的静电力}.$

\incfig{prob7-4}

答案: $F=qE=\frac{qQ}{4\pi\eqs_{0}(x^{2}-L^{2})}.$ 

分析: 考虑以棒子的中点为$O$原点, 以电荷的方向为正方向, 建立$Ox$坐标轴. 取一个电荷元$\dd q=\lambda\dd a=\frac{Q}{2L}\dd a$.
那么, $\dd E=\frac{\dd q}{4\pi\eqs_{0}r^{2}}=\frac{Q\dd a}{8\pi\eqs_{0}L(x-a)^{2}}.$
各个$\dd\vec E$的方向均向右, 同方向叠加, 大小直接总$-L\to L$积分即可. 
\end{example}
%
\begin{example}
(7(1)-5) 如图半径为$R$的半圆周上均匀分布有电荷$Q$, 求圆心$O$点的场强. 另外, 如果用不导电的细塑料棒弯成半径为$R$的圆弧,
两段空隙为$d<<R,$电量为$Q$的均匀分布在上面, 求圆心处的场强的大小和方向. 

答案: $\frac{Q}{2\pi^{2}\eqs_{0}R^{2}},\frac{Q/(2\pi r-d)d}{4\pi\eqs_{0}R^{2}},$方向指向缺口. 

分析: 由于和弧有关, 这里选取$\theta$表示量的关系. 选取电荷元$\dd q=\lambda\dd l=\lambda R\dd\theta$.
考虑两段对称的弧在$O$点的作用. $\dd E$的大小是$\frac{1}{4\pi\eqs_{0}}\frac{\dd q}{R^{2}}.$带入,
$E$的大小为$\frac{1}{4\pi\eqs_{0}R^{2}}\frac{Q}{\pi}\dd\theta,$那么$E=\int_{0}^{2\pi}\frac{Q}{4\pi^{2}\eqs_{0}R^{2}}\cos\theta\dd\theta$,
方向向下. 

对于另外的一问, 我们知道
\[
\vec{E_{O}}=\vec{E_{\text{圆}}}+\vec{E_{\text{缺口处的负电荷}}}=\vec{E_{\text{缺口处的负电荷}}}.
\]
 于是就可以知道$\lambda=\frac{Q}{2\pi R-d},E_{O}=\frac{\lambda d}{4\pi\eqs_{0}R^{2}}.$
\end{example}
%
\begin{example}
(7(1)-7) 一大平面中部有一半径为 $R$ 的小孔, 设平面均匀带电,面电荷 密度为 $\sigma_{0}$ , 求通过小孔中心并与平面垂直的直线上的场强分布. 

答案: $E=\frac{x\sigma_{0}}{2\eqs_{0}\sqrt{x^{2}+R^{2}}},$方向向右. 

分析: 可以考虑使用叠加法进行求解. 取小圆环面为微元作为$\dd q$, 因此就有$\dd E=\frac{\dd q}{4\pi\eqs_{0}\left(x^{2}+r^{2}\right)}\frac{x}{\sqrt{x^{2}+r^{2}}}.$
这时候, \textcolor{purple}{\uwave{可以将\mbox{$\dd q$}视作一段很小的厚度}}, 这样我们就可以用面密度的信息了.
具体地, 有$\dd q=\sigma_{0}2\pi r\dd r.$ 从$R$到$+\infty$积分, 就有$E=\frac{x\sigma_{0}}{2\eqs_{0}\sqrt{x^{2}+R^{2}}}.$

提示: 可以把小圆环的公式正常记忆, \textcolor{purple}{\uwave{不过加上一个要乘的\mbox{$\cos\theta$}值}}.
也就是
\[
E_{\text{环}}=\frac{Q}{4\pi\eqs_{0}\green{\left(x^{2}+r^{2}\right)}}\red{\frac{x}{\sqrt{x^{2}+r^{2}}}}.
\]
\end{example}
%
\begin{example}
(7(2)-9) 一厚为$a$无限大的平板均匀带电, 电荷体密度为 $\rho$,  求平 板面垂直距离为$b$的$P$点的电场强度. 

答案: $E=\frac{\rho a}{2\eqs_{0}}.$

分析: 考虑一薄片. 那么这个就是$\dd E=\frac{\dd\sigma}{2\eqs_{0}}.$ 考虑$\dd V$与$\sigma$的关系.
直观上感觉, $\dd V$是一个单位体含有的电荷量, 这时候如果我们乘上$\dd x,$就变成了面密度. 因此$\sigma=\rho\dd x.$积分后就有答案. 

体电流与面电流的转化: 考虑在厚度为$\dd x$的平面层切割面元$\dd s$的一个小的体积元$\dd V$, 那么这时候根据带电量是恒等的列出等式:
\[
\dd q=\rho\dd v=\rho\dd x\dd s=\sigma\dd s.
\]
\end{example}
%
\begin{example}
(7(2)-6) 半径为R的均匀带电半圆环, 电量Q, 求圆心处的电势. 

答案: $V_{O}=\frac{Q}{2\pi\eqs_{0}R}$. 

分析: 考虑
\begin{align*}
\dd V_{O} & =\frac{\dd q}{4\pi\eqs_{0}r}\\
\dd q & =\sigma\pi r\dd r
\end{align*}
就可以得到答案. 
\end{example}
%
\begin{example}
(7(2)-12) 均匀带电圆盘半径为$R$, 电荷面密度为$\sigma$, 挖去半径$R/2$的 圆片, 求轴线上的电势分布,
再求轴线上的场强. 

答案: $V=\frac{\sigma}{2\eqs_{0}}\left(\sqrt{R^{2}+x^{2}}-\sqrt{(R/2)^{2}+x^{2}}\right),E=E_{x}=-\frac{\dd V}{\dd x}=-\frac{\sigma x}{2\eqs_{0}}\left(\frac{1}{\sqrt{R^{2}+x^{2}}}-\frac{1}{\sqrt{(R/2)^{2}+x^{2}}}\right).$

分析: 根据先前的经验, 仍然去取半径为$r$, 宽为$\dd r$的细圆环. 
\begin{align*}
\dd q & =\sigma2\pi r\dd r\\
\dd V & =\frac{\dd q}{4\pi\eqs_{0}\sqrt{r^{2}+x^{2}}}
\end{align*}
之后使用电势梯度$E_{x}=-\frac{\dd V}{\dd x}$对$x$求导就可以得到场强. 

提示: 圆环轴心处的电势为
\[
V_{O}=\frac{Q}{4\pi\eqs_{0}\red{\sqrt{r^{2}+x^{2}}}}.
\]
因为不涉及到矢量分解, 于是没有后面一项. 
\end{example}

\subsection*{关于$\protect\vec D$和$\protect\vec E$的Gauss定理及其使用}
\begin{description}
\item [{1.}] $\vec E$的Gauss定理: $\oint_{S}E\cdot\dd S=\frac{1}{\eqs_{0}}\sum q_{\text{内}}.$
\begin{description}
\item [{a.}] 静电场中的导体: 
\begin{description}
\item [{静电平衡.}] 电荷无宏观的移动, 不随时间变化. 
\item [{静电平衡的条件.}] 电场分布: $E_{\text{内}}=0,E_{\text{表面}}\perp E_{\text{导体表面}}.$
导体内无净电荷分布
\end{description}
\item [{b.}] 静电场中的电介质
\begin{description}
\item [{Gauss定理.}] 仍然适用, 但是需要加上另外感应出来的电荷, 较为麻烦. 于是引入$\vec D$的Gauss定理. 
\end{description}
\end{description}
\item [{2.}] $\vec D$的Gauss定理: $\oint_{S}\vec D\cdot\dd S=\sum\green{q_{0\text{内}}}.$
\begin{description}
\item [{a.}] $\vec D$与$\vec E$的关系: $\vec D=\eqs_{0}\eqs_{r}\vec E.$
\item [{b.}] 不影响电场分布的对称性条件: (1)均匀无穷介质; (2) 分界面为等势面; (3) 分界面与等势面垂直. 
\end{description}
\item [{3.}] 电势分析: 注意吧电势变化带来的作用当做分析的起点. 
\begin{description}
\item [{a.}] 接地: 电势为0. 
\item [{b.}] 两个连在一起: 电势相等
\item [{c.}] 两个相连的无异平行板之间: 电势差相等
\end{description}
\end{description}
\begin{example}
(7(1)-12) 点电荷 $q$ 位于边长为 $a$ 的正立方体的中心, (1)通过此立方体 的每一面的电通量各是多少? (2)
若电荷移至正方体的一个顶点上,  则通过每个面的电通量又各是多少? 

答案: $\phi_{1}=\frac{q}{6\eqs_{0}},\phi_{2}=\frac{1}{24}\frac{q}{\eqs_{0}}\text{与}0.$

分析: 因为立方体 6个面电通量一样可以解答出第一问. 对于第二问, 设计边长 $2a$ 的大立方体, 将$q$包围在中心, 不过该顶点的三个小面的电通量为$\frac{1}{4}\cdot\frac{1}{6}\frac{q}{\eqs_{0}}.$过该顶点的三个小面电通量为零
. 

\incfig{prob712}
\end{example}
%
\begin{example}
(7(2)-8) 两个截面半径分别为$R_{1}$、$R_{2}$的无限长圆柱面同轴放置, 已 知内圆柱面电荷线密度为$\lambda$
, 求两圆柱面间的电势差$U$. 

答案: $\frac{\lambda}{2\pi\eqs_{0}}\ln\frac{R_{2}}{R_{1}}.$

分析: 根据Gauss定理求柱面内部的场强, 然后积分即可. 
\end{example}
%
\begin{example}
(7(1)-15) 半径$R$的球形带电体, 电荷体密度为 $\rho=Ar$, $A$为常数; 总电量$Q$, 求场强分布. 

答案: $E_{\text{内}}=\frac{Ar^{2}}{4\eqs_{0}},E_{\text{外}}=\frac{Q}{4\pi\eqs_{0}r^{2}}.$

分析: 同样取同心球面为Gauss面, 然后积分即可. 但是这时候内部不是常量, 于是取体积元为半径$r$, 厚度为$\dd r$的同心球面层.
包含的电量为$\dd q=\rho4\pi r^{2}\dd r.$ $Q_{\text{总}}=\int_{0}^{r}\rho\cdot4\pi r^{2}\dd r.$然都带入对应的限积分即可. 

注意: 这个问题同样应用了微元法, 其方法和上面总结的如出一辙, 很应该仔细体会. 
\end{example}
%
\begin{example}
(7(1)-16) 半径$R$的无限长圆柱形带电体, 电荷体密度为$\rho=Ar^{2}$, $A$为 常数;求场强分布. 

答案: $E_{\text{内}}=\frac{Ar^{3}}{4\eqs_{0}};E_{\text{外}}=\frac{AR^{4}}{4\eqs_{0}r}.$方向沿径向. 

分析: 考虑作高为$h$, 半径为$r$圆柱面为圆柱面. 仅仅有侧面有通量, 同样应用Gauss公式. 这时候其内部包含的电荷量就变成了(取一小段电荷的微元)$\dd q=2\pi rh\rho\dd r$.
然后带入Gauss公式就行了. 
\end{example}
%
\begin{example}
(8(1)-6) 不带电的导体球壳, 内有一点电荷$q$, 距离球心为$d$. 用导线将球壳接地后,  再撤除地线, 求球心的电势. 

答案: $V_{O}=\frac{1}{4\pi\eqs_{0}}\left(\frac{1}{d}-\frac{1}{R}\right).$

分析: 此时接地球壳$V_{\text{壳}}=0.$内表面带电$-q$, 外表面带电0, 球外电场为0. 只关注球壳在中心的电势和电荷在中心处的电荷的代数和即可.
\textcolor{blue}{\uwave{(接地之后, 撤掉导线为啥还是电势为0?)}}
\end{example}
%
\begin{example}
(课上例题) 现在有如下所示的问题. 原先, 球$A$带有电荷$q$, 球$B$带有电荷$Q$. 现在将球$A$接地, 这时候求球各个表面的电势分布. 

\incfig{probgex}

答案: $V_{A}=\frac{q'}{4\pi\eqs_{0}R_{0}}+\frac{-q'}{4\pi\eqs_{0}R_{1}}+\frac{Q+q'}{4\pi\eqs_{0}R_{2}}=0,q'=\frac{R_{0}R_{1}Q}{R_{0}R_{2}-R_{1}R_{2}-R_{0}R_{1}}.$

分析: 我们发现$A$的电势是0. 那么考虑$A$的电荷分布变为了$q'.$$Q_{B\text{内表面}}=-q',Q_{B\text{外表面}}=Q+q'.$
那么列出$V_{A}=0$的式子:$V_{A}=\frac{q'}{4\pi\eqs_{0}R_{0}}+\frac{-q'}{4\pi\eqs_{0}R_{1}}+\frac{Q+q'}{4\pi\eqs_{0}R_{2}}=0$;
解之即可. 
\end{example}
%
\begin{example}
(8(1)-10) 两块大导体平板, 面积$S$, 带电量分别为$Q_{1}$、$Q_{2}$, 证明相对 两个面总是带等量异号电荷,
相背两个面总是带等量同号电荷,  并求各面电荷密度. 

\incfig{prob810}

答案: $\sigma_{1}=\sigma_{4}=(Q_{1}+Q_{2})/2S;Q_{2}=-Q_{3}=(Q_{1}-Q_{2})/2S.$

分析: 取贯穿两个内部的面作为Gauss面, 因为通过Gauss面的电力线条数为0. 但是电荷包围的电荷是可以计算的. 因此我们就可以写作$\oint\vec E\cdot\dd\vec s=0=\frac{Q_{\text{内}}}{\eqs_{0}}=\frac{\sigma_{1}+\sigma_{2}}{\eqs_{0}}S_{\text{底}}$,
也就是$\sigma_{2}+\sigma_{3}=0.$ 接着, 我们考虑场强叠加加上静电平衡条件$E_{\text{内}}=0,$在介质内任取一点,
得到$\sigma_{1}-\sigma_{2}-\sigma_{3}-\sigma_{4}=0.$ 最后, 根据电荷守恒, 就得到了$\sigma_{1}+\sigma_{2}=Q_{1}/S;\sigma_{3}+\sigma_{4}=Q_{2}/S.$
\end{example}
%
\begin{example}
(8(1)-11) 三块大导体平板, 面积$S$, 中间一块带电$Q$ ,  其余条件见图. (1)当$ab$断开的时候求各面电荷密度.
(2)用导线连接$ab$, 求各面的电荷密度. 

\incfigw{prob811}

答案: (1) $\sigma_{1,3,4,6}=Q/2S;Q_{2,5}=-Q/2S$; (2) $\sigma_{1.6}=Q/2S;\sigma_{2}=-\sigma_{3}=2Q/3S;\sigma_{4}=-\sigma_{5}=Q/3S.$

分析: 对于(1)而言, 考虑选择两个类似的Gauss面, 分别贯穿$A$的下表面, $B$的上表面; $B$的下表面, $C$的上表面,
这样就与上一题有类似的$\sigma_{2}+\sigma_{3}=0\qquad(1);\sigma_{4}+\sigma_{5}=0\qquad(2).$
接下来运用$C$板的内容里面的场强之和为0的信息, 就有$\sigma_{1}+\sigma_{2}+\sigma_{3}+\sigma_{4}+\sigma_{5}-\sigma_{6}=0\qquad(3)$.
最后用电荷守恒的式子, 就有$\sigma_{3}+\sigma_{4}=0\qquad(4);\sigma_{1}+\sigma_{2}=0\qquad(5);\sigma_{5}+\sigma_{6}=0\qquad(6).$

对于(2)而言, 由于$A,C$之间连了线, 电荷可以自由跑动, 于是$(5),(6)$应当改写为$\sigma_{1}+\sigma_{2}+\sigma_{5}+\sigma_{6}=0\qquad(7).$
$A,C$两板因为相连而等势, 选取$\frac{\sigma_{3}}{\eqs_{0}}d=\frac{\sigma_{4}}{\eqs_{0}}2d\qquad(8)$.
联立求解即可. 

对于$(3),$如果改写成在$A$板子, $B$板子里面列方程, 是等效的, 这样是重复的条件(也就是线性相关的), 对我们的方程组解答没有大影响. 
\end{example}
%
\begin{example}
(8(2)-6) 平行板电容器中一半空间充满介质, 极板带电$Q$时, 求极板 间电场和电势差.

\incfig{prob86}

答案: $U=E_{2}d=\frac{2Qd}{\eqs_{0}S(\eqs_{r}+1)}.$

分析: 考虑因为两板之间的电势差是恒定的, 因此在介质的分界面上电荷会重新分布. 考虑左侧的面密度为$\sigma_{1},$右侧的为$\sigma_{2}.$
下面选取合适的Gauss面求场强. $D_{1}=\sigma_{1},D_{2}=\sigma_{2};E_{1}=\frac{\sigma_{1}}{\eqs_{0}\eqs_{r}},E_{2}=\frac{\sigma_{2}}{\eqs_{0}}.$
电势差相等, $E_{1}d=E_{2}d$, 并且$\sigma_{1}+\sigma_{2}=2Q/S.$ 由此可以得到$E_{2}=\frac{2Q}{\eqs_{0}S(\eqs_{r}+1)}$. 
\end{example}

\subsection*{电容器}
\begin{description}
\item [{1.}] 电容
\begin{description}
\item [{定义:}] $C\equiv Q/U.$
\begin{description}
\item [{通常求法:}] 假设带电$Q$, 求电场, 积分求电势$U$, 使用定义求之. 
\end{description}
\item [{击穿电压:}] 最大是沿最大场强的积分. 
\end{description}
\item [{2.}] 若干个特殊物体的电容: 
\begin{description}
\item [{孤立导体球的电容:}] $C=4\pi\eqs_{0}R$
\item [{平行板电容器的电容:}] $C=\frac{\eqs_{0}\eqs_{r}S}{d}$
\item [{柱形电容器的电容:}] $C=2\pi\eqs_{0}\eqs_{r}L\left(\ln\frac{R_{2}}{R_{1}}\right)^{-1}.$
\end{description}
\item [{3.}] 电容器的串并联
\begin{description}
\item [{并联:}] $U$相等, $C=\frac{Q_{1}+Q_{2}+\cdots+Q_{n}}{U}$
\item [{串联:}] $Q$相等, $C=\frac{1}{C_{1}}+\frac{1}{C_{2}}+\frac{1}{C_{3}}+\cdots+\frac{1}{C_{n}}.$
\item [{判别方法:}] 同一根导线连接, $U$相等, 若干个极板连着极板, $Q$相等. 
\end{description}
\item [{4.}] 电容器的储能: $W=\int\dd W=\int_{0}^{Q}U\dd q=\int_{0}^{Q}\frac{q}{C}\dd q=\frac{Q^{2}}{2C}.$
\begin{description}
\item [{另外的公式:}] 带入$Q=CU$做恒等变换, 得$W=\frac{CU^{2}}{2}=\frac{QU}{2}.$
\end{description}
\end{description}
\begin{example}
(8(2)-5) 平行板电容器中插入一介质板, 求电容. 

\incfig{prob85}

答案: $C=\frac{3\eqs_{0}\eqs_{r}S}{(2\eqs_{r}+1)d}.$

分析: 假设带电, 并且面密度$D=\frac{Q}{S}=\sigma,E_{0}=\frac{\sigma}{\eqs_{0}},E_{1}=\frac{\sigma}{\eqs_{0}\eqs_{r}}.$$U=E_{0}\frac{2d}{3}+E_{1}\frac{d}{3}.$于是就可以根据$C=\frac{Q}{U}$得到答案. 
\end{example}
%
\begin{example}
(8(2)-9) 这里有一个液位计, 内部导体圆柱半径为$r$, 外面导体圆筒的半径为$R$, 高 为$H$. 液体的相对介电常数为$\eqs_{r}$,
 求电容与液位高度$x$的关系. 

\incfig{ex829}

答案: 
\[
C=\frac{\lambda_{2}(H+(\eqs_{r}-1)x)}{\frac{\lambda}{2\pi\eqs_{0}}\ln\frac{R}{r}}
\]

分析: 设电荷线密度介质部分,真空部分分别为$\lambda_{1},\ensuremath{\lambda_{2}.}$ $D_{1}=\frac{\lambda_{1}}{2\pi r},D_{2}=\frac{\lambda_{2}}{2\pi r}$.
求出$E,$然后因为电势差相等, 就有$\int_{r}^{R}E_{1}\dd r=\int_{r}^{R}E_{2}\dd r$,
并且电荷守恒有$\lambda_{1}+\lambda_{2}(H-x)=Q$, 就可以解答. 
\end{example}
%
\begin{example}
(8(2)-12)平行板电容器面积为$S$, 间距为$d$, 放在方形金属盒内, 两 板距盒壁距离为$a$, 从两板接出的两极之间的电容是多少? 

\incfigw{ex812}

答案: $\eqs_{0}S\left(\frac{1}{2a}+\frac{1}{d}\right).$

分析: 考虑把相等电势的电容放在一起. 然后对右侧的图示做串并联分析(就像中学做电路分析一样), 就有$(\text{下}+\text{上})//\text{中}.$于是就可求得. 
\end{example}
%
\begin{example}
(8(2)-10) 两个电容器的电容之比$C_{1}:C_{2}=1:3$.把它们串联起来 接电源充电, 它们的电场能量之比 $W_{1}:W_{2}=\qst$;如果是并联起
来接电源充电, 则它们的电场能量之比 $W_{1}:W_{2}=\qst$. 

答案: 3:1; 1:3.

分析: 串联, $Q$相同, 因此可以用$Q^{2}/2C;$并联$U$相同, 因此可以用$CU^{2}/2.$
\end{example}
%
\begin{example}
(8(2)-11) 两个电容均为C的电容器带电量分别为$Q$和$2Q$, 求并联后 极板电量和能量增量. 

答案: 电量增量: 都为$\left(3/2\right)Q;$能量增量为$-Q^{2}/4C$. 

分析: 先求总电量$q_{1}+q_{2}=3Q.$由于电压相同进行重新分配, 也就是$\frac{q_{1}'}{C_{1}}=\frac{q_{2}'}{C_{2}}\implies q_{1}'=q_{2}'.$根据$W=\frac{Q^{2}}{2U}$就可以得到值. 
\end{example}

\subsection*{电场的能量}
\begin{description}
\item [{1.}] 电场的能量公式: $W=\int_{(a)}^{(b)}DE\dd V.$
\end{description}
\begin{example}
(综合复习题) 半径为$R$的导体球和内、外分别为$2R$、$3R$的导体球壳同心放 置, 其间有均匀电介质, 相对介电常数为$\eqs_{r}$.
若给外导体球壳带电 $Q$, 并将内球接地, 求:(1)各表面的带电量;(2)电场分布;(3)导体 球和球壳的电势差;(4)电场的总能量

答案: 这个答案太复杂了, 问问老师到时候算出答案能不能不带入. 

分析: (1) 设内球带电量为$q,$那么从内到外的各个电荷量为$q,-q,Q+q.$ (2) 先求$D$的大小, 再求$E$的大小.
从内到外的各个$D$分别是$E_{1}=0,E_{2}=\frac{q}{4\pi r^{2}},E_{3}=0,E_{4}=\frac{Q+q}{4\pi r^{2}};$各个电场分别是$E_{1}=0,E_{2}=\frac{q}{4\pi\eqs_{0}\eqs_{r}r^{2}},E_{3}=0,E_{4}=\frac{Q+q}{4\pi\eqs_{0}r^{2}}.$(3)电势差积分即可.
(4) 表达式为$W=\int_{0}^{\infty}\frac{1}{2}DE\green{\dd V}=\int_{R}^{2R}\frac{1}{2}D_{2}E_{2}\green{4\pi r^{2}\dd r}+\int_{3R}^{\infty}\frac{1}{2}D_{4}E_{4}\green{4\pi r^{2}\dd r}=\frac{4\pi q^{2}}{16\pi^{2}r^{4}\eqs_{0}\eqs_{r}}\cdot\frac{7R}{3}-\frac{4\pi(Q+q)^{2}}{16\pi^{2}r^{4}\eqs_{0}}\cdot\frac{27R}{3}.$但是我们现在还不知道$q$,
根据$U$导体球电势为0, 可以得到方程式$U=\frac{q}{4\pi\eqs_{0}\eqs_{r}}\left(\frac{1}{R}-\frac{1}{2R}\right)+\frac{Q+q}{4\pi\eqs_{0}}\frac{1}{3R}=0.$于是知道$q=\frac{-Q\eqs_{r}}{\eqs_{r}-3}.$
带入原方程就行了. 
\end{example}

\section{磁场中的常见问题}

\subsection*{恒定电流与其激发的磁场}
\begin{description}
\item [{1.}] 电流
\begin{description}
\item [{形成:}] 载流子的定向移动形成电流
\begin{description}
\item [{传导电流.}] 载流子在电场作用下的定向移动
\item [{运流电流.}] 带电体的机械运动
\item [{位移电流.}] 与电场的变化有关
\end{description}
\item [{强度.}] 单位时间内通过任一截面的电量. $I=\lim_{\Delta t\to0}\frac{\Delta q}{\Delta t}=\frac{\dd q}{\dd t}=neSu$,
$n$是数密度; $S$是横截面面积; $u$是电子的漂移速率. 
\item [{线,面上的电流状况}]~
\begin{description}
\item [{体电流的面密度.}] $\sigma=\frac{\dd I}{\dd S_{\perp}}$(A/m$^{2}$)
\item [{面电流的线密度.}] $j=\frac{\dd I}{dl_{\perp}}$(A/m)
\end{description}
\end{description}
\item [{2.}] 电源和电动势
\begin{description}
\item [{a.}] 电路 
\begin{description}
\item [{内电路:}] 外力克服静电场力做功, 将电荷从低电位转移到高电位
\item [{外电路:}] 电荷在电场的作用下由高电位到低电位形成电流. 
\end{description}
\item [{b.}] 电动势: 把单位正电荷从电源负极移动到正极, 非静电力做的功. 
\begin{description}
\item [{方向:}] 外电路从高到低, 内电路从低到高. 
\end{description}
\end{description}
\item [{3.}] 特殊运动情形下的电流
\begin{description}
\item [{a.}] 圆电流: 带电体绕定点的旋转
\begin{description}
\item [{均匀带电杆子.}] 圆面电流$\dd I=\frac{\lambda\dd r}{\frac{2\pi}{\omega}}.$
\item [{均匀带电圆盘.}] 圆面电流$\dd I=\frac{\sigma2\pi r\dd r}{\frac{2r}{\omega}}.$
\end{description}
\end{description}
\item [{4.}] 磁场
\begin{description}
\item [{a.}] 特性: 运动电荷周围存在磁场, \textbf{磁场对周围运动电荷有力的作用}; 是物质, 客观存在;\textbf{服从叠加原理}
\item [{b.}] 磁感应强度$\vec B$. 速度为$v$的运动电荷$q_{0}$受Lorentz力
\begin{description}
\item [{大小:}] $f_{m}=q_{0}vB\sin\theta$
\item [{方向:}] $\vec{f_{m_{\text{max}}}}\times\vec v.$
\item [{单位:}] T
\end{description}
\item [{c.}] 磁感应线. 描述磁场空间分布的假想的曲线
\begin{description}
\item [{关系.}] 切线方向为$\vec B$的方向, 数密度为$\vec B$的大小: $B=\frac{\dd\varphi_{m}}{dS_{\perp}}$,
又称为磁通密度. 
\item [{性质.}] (1) 无头无尾的闭合曲线; (2) 曲线方向与电流成右手螺旋; (3) 曲线与电流相互铰链. 
\end{description}
\item [{d.}] 曲面$S$的磁通量. $\varphi_{m}=\int_{S}\vec B\cdot\dd\vec S$.
\item [{e.}] 磁场的Gauss定理: $\oint_{S}\vec B\cdot\dd\vec S=0.$ 这表明了磁场没有原,
但是有旋. 
\end{description}
\item [{5.}] 常见内容的磁场极其叠加
\global\long\def\dl{I\dd\vec l}%

\begin{description}
\item [{a.}] 电流元$I\dd\vec l$的磁场. $\dd\vec B=\frac{\mu}{4\pi}\frac{\dl\times\vec{e_{r}}}{\green{r^{2}}},$其中,
$\vec{e_{r}}$的方向是当前的微元指向要判定的点的矢量. 
\item [{b.}] 载流直导线的磁场. $\vec B=\frac{\mu_{0}I}{4\pi a}\left(\cos\theta_{1}-\cos\theta_{2}\right)\vec e$. 
\item [{c.}] 圆环中心轴线距离$x$处上的磁场: $\frac{\mu_{0}I}{2}\frac{R}{x^{2}+R^{2}}\frac{R}{\sqrt{x^{2}+R^{2}}}$
\end{description}
\end{description}
\begin{example}
(9(1)-6) 有一闭合回路由半径为 $a$ 和 $b$ 的两个同心共面半圆连接而成,  如图其上均匀分布线密度为 $\lambda$
的电荷, 当回路以匀角速度 $\omega$ 绕 过 $O$ 点垂直于回路的轴转动时, 求圆心 $O$ 点处的磁感应强度. 

答案: $\frac{\mu_{0}\lambda\omega}{2\pi}\left(\pi+\ln\frac{b}{a}\right)$.

分析: 分为两个部分: 圆面电流和直线电流. 首先来看圆面电流: $I_{a}=\frac{q}{T}=\frac{\lambda\pi a}{2\pi/\omega}=\frac{1}{2}\lambda a\omega.$
$B_{0}=\frac{\mu_{0}I_{a}}{2a}.$同理可以看到另一个小圆环的情形. 他们的方向竖直向外. 

接下来看直线段的电流: 其上电荷元$\dd q$旋转形成等效的圆电流. 也就是$\dd I_{r}=2\frac{\dd q}{T}=2\frac{\lambda\dd r}{2\pi/\omega}=\frac{\omega}{\pi}\lambda\dd r.$这个方向向外.
然后我们求这一小段的$\dd\vec B$的大小. $\dd B_{r}=\frac{\mu_{0}\dd I}{2\pi r}$.
积分之后就可以得到. 
\end{example}
%
\begin{example}
(课堂练习17-2) 在一半径为$R$的无限长半圆柱形 金属薄片中, 自上而下地有电流强度$I$ 通过, 如图示, 试求圆柱轴线任一点$P$
处的磁感应强度. 

\incfig{ring}

答案: 沿$y$轴负向, 大小为$-\frac{\mu_{0}I}{\pi^{2}R}.$

分析: 由于这是和圆面有关的, 于是使用极(柱)坐标系考虑. 于是$\dd I=\frac{I\dd r}{\pi R}=\frac{IR\dd\theta}{\pi R}.$
那么电流元的磁感应强度大小为$\dd B=\frac{\mu_{0}\dd I}{2\pi r}.$ 从$0\to2\pi$积分即可. 
\end{example}
%
\begin{example}
(课堂练习17-4) 在半径为$R$及$r$ 的两圆周之间, 有总匝数为$N$的均匀密绕 平面螺线圈如图示, 当导线中通有电流I时,求螺线圈中心点(即两圆圆心)处的磁感应强度
. 

\incfig{ex174}

答案: $\frac{\mu_{0}NI}{2(R-r)}\ln\frac{R}{r}.$

分析: 取半径为$a$, 宽度为$\dd a$的窄环, 则电流为$\dd I=\frac{N}{R-r}I\dd a.$那么$\dd B_{\text{圆心}}=\frac{\mu_{0}\dd I}{2\pi a}.$积分即可. 
\end{example}
%
\begin{example}
(9(1)-2) 电流由长直导线 1 沿半径方向经 $a$ 点流入 一电阻均匀分布的圆环, 再由 $b$ 点沿半径方 向从圆环流出,
经长直导线 2 返回电源(如 图). 已知直导线上电流强度为 $I$, 求圆心 $O$ 点的磁感应强度. 

\incfig{eleprob}

答案: 0.

分析: 首先根据结论, 在延长线上的线,在中心的磁感应强度为0. 其次, 考虑圆周上的点, 我们知道结论$B=\varphi\frac{\mu_{0}I}{2\pi r}.$
于是有$I_{1}=\frac{I}{2\pi}(2\pi-\varphi).$一个方向向里, 另一个方向向外. 因此可以抵消. 
\end{example}

\subsection*{环路定理}
\begin{description}
\item [{1.}] 环路定理: $\oint_{L}\vec B\cdot\dd\vec l=\mu_{0}I_{\text{内}}.$
\item [{2.}] 对环路定理的正确理解: 
\begin{description}
\item [{$L$.}] 在场中任取的一闭合路径, 任意一个绕行方向
\item [{$\dd L$.}] $L$上的任意一个线元
\item [{$\vec B$.}] 线元处的磁场, 由空间所有电流共同产生
\item [{$I_{in}$.}] 与$I$铰链的电流
\item [{$\sum I_{in}$.}] 代数和, 取与$L$方向右手螺旋的电流取证
\end{description}
\end{description}
\begin{example}
(课上展示20-14) 求下列轴对称电流分布周围激发的磁场. (1) 无限长直电流; (2) 无限长圆柱面电流; (3) 无限长圆柱体电流. 

答案: (1)$B=\frac{\mu_{0}I}{2\pi r};$(2) $B=\begin{cases}
0 & r<R\\
\frac{\mu_{0}I}{2\pi r} & r>R
\end{cases};$(3)$B=\begin{cases}
\frac{\mu_{0}rI}{2\pi R^{2}} & r<R\\
\frac{\mu_{0}I}{2\pi r} & r>R
\end{cases}.$

分析: (1) 考虑半径为$r$的与电流垂直的圆环路, 如图所示. 那么有$\mu_{0}I=B2\pi r.$ (2) 考虑和上面一样的环路,
就有$\mu_{0}I_{\text{内}}=B2\pi r.$ (3) 考虑$\mu_{0}I_{\text{内}}=B2\pi r.$其中$I_{\text{内}}=\frac{r^{2}I}{R^{2}}(r<R).$
\end{example}
%
\begin{example}
(课上展示20-15) 求下列面对称电流的分布. (1) 无限大均匀载流平面, 电流线密度为$j;$(2)无限大平面层体电流(电流的线密度为$\sigma$). 

答案: $B=\frac{\mu_{0}j}{2},B=\frac{\mu_{0}\sigma h}{2}.$

分析: 对于(1), 选取与电流方向垂直的部分, 居中放置矩形线框(如长度为$l$). 矩形线框的两个贯穿次场面的两条线相互抵消,
剩下上下两个部分. 因此, 就有$B2l=\mu_{0}jl\to B=\frac{\mu_{0}j}{2}.$ (2) 同样选取类似的内容,
就有$B2l=\mu_{0}\sigma hl.$
\end{example}
%
\begin{example}
(课上展示20-15) 密绕螺线环如下所示, 求内部的磁场. 

\incfig{ring}

答案: $B=0,R<R_{1},R>R_{2};B=\frac{\mu_{0}NI}{2\pi r},R_{1}<R<R_{2}.$

分析: 对于小于$R_{1}$和大于$R_{2}$的情形, 会发现净电流总和为0. 否则, 使用圆形的线框作用, 就有$B2\pi r=\mu_{0}NI.$
\end{example}
%
\begin{example}
(9(1)-7) 长载流导体直圆管, 内半径为 $a$, 外半径为$b$, 电流强度为 $I$,  电流沿轴线方向流动, 并且均匀分布在管壁的横截面上.
空间 某点到管轴的垂直距离为$r$. 求$r<a,a<r<b,r>b$等各区 间的磁感应强度. 导体内部$\mu\approx\mu_{0}.$

答案: 分别为$B_{0}=0,B_{1}=\frac{\mu_{0}I(r^{2}-a^{2})}{2\pi r(b^{2}-a^{2})},B_{2}=\frac{\mu_{0}}{2\pi r}.$
\end{example}

\subsection*{磁通量的计算}
\begin{description}
\item [{1.}] 计算方法: 一般采用微元累计的方法. 
\end{description}
\begin{example}
(课上演示20-16) 求(1) 密绕长直螺线管(内部为均匀磁场)的垂直于竖直截面的磁通量; (2) 无限大均匀载流平面, 电流线密度为$j$(均匀磁场). 

\incfigw{magf1}

答案: $\mu_{0}nIS,BS\cos\theta.$

分析: (1) 选取和上述一样的一段矩形线框, 根据Ampere环路定理得到$B_{0}=\mu_{0}NI.$ 由于是均匀的,
$\phi_{m}=\int_{S}\vec B\cdot\dd\vec S=BS=\mu_{0}nIS.$ (2) 无限大均匀载流平面选取对应的载流平面,
$B=\frac{\mu_{0}j}{2},\phi_{m}=BS\cos\theta.$
\end{example}
%
\begin{example}
(课上演示20-17) 求(1) 无限长直电流外侧如图所示的区域磁通量的计算; (2) 密绕螺绕环单侧截面的磁通量的计算. 

\incfigw{ms2}

答案: $\frac{\mu_{o}Ib}{2\pi}\ln\frac{r_{0}+a}{r_{0}};\frac{N^{2}\mu_{0}Ih}{2\pi}\ln\frac{R_{2}}{R_{1}}.$

分析: (1) 无限长直导线周围的磁场为$B=\frac{\mu_{0}I}{2\pi r}.$选取一位置为$a$, 宽度为$\dd a$的窄竖条,
就有$\dd\vec S=b\dd r\vec{e_{n}}.$ $\phi_{m}=\int_{S}B\dd S=\int_{r_{0}}^{r_{0}+a}\frac{\mu_{0}I}{2\pi r}.$(2)
同样可以知道$B2\pi r=\mu_{0}NI.$同样根据定义$\phi_{m}=N\varphi_{m}=N\int_{S}B\dd S=N\int_{R_{1}}^{R_{2}}\frac{N\mu_{0}I}{2\pi r}h\dd r$. 
\end{example}
%
\begin{example}
(9(1)-8) 一根很长的铜导线均匀载有电流$I$,  在导线内部作一平面 $S$ , 如图示, 试计算通过 $S$ 平面的磁通量
(沿导线长度方向 取长为1米的一段作计算). 铜的磁导率$\mu\approx\mu_{0}.$

\incfig{ex917}

答案: $B=\frac{\mu_{0}Ir}{2\pi R^{2}}(0<r<R);\phi_{m}=\frac{\mu_{0}}{4\pi}I.$ 

分析: 首先求出$B$的大小, 方向沿环的切向, $r$相同的地方$B$相同. 取距轴$r$处,宽$\dd r$的窄条,
其上$B$相同, $\dd\phi_{m}=\vec B\cdot\dd\vec S=B\dd S=Bl\dd r.$积分即可得到$\phi_{m}.$

注意: 一般问磁通量是问磁通最大的地方. 
\end{example}

\subsection*{磁力与磁力矩}
\begin{description}
\item [{1.}] 带电粒子在磁场的运动
\begin{description}
\item [{a.}] Lorentz力. $\vec f=q\vec v\times\vec B$. 大小$f=qvB\sin\theta.$
方向: Lorentz力只改变速度的方向, 不改变速度的大小, Lorentz力不做功. 
\item [{b.}] 带电粒子在磁场中的运动. $R=\frac{mv_{\perp}}{Bq},h=v_{//}T=\frac{2\pi mv_{//}}{Bq}.$
\item [{c.}] Hall效应: $q\frac{\Delta U_{H}}{h}=qvB\to\text{\ensuremath{\Delta H=\frac{1}{nq}\frac{IB}{b}.} }$判定电势高低的时候注意是在电源的内部. 
\end{description}
\item [{2.}] 载流导线在磁场中的受力
\begin{description}
\item [{a.}] Ampere力: $\vec F=\dl\times\vec B.$
\end{description}
\item [{3.}] 磁力矩
\begin{description}
\item [{矩形载流线圈:}] 载流线圈在磁场中的情形
\begin{description}
\item [{合力与合力矩:}] 合力为0, 合力矩不为0. 而是
\begin{align*}
M & =\frac{1}{2}l_{1}F_{1}\sin\theta+\frac{1}{2}l_{1}F_{2}\sin\theta\\
 & =Il_{1}l_{2}B\sin\theta\\
 & =ISB\sin\theta\\
\red{\vec M} & =\green{IS\vec{e_{n}}\times\vec B}
\end{align*}
\end{description}
\item [{磁矩.}] $\vec{p_{m}}=IS\vec{e_{n}}$, 其中$e_{n}$与线圈电流方向满足右手螺旋. 
\item [{磁力矩.}] $\red{\vec M}=\green{IS\vec{e_{n}}\times\vec B}=\red{\vec{p_{m}}\times\vec B}$
\item [{结果.}] 在磁力矩的作用下, 线圈转向与磁场方向正平行. 
\end{description}
\end{description}
\begin{example}
(9(2)-6) 如图示, 一条任意形状的载流导线位于均 匀磁场中, 试证明它所受到的Ampere力等于载流 直导线$ab$所受到的Ampere力. 

分析: 电流元的受力为$\dd\vec F=\dl\times\vec B.$对路径积分即可得到$\vec F=I\left(\int_{a}^{b}\dd l\right)\times\vec B=I\vec{ab}\times\vec B.$
\end{example}
%
\begin{example}
(课堂例子20-2) 一圆柱形磁铁$N$ 极上方水平放置一半径为$R$ 的圆 电流$I$, 圆电流所在处磁场的方向处处均与竖直方向成$\alpha$角,
求圆 电流所受磁力. 

\incfig{Bexa}

答案: $2\pi RIB\sin\alpha$. 

分析: 在圆电流上去一小电流元, 在对称的时候$\dd\vec F=\dl\times\dd\vec B$. 根据对称性分析可以知道,
$\oint_{2\pi R}\dd F_{x}=0,$那么$F=\oint_{2\pi R}\dd F_{y}=2\pi RIB\sin\alpha.$
\end{example}
%
\begin{example}
(课堂例子20-3) 这里有一个电磁轨道炮. 其形成机制是有两个相距为$d$的长直的导线, 导线截面半径为$r_{0}.$在这中间有导体棒$ab$可以在上面滚动.
现在, 这两个长直导体棒被通入了大小为$I$的电流. 现在问这个导体棒的受力是多少. 

答案: $\frac{\mu_{0}I^{2}}{2\pi}\ln\frac{d-r_{0}}{r_{0}}.$

分析: 我们知道一个半无限长的导体棒在距离$r$处形成的磁场大小为$\frac{\mu_{0}I}{4\pi r}.$考虑一小段$\dd I,$其中它的长度是相同的,
这样子就得到了$2\int_{ab}I\dd rB=2\int_{r_{0}}^{d-r_{0}}\frac{\mu_{0}I^{2}}{4\pi r}\dd r.$
\end{example}
%
\begin{example}
(9(2)-11) 有一片半径为$R$, 圆心角为$\theta$的扇形薄板, 均匀带电, 电荷面 密度为 $\sigma$,
它绕过$O$点垂直于扇面的轴以匀角速率$\omega$顺 时针旋转, 旋转带电扇形薄板的磁矩大小为$\qst$,方向为$\qst.$

答案: $\frac{1}{8}\sigma\theta\omega R^{4}.$

分析: 由于是带电体的转动, 因此考虑取$r$处, 宽度为$\dd r$的细线段旋转形成的细小圆环面. 因此, 对应的电流是
\[
\dd I=\frac{\dd q}{T}=\frac{\sigma\theta r\dd r}{2\pi/\omega}.
\]
那么$\dd p_{m}=\dd I\cdot S=\frac{\sigma\theta r}{2\pi/\omega}\dd r,$总磁矩积分即可. 
\end{example}

\section{电磁感应中的常见问题}

\subsection*{Faraday的电磁感应定律}
\begin{description}
\item [{1.}] Faraday电磁感应定律
\begin{description}
\item [{感应电动势.}] $\eqs=-\frac{\dd\phi_{m}}{\dd t}=\frac{\dd}{\dd t}\int_{S}\vec B\cdot\dd\vec S$.
(负号表示和磁通量的变化方向相反 – $\eqs$与$L$的绕行方向相反. ) 其中$\phi_{m}=\oint_{S}\vec B\cdot\dd\vec S>0.$
\begin{description}
\item [{引起磁通量变化的方式:}] (1)$B$不变, 导线运动使得$S$变化, 形成动生电动势; (2)$S$不变, 导线回路的$B$发生变化.
感生电动势. 
\end{description}
\item [{磁链.}] $N$匝串联的回路的总磁通$\phi_{m}=\sum_{i}\phi_{mi}$. 
\end{description}
\item [{2.}] 纯电阻电路中的感应电流和感应电荷
\begin{description}
\item [{感应电流.}] $I=\frac{\eqs}{R}=-\frac{1}{R}\frac{\dd\phi_{m}}{\dd t}.$
\item [{通过的电量.}] $q=\int_{\Delta t}I\dd t=\int_{\Delta t}-\frac{1}{R}\frac{\dd\phi_{m}}{\dd t}\dd t=-\frac{1}{R}\Delta\phi_{m}.$
\end{description}
\end{description}
\begin{example}
(课堂例子20-2) 真空中直导线通交流电, 求:与其共面的$N$ 匝矩形回路中的 感应电动势. (1) 已知$I=I_{0}\sin\omega t$.
(2)$I$ 恒定, 平面线圈以匀速率$v$向右运动,  结果如何? 

答案: (1)$\eqs=-\frac{\mu_{0}NI_{0}l\omega}{2\pi}\ln\frac{d+a}{d}\cos\omega t$
(2) $\eqs=\mf{2\pi}N\frac{va}{(d+a+vt)(d+vt)}.$

分析. 先求$B$, 并且根据回路的方向按照右手螺旋的规则求$\dd\varphi_{m},$然后积分求磁通量
\[
\phi_{m}=N\int_{d}^{d+a}\mf{2\pi x}l\dd x=\mf{2\pi}Nl\ln\frac{d+a}{d},
\]
 最后让磁通量对时间求导即可. 

对于第二问, 选取任一时刻, 然后考虑
\[
\phi_{m}=N\int_{d+vt}^{d+a+vt}\mf{2\pi x}l\dd x
\]
即可. 
\end{example}

\subsection*{动生电动势}
\begin{description}
\item [{1.}] 动生电动势: $\eqs=-\frac{\dd\phi_{m}}{\dd t}=-\frac{\dd}{\dd t}(BS)=-Bl\frac{\dd s}{\dd t}=-Blv$,
回路$L$方向同$B$方向. 
\item [{2.}] 机制: 受到Lorentz力$\vec{f_{m}}=-e\vec v\times\vec B.$
\begin{description}
\item [{推导.}] $\eqs=\int(\vec v\times\vec B)\cdot\dd\vec l.$
\item [{转动公式.}] $\eqs=\int_{(a)}^{(b)}-\omega xB\dd x=\green{-\frac{1}{2}\omega Bl^{2}.}$
\item [{推论.}] 均匀磁场中, 任意形状导线$ab$绕$a$端转动, $\eqs=\int_{(a)}^{(b)}-\omega xB\dd x=\green{-\frac{1}{2}\omega B(\overline{ab})^{2}.}$
\end{description}
\end{description}
\begin{example}
(11-2) 在均匀磁场$B$中的导线$oab$, 形状如图示, 当其绕 $o$点以角速 度$\omega$在平面内转动,
求导线$oab$上的动生电动势的大小, 哪端电 势高? 已知弧$ab\lyxmathsym{为}$半径$R$的$3/4$圆弧,
且$oa$长度为$R$. 

答案: $\frac{5}{2}\omega BR^{2}.$

分析: 根据上述的结论, 等同于一个直杆子在运动. 于是可以计算出答案. 
\end{example}
%
\begin{example}
(11-10) 导线杆$ab$以速率$v$在导线导轨$adcb$上平行移动, 杆$ab$在 $t=0$时, 位于导轨$dc$处.
均匀磁场的磁感应强度为$B=B_{0}\sin\omega t$, 求$t$时刻导线回路中的感应电动势? 

答案: $-B_{0}lv(\omega t\cos\omega t+\sin\omega t).$

分析: 先求$\phi_{m}.$ 注意时间变量的引入, 也就是$\phi_{m}=BS=vtlB_{0}\sin\omega t.$
\end{example}

\subsection*{感生电动势}
\begin{description}
\item [{1.}] 感生电动势:
\begin{description}
\item [{性质.}] 涡旋电场 
\item [{大小.}] $\eqs_{i}=\oint_{L}\vec{E_{i}}\cdot\dd\vec l$(非静电力对单位正电荷绕导线回路做功)
\item [{方向.}] (1)$S$以$L$为边界, 成右手螺旋关系.
\end{description}
\item [{2.}] 求解感生电动势. 
\begin{description}
\item [{使用关系$\eqs_{i}=\oint_{L}\vec{E_{i}}\cdot\dd\vec l=\int_{S}-\frac{\partial\vec B}{\partial t}\cdot\dd\vec S.$}]~
\begin{description}
\item [{方向:}] (1)$S$以$L$为边界, 成右手螺旋关系. (2) 只有在$\vec{E_{i}}$有特殊分布的时候才可以计算出来. 
\item [{对称条件:}] 限制在圆柱体内的均匀磁场, 磁感强度方向 平行于轴线. 当磁场随时间变化时, 激发的感 生电场具有轴对称分布的特性. 
\end{description}
\item [{添加构成封闭区域}]~
\end{description}
\item [{3.}] 与静电场的区别: 静电场有源无旋, 是保守场. 而涡旋电场无源有旋, 是\textbf{非保守场}.
\end{description}
%
\begin{example}
(课上演示-20(3)-3) 右图的时变磁场激发的感生电场, 求它的电场分布. 

答案: $E_{i}=-\frac{S}{2\pi r}\frac{\dd B}{\dd t}.$

分析: 选取同心圆环, 有
\[
\oint_{L}\vec{E_{i}}\cdot\dd\vec l=\oint_{2\pi r}E_{i}\dd l=E_{i}2\pi r
\]

同时由于Faraday的电磁感应定律, 就有
\[
\int_{S}-\frac{\partial\vec B}{\partial t}\cdot\dd\vec S=-\frac{\dd\vec B}{\dd t}\int_{S}\dd\vec S=-S\frac{\dd B}{\dd t}.
\]

联立即可得到答案. 
\end{example}
%
\begin{example}
(课上演示20(3)-9) 截面半径为R的长直螺线管中的磁场$\frac{\dd B}{\dd t}=C.$ (1) 求沿半径方向放置的导线$oa$
上的感生电动势; (2) 求导线ab上的感生电动势; (3) 求导线cd上的感生电动势. 

答案: (1) 0, (2) $-S_{\triangle}\frac{\dd B}{\dd t}$, (3)$-S_{\text{扇}}\frac{\dd B}{\dd t}$. 

分析: 由于第一个$oa$上各处$E_{i}$方向垂直于$oa$, 因此是0. 后面两个可以通过构成回路的方法完成这样的操作.
把边界的问题转化为了区域内的问题. 

这就和Green公式, Gauss公式有很相似的地方. 
\end{example}

\end{document}
