\chapter{Manipulating sums}

\section{Basic Notations}

\mkpt{Summation notation} We can use $\sum$ to denote add things together.
\begin{itemize}
    \item $\sum_{a}$: sum over all items in $a$
    \item $\sum_{i=1}^{n} f(i)$: sum from $i=1$ to $i=n$, substituting all $i$s
        in the right $f(i)$.
\end{itemize} 

\mkpt{Iverson's Bracket} We have the notation $[p]$: $p$ is a proposition, and
when $p$ is true, the equation is evaluated 1, otherwise evaluated as 0. 

\mkpt{Using Aversion's bracket to simplifying sums} For example we have 
\[
    \sum_{p\text{ prime},p \leq N} p = \sum_p [p\text{ prime}][p \leq N] p.
\]

\section{Manipulation of sums}

\mkpt{Distributive law} $\sum_{k\in K} ca_k=c\sum_{k\in K}a_k$.

\mkpt{Associative law} $\sum_{k\in K} (a_k+b_k)=\sum_{k\in K}a_k+\sum_{k\in K}b_k$.

\mkpt{Commutative law} $\sum_{k\in K} a_k=\sum_{p(k)\in K}a_{p(k)}$. 

This rule is for mainly substitution use. When the condition is not specified, the
$p(k)$ should be a permutation of all integers. 

Example. We have 
\[
    \sum_{k\in K, k \texttt{ even}}^{} a_k = \sum_{n\in K, n \texttt{ even}}^{}a_n
    =\sum_{2k \in k}^{a_{2k}}
\]


\mkpt{Example 1: Arithmetic's progression} We have $S=\sum_{0\leq k\leq
n}(a+bk)$. By commutative law, replace $k$ by $n-k$, obtaining $S=\sum_{0\leq
n-k\leq n}(a+b(n-k))=\sum_{0\leq k \leq n} (a+bn-bk)$. These two can be added
up to $2S=\sum_{0 \leq k\leq n} (2a+bn)$. Then this problem would be trivial. 

\mkpt{Exclusion and Inclusive Principle with Sums} Here is a important rule for combining
different set of indicies. 

Suppose $K$ and $K'$ are any set of integers, then 
\[
    \sum_{k \in K}^{}a_k + \sum_{k\in K'}^{}a_k = \sum_{k\in k \cap K'}^{}a_k +
    \sum_{K\in K \cup K'}^{}a_k.
\]
In the case of Iverson's bracket, we have 
\[
    [k\in K]+[k\in K'] = [k \in K \cap K'] + [K \in K \cup K']
\]
\mkpt{Perturbation method} We first write $S_n = \sum_{0 \leq k \leq n}^{}a_k$,
then we rewrite $S_{n+1}$ in 2 ways: 
\begin{align*}
    S_n +a_{n+1} &= \sum_{0 \leq k \leq n+1}^{} &= a_0 + \sum_{1\leq k\leq n+1}^{}a_k  \\
                 & &= a_0 + \sum_{1\leq k+1 \leq n+1}^{}a_{k+1}\\
                 & &= a_0 + \sum_{0 \leq k \leq n}^{}a_{k+1}.
\end{align*}

Now we can work on the last term trying to solving the closed form for it. 

\mkpt{Example 2. Geometry progression} We have $S_n = \sum_{0 \leq k \leq
n}^{}ax^k$. And the sum is needed. Using {\it perturbation} method we have 
\[
    S_n + ax^{n+1} = ax^0 + \sum_{0\leq k\leq n}^{}ax^{k+1}
\]

Factoring out an $x$, we have
\begin{align*}
    S_n + ax^{n+1} &= a + xS_n \\
    \sum_{k=0}^{n} ax^k = \frac{a-ax^{n+1}}{1-x}, \qquad \text{for } x \neq 1
\end{align*}

\mkpt{Example 3. Arithmetic Geometric progression} We have the sequence 
\[
    S_n = \sum_{0 \leq k \leq n}^{}k 2^k
\]

Using the perturbation technique, we have 
\[
    S_n + (n+1)^{2n+1} = \sum_{0\leq k\leq n}^{}(k+1)2^{k+1}
\]
Rewrite RHS as sums, we have 
\[
    S_n + (n+1)^{2n+1} = \sum_{0\leq k\leq n}^{}k 2^{k+1} + 
    \sum_{0\leq k\leq n}^{}2^{k+1}
\]
Hence $\sum_{0\leq k\leq n}^{}k 2^k = (n-1)2^{n+1}+2$.

\section{Multiple sums}

\mkpt{Notations} Stacking multiple sums in a row helps to derivate more
complex sums. 

Example: $\sum_{i=1}^{3}\sum_{j=1}^{3}a_{ij} = a_{11}+a_{12}+a_{13}+\cdots
+a_{33}$.

Using the simplified Iverson's bracket, getting another way of expressing
multiple sums: $\sum_{1\leq j,k\leq 3}^{}a_j b_k = \sum_{j,k}^{}a_jb_k[1\leq
j\leq 3] [1\leq k\leq 3]$. 

\mkpt{General distributive law} For distinct lower set index $i,j$, we have the
following distributive law: 
\[
    \sum_{j\in J,k\in K}^{}a_jb_k = \big( \sum_{j\in J}^{}a_j \big) + \big(
    \sum_{k\in K}^{}b_k \big)
\]
Another general form for this is $\sum_{j\in J}^{}\sum_{k\in
K(j)}^{}a_{j,k}=\sum_{k \in K'}^{} \sum_{j\in J'(k)}^{}a_{j, k}$. Here, the sets
should satisfy $[j \in J][k \in K(j)] = [k \in K'][j \in J'(k)]. $

\mkpt{Example 1. Consecutive integers} We may rewrite $[1\leq j\leq n][j\leq
k\leq n] = [1\leq j\leq k\leq n] = [1\leq k\leq n][1\leq j\leq k]$. 

\mkpt{Example 2. Sum over an matrix} 
We try to find a simplified idea of the matrix
\[
    \left(\begin{matrix}
            a_1a_1& a_1a_2 & \cdots & a_1a_n \\
            a_2a_1&a_2a_2&\cdots & a_2a_n\\
            \vdots & \vdots & \ddots & \vdots \\
            a_na_1 & a_na_2&\cdots & a_na_n
    \end{matrix}\right)
\]
we shall find a sum $S_1 = \sum_{1\leq j\leq k\leq n}^{}a_ja_k$. 

According to the associativity of the multiplication, we find that $a_j a_k= a_k
a_j$. Hence we have $S_\Delta = \sum_{1\leq j\leq k\leq n}^{}a_ja_k =
\sum_{1\leq k\leq j\leq n}^{}a_ka_j=\sum_{1\leq j\leq n}^{}a_ja_k=S_\nabla$.
According to the problem above, we have $[1\leq j\leq k\leq n]+[1\leq k\leq
j\leq n]=[1\leq j,k\leq n]+[1\leq j=k\leq n]$. We have 
\[
    2S_\nabla = S_\nabla + S_\Delta = \sum_{1\leq j,k\leq n}^{}a_ja_k + 
    \sum_{1\leq j=k\leq n}^{} a_j a_k. 
\]

The first sum is $\left( \sum_{j=1}^{n}a_j \right) \left( \sum_{k-1}^{n} a_k \right) = \left( \sum_{k=1}^{n}a_k  \right)^2$, and the second sum is that $\sum_{k-1}^{n}a_k^2$. 

So we have the sum 
\[
    \sum_{\nabla}^{} = \sum_{1\leq j\leq k\leq n}^{}a_ja_k = \frac{1}{2} \left(
    \left( \sum_{k=1}^{n} a_k \right)^2 + \sum_{k=1}^{n}a_k^2\right).
\]

The idea is like the aligning the data and subtracting the overlapped off.
That's a more religious form. 

\mkpt{Another double sum} We try to evaluate 
\[
    S=\sum_{1\leq j< k\leq n}^{} (a_k-a_j)(b_k-b_j).
\]

We find that this sum satisfies summation: 
\[
    S=\sum_{1\leq j< k\leq n}^{} (a_k-a_j)(b_k-b_j)=\sum_{1\leq j<k\leq n}^{}(a_j-a_k)(b_j-b_k).
\]

We add $S$ to itself, with equation
\[
    [1\leq j<k\leq n]+[1\leq k<j\leq n]=[1\leq j,k\leq n]-[1\leq j=k\leq n].
\]
yields 
\[
    2S=\sum_{1\leq j,k\leq n}^{}(a_j-a_k)(b_j-b_k) - \sum_{1\leq j=k\leq n}^{} (a_j-a_k)(b_j-b_k)
\]

The second part is 0, so we have to evaluate the first sum.
The first sum expand it and swapping the summation index
we have the identity 
\[
    2 \sum_{1\leq j,k\leq n}^{n}a_kb_k + 2 \sum_{1\leq j,k\leq n}^{} a_jb_k
\]
We have sums over $j,k$, extracting $j$ out, we have 
\[
    2n \sum_{1\leq j\leq n}^{n}a_kb_k + 2 \left(\sum_{1\leq j\leq n}^{} a_j  \right)^2 \left( \sum_{i\leq k\leq n}^{} b_k\right)^2
\]

\mkpt{Chebyshev's monotonic inequalities } This is a special case for the 
last case as an example, that is 
\begin{align*}
    \left( \sum_{i=1}^{n} a_k \right) \left( \sum_{k=1}^{n} b_k \right) 
    &\leq n \sum_{k=1}^{n} a_k b_k \\
\end{align*}
if $a_1\leq a_2\leq \cdots\leq a_n$ and $b_1\leq b_2\leq \cdots\leq b_n$. 
and vice versa. 

\mkpt{Changing the index as the single sums} In the index is changed in 
single sums, we have that 
\[
    \sum_{k \in K}^{}a_k = \sum_{p(k) \in k}^{} a_k,
\]
if $p(k)$ as the permutation of the integers. 

We now generalize $k$ by $f(j)$, where $f$ is an arbitrary function: 
\[
    f:j\to K
\]
that takes an integer $j \in J$ into an integer $f(j) \in K$. In this case
we have the replace formula:
\[
    \sum_{j \in J}^{}a_{f(j)} = \sum_{k \in K }^{} a_k \#f^-(k)
\]
where the $\#f^-(k)$ stands for number of sets in the set $f^-(k)=\{ j|f(j)=k \}$. 

Proof. $\sum_{j \in J}^{}a_{f(j)} = \sum_{j \in J, k \in K}^{}a_k[f(j)=k]=
\sum_{k \in K}^{}a_k \sum_{j \in J}^{}[f(j)=k]$. This yields the answer. 

\mkpt{Example: A fraction sum} We wish to sum 
\[
    S_n = \sum_{1\leq j<k\leq n}^{}\frac{1}{k-j}. 
\]
An attempt shows that 
\begin{align*}
    S_n &= \sum_{j=1}^{n} \sum_{k=j}^{n}\frac{1}{k-j} \\
        &=  \sum_{1\leq k\leq n}^{1\leq k-j<k} &\text{Replacing $j$ by $k-i$} \\
        &= \sum_{1\leq k\leq n}^{0\leq k\leq j-1} \frac{1}{j}\\
        &= \sum_{0\leq k\leq n}^{}H_k.
\end{align*}

The harmonic number must be evaluated to make sure it is evaluated. 

\section{Finite Calculus}

\mkpt{The diff operator} By definition, the Difference operator is defined as 
    \[
        \Delta f(x)=f(x+1)-f(x).
    \]

\mkpt{Falling factorial power}

The falling factorial power is defined as (positive int)
\[
    \fp{x}{m} = x(x-1)\cdots (x-m+1)
\]

The rising factorial power is defined as (positive int) 
\[
    \rp{x}{m}=x(x+1)\cdots (x+m-1)
\]

We have the property of 

\begin{align*}
    \Delta(\fp{x}{m}) &= \fp{x+1}{m} - \fp{x}{m} \\
                      &= mx(x-1)(x-2)\cdots(x-m+2)\\
                      &= m\fp{x}{m-1}
\end{align*}

And the inverse operator of $\Delta$ is $\sum$. 
Still we have the fundamental theorem of differential. 
\[
    \sum_{a}^{b}g(x)\delta x = f(x)_{a}^b = f(b)-f(a).
\]

Example: calculating a falling sum. 
\[
    \sum_{0\leq k< n}^{}\fp{k}{m} = \frac{\fp{k}{m+1}}{m+1} |_0^n = 
    \frac{\fp{n}{m+1}}{m+1} 
\]
and it's easy to analoog to the integrate formula. 

\mkpt{The conversion between ordinary powers and falling powers} We observe the following fact: 
\[
    k^2=\fp{k}{3}/3 + \fp{k}2/2
\]
and the fact that 
\[
    k^3=\fp{k}{3}+3\fp{k}{2}+\fp{k}{1}
\]
The general method of converting can be found at the chapter 
Stirling numbers. 

The falling/rising sum also satisfies binomial theorem. 
\[
    \fp{(a+b)}{m} =\sum_{i=1}^{n}{n\choose i} \fp{a}{i} \fp{b}{m-i},
\]
and 
\[ 
    \rp{(a+b)}{m} =\sum_{i=1}^{n}{n\choose i} \rp{a}{i} \rp{b}{m-i},
\]
\mkpt{Generalization fp and rp to neg int} We observe as the degree decrease
 the division are made in the context. 
 hence we define 
\[
    \fp{x}{-1}=\frac{1}{x+1}
\]
and so on. In general, $\fp{f}{-m}=\frac{1}{(x+1)(x+2)\cdots(x+m}$, for $m>0.$

And we can also define real or even complex valued values here, omitted. 

\mkpt{Properties about fp and rp} 
Law 1. $\fp{x}{m+n}=\fp{x}{m}\fp{(x-m)}{n}$. 

Law 2. The $\Delta \fp{x}{m}=m\fp{x}{m-1}$, forall $m$. 

Law 3. The summation 
\[
    \sum_{a}^{b} \fp{x}{m} \delta x = 
    \begin{cases}
        \frac{\fp{x}{m+1}}{m+1} | _a ^ b & \text{if } x \neq 1 \\
        H_x|_a^b & \text{if} m=-1
    \end{cases}
\]
This is similar to the derivative formula for $1/x$. 

Law 4. $\Delta 2^x=2^x$. We can see that $f(x+1)-f(x)=f(x)$ hence getting 
$f(x+1)=2f(x)$. 

Now have a look at the ways of manipulation: 

Compared to calculus, there's no such thing as chain rule, but we still have

Rule 1. $\Delta cf=c\Delta  f$. This can be proved by extracting the constant
around the function $f$. 

Rule 2. $\Delta (f+g)=\Delta f+\Delta g$. This can also be proved via 
a similar technique. 

Rule 3. $\Delta fg = f\Delta g+g\Delta f$. This can be proved with the 
technique of adding and subtracting the terms: 
\begin{align*}
   \Delta (u(x)v(x)) &= u(x+1)v(x+1)-u(x)v(x) \\
   &= u(x+1)v(x+1)-u(x)v(x+1)+u(x)v(x+1)-u(x)v(x) \\
   &= u(x)\Delta v(x)+v(x+1)-\Delta u(x) \\
\end{align*}

Making the shifting operator $E$ into this formula will simplify it a lot
namely defined as 
\[
    Ef(x)=f(x+1).
\]
this will lead to $\Delta (uv)=u\Delta v +Ev\Delta u$

Rule 4. Summation by parts: $\sum u\Delta v=uv-\sum v\Delta u$. This rule is
similar to the one by integrating by parts. 

Example. Do summation to $x 2^x \delta x$. 
\begin{align*}
    \sum x 2^x \delta x &= x 2^x-\sum 2^{x+1} \delta x \\
                        &= x 2^x - 2^{x+1} +C
\end{align*}
We may get the result by attaching limits.

Another example: find $\sum xH_x\delta x$. 
\begin{align*}
    \sum xH_x \delta x &= {\fp x 2\over 2} \fp x {-1} \delta x \\
                       &= {\fp x 2 \over 2} H_x - \frac{1}{2} \sum \fp x 1 
                       \delta x \\
                       &= {\fp x 2 \over 2} H_x - {\fp x 2 \over 4}+C.
\end{align*}

\section{Infinite series} 

We cannot manipulate things with the restriction of infinite sums. 
Here, associativity and exchanging things may not be true. However, we 
can observe the following sequence: 

\mkpt{Good summation sequences}
There's a general sum $\sum_{k \in K}^{}a_k$ as $K$ can be infinite. 
We assume that all the $a_k\geq 0$
, if there's a bounding constant $A$ s.t. $\sum_{k\in F}a_k\leq A$ holds
for all finite subsets $F \subset K$, we define then $\sum_{k\in K}a_k$
to be the least such $A$. If there's no bounding constant, we say 
the result is $\infty$.

This implies that $\sum_{k\geq 0}a_k=\lim_{n\to \infty} \sum_{k=0}^{n}a_k$,
for example, 
\[
    \sum_{k\geq 0}^{}x^k = \lim_{n \to \infty} {1-x^{n+1}\over 1-x}
    =
    \begin{cases}
        {1\over 1-x}, & 0\leq x< 1;\\
        \infty, &x\geq 1.
    \end{cases}
\]
\mkpt{Simplified definition on infinite sums}
Let $K$ be any set, and let $a_k$ be a real-valued term defined for each 
$k\in K$, any real number $x$ can be written as the difference of its 
positive and negative parts, that is: $x=x^+-x^-$. Hence, we can come up 
with ideas like converges and converges absolutely and so on. More in the 
Calculus book. 

\section{The repertoire method}

\mkpt{Introductory Example} We have the fact that
\[
    S_n =\sum_{k=0}^{n}a_k
\]
that is we have $S_0 = a_0$, $S_n = S_{n-1}+a_n (n>0)$. 
Now this recurrence gives all the prefix sum in the series. 

Suppose that we have the linear combination form $a_n=\beta + \gamma n$,
and this will define $a$ as: 
\begin{align*}
    R_0 &= \alpha  \\
    R_n &= R_{n-1}+\beta+\gamma r, \qquad n>0 \\
\end{align*}

The key idea is that if we can convince ourselves that the final solution
is like the form of
\[
    R_n = A(n)\alpha+B(n)\beta+C(n)\gamma 
\]
with independent solutions for three different relations as what 
$\alpha,\beta,\gamma$ is, then we can claim what our closed forms 
are! 

\mkpt{Example 1} 

Assume that we have that 
\begin{align*}
    R_0 &= \alpha \\
    R_n &= R_{n-1}+\beta+\gamma n \quad n>0 \\
\end{align*}
and has the table that 
\begin{tabular}{c|l}
    $n$ & $R_n$ \\
    0 & $\alpha$ \\
    1 & $\alpha + \beta + \gamma$ \\
    2 & $\alpha + 2\beta + 3\gamma$ \\
    3 & $\alpha + 3\beta + 6\gamma$ \\
    4 & $\alpha + 4\beta + 10\gamma$ \\
    5 & $\alpha + 5\beta + 15\gamma$ \\
\end{tabular}

Then we can start guessing what the solution of the function is
$A(n)=1, \forall n$, and that $B(n)=n$, and that $C(n)=n(n+1)/2$. 
If the function is too complicated, we cant guess that out. 
Hence we need the repertoire method. 

The idea is to take the linear combination of the solutions like 
follows: 

Basic setting 1: Consider that when $\alpha=1,\beta=\gamma=0$, the equation will be 
easy to solve -- that's an constant solution. 

Basic setting 2: $\alpha=0,\beta=1,\gamma=0$, this yields $R_n = R_{n-1}+1$
. This makes a simple sequence. This can be used reversed the finding 
progress. 

Basic setting 3: $R_n=n^2$, guessing its recurrence relation: 
\begin{align*}
    1&=\beta+\gamma\\
    4 &= 1+\beta+2\gamma \\
\end{align*}
and this implies $\alpha=0,\beta=-1,\gamma=2$. Then by induction
we found it matches the pattern of all $n^2$ s. 
so setting all these three cases we get 
\begin{align*}
    1 &= 1A(n) + 0B (n)+ 0C(n) \\
    n &= 0A(n) + 1B (n)+ 0C(n) \\
    1 &= 1A(n) + (-1)B (n)+ 2C(n) \\
\end{align*}


Thus, after guessing the constants, we can solve all the $A(n), B (n) $ 
and $C(n)$. That is 
\begin{align*}
    A(n) &= 1 \\
    B(n) &= n \\
    C(n) &= \frac{n^2 +n}{2}\\
\end{align*}





\mkpt{The main idea} This method is to find coefficients for a linear 
combination. We guess that the solutions can be expressed 
as a sum of coefficients multiplied by functions of $n$ . 

\mkpt{The recipe of repertoire method} 
\begin{itemize}
    \item Relax the recurrence by adding an extra function term.
    \item Substitute known function into the recurrence to derive 
        identities similar to the recurrence. 
    \item Take linear combination of such identities to derive an equation
        identical to the recurrent. 
\end{itemize}

That is, firstly check $a_n =a_{n-1}+\text{sth}$, generalize sth to $f(n)$.
For example, $a_n = a_{n-1}$ will be converted to $a_n=a_{n-1}+f(n)$.
We easily retrieve $f(n)=a_n-a_{n-1}.$ Next we build a table of indigents
in which we can construct $f(n)$. Finally, we determine coefficients of 
each component so that they satisfy the basis of the basis and 
recurrence. 

\mkpt{Example 1} Find the closed form sum of 
\[
   S_n = \sum_{k=0}^{n}k^3.  
\]

The sum can be seen as $a_0=0$, and $a_n = a_{n-1}+n^3$. Next we build 
a table of $a_n$ and $a_n - a_{n-1 }$. 

We find some candidates that may be useful.

\begin{table}[htpb]
    \centering
    \caption{The table built}
    \label{tab:rep-fig}

    \begin{tabular}{c|l}
        $a_n$ & $a_n-a_{n-1}$ \\
        \hline 
        1 & 0\\
        n & 1\\
        $n^2$ & $2n-1$\\
        $n^3$ & 3$n^2-3n+1$\\
        $n^4$ & $4n^3-6n^2+4n-1$
    \end{tabular}
\end{table}

How is the coeffs useful? The simplest way we come up is 
\[
    n^3=\alpha \left( 4n^3-6n^2+4n-1 \right) \implies \alpha=1/4.
\]
We shall also eliminate $n^2$ and so on and we have that
\[ 
    \alpha(4n^3-6n+4n-1)+\beta(3n^2-3n+1)+\gamma(2n-1)=n^3
\]

Solve this will be useful. 


\section{Exercises}

\mkpt{W 1} This means $0$. For the lower bound is larger than upper bound.

\mkpt{W 2} We do this by splitting the cases: 

If $x=0$, that is simply 0. If $x>0$, it will become $x$; otherwise, it 
will become $-x$. 

To sum up, this is equivalent to absolute value. Namely $|x|$. 

\mkpt{W 3} The first one is 
\[
    a_1 + a_2+a_3+a_4+a_5
\]
while the second one is
\[
    a_{1^2}+a_{\sqrt{2}^2}+a_{\sqrt{3}^2}+a_{\sqrt{4}^2}+a_{\sqrt{5}^2}. 
\]

\mkpt{W 4} (a) We have $\sum[1\leq i<j<k\leq 4]=\sum[1\leq k\leq 4][k>j][j>i]=\sum_{k=1}^{4}\sum_{j=k}^{4}\sum_{i=j}^{4} a_{ijk}$. 

(b) We have $\sum_{i=1}^{4} \sum_{j=1}^{4} \sum_{j=1}^{4} a_{ijk}.$ This 
is pretty similar to the determination in double integral and triple 
integral. 

\mkpt{W 5} The change of variable is wrong, for $i$ and $j$ represents 
different meaning. The change of variables must be fresh. That is it's 
not able to be collision to the existing value. 

\mkpt{W 6} The value of that is $n-j+1$, as a counting problem. 

\mkpt{W 7} We have the fact that $\rp x m = x(x+1)\cdots (x+m-1)$, and that 
$\rp x {m-1} = x(x+1)\cdots (x+m-2)$. So the answer is 
\[
    (x+m-2) \rp x {m-1}
\]

\mkpt{W 8} According to the definition, we have that the answer is 0. 
For whatever we do the answer must be 0. 

\mkpt{W 9} We define that $\rp x {-n}$ according to the division rule that 
\[
    \rp x {-1} = {1\over x-1}, 
\]
and more generally, we have 
\[
    \rp x {-m} = {1\over (x-1)(x-2)\cdots (x-m)}.  
\]
\mkpt{W 10} If we swap this, getting 
\[
    \Delta (uv) = u\Delta v + Ev\Delta u
\]
and have 
\[
    \Delta (vu) = v\Delta u + Eu\Delta v
\]
subtracting both, we have 
\begin{align*}
    \Delta (uv)-\Delta (vu) &= u\Delta v +Ev\Delta u - (v\Delta u + Eu\Delta v) \\
                            &= (u-Eu)\Delta v +(Ev - v)\Delta u\\
                            &= (u(x)-u(x+1))(v(x+1)-v(x))+(v(x+1)-v(x))
                            (u(x+1)-u(x))\\
                            &= 0
\end{align*}
and the derivation is symmetric, although. 

\mkpt{B 11} We first note this by setting $n = 3$, and we have 
\[
    a_1b_0-a_0b_0+a_2b_1-a_1b_1+a_3b_2-a_2b_2,
\]
that is 
\begin{align*}
    \sum_{0\leq k\leq n}^{} (a_{k+1}-a_k)b_k &= 
    -\sum_{0\leq k<n}^{}a_{k+1} (b_{k+1}-b_k)-\sum_{0\leq k<n}^{} a_{k+1}b_k
    -\sum_{0\leq k<n}^{} a_kb_k\\
    &= -\sum_{0\leq k<n}^{}a_{k+1}(b_{k+1}-b_k) - 
      \sum_{0\leq k<n}^{} b_k (a_{k+1} + a_k)\\
    &=  -\sum_{0\leq k<n}^{}a_{k+1}(b_{k+1}-b_k) + (a_n b_n - a_0b_0). 
\end{align*}

\mkpt{B 12} A permutation of integer means that $p(k)$ and all integers 
are bijection. That is: Consider $\mathbb R \to \mathbb R$, $x\mapsto x+(-1)^kc$, the result is 
unique. Substituting back, we have that: if $k$ is odd, we have $k=f(x)-c$; 
if $ k$ is even, we have $k=f(x)+c$. We can find that these are identical
and is able to make a bijection, hence there's a permutation on every
integer. 


\mkpt{B 13} 



