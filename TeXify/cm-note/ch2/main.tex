\chapter{Manipulating sums}

\section{Basic Notations}

\mkpt{Summation notation} We can use $\sum$ to denote add things together.
\begin{itemize}
    \item $\sum_{a}$: sum over all items in $a$
    \item $\sum_{i=1}^{n} f(i)$: sum from $i=1$ to $i=n$, substituting all $i$s
        in the right $f(i)$.
\end{itemize} 

\mkpt{Iverson's Bracket} We have the notation $[p]$: $p$ is a proposition, and
when $p$ is true, the equation is evaluated 1, otherwise evaluated as 0. 

\mkpt{Using Aversion's bracket to simplifying sums} For example we have 
\[
    \sum_{p\text{ prime},p \leq N} p = \sum_p [p\text{ prime}][p \leq N] p.
\]

\section{Manipulation of sums}

\mkpt{Distributive law} $\sum_{k\in K} ca_k=c\sum_{k\in K}a_k$.

\mkpt{Associative law} $\sum_{k\in K} (a_k+b_k)=\sum_{k\in K}a_k+\sum_{k\in K}b_k$.

\mkpt{Commutative law} $\sum_{k\in K} a_k=\sum_{p(k)\in K}a_{p(k)}$. 

This rule is for mainly substitution use. When the condition is not specified, the
$p(k)$ should be a permutation of all integers. 

Example. We have 
\[
    \sum_{k\in K, k \texttt{ even}}^{} a_k = \sum_{n\in K, n \texttt{ even}}^{}a_n
    =\sum_{2k \in k}^{a_{2k}}
\]


\mkpt{Example 1: Arithmetic's progression} We have $S=\sum_{0\leq k\leq
n}(a+bk)$. By commutative law, replace $k$ by $n-k$, obtaining $S=\sum_{0\leq
n-k\leq n}(a+b(n-k))=\sum_{0\leq k \leq n} (a+bn-bk)$. These two can be added
up to $2S=\sum_{0 \leq k\leq n} (2a+bn)$. Then this problem would be trivial. 

\mkpt{Exclusion and Inclusive Principle with Sums} Here is a important rule for combining
different set of indicies. 

Suppose $K$ and $K'$ are any set of integers, then 
\[
    \sum_{k \in K}^{}a_k + \sum_{k\in K'}^{}a_k = \sum_{k\in k \cap K'}^{}a_k +
    \sum_{K\in K \cup K'}^{}a_k.
\]
In the case of Iverson's bracket, we have 
\[
    [k\in K]+[k\in K'] = [k \in K \cap K'] + [K \in K \cup K']
\]
\mkpt{Perturbation method} We first write $S_n = \sum_{0 \leq k \leq n}^{}a_k$,
then we rewrite $S_{n+1}$ in 2 ways: 
\begin{align*}
    S_n +a_{n+1} &= \sum_{0 \leq k \leq n+1}^{} &= a_0 + \sum_{1\leq k\leq n+1}^{}a_k  \\
                 & &= a_0 + \sum_{1\leq k+1 \leq n+1}^{}a_{k+1}\\
                 & &= a_0 + \sum_{0 \leq k \leq n}^{}a_{k+1}.
\end{align*}

Now we can work on the last term trying to solving the closed form for it. 

\mkpt{Example 2. Geometry progression} We have $S_n = \sum_{0 \leq k \leq
n}^{}ax^k$. 


