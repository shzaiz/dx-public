%\subsection{知识引导}
%
%\ti{余子式.} 如果$A$是一个矩阵, 其中 $M_{1j}$ 是划去 $A$ 的第 $1$ 行第 $j$ 列后所得的 $n-1$ 阶方阵的行列式,称为 $A_{1j}$ 的 余子式
%
%\ti{行列式的性质.} 
%
%(1)行列式中两行(列)互挟,行列式变号. 证明可以使用每次交换逆序数奇偶性改变.
%
%(1-1)一个方阵和它的转置的行列式相等. 
%
%(2)行列式某行(列)来以常数 $c$,则新的行列式是原行列式来以 $c$.
%
%(3)行列式具有行(列)的线性性. 
%
%(3-1)若 $A$ 中有两行成比例,特别地,一行为零或两行相同, 那么$|A|=0$.
%
%(4)行列式的某一行(列)加上另一行(列)的 $c$ 倍,行列式不变.
%
%\ti{多行Laplace展开. } 
%
%一个$A\in\F^{n\times n}$, 任取$1\leq i_1<i_2<\cdots<i_p\leq n$, 则$A$ 中所有$i_1,i_2, \cdots ,i_p$行上的 $p$ 阶子式与其代数余子式的积之和为 $|A|$ ,即
%$$
%|A|=\sum_{\left(j_{1} \cdots j_{p}\right)}(-1)^{\sum_{t=1}^{p}\left(i_{t}+j_{t}\right)} A\left(\begin{array}{l}
%    i_{1} \cdots i_{p} \\
%    j_{1} \cdots j_{p}
%    \end{array}\right) A\left(\begin{array}{l}
%    i_{p+1} \cdots i_{n} \\
%    j_{p+1} \cdots j_{n}
%    \end{array}\right)
%$$
%这里,求和取遍所有$1\leq j_1<j_2<\cdots\leq n$. 
%
%注意是固定行, 取遍列, 或者固定列, 取遍行. 