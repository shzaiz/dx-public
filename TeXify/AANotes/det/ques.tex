\subsection{常见例子}
\begin{prob} 
    使用Laplace展开可以有很多作用. 
$$
\left|\begin{array}{cccccc}
    a_{11} & \cdots & a_{1 m} & c_{11} & \cdots & c_{1 n} \\
    \vdots & & \vdots & \vdots & & \vdots \\
    a_{m 1} & \cdots & a_{m m} & c_{m 1} & \cdots & c_{m n} \\
    0 & \cdots & 0 & b_{11} & \cdots & b_{1 n} \\
    \vdots & & \vdots & \vdots & & \vdots \\
    0 & \cdots & 0 & b_{n 1} & \cdots & b_{n n}
    \end{array}\right|=\left|\begin{array}{cccccc}
    a_{11} & \cdots & a_{1 m} & 0 & \cdots & 0 \\
    \vdots & & \vdots & \vdots & & \vdots \\
    a_{m 1} & \cdots & a_{m m} & 0 & \cdots & 0 \\
    d_{11} & \cdots & d_{1 m} & b_{11} & \cdots & b_{1 n} \\
    \vdots & & \vdots & \vdots & & \vdots \\
    d_{n 1} & \cdots & d_{n m} & b_{n 1} & \cdots & b_{n n}
    \end{array}\right|=|A||B| .
$$ 
\end{prob} 

\begin{proof} 
 
    %TODO
 
\end{proof} 


%%%%%

\begin{prob} 

    使用Laplace展开证明$|AB|=|A||B|$:

    $$
    |A||B|=\left|\begin{array}{cccccc}
        a_{11} & \cdots & a_{1 n} & 0 & \cdots & 0 \\
        \vdots & & \vdots & \vdots & & \vdots \\
        a_{n 1} & \cdots & a_{n n} & 0 & \cdots & 0 \\
        -1 & \cdots & 0 & b_{11} & \cdots & b_{1 n} \\
        \vdots & & \vdots & \vdots & & \vdots \\
        0 & \cdots & -1 & b_{n 1} & \cdots & b_{n n}
        \end{array}\right| .
    $$

\end{prob} 

\begin{proof} 
 
    %TODO
 
\end{proof} 

%%%%

下面这两个例子展示了矩阵和多项式之间的联系:

\begin{prob} 
证明:
$$
\begin{array}{|cccccc|}
    x & 0 & 0 & \cdots & 0 & a_{0} \\
    -1 & x & 0 & \cdots & 0 & a_{1} \\
    0 & -1 & x & \cdots & 0 & a_{2} \\
    \vdots & \vdots & \vdots & & \vdots & \vdots \\
    0 & 0 & 0 & \cdots & x & a_{n-1} \\
    0 & 0 & 0 & \cdots & -1 & a_{n}
\end{array} =\sum_{i=0}^{n} a_{i} x^{i} ;
$$ 

\end{prob} 

\begin{proof} 
 
    %TODO
 
\end{proof} 


%%%%%

\begin{prob} 
    证明:
    $$
    \left|\begin{array}{cccccc}
        x & 0 & 0 & \cdots & 0 & a_{0} \\
        -1 & x & 0 & \cdots & 0 & a_{1} \\
        0 & -1 & x & \cdots & 0 & a_{2} \\
        \vdots & \vdots & \vdots & & \vdots & \vdots \\
        0 & 0 & 0 & \cdots & x & a_{n-2} \\
        0 & 0 & 0 & \cdots & -1 & x+a_{n-1}
        \end{array}\right|=x^{n}+a_{n-1} x^{n-1}+\cdots+a_{0}
    $$ 
    
\end{prob} 

\begin{proof} 
    
    %TODO
    
\end{proof} 

%%%%%

下面这个问题产生的Vandermonde行列式可以应用于快速Fourier变换(见《算法导论》), 也有很多实际的背景. 

\begin{prob} 
    证明Vandermonde行列式: 
    $$
    \text { 设 } A=\left(\begin{array}{cccc}
        1 & 1 & \cdots & 1 \\
        x_{1} & x_{2} & \cdots & x_{n} \\
        \vdots & \vdots & & \vdots \\
        x_{1}^{n-1} & x_{2}^{n-1} & \cdots & x_{n}^{n-1}
        \end{array}\right) \text {, 证明: }|A|=\prod_{1 \leqslant i<j \leqslant n}\left(x_{j}-x_{i}\right) \text {. } 
    $$
\end{prob} 

\begin{proof}
    %TODO
\end{proof}