


    \section{数域}
    
    \subsection{课后练习}

    \begin{prob}
        设$p_1, p_2, \cdots, p_n$为不同的素数, $n\geq 2$, 证明, $\sqrt[n]{p_1\cdots p_k}$是无理数. 
    \end{prob}

    \begin{proof}
        由题意, 假设
        $$
        p_1p_2\dots p_k = (a/b)^n, \gcd(a, b) =1, a, b\in \mathbb{Z},
        $$
        那么有$b^np_1p_2\cdots p_=a^n$, 由于$p_1,\cdots, p_n$都是质数, $a$, $b$互质, 且$n\geq 2$, 最终可以写成如下形式:
        $$
        \prod_{i=1}^{n_1} q_i^{\alpha_i} \prod_{i=1}^{n_2} p_i = \prod_{i=1}^{n_2} r_i^{\beta_i}
        $$

        分类讨论: 

        \begin{itemize}
            \item 如果$p_i$不与相同, 等式不可能成立.  
            \item 如果$p_i$与$q_i$部分相同, 那么$p_i$的次数只有$1$, 与剩余部分的$n$次($n\geq 2$)不同, 因此也不成立.  
        \end{itemize}

        综上, $p_1p_2\dots p_k \neq (a/b)^n$, 意味着$\sqrt[n]{p_1\cdots p_k}$是无理数, 在$p_1, p_2, \cdots, p_n$为不同的素数的条件下. 
    \end{proof}

    \begin{prob}
        试求所有$\{t\in \mathbb{C}\}$使得$\{a+bt|a,b\in \mathbb{Q}\}$是数域.
    \end{prob}

    \begin{sol}
        假设有$x_1=a+bt$, $x_2=c+dt$, $x_1x_2=(a+bt)(c+dt)=t (a d+b c)+a c+b d t^2 \in \mathbb{Q}$. 这就要求$t$是二次有理方程系数的根.  
    \end{sol}

    \begin{prob}
        证明: 真包含$\mathbb{R}$的数域只有复数域$\mathbb{C}$.
    \end{prob}

    \begin{proof}
        考虑反证法: 假设存在一个数域$\mathbb{F}$使得$\mathbb{R}\subseteq \mathbb{F} \subseteq \mathbb{C}$. 

        取得$x = a+bi \in \mathbb{F}$, 且要求$b=0$. 由于$R\subseteq \mathbb{F}$, $a\in P$. 注意到加法不具有封闭性. 因此与假设矛盾.   
    \end{proof}

    \begin{prob}
        设$\mathbb{E,F}$为数域, 称映射$\varphi: \mathbb{E}\rightarrow \mathbb{F}$为$\mathbb{E}$到$\mathbb{F}$的自同态, 如果

        $$\varphi(1)=1, \varphi(a+b)=\varphi(a)+\varphi(b),\varphi(ab)=\varphi(a)\varphi(b)$$

        特别的, 若$\mathbb{F}=\mathbb{E}$, 称$\varphi$为$\mathbb{E}$的自同构. 证明: 同态$\varphi$一定是单射. 
    \end{prob}

    \begin{proof}
        考虑使用反证法, 假设$\varphi$不是一个单射, 也就是$\exists x_1,x_2, x_1 \neq x_2, \varphi(x_1) = \varphi(x_2)$.

        由题意, 我们可以表达$x_1$为
        $$
        \varphi(x_1)=\varphi(x_2+(x_1-x_2))=\varphi(x_2)+\varphi(x_1-x_2)
        $$
        同样可以表达$x_2$为
        $$
        \varphi(x_2)=\varphi(x_1+(x_2-x_1))=\varphi(x_1)+\varphi(x_2-x_1)
        $$

        由于$\varphi(x_1)=\varphi(x_2)$, 那么$\varphi(x_2)+\varphi(x_1-x_2)=\varphi(x_1)+\varphi(x_2-x_1)$. 也就是$\varphi(x_2-x_1)=0$. 如果$p=(x_2-x_1)\neq 0$, 那么$1=\varphi(1)=\varphi(p\cdot 1/p)=\varphi(p)\varphi(1/p)=0$, 矛盾. 因此假设不成立. 

        
    \end{proof}

    \begin{prob}
        设  $\mathbb{E}, \mathbb{F}$  为数域, 称  $\mathbb{E}, \mathbb{F}$  同构, 如果存在可逆映射  $\varphi: \mathbb{E} \rightarrow \mathbb{F}$  使得$$\varphi(\alpha+\beta)=\varphi(\alpha)+\varphi(\beta), \quad \varphi(\alpha \beta)=\varphi(\alpha) \varphi(\beta) .$$

        称  $\varphi$  为  $\mathbb{E}$  到  $\mathbb{F}$  的同构. 特别地, 若  $\mathbb{F}=\mathbb{E}$ , 称  $\varphi$  为  $\mathbb{E}$  的自同构.
        
        (1) 证明: 存在无穷多个不同构的数域;

        (2) 设$ \varphi:\mathbb{E} \rightarrow \mathbb{F} $为同构, 证明对任意 $\alpha \in Q$, 有$\varphi(\alpha)=\alpha$. 

        (3) 试求$F=Q(\sqrt{2})$的所有自同构
    \end{prob}

    \begin{proof}
        % TODO
    \end{proof}
    


