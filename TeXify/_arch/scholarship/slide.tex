\documentclass[11pt]{beamer}

\usepackage[utf8]{inputenc}
% \usepackage[T1]{fontenc}
%\usepackage{lmodern}
\usepackage{geometry}
\usepackage{fancyhdr}
\usepackage{graphicx}
\usepackage{array}
\usepackage{amsmath}
\usepackage{listings}
\usepackage{xcolor}
\usepackage{fontspec}
\usefonttheme[fontset = none]{serif}
\usepackage{ctex}
\usetheme{EastLansing}

\begin{document}
	\author{张桄玮}
	\title{获得真正的力量: 探索与实践}
	\subtitle{华为``育苗''奖学金评选答辩}
	\institute{中国地质大学(武汉)\\计算机学院}
	\titlegraphic{\includegraphics[width=3cm]{logo.png}}
	%\setbeamercovered{transparent}
	%\setbeamertemplate{navigation symbols}{}
	\AtBeginSection[]
	{
		\begin{frame}{主要内容}
			\transfade%淡入淡出效果
			\tableofcontents[sectionstyle=show/shaded,subsectionstyle=show/shaded/hide] 
		\end{frame}
	}
	\begin{frame}[plain]
		\maketitle
	\end{frame}
	
	\section{大学以前: 自由地探索}
	\begin{frame}
		\frametitle{进入这个领域}
		\framesubtitle{小学的时候}
		\textbf{小学的时候}
		\begin{itemize}
			\item 感觉神奇
			\item 学了点VB, 造了点简单的小游戏
			\item 学了点PPT, 很多老师找我帮他们改一改公开课要用的PPT
		\end{itemize} 
	\end{frame}


	\begin{frame}
		\frametitle{进入这个领域}
		\framesubtitle{初中的时候}
		\textbf{初中的时候}
		\begin{itemize}
			\item 还是经常帮老师们改PPT
			\item 因为很好玩, 继续探索
			\item 了解了一点Java, 帮助老师制作了很多有用的小工具
				\item 但是对于只能``拼搭积木''这样的模式感到不满足
			\item 于是对``算法''的渴求恰如其时的出现了
		\end{itemize} 
	\end{frame}

	\begin{frame}
		\frametitle{进入这个领域}
		\framesubtitle{高中的时候}
		\textbf{高中的时候}
		\begin{itemize}
			\item 进入了郑州市第一中学信息竞赛队
			\item 了解到了广阔的算法天地
			\item 幸运的是得到了CSP2020一等奖, NOIP2020二等奖
			\item 迫于实力和河南省的高考压力, 回到文化课准备高考
			\item 对计算机科学持续的热爱
		\end{itemize} 
	\end{frame}

	\section{大学: 希望在专业的人员指导下继续探索}
	\begin{frame}
		\frametitle{代价: 很多的弯路}
		\framesubtitle{和一部分无意义的时间浪费}
		
		\textbf{具体的代价}
		\begin{itemize}
			\item 知识体系较为零散
			\item 数学方面的训练不是很足
			\item 不能善于阅读/编写多人工程协作项目
		\end{itemize}
		
		\begin{quote}
			我们都是活生生的人, 从小就被不由自主地教导用最小的付出获得最大的得到, 经常会忘记我们究竟要的是什么. 我承认我完美主义, 但我想每个人心中都有那一份求知的渴望和对真理的向往, ``大学"的灵魂也就在于超越世俗, 超越时代的纯真和理想 -- 我们不是要讨好企业的毕业生, 而是要寻找改变世界的力量.  --- by  \texttt{ Yanyan Jiang } \textit{on ICS2020}
		\end{quote}
		\begin{centering}
		最主要的理由: 要让自己变成一个真正\textbf{实力过硬}的学生.
		\end{centering}
	\end{frame}

	\begin{frame}
		\frametitle{期望的解决方案}
		\framesubtitle{接受专业人员的指导}
		\pause
		\textbf{(I)接受专业人员的指导}
		
		\begin{itemize}
			\item 来自学校课程/老师的教导
			\item \textbf{来自社团巨佬们的指导}
			\item 来自网络上公开课的指导
			\item ...
		\end{itemize}
		
		
		\begin{quote}
			将已有的知识和方法重新消化, 为大家建立好``台阶'', 在有限的时间里迅速赶上数十年来建立起的学科体系.  --- by  \texttt{ Yanyan Jiang } \textit{on Operating Systems: Design and implementation}
		\end{quote}
	
	
	\end{frame}



	\begin{frame}
		\frametitle{期望的解决方案}
		\framesubtitle{打牢数学/计算机科学基础}
		\textbf{(II)打牢数学/计算机科学基础}
		
		\begin{quote}
			在打好基础之前就不要奢望有什么成就了. \\--- \texttt{Haonan Huang}(B站up: 數心)
		\end{quote}
		
		
		数学课程:
		\begin{itemize}
			\item 代数: \textbf{线性代数}$\rightarrow$\textbf{高等代数}$\rightarrow$抽象代数$\rightarrow \cdots$
			\item 分析: \textbf{高等数学}$\rightarrow$\textbf{数学分析}$\rightarrow$实分析$\rightarrow\cdots$
			\item \textbf{离散数学}: 很杂, 主要包括具体数学, 组合数学... (NJU离散数学公开课有一点学一点吧)
			\item ...
		\end{itemize} 
		计算机科学课程:
		\begin{itemize}
			\item \textbf{SICP(MIT)系列}
			\item \textbf{计算机系统基础(简化版的CSAPP)}
			\item ...
		\end{itemize} 
	\end{frame}

	\begin{frame}
		\frametitle{期望的解决方案}
		\framesubtitle{做更多的实验}
		\textbf{(III)做更多的实验}\\
		目前正在进行: 
		\begin{itemize}
			\item \textbf{NJU问题求解(第一学期)书面作业}
			\begin{itemize}
				\item 进度: 论题1-4
			\end{itemize}
			\item \textbf{NJU计算机系统基础实验(PA)}: 在框架代码的指引下造一个NEMU, 在上面跑Super Mario 
			\begin{itemize}
				\item 进度: PA2.2
			\end{itemize}
		\end{itemize} 
	以后可能会进行:
	
	\begin{itemize}
		\item NJU操作系统实验
		\item CSAPP实验
		\item ...
	\end{itemize} 
	\end{frame}

	\section{目标}

	\begin{frame}
		\frametitle{承担更多提升能力的训练}
		\framesubtitle{让自己变成更加强大的学生}
		
		\begin{quote}
			教育除了知识的记忆之外, 更本质的是能力的训练, 即所谓的training. 而但凡training就必须克服一定的难度, 否则你就是在做重复劳动, 能力也不会有改变. 如果遇到难度就选择退缩, 或者让别人来替你克服本该由你自己克服的难度, 等于是自动放弃了获得training的机会, 而这其实是大学专业教育最宝贵的部分. --- by  \texttt{ Yitong Yin }
		\end{quote}
		\pause
		\begin{itemize}
			\item 为未来的研究打下一些坚实的基础
			\item 回馈社会
			\begin{itemize}
				\item 以前在B站(\color{blue}\href{https://space.bilibili.com/13246364}{AUGPath}\color{black})经常投科普性质的稿件
				\item 但是发现知识不够系统, 无法满足自己对于高质量科普稿件的定义
				\item 于是希望学习更多的知识
				\item \url{shzaiz.github.io}
			\end{itemize}
		\end{itemize} 
		
		
	\end{frame}
	
	\section*{致谢与参考}
	
	\begin{frame}
		\frametitle{参考资料}
		\textbf{刚刚列出的一些课程}
		\begin{itemize}
			\item \color{blue}\href{https://space.bilibili.com/1632276842}{Maki's Lab数学分析公开课}
			\item \color{blue}\href{https://space.bilibili.com/509086270}{NJU 高等代数公开课}
			\item \color{blue}\href{https://space.bilibili.com/479141149/channel/seriesdetail?sid=490582}{NJU 离散数学公开课}
			\item \color{blue}\href{https://www.bilibili.com/video/BV1Xx41117tr/}{MIT SICP公开课}
			\item \color{blue}\href{https://www.bilibili.com/video/BV1kE411X7S5/}{NJU 计算机系统基础}
			
			
		\end{itemize}
	\end{frame}
	
	\begin{frame}
		
		$$\Huge \text{谢谢!}$$
	\end{frame}
	
\end{document}