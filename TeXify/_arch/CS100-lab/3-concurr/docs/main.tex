% 1-2-reasoning.tex

%%%%%%%%%%%%%%%%%%%%
\documentclass[a4paper, justified]{tufte-handout}
% hw-preamble.tex

% geometry for A4 paper
% See https://tex.stackexchange.com/a/119912/23098
\geometry{
  left=20.0mm,
  top=20.0mm,
  bottom=20.0mm,
  textwidth=130mm, % main text block
  marginparsep=5.0mm, % gutter between main text block and margin notes
  marginparwidth=50.0mm % width of margin notes
}

% for colors
\usepackage{xcolor} % usage: \color{red}{text}
% predefined colors
\newcommand{\red}[1]{\textcolor{red}{#1}} % usage: \red{text}
\newcommand{\blue}[1]{\textcolor{blue}{#1}}
\newcommand{\teal}[1]{\textcolor{teal}{#1}}

% heading
\usepackage{sectsty}
\setcounter{secnumdepth}{2}
\allsectionsfont{\centering\huge\rmfamily}

% for fonts
\usepackage{fontspec}
\newcommand{\song}{\CJKfamily{song}} 
\newcommand{\hei}{\CJKfamily{hei}} 
\newcommand{\kai}{\CJKfamily{kai}} 
\newcommand{\fs}{\CJKfamily{fs}} % 仿宋

% To fix the ``MakeTextLowerCase'' bug:
% See https://github.com/Tufte-LaTeX/tufte-latex/issues/64#issuecomment-78572017
% Set up the spacing using fontspec features
\renewcommand\allcapsspacing[1]{{\addfontfeature{LetterSpace=15}#1}}
\renewcommand\smallcapsspacing[1]{{\addfontfeature{LetterSpace=10}#1}}

% for url
\usepackage{hyperref}
\hypersetup{colorlinks = true, 
  linkcolor = teal,
  urlcolor  = teal,
  citecolor = blue,
  anchorcolor = blue}

\newcommand{\me}[4]{
    \author{
      {\bfseries Name:}\underline{#1}\hspace{2em}
      {\bfseries Student ID:}\underline{#2}\hspace{2em}\\[10pt]
  }
}



% for math
\usepackage{amsmath, mathtools, amsfonts, amssymb}
\newcommand{\set}[1]{\{#1\}}

% define theorem-like environments
\usepackage[amsmath, thmmarks]{ntheorem}

\theoremstyle{break}
\theorempreskip{2.0\topsep}
\theorembodyfont{\song}
\theoremseparator{}
\newtheorem*{problem}{Problem}
\newtheorem{ot}{Open Topics}

\theorempreskip{3.0\topsep}
\theoremheaderfont{\kai\bfseries}
\theoremseparator{:}
\theorempostwork{\bigskip\hrule}
\newtheorem*{solution}{Answer}
\theorempostwork{\bigskip\hrule}
\newtheorem*{revision}{Revision}

\theoremstyle{plain}
\newtheorem*{cause}{Cause}
\newtheorem*{remark}{Note}

\theoremstyle{break}
\theorempostwork{\bigskip\hrule}
\theoremsymbol{\ensuremath{\Box}}
\newtheorem*{proof}{证明}

% \newcommand{\ot}{\blue{\bf [OT]}}

% for figs
\renewcommand\figurename{Fig}
\renewcommand\tablename{Diag}

% for fig without caption: #1: width/size; #2: fig file
\newcommand{\fig}[2]{
  \begin{figure}[htbp]
    \centering
    \includegraphics[#1]{#2}
  \end{figure}
}
% for fig with caption: #1: width/size; #2: fig file; #3: caption
\newcommand{\figcap}[3]{
  \begin{figure}[htbp]
    \centering
    \includegraphics[#1]{#2}
    \caption{#3}
  \end{figure}
}
% for fig with both caption and label: #1: width/size; #2: fig file; #3: caption; #4: label
\newcommand{\figcaplbl}[4]{
  \begin{figure}[htbp]
    \centering
    \includegraphics[#1]{#2}
    \caption{#3}
    \label{#4}
  \end{figure}
}
% for margin fig without caption: #1: width/size; #2: fig file
\newcommand{\mfig}[2]{
  \begin{marginfigure}
    \centering
    \includegraphics[#1]{#2}
  \end{marginfigure}
}
% for margin fig with caption: #1: width/size; #2: fig file; #3: caption
\newcommand{\mfigcap}[3]{
  \begin{marginfigure}
    \centering
    \includegraphics[#1]{#2}
    \caption{#3}
  \end{marginfigure}
}

\usepackage{fancyvrb}

% for algorithms
\usepackage[]{algorithm}
\usepackage[]{algpseudocode} % noend
% See [Adjust the indentation whithin the algorithmicx-package when a line is broken](https://tex.stackexchange.com/a/68540/23098)
\newcommand{\algparbox}[1]{\parbox[t]{\dimexpr\linewidth-\algorithmicindent}{#1\strut}}
\newcommand{\hStatex}[0]{\vspace{5pt}}
\makeatletter
\newlength{\trianglerightwidth}
\settowidth{\trianglerightwidth}{$\triangleright$~}
\algnewcommand{\LineComment}[1]{\Statex \hskip\ALG@thistlm \(\triangleright\) #1}
\algnewcommand{\LineCommentCont}[1]{\Statex \hskip\ALG@thistlm%
  \parbox[t]{\dimexpr\linewidth-\ALG@thistlm}{\hangindent=\trianglerightwidth \hangafter=1 \strut$\triangleright$ #1\strut}}
\makeatother

% for footnote/marginnote
% see https://tex.stackexchange.com/a/133265/23098
\usepackage{tikz}
\newcommand{\circled}[1]{%
  \tikz[baseline=(char.base)]
  \node [draw, circle, inner sep = 0.5pt, font = \tiny, minimum size = 8pt] (char) {#1};
}
\renewcommand\thefootnote{\protect\circled{\arabic{footnote}}} % feel free to modify this file
\newcommand{\ttt}[0]{\texttt}
%%%%%%%%%%%%%%%%%%%%
\title{Parallel Threads}
\me{Guangwei Zhang}{20221001980}{}{}
%%%%%%%%%%%%%%%%%%%%
\begin{document}
\maketitle

\section{Exercise 1.}

\begin{verbatim}
class Algo1:
  halt = 0
  x = 0
 
  def t1(self):
    a = self.x
    a = a + 5
    self.x = a
 
  def t2(self):
    b = self.x
    b = b + 10
    self.x = b
\end{verbatim}

This is possible by $\ttt{t1}\rightarrow \ttt{t2} \rightarrow \ttt{t2}\rightarrow \ttt{t2}\rightarrow \ttt{t1}\rightarrow \ttt{t1}$.

This is the rightmost node at the graph shown in $\ttt{1.html}$

\section{Exercise 2.}

\begin{verbatim}
class Algo2:
    halt = 0
    x = 5
    y = 10
    z = 30
    sum = 0
   
    def t1(self):
      val1 = self.sum + self.x
      self.sum = val1
   
    def t2(self):
      val1 = self.sum + self.y
      self.sum = val1
   
    def t3(self):
      val1 = self.sum + self.z
      self.sum = val1
\end{verbatim}

\textbf{Part 1.} This is possible by $\ttt{t2}\rightarrow \ttt{t3} \rightarrow \ttt{t3}\rightarrow \ttt{t2}\rightarrow \ttt{t1}\rightarrow \ttt{t1}$, which got the sum 15. 




\textbf{Part 2.} This is possible by $\ttt{t2}\rightarrow \ttt{t3} \rightarrow \ttt{t1}\rightarrow \ttt{t1}\rightarrow \ttt{t2}\rightarrow \ttt{t3}$

This is verified by states on $\ttt{2.html}$. 

\section{Exercise 3.}

\begin{verbatim}
class Algo3:
  halt = 0
  turn = ''
  x = 0
 
  def t1(self):
    self.turn = 't1'
    while self.turn == 't2':
      pass
    a = self.x
    a = a + 5
    self.x = a
    self.turn = 't2'
 
  def t2(self):
    self.turn = 't2'
    while self.turn == 't1':
      pass
    b = self.x
    b = b + 10
    self.x = b
    self.turn = 't1'
\end{verbatim}

One possible path is $\ttt{t2}\rightarrow \ttt{t2}\rightarrow\ttt{t1}\rightarrow\ttt{t1}\rightarrow\ttt{t1}\rightarrow\ttt{t1}\rightarrow\ttt{t2}\rightarrow\ttt{t2}\rightarrow\ttt{t1}\rightarrow\ttt{t2}\rightarrow\ttt{1}\rightarrow\ttt{2}$

This is verified by states on $\ttt{3.html}$. As red nodes on $\ttt{3-marked.html}$ shows the problem in critical section(marked by CS).

Using Peterson algorithm can make sure the addition is atomic. 


\section{Note}

The model-checker is by teacher Yanyan Jiang from Nanjing University on the operating system course. 

\begin{itemize}
    \item Model checker by Yanyan Jiang(Download at \url{http://jyywiki.cn/pages/OS/2022/demos/model-checker.py} and \url{http://jyywiki.cn/pages/OS/2022/demos/visualize.py}), slightly modified by Guangwei Zhang (added halting indication)
    \item Example programs \url{http://jyywiki.cn/pages/OS/2022/demos/mutex-bad.py} 
    \item Use command \texttt{./model-checker.py | ./visualize.py > output.html} to visualize. 
\end{itemize}

Adopted version is at \texttt{prog-visualize} folder. 

\end{document}