% 1-2-reasoning.tex

%%%%%%%%%%%%%%%%%%%%
\documentclass[a4paper, justified]{tufte-handout}

% hw-preamble.tex

% geometry for A4 paper
% See https://tex.stackexchange.com/a/119912/23098
\geometry{
  left=20.0mm,
  top=20.0mm,
  bottom=20.0mm,
  textwidth=130mm, % main text block
  marginparsep=5.0mm, % gutter between main text block and margin notes
  marginparwidth=50.0mm % width of margin notes
}

% for colors
\usepackage{xcolor} % usage: \color{red}{text}
% predefined colors
\newcommand{\red}[1]{\textcolor{red}{#1}} % usage: \red{text}
\newcommand{\blue}[1]{\textcolor{blue}{#1}}
\newcommand{\teal}[1]{\textcolor{teal}{#1}}

% heading
\usepackage{sectsty}
\setcounter{secnumdepth}{2}
\allsectionsfont{\centering\huge\rmfamily}

% for fonts
\usepackage{fontspec}
\newcommand{\song}{\CJKfamily{song}} 
\newcommand{\hei}{\CJKfamily{hei}} 
\newcommand{\kai}{\CJKfamily{kai}} 
\newcommand{\fs}{\CJKfamily{fs}} % 仿宋

% To fix the ``MakeTextLowerCase'' bug:
% See https://github.com/Tufte-LaTeX/tufte-latex/issues/64#issuecomment-78572017
% Set up the spacing using fontspec features
\renewcommand\allcapsspacing[1]{{\addfontfeature{LetterSpace=15}#1}}
\renewcommand\smallcapsspacing[1]{{\addfontfeature{LetterSpace=10}#1}}

% for url
\usepackage{hyperref}
\hypersetup{colorlinks = true, 
  linkcolor = teal,
  urlcolor  = teal,
  citecolor = blue,
  anchorcolor = blue}

\newcommand{\me}[4]{
    \author{
      {\bfseries Name:}\underline{#1}\hspace{2em}
      {\bfseries Student ID:}\underline{#2}\hspace{2em}\\[10pt]
  }
}



% for math
\usepackage{amsmath, mathtools, amsfonts, amssymb}
\newcommand{\set}[1]{\{#1\}}

% define theorem-like environments
\usepackage[amsmath, thmmarks]{ntheorem}

\theoremstyle{break}
\theorempreskip{2.0\topsep}
\theorembodyfont{\song}
\theoremseparator{}
\newtheorem*{problem}{Problem}
\newtheorem{ot}{Open Topics}

\theorempreskip{3.0\topsep}
\theoremheaderfont{\kai\bfseries}
\theoremseparator{:}
\theorempostwork{\bigskip\hrule}
\newtheorem*{solution}{Answer}
\theorempostwork{\bigskip\hrule}
\newtheorem*{revision}{Revision}

\theoremstyle{plain}
\newtheorem*{cause}{Cause}
\newtheorem*{remark}{Note}

\theoremstyle{break}
\theorempostwork{\bigskip\hrule}
\theoremsymbol{\ensuremath{\Box}}
\newtheorem*{proof}{证明}

% \newcommand{\ot}{\blue{\bf [OT]}}

% for figs
\renewcommand\figurename{Fig}
\renewcommand\tablename{Diag}

% for fig without caption: #1: width/size; #2: fig file
\newcommand{\fig}[2]{
  \begin{figure}[htbp]
    \centering
    \includegraphics[#1]{#2}
  \end{figure}
}
% for fig with caption: #1: width/size; #2: fig file; #3: caption
\newcommand{\figcap}[3]{
  \begin{figure}[htbp]
    \centering
    \includegraphics[#1]{#2}
    \caption{#3}
  \end{figure}
}
% for fig with both caption and label: #1: width/size; #2: fig file; #3: caption; #4: label
\newcommand{\figcaplbl}[4]{
  \begin{figure}[htbp]
    \centering
    \includegraphics[#1]{#2}
    \caption{#3}
    \label{#4}
  \end{figure}
}
% for margin fig without caption: #1: width/size; #2: fig file
\newcommand{\mfig}[2]{
  \begin{marginfigure}
    \centering
    \includegraphics[#1]{#2}
  \end{marginfigure}
}
% for margin fig with caption: #1: width/size; #2: fig file; #3: caption
\newcommand{\mfigcap}[3]{
  \begin{marginfigure}
    \centering
    \includegraphics[#1]{#2}
    \caption{#3}
  \end{marginfigure}
}

\usepackage{fancyvrb}

% for algorithms
\usepackage[]{algorithm}
\usepackage[]{algpseudocode} % noend
% See [Adjust the indentation whithin the algorithmicx-package when a line is broken](https://tex.stackexchange.com/a/68540/23098)
\newcommand{\algparbox}[1]{\parbox[t]{\dimexpr\linewidth-\algorithmicindent}{#1\strut}}
\newcommand{\hStatex}[0]{\vspace{5pt}}
\makeatletter
\newlength{\trianglerightwidth}
\settowidth{\trianglerightwidth}{$\triangleright$~}
\algnewcommand{\LineComment}[1]{\Statex \hskip\ALG@thistlm \(\triangleright\) #1}
\algnewcommand{\LineCommentCont}[1]{\Statex \hskip\ALG@thistlm%
  \parbox[t]{\dimexpr\linewidth-\ALG@thistlm}{\hangindent=\trianglerightwidth \hangafter=1 \strut$\triangleright$ #1\strut}}
\makeatother

% for footnote/marginnote
% see https://tex.stackexchange.com/a/133265/23098
\usepackage{tikz}
\newcommand{\circled}[1]{%
  \tikz[baseline=(char.base)]
  \node [draw, circle, inner sep = 0.5pt, font = \tiny, minimum size = 8pt] (char) {#1};
}
\renewcommand\thefootnote{\protect\circled{\arabic{footnote}}} % feel free to modify this file
%%%%%%%%%%%%%%%%%%%%
\title{Formal Languages}
\me{Guangwei Zhang}{20221001980}{}{}
%%%%%%%%%%%%%%%%%%%%
\begin{document}
\maketitle
\usetikzlibrary{automata, positioning, arrows}
\section{Sentence making}

Possible generated sentences are:  
\begin{verbatim}
a quick cow happily jumps the sad fox
the sad fox slowly eats a brown owl
a quick owl slowly jumps a brown fox

\end{verbatim}

\section{Grammar generating}

The rules are :

(1)


\begin{align}
    \text{VP}& \rightarrow \text{V} \\
    \text{VP}& \rightarrow \text{V  VP}
\end{align}

Example\footnote{The core of the language is to recursion}: 
\begin{verbatim}
    a quick cow happily eats slowly jumps the quick fox
\end{verbatim} 

(2)

\begin{align}
    \text{NP} &\rightarrow \text{Det NONS QST}\\
    \text{S} &\rightarrow \text{Aux NP VPO} \\
    \text{VPO} &\rightarrow \text{VerbO N}\\
    \text{NONS} &\rightarrow \text{N CONJ NONS}\\
    \text{Aux} &\rightarrow \text{do}\\
    \text{VerbO} &\rightarrow \text {jump | eat | catch}\\
    \text{CONJ} &\rightarrow \text{and}\\
    \text{QST} &\rightarrow \texttt{?}
\end{align}

Example\footnote{Now we can ask questions can't we? }: 

\begin{verbatim}
    do the cow and the fox eat the owl?
\end{verbatim}

\section{FSA Problem}

(1) The answer is filled in the following sheet: 
\begin{center}
    \begin{tabular}{|l|l|l|l|l|l|l|l|l|l|}
    \hline
    1 & 2 & 3 & 4 & 5 & 6 & 7 & 8 & 9 & 10 \\ \hline
    Y & N & N & Y & N & N & N & Y & Y & N  \\ \hline
    \end{tabular}
\end{center}

(2) The formal grammar can be written as the following:  

With the finite state automaton $M=\{\{S_0, S_1, S_2, S_3, S_4\}, \{\texttt{run}, \texttt{faster}, \texttt{forrest} \}, t, S_0, \{S_3, S_4\}\}$, where $t$ is the all possible form of $(A, b)\rightarrow C$ where node $A$ is connected by edge $b$ to node $C$. 

\begin{align}
    S_0 & \rightarrow  \texttt{run } S_1 | \texttt{forrest } S_3\\
S_1 & \rightarrow  \texttt{run } S_0 | \texttt{faster} S_2 | \texttt{forrest } S_3\\
S_2 & \rightarrow  \texttt{forrest } S_3 |\texttt{run } S_1 \\
S_3 & \rightarrow  \texttt{run } S_4 | \emptyset \\
S_4 & \rightarrow  \emptyset
\end{align}


\section{Build an automation}
\tikzset{
->, % makes the edges directed
>=stealth, % makes the arrow heads bold
node distance=3cm, % specifies the minimum distance between two nodes. Change if necessary.
every state/.style={thick, fill=gray!10}, % sets the properties for each ’state’ node
initial text=$ $, % sets the text that appears on the start arrow
}
We have the following automation with $s_2$ and $s_4$ are accepting states\footnote{Regarding 3 letters as a whole for simplicity without affecting its correctness. If the length of the string is not divisible by 3, then add minimum spaces to make it divisible by 3.}. 
\begin{center}
    \begin{tikzpicture}
        \node[state, initial] (s) {$S$};
        \node[state, right of=s] (s0) {$s_0$};
        \node[state, below left of=s0] (s1) {$s_1$};
        \node[state, right of=s0, accepting] (s2) {$s_2$};
        \node[state, below of=s0] (s3) {$s_3$};
        \node[state, right of=s3, accepting] (s4) {$s_4$};
        \node[state, right of=s0] (s2) {$s_2$};
        \node[state, right of=s4] (s5) {$s_5$};
    
    
        \draw (s) edge[above] node{ATG} (s0)
        (s) edge[above] node{not ATG}(s1)
        (s0) edge[above] node{TAG} (s2)
        (s0) edge[above, bend right] node{not TAG} (s3)
        (s3) edge[loop above] node{not TAG} (s3)
        (s3) edge[above] node {TAG} (s4)
        (s4) edge[above] node {any} (s5);
        % (q1) edge[above] node{1} (q2)
        % (q2) edge[loop above] node{1} (q2)
        % (q2) edge[bend left, above] node{0} (q3)
        % (q3) edge[bend left, below] node{0, 1} (q2);
    \end{tikzpicture}
\end{center}


\vspace*{1pt}

\begin{center}
    \texttt{End of lab assignment.}
\end{center}

\end{document}