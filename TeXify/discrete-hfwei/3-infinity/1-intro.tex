我们在高中的学习中, 对于极限的了解仅仅停留在表面和直观感受上. 那么, 我们有没有办法把这件事情严格化呢? 这是我们这次主要需要的事情. 

\begin{quote}
    From his paradise that Cantor with us unfolded,
    we hold our breath in awe; knowing, we shall not be expelled.

    \hfill --- David Hilbert
\end{quote}

\section{在Cantor以前}

在《几何原本》中, Euclid明确提出: ``整体大于部分''的公理. 但我们对无穷多个元素的集合而言, 部分是可以``等于''整体的. 

考虑下面的两个集合: 
\[
S_1 = \set{1, 2, 3, \cdots, n, \cdots}
\]
\[
S_2 = \set{1, 4, 9, \cdots, n^2, \cdots}
\]
我们可以给$S_1$的每一个$S_1$中的元素$a$做映射$a^2$, 使得$a^2 \in A_2$, 因此可以认为这是相等的. 但是明显的, $S_2 \subset S_1$. 这就是所说的``部分等于全体''. 用我们有限的心智来讨论无限 $\cdots$

\begin{quote}
    ``说到底,`等于'、`大于'和`小于'诸性质不能用于无限,而只能用于有限的数量。'' 
    \hfill --- Galileo Galilei
\end{quote}

当时也有人反对使用``无穷''的论点, 认为这是一派胡言. 
\begin{quote}
    ``无穷数是不可能的。'' \hfill --- Gottfried Wilhelm Leibniz
\end{quote}

但是Cantor坚定地认为, 以前我们的``大于'', ``小于''那一套东西已经过时了, 要讨论``无穷'', 当然要建立一套新的体系, 来帮助我们理解这一套内容. 

\begin{quote}
    ``这些证明一开始就期望那些数要具有有穷数的一切性质,
    或者甚至于\blue{把有穷数的性质强加于无穷}。

    相反,这些无穷数,如果它们能够以任何形式被理解的话,
    倒是由于它们与有穷数的对应,\red{它们必须具有完全新的数量特征}。

    \teal{这些性质完全依赖于事物的本性},$\cdots$而并非来自我们的主观任意性
    或我们的偏见。''

    \hfill --- Georg Cantor (1885)
\end{quote}

于是他们尝试使用映射的观念来定义无穷: 
\begin{definition}[Dedekind-infinite \& Dedekind-finite (Dedekind, 1888)]
    A set $A$ is \purple{\it Dedekind-infinite}
    if there is a bijective function from $A$ to some \red{proper} subset $B$ of $A$. 

    A set is \purple{\it Dedekind-finite} if it is not Dedekind-infinite.
\end{definition}

但是我们还没有定义``finite''和``infinite''. 所以下面我们要用函数的观念来比较集合. 

\section{集合的比较}

\subsection{集合的个数相等}

从开始的问题中, 我们可以看到, 只要出现了一个双射函数, 我们就可以说两个集合的元素个数相等. 于是给出如下定义:
\begin{definition}[$|A| = |B| \;(A \approx B)$ (1878)]
    $A$ and $B$ are \red{\it equipotent (等势)}
    if there exists a \purple{\it bijection} from $A$ to $B$.
\end{definition}

一个集合其实是不关心这个集合元素的次序,在``势''的考量下, 也不关心集合中的元素是什么, 就是两层抽象. 有时候也写作$\overline{\overline{A}}$. 

那么, 等势关系是一个等价关系吗? 不要关注太多, 要想证明远没有想象的那么简单.

\begin{theorem}{\purple{The ``Equivalence Concept'' of Equipotent}}
    For any sets $A$, $B$, $C$:
    \begin{enumerate}
      \item $A \approx B$
      \item $A \approx B \implies B \approx A$
      \item $A \approx B \land B \approx C \implies A \approx C$
    \end{enumerate}
\end{theorem}

有了``势''的概念, 我们就可以对``有限''进行定义了. 
\begin{definition}[Finite]
    $X$ is finite if
    \[
      \exists n \in \N: |X| = |\red{n}| = |\red{\set{0, 1, \dots, n-1}}|.
    \]
\end{definition}
我们很多时候写作$|X| = n$. 这就意味着集合 $X$ 是有穷的当且仅当它与某个自然数等势. 

相反的, 我们可以定义无穷. 

\begin{definition}[Infinite]
    $X$ is infinite if it is not finite:
    \[
      \forall n \in \N: |X| \neq n.
    \]
\end{definition}

我们既然定义了, 当然要说明它是存在的. 于是我们有这样的一个定理: 

\begin{theorem}[\red{Existence of Infinite Sets!}]
    $\N$ is infinite.{\cyan{(So are $\Z$, $\Q$, $\R$.)}}
\end{theorem}

我们可以使用反证法证明. 

\begin{proof}
    假设$\exists n \in \N$, 使得$|\N|=n$, 那么我们就存在$f:\N \xleftrightarrow[onto]{1-1} \set{0, 1, \cdots, n-1}$. 

    下面, 我们构造限制映射$g \triangleq \purple{f|_{\set{0, 1, \ldots, n}}}: \set{0, 1, \cdots, n} \to \set{0, 1, \cdots, n-1}$. 由于抽屉原理, $g$不是一一映射, 那么$f$也不是一一映射. 

\end{proof}

我们注意到无穷也有几种. 比如有的无穷是可以一个一个计数清楚的, 有的无穷是不行的. 

\begin{definition}[Infinite]
    For any set $X$,
    \begin{itemize}
      \item Countably Infinite: 
        \[
          |X| = |\N| \red{\;\triangleq \aleph_{0}}
        \]
      \item Countable.
        \[
          \text{(finite $\lor$ countably infinite)}
        \]
      \item Uncountable. 
        \[
          \text{($\lnot$ countable)}
        \]
        \[
          \text{(infinite) $\land$ \Big($\lnot$ (countably infinite)\Big)}
        \]
    \end{itemize}
  \end{definition}

自然的, 我们会发现$\Z$是可数的. 如果我们把$\N$的每一个元素扩大到原来了2倍, 中间就会稀疏一些, 可以容纳下负数. 

Cantor发现$\Q$也是可数的. 因为Cantor是发现了一种数出有理数的方法. 其一句就是任何一个有理数(quotient)都可以成为形如$a/b, \gcd(a, b)=1$成比例的数. 如此: 
$$
\begin{aligned}
  1/1\quad &\quad &\quad & \quad\\
  1/2\quad &2/1\quad&\quad &\quad\\
  1/3\quad &2/2\quad&3/1\quad&\quad\\
  1/4\quad&2/3\quad&3/2\quad&4/1\quad \\
\end{aligned}
$$

因此, 有理数是可数的. 更进一步的, $\N\times \N$也是可数的. 因为我们只要做一个映射就行了. 具体的, 可以把$\N \times \N $压缩到$\N$上. 也就是$\pi(k_1, k_2) = \frac{1}{2} (k_1 + k_2)(k_1 + k_2 + 1) + k_2$. 

按照归纳的方法, 也就是$\pi^{(n)}(k_1, \ldots, k_{n-1}, k_n) = \pi (\red{\pi^{(n-1)}}(k_1, \ldots, k_{n-1}) , k_n)
\quad (n \ge 3)$, 有如下的定理: 
\begin{theorem}[$\N^{n}$ is Countable.]
  \[
    |\N^{n}| = |\N|
  \]
\end{theorem}

进一步的推广, 我们有:
\begin{theorem}
  The Cartesian product of \red{finitely many} countable sets is countable.
\end{theorem}

另外, 任意有限集的并集都是可数的, 我们还可以用刚刚的想法, 使用对角线计数. 

再后来的研究中, 我们惊奇的发现, 有些无穷的大小之间是有着深刻的差别的. 例如, $\R$是不能够被数出来的. 同样也是用对角线证明法得到的结论. 

\begin{theorem}[$\R$ is Uncountable. (Cantor 1873-12; Published in 1874)]
  \[
    |\R| \neq |\N|
  \]
\end{theorem}

这个定理就告诉我们了实数不能被表示成有理数的有序对. 这就说明了$\R$是一个连续统. 另外一个惊人的事情是, $(0,1)$之间的实数和$\R$, $\R\times\R$的势是一致的. 
\begin{theorem}[$|\R|$ (Cantor 1877)]
    \[
      |(0,1)| = \red{|\R| = |\R \times \R|} \blue{\;= |\R^{n \in \N}|}
    \]
\end{theorem}
  
一个可能的证明方式是考虑
\[
    (x = 0.\blue{a_1 a_2 a_3 \cdots}, y = 0.\red{b_1 b_2 b_3 \cdots}) \mapsto 0.\blue{a_1}\red{b_1}\blue{a_2}\red{b_2}\blue{a_3}\red{b_3}\cdots
\]
但是很不幸, 这个证明是错误的. 严格的证明需要系统地学习数学分析才可以知道. 

在成功证明这件事情之后, Cantor写给Dedekind的一封信里面说到: ``Je le vois, mais je ne le crois pas !'' 翻译成英语就是``I see it, but I don’t believe it !''. 可见这样对于``无穷''的探讨是十分激动人心同时也是令人感到违背常识的. 

更惨的还在后面: 我们在《线性代数》课程和之前的理解上, 维数好像总是有限的. 但是Cantor的论述让我们对于``维数''的理解出现了很多问题. 其中一个问题是, 我们能不能通过一个双射把$m$维的东西映射到$n$维去? 

\begin{theorem}[Brouwer (Topological Invariance of Dimension)]
    There is no \red{continuous} bijections between $\R^m$ and $\R^n$ for $m \neq n$.
\end{theorem}

什么是continuous的双射? 这个就需要更多数学专业的知识了. 在这里不做阐述了. 

\section{连续统假设}
 连续统假设说明的是
 \[
    \red{\fbox{\text{ $\nexists A: \aleph_{0} < |A| < \mathfrak{c}$}}}
  \]
  
  事实上, 这是一个没有办法证明的命题. 

