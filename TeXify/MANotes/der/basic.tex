\subsection{Cheat Sheet}
\ti{导数表.}
(1) 和三角函数相关

\begin{eqnarray}
    (\cos x)^{\prime} & = & -\sin x .\\
    (\tan x)^{\prime} & = & \frac{1}{\cos ^{2} x} = \cot{x}.\\
    (\cot x)^{\prime} & = & -\frac{1}{\sin ^{2} x}.\\ 
    (\arcsin x)^{\prime} & = & \frac{1}{\sqrt{1-x^{2}}} .\\
    (\arccos x)^{\prime} & = & -\frac{1}{\sqrt{1-x^{2}}} .\\
    (\arctan x)^{\prime} & = & \frac{1}{1+x^{2}} .\\
    (\operatorname{arccot} x)'&=&-{1\over \sqrt{1-x^2}}.
\end{eqnarray}

(2) 其余从略. 

\ti{常见高阶导数.}

1. 分式形高阶导数

$$
\left(\frac{1}{x+a}\right)^{(n)}=\left[(x+a)^{-1}\right]^{(n)}=-\left[(x+a)^{-2}\right]^{(n-1)}=\cdots=\frac{(-1)^{n} n !}{(x+a)^{n+1}}
$$

2. 对数形
$$
[\ln (x+a)]^{(n)}=\left(\frac{1}{x+a}\right)^{(n-1)}=\frac{(-1)^{n-1}(n-1) .}{(x+a)^{n}}
$$

3. 三角函数形
$$
\cdot[\sin (k x+a)]^{(n)}=k\left[\sin \left(k x+a+\frac{\pi}{2}\right)\right]^{(n-1)}=\cdots=k^{n} \sin \left(k x+a+\frac{n \pi}{2}\right) 
$$
同样, 有: 
$$
\cdot[\cos (k x+a)]^{(n)}=k\left[\cos \left(k x+a+\frac{\pi}{2}\right)\right]^{(n-1)}=\cdots=k^{n} \cos \left(k x+a+\frac{n \pi}{2}\right)
$$

4. 多项式若干次方
$$
\left[(x+a)^{m}\right]^{(n)}=\left\{\begin{array}{cc}
    m(m-1)(m-2) \cdots(m-n+1)(x+a)^{m-n} & , m \geq n \\
    0 & , m<n
    \end{array}\right.
$$

5. Newton二项式定理(略).