\subsection{习题}

\begin{prob}% P188 5.15 例子题目
    证明:
    $$
    \sum_{k=0}^n(-1)^{n-k}{n\choose k}k^m=\begin{cases}
        0, m=0,1,\cdots, n-1\\
        n!, m=n
    \end{cases}
    $$
\end{prob}

\begin{proof}
    考虑构造多项式$f(x)=x^m$, 那么做一次离散差分, 也就是
    $$
    \textbf{D}f(x) \triangleq (x+1)^{m}-x^m = {m\choose 1}x^{m-1}+{m\choose 2}x^{m-2}+\cdots+1
    $$ 
    接着在做一次离散差分
    $$
    \textbf{D}^2 f(x) \triangleq \textbf{D}(f(x+1))-\textbf{D}(f(x)) = {m\choose 1}{m-1\choose 1}x^{m-2}+\cdots+c
    $$

    经过$n=m$次离散差分之后, 每次都会消去最后的一项. 在$n=m$次差分之后, 函数中已经会没有$x$, 因此是$n!$. 否则, 要么式子中会存在$x$, 要么值已经是0. 因此可以解释. 

    
\end{proof}

\begin{remark}
    我们可以把离散差分推广到一般的形式. 

    首先观察到对于如下的式子有着很好的结论:
    $$
    \begin{aligned}
        \textbf{D}(\fpow{x}{m})\fpow{x+1}{m}-\fpow{x}{m}
        &= (x+1)(x)\cdots (x-m+2) - x(x-1)\cdots (x-m+1)\\ 
        &= mx(x-1)\cdots(x-m+2) \\
        &= m\fpow{x}{m-1}
    \end{aligned}
    $$

    并且观察到, 
    $$
    \begin{aligned}
        \opbold{D} f(x)&=f(x+1)-f(x) \\ 
        \opbold{D}^2 f(x)&=f(x+2)-2f(x+1)+f(x)\\ 
        \opbold{D}^3 f(x)&=f(x+3)-3f(x+2)+2f(x+1)+f(x) 
    \end{aligned}
    $$

    因此这样就可以总结出
    $$
    \opbold{D}^n f(x)=\sum_{k} {n\choose k} (-1)^k f(x+k)
    $$
    这样的结论. 
\end{remark}

%%%