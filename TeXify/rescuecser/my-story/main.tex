

    我是出生于河南省的一名普通学生. 很幸运在郑州市第一中学这一所崇尚自由的中学中读完了高中三年. 下面我将简要描述一下我的过去, 然后试图从中得到一点内容. 

    \section{我的家庭环境}
    \subsection{我和父亲}

    在我心里面父亲一直都是一个很喜欢买书, 阅读书籍的人, 小时候在家里面的时候注意到他买了很多关于育儿的书籍(当时也看了一些, 可能是现在我很多时候会更好的换位思考, 产生共情的来源).  

    但是父亲很多时候管理自己的情绪的能力不是很好. 记得小时候总是因为一些比较基础的事情违背了自己先前订下的约定. 这就一定程度上让我有一点``所有的约定必须遵守''的行为了. 

    我记得小时候生病或者受伤的时候, 父亲可能不会有什么责骂, 但是到了高三的时候, 很多时候生病之后, 通常表现的那一点任性可能会让他感到愤怒. 虽然可以理解, 但是本质上还是缺乏素养的表现. 因此这就启发我在遇见同样问题的时候, 应当淡然处之. 

    在开心的时候, 通常父亲只能理解到``开心''的表面因素, 而不知道我真正在开心什么. 这当然是限于当时的时代因素和个人经历等因素的多重综合, 据我了解, 能够不被反对已经是很好的一个尝试了.

    

    \subsection{我和母亲}

    母亲在我的小时候一直是很慈爱的. 她会提出父亲的逻辑漏洞, 很多时候让父亲难以发火. 这就非常的有意思. 到了小学六年级, 我要学习在那个阶段非常难的英语以获得去当时最好的中学的机会, 但当时的我是不喜欢这件事情的. 所以大部分时间就都摆烂了. 但是父亲和母亲的做法是不一样的. 父亲总是尝试通过施压的方式, 而母亲总是尝试用对话(虽然很多内容逻辑并不自洽, 当时太小了数理逻辑并不完善)的方式解决问题.
    
    当然, 决定权总是在我的. 其实总体而言, 我可以看出来父母已经尽可能的理解我了, 并且提供给我了足够多的自由空间, 这一点而言, 我也是对他们没有任何怨恨的. 

    但这同样也造成了一些不足之处. 比如我在难过的时候很多时候倾诉是被动的. 因为有时候很害怕如果倾诉了之后被批评. 所以这就是一个启示就是任何人找我来聊天的时候我总是不会用说教的形式把自己的观念强加给别人. 

    \section{在信息竞赛(OI)中, 那些重要的人和事}

    \subsection{初级中学}

    当时正值小升初完成, 我考到了郑州市第一中学, 当时买了Minecraft, 就有自己写一个游戏的冲动. 因此简单学一点Java语法, 并且做了一些现在看上去很平凡的工作. 

    这段时光在我现在看上去算是荒废了. 于是zgw\footnote{张桄玮, 笔者}每一次回望这段时间的时候, 都要质问一下为什么ZZSYFLS\footnote{郑州实验外国语中学}没有OI社团. 不过那时候大部分老师还是很支持我的, 有什么想法他们也是很鼓励的. 

    \subsection{高中时代}

    中考本来以为没进入当地所谓的``最好''的高中(郑州外国语中学), 但是现在回想起来, 郑州一中(第二好的中学)的理念至少让我看到了什么是``正确''的教育. 高中的文化课是十分平凡的, 因为知识点容易理解, 大家就是在卷做题, 没有美感, 自然没有任何意思. 因此我们来简单看一下我在OI时候那些不可或缺的经历. 

    \subsubsection{加入OI}

    凭借之前的学习的经历和基础, 我轻松获得了竞赛班的名额, 进去之后, 我发现他们真的差我好一大截, 于是就心安理得的开摆. 因为我也不知道这个东西除了考学有什么意义\footnote{好吧, 后来在高一寒假的时候因为羟基计划OI取消了自主招生, 就不太好了. 不过也激发了我思考为什么学习造成了一个契机.}. 因此, 在这样的彷徨, 每天还要疲于应付学校9科作业, 显得实在是力不从心\footnote{让我理解了人的力量是有限的. }. 学习效率也很低下.

    举个例子, 当时郑州市有一个ACM\footnote{是的, 就是和当时的高中生组队打}比赛, 我居然连高精度的A+B Problem\footnote{这是很简单的问题, 学过的大家都应该会}都写不出来! 

    \begin{quote}
        听说往年他们都轻松市一, 结果我们什么也没有. \hfill --- Bingzhen Xie
    \end{quote}

    \subsubsection{继续摆烂}

    ``这些算法原理好像很简单啊!''zgw想, 但是一旦真正写下来就像火葬场一样, 漏洞百出. 彼时我还没有``出了问题也是很正常''的观念, 一心只想隐藏. 因此, 在Bingzhen Xie, Yufanxing Liu\footnote{机房另外两位大佬}努力学习图论的时候, 我却在摆烂, 回想起来算是花时间买教训了吧\footnote{后来听到了Yanyan Jiang在南京大学的操作系统课程, 突然发现了有问题, 不过那时候已经高三了, 问题发现的很晚}.
    \newpage
    \subsubsection{北京集训}

    当时文化课当然是没有经历完全顾及到, 整体都是得过且过, 考完就忘的一种心态. 竞赛也在吃老本. 在这里, 讲师讲得十分的快, 因为他默认这些选手有基础, 而我什么也没有! 这就让我体会到了深深的无力感! 更重要的是, 模拟赛场场爆炸. 于是, 刚刚燃起的奋斗的热情又被浇灭了\footnote{其实让当时的我重新恢复热情也是简单的, 只要有个人告诉我``我当时学的时候也这么难'', 后来证明这个人是南京大学的Yanyan Jiang老师在一次课上告诉我的}.
    
    这时候也加了很多神仙, 其中一个神仙是BLUESKY007(Nongyu Di), 当时我在``为什么要加我''一栏中浩然填下了``AKCSP''\footnote{表示在某门比赛取得满分} 的豪言壮语. 
    
    \subsubsection{BLUESKY007是我最好的朋友}

    其实说来也简单, 就是我在他出的模拟赛里面过了一道当时看起来非常难的题目, 要用到数位DP. 他当时告诉我``很有希望进省队''\footnote{显然我最后没有进省队. 可能算是对不起他的一个地方. 但是也不一定, 今天的我可以整天阻止他无限制的摆烂, 也是挺好的一件事}. 这让我很感激, 并且我也会记得任何一个无偿的给予我温柔的一些人, 一定程度上也让我更加的温柔, 愿意奉献和付出. 

    后来我惊喜的发现我能和他聊天这么顺利的原因, 是我们都曾经就读于ZZSYFLS中学\footnote{本来还想要跟他读一个大学的, 可惜差了两分. 还是拥抱生活的不确定性吧}.

    BLUESKY007大我两届, 现在是LZU竞赛的主力军, 大一的他带领LZU的ACM队征战全场(不愧是NOI银牌获得者). 于是顺利的, 今年他也在CCPC中取得了银牌的好成绩. 

    \begin{quote}
        还没有拿金牌, 不退役. \hfill --- BLUESKY007
    \end{quote}

    后来BLUESKY007在很多地方做的都比我强, 因此我要向他学习. 

    基本上遇见的冲突也不算多, 因为很多时候我们的想法是很相似的. 网络上总是还是能够让冲突的频率和概率降级的. 我想大概是这样的. 

    总之, 我认为, BLUESKY007是我最好的朋友和老师, 也是教会我乐观的学长, 这是一笔宝贵的财富. 

    \subsubsection{NOIP/CSP之前}

    说实话, 当时``已经没有什么好怕的了''---因为什么都不会! 为什么呢? 这还是和ZZYZ\footnote{郑州市第一中学, 高中}和ZZSYFLS\footnote{郑州实验外国语, 初中}的教育模式不一样有关的. 

    ZZYZ是一所自由的学校, 老师和同学都不会给很多压力, 这样就会导致自主和能动性不强. ZZSYFLS呢? 那里是一个别人会过度督促你的地方---一次轻微的测试成绩下滑就有可能被老师批评. 因此, 在这样的环境的差异下学习, 就像潜水员从深海一下上升到了海平面, 根本不适应. 

    
    \subsubsection{总结我的OI生涯}

    总体而言, 整个过程的锅还是不少的. 

    --- 倘若当时真的少点做作, 真心踏实下来, 可能经历会更丰富吧! 
    
    --- 倘若当时可以多帮同学调一点代码, 应该就不是现在这个样子了吧! 

    --- 倘若当时可以放下面子多请教学长, 我应该就可以有更加丰富的想法了吧!

    ...

    以上是我自己认为的一些失去. 但是得到的最重要的事情, 就是``学习没有捷径'', 要遵循客观规律行事. 

    \subsubsection{领悟的高三}

    由于我是前信息学竞赛的学生, 自然有每天出入机房的权利. 一天, 一个名字叫``绿导师原谅你了''的ID闯进了我的主页.\footnote{他就是NJU的Yanyan Jiang老师. } 题目是``$\sum$''求和---一个求和程序. 

    当时由于我在OI的时候数学实在太差了, 因此自然想补充一下. 点进去一看, 豁然开朗. 原来以前真的是遵循了完全错误的学习方法来学的啊! 那一刻真的感觉到了好课的重要性. 

    但是可能还是因为ZZSYFLS那种批评教育的模式的阴影下, 我的内心依旧十分脆弱. 好在BLUESKY007可以乐观的让我看问题, Yanyan Jiang能够提出一个强大的价值观, 让我的内心得到一番宁静. 虽然素未谋面, 但是总是让我的未来有了点方向. \footnote{他们两位是最重要的人了.}
    
    \subsubsection{Yanyan Jiang的鼓励}

    当时高考爆炸了, 因此就给Yanyan Jiang老师发了这样一条私信:
    \begin{quote}
        今年的高考可能没办法成为您的学生了, 只有考研这一关再战了! 谢谢老师! \hfill ---AUGPath
    \end{quote}

    令人惊讶的是, 他居然回复了我的私信. 

    \begin{quote}
        哈哈哈 高中生 前途无量啊 
        
        衷心感谢 :)做老师的快乐也莫过于此了。世界很大,一定要出去看看,国外好老师多!有心能检查\footnote{Yanyan Jiang笔误了, 是``坚持''}下去,就一定能做出了不起的成绩!

        \hfill --- Yanyan Jiang
    \end{quote}

    这也是我每天都要进行一些学习, 让自己变得更强的动力源泉. 从暑假到现在, 每天日拱一卒, 看看能不能做一点好玩的事情. 

    \section{接纳自己}

    渐渐的, 我意识到很多人之所以看上去`反应灵敏', 并不是真的聪明, 可能只是因为提前学过. 这样一来, 我就可以从容应对很多内容. 最近在看Maki's Lab的学习方法和认知理论, 确实和我的观察是一致的. 如果当时我的内心足够强大, 以至于真的能够把它们提出来, 那么我想, 我走上进步之路的时间, 可以更加早一些.

    不过还是可以庆幸的. Ayumu是在三十多岁的时候, 为了寻找自己内心的一份宁静, 学习的数学, 我比他早了十多年. 

    比来比去终究还是没有意义的. 现在我的衡量标准, 仅仅是把今天的自己, 和昨天的自己放在一起. 多学一个知识, 对某一个知识产生了更深刻的理解, 很多时候都算是一种进步, 就值得肯定. 这就是很多时候心情很平和, 稳定的原因.

    随着越来越多同学认识到国内计算机科学教育的落后, 很多老师, 学生都自发的希望改变这一个情形. 

    此外, 我认为我还要做点什么. 据我目前观察, 高中和大学教育很多时候是脱节的. 即使在自由之情洋溢的ZZYZ, 某些学生也会难以适应大学的生活(通常是卷王)! 所以我(现在还很菜)准备和Maki's Lab的独不迁老师一起构建一门``缺失的那一学期''一样的课程, 一起铺平计算机科学生的学习道路, 同时提高自己. 现在还在课程的筹备阶段. 预计明年新的一批高三学生高考完的时候, 课程就要建设好了. 这个价值我认为是很大的, 比ACM得个奖项真的要有意义多了. 

    \begin{quote}
        “我能力很强、参加了很多竞赛”:我自己是竞赛出身 (ICPC World Finalist)、参加过数学建模也目睹过其中各种不靠谱的黑幕,也受过一点理论计算机科学的训练。竞赛给我的意义是看到更广阔的理论计算机科学天地,并且看到自己和其他人在解题能力上的差距。 “我发表了论文/专利/软著,我有过良好的科研训练”:我对论文的标准是 “教会你的大/小专家同行一些不 trivial 的事情”。据我近年观察,很多发表的论文都是减分项。相比于发表了 “错误” 论文或是在大创项目里学会了一本正经胡说八道的同学,我更偏好能耐心读论文和写代码的 “一张有潜力的白纸”。科学研究是脚踏实地的,前人所做到的 (内卷程度) 可能远比你想象得要大,认真读了 3-4 年博士依然没有论文的也大有人在,完全不必急功近利。\hfill --- Yanyan Jiang
    \end{quote}

    今年暑假的时候给一名刚准高中生用一个非传统的视角体会了解说了一点高中数学, 感觉还是很开心的. 

    \section*{一些链接和参考}

    \begin{itemize}
        \item 《高中数学入门》 讲稿 (Guangwei Zhang) \url{https://shzaiz.github.io/lecture/SHMathA/-1LN.html}
        \item 《NJU程序分析》(Yue Li, Tian Tan)\url{https://zhuanlan.zhihu.com/p/417187798}
        \item Yanyan's Wiki (Yanyan Jiang)\url{http://jyywiki.cn/}
        \item 张桄玮: B站个人主页 \url{https://space.bilibili.com/13246364}
        \item Ayumu: B站个人主页 \url{https://space.bilibili.com/1632276842}
        \item 独不迁: B站个人主页 \url{https://space.bilibili.com/14644161}
    \end{itemize}
