\subsection{函数: 作为关系的一个子集}

函数不允许一对多, 这就是它和``关系''最大的去区别. 

回想我们以前学习过的东西, 好像``关系''和方程的图像有点相似. 也就是我们在高中的时候在平面直角坐标系中做的椭圆的图线: $x^2/a^2+y^2/b^2=1(a, b>0)$. 接下来, 我们不妨来看一种特殊的``关系'': 函数. 

\begin{bonus}
    为什么函数不允许``一对多''? 在定义上有什么合理性?
\end{bonus}

让我们重新定义一下以前学过的函数. 

\begin{definition}[Function]
    $f \subseteq A \times B$ is a \red{\it function} \blue{from $A$ to $B$} if

    \[
      \red{\forall} a \in A.\; \red{\exists!} \;b \in B.\; (a, b) \in f.
    \]
\end{definition}

在函数中, 除了定义域和值域之外, 还有陪域. 通常被称作``cod''. 比如对于一个映射(函数)$f: A \to B$, $\dom(f) = A , \purple{\cod}(f) = B$, 对于一个函数的值域$\ran(f) = f(A) \subseteq B$. 

为什么这样定义? 值域为什么不是$B$? 原因是很多时候函数的值域难以求解, 这样就使得我们的表达造成很多不便. 而且很多时候如果强行把$B$当作值域很多时候可能会出现运算不封闭的问题, 在研究某些问题的时候非常不方便. 因此, 我们不妨把这个值域扩大一些, 这样才可以更方便一些. 因此, $B$就叫做``陪域''. 值域只不过是陪域的一个子集. 

对于证明而言, 我们同样有一套形式化的证明语言: 
$\red{\forall a \in A.}, \forall a \in A.\; \exists b \in B. (a, b) \in f, \red{\exists! b \;\in B.}, \forall b, b' \in B.\; (a, b) \in f \land (a, b') \in f \implies b = b'.$

当然我们可以看一些有趣的函数. 

\textbf{1. 恒等函数. } ``恒等''在数学的各个领域里面都是重要的. ``恒等函数''的地位有时候和加法意义下的`0', 乘法意义下的`1'很相似, 其特点是经过一次复合之后还是一样的. 我们一般用$I_X$表示, I的意思是identity的缩写. 其中, 
$$\forall x \in X.\; I_{X}(x) = x.$$

\begin{fun}
    Weierstrass构造了一个处处连续, 处处不可导的函数. 
    $$
    f(x)=\sum_{n=0} ^\infty a^n \cos(b^n \pi x),
    $$
    其中, $0 < a < 1,\; b \text{ is a positive odd integer},\; ab > 1+\frac{3}{2} \pi$. 
\end{fun}

当然, 我们也可以把相似的函数放在一个集合里面. 

\begin{definition}[$Y^{X}$]
    The \red{\it set} of all functions \blue{from $X$ to $Y$}:
    \[
      Y^{X} = \set{f \mid f: X \to Y}
    \]
\end{definition}

举一些例子, $|X|=x, |Y|=y, |Y^X|=x^y$. 

\begin{eg}
    \begin{itemize}
        \item $\forall Y.\; Y^{\emptyset} = \set{\emptyset}$
        \item $\emptyset^{\emptyset} = \set{\emptyset}$
        \item $\forall X \neq \emptyset.\; \emptyset^{X} =  \emptyset$
        \item $2^{X} = \set{0, 1}^{X} \cong \ps{X}$
    \end{itemize}
    
\end{eg}

类似的, 我们可以问: 是否存在由所有函数组成的集合? 像Russell一样, 我们的答案是否定的. 

\begin{theorem}
    There is no set consisting of all functions.
\end{theorem}

\begin{proof}
    Suppose \red{by contradiction} that $A$ is the set of all functions. 
    For every set $X$, there exists a function $I_{\red{\set{X}}}: \set{X} \to \set{X}$.
    \[
      \bigcup_{I_{X} \in A} \dom(I_{X})
        \text{ would be the \purple{universe} that does not exist!}
    \]
\end{proof}

既然函数和集合的结论如此相似, 我们自然地想到函数有没有和集合一样的性质? 

\subsection{作为集合的函数}

\begin{theorem}[函数的外延性原理 (The Principle of Functional Extensionality)]
    
      $f, g$  are functions:
    
    \[
      f = g \iff \dom(f) = \dom(g)
        \land \big(\forall x \in \dom(f).\; f(x) = g(x) \big)
    \]
\end{theorem}

注意定义并没有要求陪域相同, 只要$f = g \iff \forall (a, b).\; ((a, b) \in f \leftrightarrow (a, b) \in g)$满足, 我们就认为这是相等的. 

既然是集合, 我们就要考察一些集合的运算. 如果$f$和$g$是函数, $f \cap g, f \cup g$是函数吗? 因此我们有如下的定理: 

\begin{theorem}[Intersection of Functions]
    \[
      A = \set{x \mid x \in A \cap C \land f(x) = g(x)}
    \]
    \[
      f \cap g = \set{(x, y) \mid x \in A, y = f(x) = g(x)}
    \]
\end{theorem}

\begin{theorem}[Union of Functions]
    {\[
      f \cup g: (A \cup C) \to (B \cup D) \iff
      \forall x \in \dom(f) \cap \dom(g).\; f(x) = g(x)
    \]}
\end{theorem}

举几个例子. 如果我们有$f: \ps{\R} \to \Z$, $f(A) = \left\{\begin{array}{ll}
    \min(A \cap \N) & \text{if } A \cap \N \neq \emptyset \\
      -1 & \text{if } A \cap \N = \emptyset
  \end{array}\right.$. 
注意$\N$的良序原理, $\dom{(f)}\cap \dom{(g)}=\emptyset$. Dichlet函数也可以看作函数的并. 它是$f:\R \to \R$的一个映射. 表达式写做: $D(x) = \left\{\begin{array}{ll}
    1 & \text{if } x \in \Q \\
    0 & \text{if } x \in \R \setminus \Q
\end{array}\right.$注意到这个函数是``处处不连续''的. 

\subsection{特殊函数关系}

有时候函数之间的映射关系也是重要的. 比如, $f:A\to B$, $A$在$B$中的对应元素是不是都是不同的? $B$中的元素有没有全部对应上$A$中的元素(可能不止被对应了一次)? $A$有没有和$B$中元素一一对应? 这样我们就有了单射, 满射的概念. 

\begin{definition}[Injective (one-to-one; 1-1) 单射函数]
    \[
      f: A \to B \qquad {f: A \red{\;\rightarrowtail\;} B}
    \]

    \[
      \forall a_1, a_2 \in A.\; a_1 \neq a_2 \to f(a_1) \neq f(a_2)
    \]
\end{definition}

对于证明而言, 我们可以这样写: $\forall a_1, a_2 \in A.\; f(a_1) = f(a_2) \to a_1 = a_2$. 证明一个函数不是单射函数, 就可以这样写: $\red{\exists} a_1, a_2 \in A.\; a_1 \neq a_2 \land f(a_1) = f(a_2)$. 

\begin{definition}[Surjective (onto) 满射函数]
    \[
      f: A \to B \qquad {f: A \red{\;\twoheadrightarrow\;} B}
    \]
    \[
      \ran(f) = B
    \]
\end{definition}

同样的, 对于证明给定的函数是满射而言, 我们可以这样写: $\forall b \in B. \; \big(\red{\exists} a \in A.\; f(a) = b \big)$, 反之, 我们可以这样写: $\red{\exists} b \in B.\; \big(\red{\forall} a \in A.\; f(a) \neq b \big)$. 

既是双射又是满射的函数一定很特殊, 因为它有一个一一对应的关系. 因此我们给出如下定义: 

\begin{definition}[Bijective (one-to-one correspondence) 双射; 一一对应]
    \[
      f: A \to B \qquad {f: A \red{\;\xleftrightarrow[onto]{1-1}\;} B}
    \]
    \begin{center}
      {1-1 \& onto}
    \end{center}
\end{definition}

那么, 一个集合和它的幂集之间可不可以找到一个满射呢? 其实是不行的. Cantor给出了一个证明.

\begin{theorem}[Cantor Theorem]
    If $f: A \to 2^{A}$, then $f$ is \red{not} onto.
\end{theorem}

\begin{proof}
    Let $A$ be the set and let $f:A\rightarrow 2^A$. To show that $f$ is not onto, we must find $B\in 2^A(i.e. B\subseteq A )$ for which there is no $a\in A$ with no $f(a)=B$. In other words, $B$ is a set that $f$ ``misses''. To this end, let 
    $$
    B={x\in A | x \notin f(x)}
    $$
    We claim there is no $a\in A $ with $f(a)=B$. 
    Suppose, for the sake of contradiction, there is an $a\in A$ such that $f(a)=B$, we pounder: Is $a\in B$?
    \begin{itemize}
        \item if $a\in B$, then, since $B=f(a)$, we have $a\in f(a)$. So by the definition of $B$ , $a\notin f(a)$. that is $a\notin B$. Contradiction!
        \item If $a\notin B=f(a)$, then by the definition of $B$, $a\in B$. Contradiction!  
    \end{itemize} 
    To sum up, it can't be onto. 
\end{proof}

除了反证之外, 还有一个构造性的证明, 我们一并给出. 

\begin{proof}[对角线论证 (Cantor's diagonal argument){(以下仅适用于可数集合 $A$)}]
  \begin{center}[]
    \begin{tabular}{|c||c|c|c|c|c|c|}
      \hline
      $a$      & \multicolumn{6}{c|}{$f(a)$} \\ \hline
            & 1      & 2      & 3      & 4      & 5      & $\cdots$ \\ \hline \hline
      1      & 1      & 1      & 0      & 0      & 1      & $\cdots$ \\ \hline
      2      & 0      & 0  & 0      & 0      & 0      & $\cdots$ \\ \hline
      3      & 1      & 0      & 0      & 1      & 0      & $\cdots$ \\ \hline
      4      & 1      & 1      & 1      & 0      & 1      & $\cdots$ \\ \hline
      5      & 0      & 1      & 0      & 1      & 0      & $\cdots$ \\ \hline
      $\vdots$ & $\vdots$ & $\vdots$ & $\vdots$ & $\vdots$ & $\vdots$ & $\cdots$ \\ \hline
    \end{tabular}
  \end{center}
\end{proof}

\subsection{作为关系的函数}

\subsubsection{函数的限制}
和关系一样, 函数也有限制等操作. 我们来看看. 

\begin{definition}[Restriction]
  The \red{\it restriction} of a function $f: A \to B$ to $X$
  is the \blue{function}:

  \[
    f|_{X} = \set{(x, y) \in f \mid \purple{x \in X}}
  \]
\end{definition}

注意$X \subseteq A$并不是必要的(虽然平时经常这样用). 

\subsubsection{像和逆像}

\begin{definition}[像 (Image)]
  The \red{\it image} of $X$ under a function $f: A \to B$ is the set
  \[
    {f(X) = \set{y \mid \exists \purple{x \in X}.\; (x, y) \in f}}
  \]
\end{definition}

同样, $X \subseteq \dom(f) = A$也不是必要条件, 尽管通常是这样的. 记号层面, $ f(\set{a}) = \set{b} \quad \text{简记为} \quad f(a) = b$. 

也就是$y \in f(X) \iff \exists x \in X.\; y = f(x)$.

\begin{definition}[逆像 (Inverse Image)]
  The \red{\it inverse image} of $Y$ under a function $f: A \to B$ is the set
  \[
    {f^{-1}(Y) = \set{x \mid \exists \purple{y \in Y}.\; (x, y) \in f}}
  \]
\end{definition}

注意不一定要$Y \subseteq \ran(f)$, 但是很多情况都是满足这样的. 但是注意
\[
    f^{-1}(\set{b}) = \set{a}
      \quad\text{\blue{可简记为}}\quad f^{-1}(b) = \set{a}
      \quad\text{\red{不能简记为}}\quad f^{-1}(b) = a
\]
想一想, 为什么会这样? 

\begin{definition}[逆像 (Inverse Image)]
  The \red{\it inverse image} of $Y$ under a function $f: A \to B$ is the set
  \[
    {f^{-1}(Y) = \set{x \mid \exists \purple{y \in Y}.\; (x, y) \in f}}
  \]
\end{definition}

这样一来, 我们就有这样的关系 :$y \in f(X) \iff \exists x \in X.\; y = f(x), x \in f^{-1}(Y) \iff f(x) \in Y$. 

需要注意的是, 如果有$f: a\rightarrow b$, $a \in A_{0} \red{\;\centernot\implies\;} f(a) \in f(A_{0})$, 这个式子才可以成立: $a \in A_{0} \purple{\;\cap\; A} \blue{\;\implies\;} f(a) \in f(A_{0})$.

关于求逆也有很多的性质. 很多时候我们可能会想当然的误用. 所以使用之前一定要小心, 小心, 再小心. 

幸运的是, 它还保留有很多性质. 我们一一罗列, 并给出一些证明. 

\begin{theorem}[Properties of $f$ and $f^{-1}$]
  \[
    f: A \to B \qquad
    {A_1, A_2 \subseteq A,\; B_1, B_2 \subseteq B}
  \]

  \begin{enumerate}
    \item \purple{$f$ preserves only $\subseteq$ and $\cup$:}
      \begin{enumerate}
        \item $A_1 \subseteq A_2 \implies f(A_1) \subseteq f(A_2)$
        \item \teal{$f(A_1 \cup A_2) = f(A_1) \cup f(A_2)$}
        \item $f(A_1 \cap A_2) \subseteq f(A_1) \cap f(A_2)$
        \item $f(A_1 \setminus A_2) \supseteq f(A_1) \setminus f(A_2)$
      \end{enumerate}
    \item \purple{$f^{-1}$ preserves $\subseteq$, $\cup$, $\cap$, and $\setminus$:}\begin{enumerate}
      \item $B_1 \subseteq B_2 \implies f^{-1}(B_1) \subseteq f^{-1}(B_2)$
      \item \teal{$f^{-1}(B_1 \cup B_2) = f^{-1}(B_1) \cup f^{-1}(B_2)$}
      \item $f^{-1}(B_1 \cap B_2) = f^{-1}(B_1) \cap f^{-1}(B_2)$
      \item \teal{$f^{-1}(B_1 \setminus B_2) = f^{-1}(B_1) \setminus f^{-1}(B_2)$}
    \end{enumerate}
  \end{enumerate}
\end{theorem}

对于$A_1 \subseteq A_2 \implies f(A_1) \subseteq f(A_2)$, 证明如下: 

\begin{proof}
  \setcounter{equation}{0}
  \begin{align}
    &b \in f(A_{1}) \\
    \iff & \exists a \in A_{1}.\; b = f(a) \\
    \red{\implies} & \exists a \in A_{2}.\; b = f(a) \\
    \iff & b \in f(A_{2})
  \end{align}
\end{proof}

对于$f(A_1 \cap A_2) \red{\;\subseteq\;} f(A_1) \cap f(A_2)$, 证明如下. 注意是哪一步变换, 使得它的箭头方向变为单向了, 为什么? 

\begin{proof}
  \red{对任意 $b$,}
  \setcounter{equation}{0}
  \begin{align}
    &b \in f(A_1 \cap A_2) \\[6pt]
    \iff & \exists a \in A_1 \cap A_2.\; b = f(a) \\
    \red{\implies}& \big(\exists a \in A_{1}.\; b = f(a)\big)
      \land \big(\exists a \in A_{2}.\; b = f(a)\big) \\
    \iff & b \in f(A_{1}) \land b = f(A_{2}) \\
    \iff & b \in f(A_1) \cap f(A_2)
  \end{align}
\end{proof}

对于$f(A_1 \setminus A_2) \red{\;\supseteq\;} f(A_1) \setminus f(A_2)$ : 

\begin{proof}
  \red{对任意 $b$,}
  \setcounter{equation}{0}
  \begin{align}
    &b \in f(A_{1}) \setminus f(A_{2}) \\[6pt]
    \iff & b \in f(A_{1}) \land b \notin f(A_{2}) \\
    \iff & (\exists a_{1} \in A_{1}.\; b = f(a_{1})) \land
      (\forall a_{2} \in A_{2}.\; b \neq f(a_{2})) \\
    \red{\implies} & \exists a \in A_{1} \setminus A_{2}.\; b = f(a) \\
    \iff & b \in f(A_{1} \setminus A_{2})
  \end{align}
\end{proof}

对于$B_1 \subseteq B_2 \implies f^{-1}(B_1) \subseteq f^{-1}(B_2)$, 证明如下:

\begin{proof}
  \red{对任意 $a$,}
  \setcounter{equation}{0}
  \begin{align}
    &a \in f^{-1}(B_{1}) \\
    \iff & f(a) \in B_{1} \\
    \red{\implies} & f(a) \in B_{2} \\[6pt]
    \iff & a \in f^{-1}(B_{2})
  \end{align}
\end{proof}

对于$f^{-1}(B_1 \cap B_2) = f^{-1}(B_1) \cap f^{-1}(B_2)$, 证明如下: 

\begin{proof}
  \red{对任意 $a$,}
  \setcounter{equation}{0}
  \begin{align}
    &a \in f^{-1}(B_{1} \cap B_{2}) \\
    \iff & f(a) \in B_{1} \cap B_{2} \\
    \iff & f(a) \in B_{1} \land f(a) \in B_{2} \\
    \iff & a \in f^{-1}(B_{1}) \land a \in f^{-1}(B_{2}) \\
    \iff & a \in f^{-1}(B_{1}) \cap f^{-1}(B_{2})
  \end{align}
\end{proof}

对于$A_0 \subseteq A \implies A_0 \subseteq f^{-1}(f(A_0))$, 有
\begin{proof}
  \red{对任意 $b$,}
  \setcounter{equation}{0}
  \begin{align}
    &a \in A_{0} \\
    \implies & a \in A_{0} \purple{\;\cap\; A} \\
    \red{\implies} & f(a) \in f(A_{0}) \\
    \iff & a \in f^{-1}(f(A_{0}))
  \end{align}
\end{proof}

对于$B_0 \supseteq f(f^{-1}(B_0))$, 思考什么时候这个条件是充要的($\iff$)?

\begin{proof}
  \red{对任意 $b$,}
  \setcounter{equation}{0}
  \begin{align}
    &b \in f(f^{-1}(B_0)) \\
    \iff & \exists a \in f^{-1}(B_0).\; b = f(a) \\[6pt]
    \iff & \exists a \in A.\; f(a) \in B_0 \land b = f(a) \\[6pt]
    \red{\implies} & b \in B_0
  \end{align}
  ``iff'' when $f$ is surjective and $$B_0 \subseteq \ran(f)$$.
\end{proof}

\subsubsection{函数的复合}
作为化简单为复杂的利器, 和关系一样, 函数也有复合. 

\begin{definition}[Composition]
  \[
    f: A \to B \qquad g: C \to D
  \]
  \[
    \red{\ran(f) \subseteq C}
  \]

  The \red{\it composite function} \blue{$g \circ f: A \to D$} is defined as

  \[
    (g \circ f) (x) = g(f(x))
  \]
\end{definition}

\begin{bonus}
  回顾关系复合的定义: 
  The {\it composition} of relations $R$ and $S$ is the relation
    \[
      R \circ S = \set{(a, c) \mid \red{\exists b}: (a, b) \in S \land (b, c) \in R}
    \]

  和函数的有什么不同? 为什么是存在? 
\end{bonus}

\begin{theorem}[Associative Property for Composition]
  \[
    f: A \to B \quad g: B \to C \quad h: C \to D
  \]

  \[
    h \circ (g \circ f) = (h \circ g) \circ f
  \]
\end{theorem}

\begin{proof}
  我们只需证明: 
  \begin{enumerate}
    \item
      \[
        \dom(h \circ (g \circ f)) = \dom((h \circ g) \circ f)
      \]
    \item
      \[
        \forall x \in A.\; (h \circ (g \circ f))(x) = ((h \circ g) \circ f)(x)
      \]
  \end{enumerate}


对于$(h \circ (g \circ f))(x) = ((h \circ g) \circ f)(x)$:


  \setcounter{equation}{0}
      \begin{align}
        &(h \circ (g \circ f))(x) \\[6pt]
        =\; &h((g \circ f) (x)) \\[6pt]
        =\; &h(g(f(x)))
      \end{align}
      \setcounter{equation}{0}
      \begin{align}
        &((h \circ g) \circ f)(x) \\[6pt]
        =\; &((h \circ g) (f(x))) \\[6pt]
        =\; &h(g(f(x)))
      \end{align}
\end{proof}

\begin{theorem}
  $f: A \to B \qquad g: B \to C$, 
  \begin{itemize}
    \item If $f, g$ are injective, then $g \circ f$ is injective.
    \item \teal{If $f, g$ are surjective, then $g \circ f$ is surjective.}
    \item If $f, g$ are bijective, then $g \circ f$ is bijective.
  \end{itemize}
\end{theorem}

对于第一条, 我们写出``injective''的定义, 然后完成逻辑推演. 

\begin{proof}
  $$\forall a_1, a_2 \in A.\;
      \Big( (g \circ f)(a_1) = (g \circ f)(a_2) \to a_1 = a_2 \Big)$$

      \begin{align}
        &(g \circ f)(a_1) = (g \circ f)(a_2) \\
        \iff &g(f(a_{1})) = g(f(a_{2})) \\[6pt]
        \implies &f(a_{1}) = f(a_{2}) \\[6pt]
        \implies &a_{1} = a_{2}
      \end{align}


\end{proof}

对于第二条, 我们写出``surjective''的定义. 


\begin{proof}
  $$\forall c \in C.\; \Big( \exists a \in A.\; (g \circ f)(a) = c \Big)$$
\end{proof}

对于第三条, 如出一辙. 

\begin{proof}
  \red{对任意 $a_{1}, a_{2}$,}
  \setcounter{equation}{0}
  \begin{align}
    &f(a_{1}) = f(a_{2}) \\
    \implies & g(f(a_{1})) = g(f(a_{2})) \\
    \implies & (g \circ f) (a_{1}) = (g \circ f) (a_{2}) \\[6pt]
    \implies & a_{1} = a_{2}
  \end{align}
\end{proof}

\begin{theorem}[]
  \[
    f: A \to B \qquad g: B \to C
  \]

  \begin{enumerate}
    \item \teal{If $g \circ f$ is injective, then $f$ is injective.}
    \item If $g \circ f$ is surjective, then $g$ is surjective.
  \end{enumerate}
\end{theorem}
因为(1)和(2)很像, 因此只证明(2). 注意充要条件是在哪一步消失的. 
\begin{proof}
  \red{对任意 $a_{1}, a_{2}$,}
  \setcounter{equation}{0}
  \begin{align}
    & g \circ f \text{ is surjective}\\
    \iff & \forall c \in C.\; \exists a \in A.\; (g \circ f)(a) = c \\
    \iff & \forall c \in C.\; \exists a \in A.\; g(f(a)) = c \\
    \red{\implies} & \forall c \in C.\; \exists b \in B.\; g(b) = c \\
    \iff & g \text{ is surjective}
  \end{align}
\end{proof}

\subsubsection{反函数}

什么时候有反函数? 反函数具有哪些性质? 在高中的时候老师可能不会和我们讲, 现在我们来探索一下反函数的性质. 

\begin{definition}[反函数 (Inverse Function)]
  Let $f: A \to B$ be a function. 

  The \red{\it inverse} of $f$ is a \teal{function}
  from $B$ to $A$, denoted \blue{$f^{-1}: B \to A$} \\
  if $f$ is bijective. \\

  We call $f^{-1}$ \blue{the} \red{\it inverse function} of $f$.
\end{definition}


\begin{definition}[Invertible]
      $f: X \to Y$ is {\it invertible} if there exists $g: Y \to X$ such that
  
      \[
        f(x) = y \iff g(y) = x.
      \]
\end{definition}

下面这个定理展示了什么时候具有反函数. 
\begin{theorem}
  $f$ is invertible $\iff$ $f$ is bijective.
\end{theorem}

\begin{theorem}
  Suppose that $f: A \to B$ is bijective.
  Then, its inverse function $f^{-1}: B \to A$ is unique.
\end{theorem}

\begin{proof}
  By contradiction. Omitted.
\end{proof}

\begin{theorem}[]
  \[
    \red{\boxed{f: A \to B \text{ is bijective}}}
  \]

  \begin{enumerate}
    \item $f \circ f^{-1} = I_B$
    \item $f^{-1} \circ f = I_A$
    \vspace{0.30cm}
    \item \teal{$f^{-1} \text{ is bijective}$}
    \vspace{0.30cm}
    \item $g: B \to A \land f \circ g = I_B \implies g = f^{-1}$
    \item $g: B \to A \land g \circ f = I_A \implies g = f^{-1}$
  \end{enumerate}
\end{theorem}

这些定理是帮助我们找到反函数/说明反函数不存在的一些好的结论.

对于(1). 
\begin{proof}
  \red{对任意 $b \in B$,}
  \[
    (f \circ f^{-1})(b) = f(f^{-1}(b))
  \]

    Suppose that $a = f^{-1}(b)$
    \[
      \red{\boxed{a = f^{-1}(b) \iff f(a) = b}}
    \]
    \[
    (f \circ f^{-1})(b) = f(f^{-1}(b)) = \red{f(a)} = b
  \]
\end{proof}

对于(2).
\begin{proof}
  $g = \purple{(f^{-1} \circ f) \circ g = f^{-1} \circ (f \circ g) = f^{-1} \circ I_B} = f^{-1}$
\end{proof}

我们当然可以看一看反函数的复合是怎样的一个情况. \footnote{理由不充分, 需要补充证明}

\begin{theorem}[Inverse of Composition]
  \[
    \text{Both } f: A \to B \text{ and } g: B \to C \text{ are bijective}
  \]

  \begin{enumerate}
    \item $g \circ f \text{ is bijective}$
    \item $(g \circ f)^{-1} = f^{-1} \circ g^{-1}$
    \item $f \circ g = I_B \land g \circ f = I_A \implies g = f^{-1}$
  \end{enumerate}
\end{theorem}

那么, 我们就从集合论的角度构建了我们高中学习过的内容. 

