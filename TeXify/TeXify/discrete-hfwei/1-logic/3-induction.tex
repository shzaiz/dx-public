\begin{theorem}[第一数学归纳法 (The First Mathematical Induction)]
    设 $P(n)$ 是关于自然数的一个性质. 
    如果
    \begin{enumerate}
      \item $P(0)$ 成立;
      \item 对任意自然数 $n$, 如果 $P(n)$ 成立, 则 $P(n+1)$ 成立. 
    \end{enumerate}
    那么, $P(n)$ 对所有自然数 $n$ 都成立. 
\end{theorem}

翻译成形式化的方法, 也就是
  \[
    \nd{P(0) \qquad \forall n \in \N.\; \Big(P(n) \to P(n+1) \Big)}{
      \forall n \in \N.\; P(n)}{\text{第一数学归纳法}}
  \]

  \[
    \biggl(P(0) \land \forall n \in \N.\; \Big(P(n) \to P(n+1) \Big) \biggr)
	  \to \forall n \in \N.\; P(n).
  \]

  \begin{theorem}[第二数学归纳法 (The Second Mathematical Induction)]
    设 $Q(n)$ 是关于自然数的一个性质. 
    如果
    \begin{enumerate}
      \item $Q(0)$ 成立;
      \item 对任意自然数 $n$, 如果 $Q(0), Q(1), \dots, Q(n)$ 都成立, \\ 则 $Q(n+1)$ 成立. 
    \end{enumerate}
    那么, $Q(n)$ 对所有自然数 $n$ 都成立. 
  \end{theorem}
  同样的, 翻译成形式化的表示形式, 也就是
  \[
    \nd{Q(0) \;\; \forall n \in \N.\; \Big(\big(Q(0) \land \dots \land Q(n)\big) \to Q(n+1) \Big)}{
      \forall n \in \N.\; Q(n)}{\text{第二数学归纳法}}
  \]
  \[
    \biggl(Q(0) \land \forall n \in \N.\; \Big(\big(Q(0) \land \cdots \land Q(n)\big) \to Q(n+1) \Big) \biggr)
      \to \forall n \in \N.\; Q(n).
  \]

  \begin{theorem}[数学归纳法]
    第一数学归纳法与第二数学归纳法等价. 
  \end{theorem}
  第二数学归纳法也被称为\red{\bf ``强'' (Strong)} 数学归纳法, 它强在可以使用的条件更多了. 我们可以来证明这件事情. 

  \begin{lemma}
    第一数学归纳法蕴含第二数学归纳法. 
  \end{lemma} 
  \begin{proof}
    要证第二类数学归纳法,也即任给一个命题  $F$  ,若满足  $F(1)$  及  $(F(1) \wedge F(2) \wedge \cdots \wedge F(n)) \Rightarrow F(n+1)$  ,则有  $\forall k \in \mathbb{N} . F(k)$  . 
    那么,我们可以构造命题  $G(n):=F(1) \wedge F(2) \wedge \cdots \wedge F(n)$  . 
    显然,  $G(n) \Rightarrow F(n+1)$  ,又有  $G(n) \Rightarrow G(n)$  ,则  $G(n) \Rightarrow(F(n+1) \wedge G(n))$  ,而 后者即为  $G(n+1)$  . 
    故,命题  $G$  满足第一类数学归纳法的条件,所以  $\forall k \in \mathbb{N}$ . $G(k)$  成立. 
    而  $G(k) \Rightarrow F(k)$  ,故  $\forall k \in \mathbb{N}$ . $F(k)$  ,也即第二类数学归纳法成立. 
  \end{proof}
  \begin{lemma}
    第二数学归纳法蕴含第一数学归纳法. 
  \end{lemma} 
  \begin{proof}
    要证第一类数学归纳法,也即任给一个命题  $F$  ,若满足  $F(1)$  及  $F(n) \rightarrow F(n+1)$  ,则有  $\forall k \in \mathbb{N}$ . $F(k)$  . 
    显然,  $F$  是满足第二类数学归纳法的条件的 (因为 1 的条件比 2 强),故根据第二类数学归纳法,  $F(k)$  对所有正整数  $k$  成立,也即第一类数学归纳法成立. 
  \end{proof}

数学归纳法的更深层次的结果是自然数的Peano公理. Peano 公理体系刻画了\red{\bf 自然数的递归结构}. 

\begin{definition}[Peano Axioms]
    自然数的Peano公理有如下几条: 
    \begin{enumerate}
      \item 0 是自然数;
      \item 如果 $n$ 是自然数, 则它的后继 ${\bf S}n$ 也是自然数;
      \item 0 不是任何自然数的后继;
      \item 两个自然数相等当且仅当它们的后继相等;
      \item \red{\bf 数学归纳原理:} 如果
        \begin{enumerate}
          \item $P(0)$ 成立;
          \item 对任意自然数 $n$, 如果 $P(n)$ 成立, 则 $P(n+1)$ 成立. 
        \end{enumerate}
        那么, $P(n)$ 对所有自然数 $n$ 都成立. 
    \end{enumerate}
\end{definition}

自然数集具有良序原理. 

\begin{definition}[良序原理 (The Well-Ordering Principle)]
    \red{自然数集}的任意\blue{非空}子集都有一个最小元. 
\end{definition}

\begin{theorem}{}
    良序原理与(第一)数学归纳法等价. 
\end{theorem}

\begin{lemma}
    (第一)数学归纳法蕴含良序原理. 
  \end{lemma}
  \begin{proof}
      \red{By mathematical induction on the size $n$ of non-empty subsets of $\mathbb{N}$.}

      \[
        P(n): \text{All subsets of size $n$ contain a minimum.}
      \]

      Inductive Hypothesis:
      \begin{itemize}
        \item Basis Step: $P(1)$
        \item\textcolor{cyan}{Inductive Hypothesis:} $P(n)$
        \item Inductive Step: $P(n) \to P(n+1)$
          
            \begin{itemize}
              \item $A' \gets A \setminus {a}$
              \item $x \gets \min A'$
              \item Compare $x$ with $a$
            \end{itemize}
          
      \end{itemize}
    \end{proof}

    \begin{example}
      Of the 1000 islanders,
    it turns out that \blue{\bf 100 of them have blue eyes}
    and \brown{\bf 900 of them have brown eyes},
    although the islanders are not initially aware of these statistics
    (each of them can of course only see 999 of the 1000 tribespeople). 

    One day, a \purple{\bf blue-eyed foreigner} visits to the island
    and wins the complete trust of the tribe. 

    One evening, he addresses the entire tribe to thank them
    for their hospitality. 

    However, not knowing the customs,
    the foreigner makes the mistake of mentioning eye color in his address,
    remarking ``\teal{\bf how unusual it is to see another blue-eyed person
    like myself in this region of the world''}. 

    \violet{\bf What effect, if anything, does this {\it faux pas (失礼)} have on the tribe?}
    \end{example}

    \begin{theorem}
      Suppose that the tribe had $n > 0$ blue-eyed people. 
      
      Then $n$ days after the traveller's address, all $n$ blue-eyed people commit suicide.
    \end{theorem}

    \begin{proof}
      \item[基础步骤:] $n = 1$. \\ 
        这个\blue{\bf 唯一的蓝眼人}的内心独白: \teal{\bf ``你直接念我身份证吧''} 
      \item[归纳假设:] 有$n$个蓝眼人时, 前 $n-1$ 天无人自杀, 第$n$天集体自杀. 
      \item[归纳步骤:] 考虑恰有 $n+1$ 个蓝眼人的情况.   \\
        每个\blue{\bf 蓝眼人}都如此推理: 我看到了 $n$ 个蓝眼人, 他们应该在第 $n$ 天集体自杀. 
         \\
        但是, 每个蓝眼人都在等其它$n$个蓝眼人自杀, 因此, 第 $n$ 天无人自杀. 
         \\
        每个\blue{\bf 蓝眼人}继续推理: 一定不止 $n$ 个蓝眼人, 但是我看到的其余人都不是蓝眼. 
         \\
        所以, \purple{\bf ``小丑竟是我自己''}. 
    \end{proof}

    这就像是考虑 $n = 1, n = 2$ 的简单情况, 出现了类似``\purple{我知道}\teal{你知道}\cyan{我知道} $\dots$''的思维递归情形. 
    