\subsection{公理体系}
在中学的时候, 我们定义的集合是如下的一个数学对象: \red{\bf 集合}就是任何一个\blue{有明确定义的}对象的\blue{整体}. 
\begin{definition}[集合]
    我们将\red{\bf 集合}理解为任何将\blue{我们思想中那些确定而彼此独立的对象}放在一起而形成的\blue{聚合}. 
\end{definition}

这也引出了概括原则: 

\begin{theorem}[概括原则]
    对于任意性质/谓词 $P(x)$, 都存在一个集合 $X$:
    \[
      X = \set{x \mid P(x)}
    \]
\end{theorem}

很多时候我们需要判别两个集合是不是相等, 那么我们有外延性原理: 
\begin{definition}[外延性原理 (Extensionality)]
    两个集合相等 $(A = B)$ 当且仅当它们包含相同的元素. 
    \[
      \forall A.\; \forall B.\;
        \Big(\big(\forall x.\; (x \in A \leftrightarrow x \in B)\big)
          \leftrightarrow A = B \Big)
    \]
\end{definition}

这条公理意味着集合这个对象完全由它的元素决定. 

有时候我们需要从一个集合里面抽出一部分, 也就是寻找一个集合的子集. 因此我们有如下的定义. 

\begin{definition}[子集]
    设 $A$、$B$ 是任意两个集合. 

    $A \subseteq B$ 表示 $A$ 是 $B$ 的\red{子集} (subset):
    \[
      A \subseteq B \iff \forall x \in A.\; (x \in A \to x \in B)
    \]

    $A \subset B$ 表示 $A$ 是 $B$ 的\red{真子集} (proper subset):
    \[
      A \subset B \iff A \subseteq B \land A \neq B
    \]
\end{definition}

我们还可以证明两个集合相等, 当二者互为对方的子集时候. 

\begin{theorem}
    两个集合相等当且仅当它们互为子集. 
    \[
      A = B \iff A \subseteq B \land B \subseteq A
    \]
\end{theorem}

\subsection{简单操作}

现在我们不妨把高中定义的文字性的内容重新定义一下: 

\begin{definition}[集合的并 (Union)]
    \[
      A \cup B \triangleq \set{ x \mid x \in A \lor x \in B}
    \]
\end{definition}

\begin{definition}[集合的交 (Intersection)]
    \[
      A \cap B \triangleq \set{x \mid x \in A \land x \in B}
    \]
\end{definition}

\begin{theorem}[分配律 (Distributive Law)]
    \[
      A \cup (B \cap C) = (A \cup B) \cap (A \cup C)
    \]
    \[
      \teal{A \cap (B \cup C) = (A \cap B) \cup (A \cap C)}
    \]
\end{theorem}

对于这样的命题, 我们同样给出证明. 

\begin{proof}
    对任意$x$,
    \begin{align}
        &x \in A \cup (B \cap C) \\
        \iff & (x \in A) \lor (x \in B \land x \in C) \\
        \red{\iff} & (x \in A \lor x \in B) \land (x \in A \lor x \in C) \\
        \iff & (x \in A \cup B) \land (x \in A \cup C) \\
        \iff & x \in (A \cup B) \cap (A \cup C)
      \end{align}
\end{proof}

同样, 像命题符号一样, 集合的运算也遵循吸收率: 

\begin{theorem}[吸收律 (Absorption Law)]
    \[
      A \cup (A \cap B) = A
    \]
    \[
      \teal{A \cap (A \cup B) = A}
    \]
\end{theorem}

\begin{proof}
    对任意$x$,
    \begin{align}
        &x \in A \cup (A \cap B) \\
        \iff & x \in A \lor (x \in A \land x \in B) \\
        \red{\iff} & x \in A
      \end{align}
\end{proof}

有了这个我们就可以使用这个证明一个比较重要的习题. 

\begin{theorem}
    \[
      A \subseteq B \iff A \cup B = B \;\purple{\iff A \cap B = A}
    \]
  \end{theorem}
\begin{proof}
    
  对任意 $x$
    
  \setcounter{equation}{0}
  \begin{align}
    &x \in B \\
    \red{\implies} &x \in A \lor x \in B \\
    \implies &x \in A \cup B
  \end{align}
\end{proof}

\begin{definition}[集合的差 (Set Difference); \blue{相对补} (Relative Complement)]
  \[
    A \setminus B = \set{x \mid x \in A \land x \notin B}
  \]
\end{definition}

\begin{definition}[绝对补 (Absolute Complement); \purple{$\overline{A}, A', A^{c}$}]
  \red{设全集为$U$. }
  \[
    \overline{A} = U \setminus A = \set{x \in U \mid x \notin A}
  \]

  期中, 全集 $U$ (Universe) 是当前正在考虑的所有元素构成的集合. 一般均默认存在. 注意: 不存在``包罗万象''的全集. 
\end{definition}

相对补和绝对补之间存在一些联系. 

\begin{theorem}[``相对补''与``绝对补''之间的关系]
  \red{设全集为 $U$. }
  \[
    \red{\boxed{A \setminus B = A \cap \overline{B}}}
  \]
\end{theorem}

\begin{proof}
  对任意 $x$,
  \setcounter{equation}{0}
  \begin{align}
    &x \in A \setminus B \\
    \iff & x \in A \land x \notin B \\
    \red{\iff} & x \in A \land (x \in U \land x \notin B) \\
    \iff & x \in A \land x \in \overline{B} \\
    \iff & x \in A \cap \overline{B}
  \end{align}
\end{proof}

\begin{theorem}[德摩根律 (绝对补)]
  \red{设全集为 $U$. }

  \[
    \overline{A \cup B} = \overline{A} \cap \overline{B}
  \]
  \[
    \teal{\overline{A \cap B} = \overline{A} \cup \overline{B}}
  \]
\end{theorem}
\begin{proof}
    对任意 $x$, 
  \setcounter{equation}{0}
  \begin{align}
    &x \in \overline{A \cup B} \\
    \iff & x \in U \land \lnot (x \in A \lor x \in B) \\
    \red{\iff} & x \in U \land x \notin A \land x \notin B \\
    \iff & (x \in U \land x \notin A) \land (x \in U \land x \notin B) \\
    \iff & x \in \overline{A} \land x \in \overline{B} \\
    \iff & x \in \overline{A} \cap \overline{B}
  \end{align}
\end{proof}

\begin{theorem}[德摩根律 (相对补)]
  \[
    C \setminus (A \cup B) = (C \setminus A) \cap (C \setminus B)
  \]
  \[
    \teal{C \setminus (A \cap B) = (C \setminus A) \cup (C \setminus B)}
  \]
\end{theorem}

\begin{proof}
  \setcounter{equation}{0}
  \begin{align}
    &C \setminus (A \cup B) \\
    \red{\iff} & C \cap \overline{A \cup B} \\
    \iff & C \cap (\overline{A} \cap \overline{B}) \\
    \iff & (C \cap \overline{A}) \cap (C \cap \overline{B}) \\
    \iff & (C \setminus A) \cap (C \setminus B)
  \end{align}
\end{proof}

由此, 我们可以在集合的操作的层面上证明如下四个定理而不需要取集合中的一个元素进行证明.  

\begin{theorem}
  \[
    A \cap (B \setminus C) = (A \cap B) \setminus C
                           = (A \cap B) \setminus (A \cap C)
  \]
  \[
    A \setminus (B \setminus C) = (A \cap C) \cup (A \setminus B)
  \]
  \[
      A \subseteq B \implies \overline{B} \subseteq \overline{A}
    \]
    \[
      \teal{A \subseteq B \implies (B \setminus A) \cup A = B}
    \]
\end{theorem}

这里面有一个类似一个异或操作的运算符: 对称差. 
\begin{definition}[对称差 (Symmetric Difference)]
  \[
    A \oplus B = (A \setminus B) \cup (B \setminus A)
      = (A \cap \overline{B}) \cup (B \cap \overline{A})
  \]
\end{definition}

\subsection{高级集合操作}

既然集合的对象是一组元素, 那么集合也是对象, 集合中的元素自然也可以被传进去看作运算. 由此, 我们需要定义关于集合的集合的运算. 

\begin{definition}[广义并 (Arbitrary Union)]
  设 $\mathbb{M}$ 是一组集合 (a {\it collection} of sets)
  \[
    \bigcup \mathbb{M} = \Bset{x \mid \red{\exists} A \in \mathbb{M}.\; x \in A}
  \]
\end{definition}

举一些例子, 比如$\mathbb{M} = \Bset{\set{1, 2}, \set{\set{1, 2}, 3}, \set{4, 5}}$, 那么$\bigcup \mathbb{M} = \Bset{1, 2, 3, 4, 5, \red{\set{1, 2}}}$. 注意元素只被解开了一次而不是一次解包到我们认为的``基本元素''. 因为有时候``基本元素''也是用集合定义的. 我们后来会发现我们可以把整个数学基础建立到集合论的基础上. 

和求和记号一样, 为了方便书写, 我们也有类似的记号: 
  \[
    \bigcup_{j = 1}^{n} A_j \triangleq A_1 \cup A_2 \cup \cdots \cup A_n
  \]
  \[
    \bigcup_{j = 1}^{\infty} A_j \triangleq A_1 \cup A_2 \cup \cdots
  \]
  \[
    \bigcup_{\blue{\alpha \in I}} A_{\alpha} \triangleq
      \Big\{x \mid \red{\exists} \alpha \in I.\; x \in A_{\alpha}\Big\}
  \]

和广义并一样, 我们还有广义交. 定义如下: 
\begin{definition}[广义交 (Arbitrary Intersection)]
  设 $\mathbb{M}$ 是一组集合 (a {\it collection} of sets)
  \[
    \bigcap \mathbb{M} = \Bset{x \mid \red{\forall} A \in \mathbb{M}.\; x \in A}
  \]
\end{definition}

同样的, 如果$\mathbb{M} = \Bset{\set{1, 2}, \set{\set{1, 2}, 3}, \set{4, 5}}$是全集,  $\bigcap \mathbb{M} = \emptyset$. 同样只是展开一次就行了. 注意一个有趣的情况: $\bigcap \emptyset =U$. ``包含所有元素的集合''在后面会发现会导出一个矛盾, 有时候我们也会认为这样说的结果是未定义的. 

那么类似的, 我们也希望广义集合里面有没有像普通集合的一些操作. 答案是肯定的. 下面我们来探讨一些有趣的内容. 

\begin{theorem}[德摩根律]
  \[
    X \setminus \bigcup_{\alpha \in I} A_{\alpha} = \bigcap_{\alpha \in I} (X \setminus A_{\alpha})
  \]

  \[
    \teal{X \setminus \bigcap_{\alpha \in I} A_{\alpha} = \bigcup_{\alpha \in I} (X \setminus A_{\alpha})}
  \]
\end{theorem}

\begin{proof}
  对任意 $x$,
  \setcounter{equation}{0}
  \begin{align}
    &x \in X \setminus \bigcup_{\alpha \in I} A_{\alpha} \\
    \iff & x \in X \land \lnot(\exists \alpha \in I.\; x \in A_{\alpha}) \\
    \iff & x \in X \land (\forall \alpha \in I.\; x \notin A_{\alpha}) \\
    \red{\iff} & \forall \alpha \in I.\; (x \in X \land x \notin A_{\alpha}) \\
    \iff & x \in \bigcap_{\alpha \in I} (X \setminus A_{\alpha})
  \end{align}
\end{proof}

我们同样可以用这条规律来化简集合, 而不用真正在一个集合的集合里面取出来一个元素. 

\begin{eg}
  如果
  \[
    X_n = \set{-n, -n+1, \cdots, 0, \cdots, n-1, n}
  \]
  请化简: 
  $$A = \mathbb{R} \setminus \bigcap_{n \in \mathbb{Z}^{+}} (\mathbb{R} \setminus X_n) $$
\end{eg}
\begin{proof}
  \begin{align*}
    A &= \mathbb{R} \setminus \bigcap_{n \in \mathbb{Z}^{+}} (\mathbb{R} \setminus X_n) \\
    &= \mathbb{R} \setminus \Big(\mathbb{R} \setminus \bigcup_{n \in \mathbb{Z}^{+}} X_n \Big) \\
    &= \mathbb{R} \setminus \Big(\mathbb{R} \setminus \mathbb{Z} \Big) \\
    &= \mathbb{Z}
  \end{align*}
\end{proof}

\subsection{集合的操作: 排列的力量}

在高中, 我们学习了排列组合. 如果对于集合中的元素进行``选择性缺席'', 这样就可以让我们构造出更加复杂而全面的集合了. 

\begin{definition}[幂集 (Powerset)]
  \[
    \ps{A} = \Bset{X \mid X \subseteq A}
  \]
\end{definition}

这个之所以强大, 是因为给定一个A, 就有如下的内容可以被生成. 
  \[
    A = \set{1, 2, 3}
  \]
  \[
    \ps{A} = \set{\emptyset,
      \set{1}, \set{2}, \set{3},
      \set{1, 2}, \set{1, 3}, \set{2, 3},
      \set{1, 2, 3}}
  \]

因为对于$|A| = n$的句子, $|\ps{A}| = 2^{n}$, 因此有时候也写做$2^{A}$或者$\set{0,1}^{A}$. 

接下来看一个(看似)没啥用的定理: 

\begin{theorem}
  $$S \in \ps{X} \iff S \subseteq X$$
\end{theorem}

这个定理的作用是在$\in$和$\subseteq$之间转换, 同时脱去一层$\ps{}$记号. 

\begin{eg}
  请证明: 
  $$\set{\emptyset, \set{\emptyset}} \in \ps{\ps{\ps{S}}}$$
\end{eg}

\begin{proof}
  根据上面的定理, 我们有
  $$\set{\emptyset, \set{\emptyset}} \in \ps{\ps{\ps{S}}} \iff \set{\emptyset, \set{\emptyset}} \subseteq \ps{\ps{S}}.$$

  分别证明之: 
  \begin{align*}
    &\red{\emptyset \in \ps{\ps{S}}}\\
    &\iff \emptyset \subseteq \ps{S}
  \end{align*}

  \begin{align*}
    &\red{\set{\emptyset} \in \ps{\ps{S}}}\\
    &\iff \set{\emptyset} \subseteq \ps{S}\\
    &\iff \emptyset \in \ps{S}\\
    &\iff \emptyset \subseteq S
  \end{align*}

\end{proof}

其实幂集生成之间也有一些关系. 不妨看一看. 

\begin{theorem}
  证明: 
  \[
      \ps{A} \cap \ps{B} = \ps{A \cap B}
  \]
\end{theorem}

\begin{proof}
  对于任意 $x$,
  \begin{align*}
    &\textcolor{white}{\iff}\;\; x \in \ps{A} \cap \ps{B} \\
    &\iff x \in \ps{A} \land x \in \ps{B} \\
    &\iff x \subseteq A \land x \subseteq B \\
    &\iff x \subseteq A \cap B \\
    &\iff x \in \ps{A \cap B}
  \end{align*}
\end{proof}

\begin{theorem}
  证明: 
  \[
    \bigcap_{\alpha \in I} \ps{A_{\alpha}} = \ps{\bigcap_{\alpha \in I} A_{\alpha}}
  \]
\end{theorem}

\begin{proof}
  对于任意 $x$,
  \begin{align*}
    &\textcolor{white}{\iff}\;\; x \in \bigcap_{\alpha \in I} \ps{A_{\alpha}} \\
    &\iff \forall \alpha \in I.\; x \in \ps{A_{\alpha}} \\
    &\iff \forall \alpha \in I.\; x \subseteq A_{\alpha} \\
    &\iff x \subseteq \bigcap_{\alpha \in I} A_{\alpha} \\
    &\iff x \in \ps{\bigcap_{\alpha \in I} A_{\alpha}}
  \end{align*}
\end{proof}

\subsection{悖论的出现}

前面我们提到``不存在含有任何东西的集合''. 这就是我们以前知道的通俗讲述的``理发师悖论''. 形式化的, 根据概括原则, 如果性质$P$是$P(x) \triangleq ``x \notin x"$, 集合$R = \set{x \mid x \notin x}$, 那么$R \in R\;$吗? 

\begin{quote}
  ``悖论出现于数学的边界上, 而且是靠近哲学的边界上'' \\
  \hfill --- 哥德尔
\end{quote}

之后, 数学家们提出了ZF(ZFC)公理化集合论, 避免了这样的内容. 通过粗暴的避免了这种情况, 我们得到了一个还可以使用, 但是丧失了一部分确定性的集合. 

\begin{theorem}[Russell's Paradox]
  \[
    \set{x \mid x \notin x} \text{ is \red{\it not} a set.}
  \]
\end{theorem}