\section{问题的提出: 为什么要研究命题的逻辑}

\begin{dialogue}
	A: 问一个问题: 如果我有一个命题叫做``若$p$则$q$'', 那么这个命题的否定是什么?
	
	B: 我研究这个干什么? 闲着找事情吗? 
	
	A: 想一想你学习的极限理论. 如果我希望对于``一个数列的极限存在''这个事情做否定, 你会怎样否定? 
	
	B: 数列极限的定义是如果一个数列的极限是 $A$ , 那么就是说$\forall n>N, \exists |\epsilon|\geq 0, \text{s.t.} |a_n-A|<\epsilon$. 要是否定还是真的不是一件容易的事情啊...
	
	A: 这就是我们学习命题逻辑的原因. 以后我们会遇见成百上千的命题等待我们的操作, 如何从中找到逻辑就至关重要. 
\end{dialogue}

事实上, 对于命题逻辑的研究一开始肯定是在数学中接受的. 但是对于数学而言, 我们当中的很多人会发现: 数学仅仅是为了对付高考这样的考试. 那么希望在这一节以及以后的生活中, 慢慢体会数学带给我们的潜移默化的影响. 

这一部分我们建议参看教科书《Reading, Writing, and Proving A Closer Look at Mathematics》的第一章. 我们会在这一章主要概括一下它的主要意思. 由于是为了本科生写的数学课本, 所以句子十分的容易懂. 就当做一个小练习吧. 

\begin{prob}
	结合自己的数学学习经历, 阅读《Reading, Writing, and Proving A Closer Look at Mathematics》的第一章, 然后与下文进行比对. 看一看自己的英语理解能力如何. 不设置时间限制, 因为我们需要做的是尽可能的联系自己过去的数学学习经历, 然后去体会这段文本. 
\end{prob}


我们在生活中经常看到这样的对话: 
\begin{dialogue}
	学那么多的数学有什么用? 买菜又用不到这样的数学, 学这些还有用吗? 
\end{dialogue}

再后来, 我们发现所有的数学问题都可以通过一种``程序化''的手段来解决. 比如, 我们在上一学期学习的Gram-Schmidt正交化矩阵向量的基、解其次线性方程组、求导数等等这样的操作, 都有一系列的明确的步骤. 

那么数学仅仅是局限于此吗? 我们来看一看那些伟大的数学家的思想是什么样的: 

伟大的数学家、教育学家George Polya专门出了一本书叫做``如何解题''. 于是, 学习数学一个很重要的目的可能就是教会我们:

(1) 如何解决一个问题;

(2) 为什么这样做是对的;

(3) 这个方法什么时候是对的. 

\begin{example}
	我们是如何解含有未知数的等式(通常叫为方程), 其中一个比较重要的方法是消去律. 
	
	对于实数构成的方程, 消去律大多数都是成立的(只要等式两端不除以0), 但是对于含有未知矩阵的方程, 这样的方法很多时候就不灵了. 
\end{example}

在我们遇到一个难以解答的数学问题的时候, 还是回过头来看看George Polya为我们总结的How to solve it的一个list吧. 

% TODO: 添加Polya的How to solve it

在解答完这些问题之后, 我们往往会感到满足. 很多时候这也是我们去学习数学的一个很重要的原因. 

\begin{dialogue}
A: 可我一点感觉开心也没有啊! 

B: 可能是把做题看得太重了. 高考的``把题目作对''的观念在大学里面就应该淡化掉了. 

A: 此话怎讲?

B: 来看一看朱富海老师的文章就知道了. 

\end{dialogue}

\begin{pas}
	\begin{center}
		\large \textbf{高中数学与大学数学}
	\end{center}
	\begin{center}
		南京大学~朱富海~~节选自数林广记微信公众号
	\end{center}
	
	美国大学的数学研究者们对于学生包括中学生的培养的确非常有热情, 比如一些名校的博士生在暑假期间常常有打工的机会, 主要任务是指导一些高中生尝试做科研. 2011 年, MIT 的 Pavel Etingof 教授与另外六位作者合作出版了一本书, 题目是 Introduction to Representation Theory.
	
	这本书的内容包括代数、有限群、quiver(箭图)表示论, 以及范畴论和有限维代数结构理论, 其中的大部分内容在国内高校数学院系的本科甚至研究生课程中都讲不到. 在 Etingof 的主页可以找到这本书的 PDF 文档. 他在前言中说, 这本书是他在 2004 年给其他六位合作者的授课讲稿, 而这六位听众当时都是高中生! 其中的 Tiankai Liu 应该是华人, 在 2001, 2002, 2004 年三次代表美国队参加国际数学奥林匹克都获得金牌. 还有一位合作者是来自 South Eugene 高中的 Dmitry Vaintrob, 他在 2006 年获得面向高中生的 Siemens 竞赛的第一名, 论文题目是 The string topology BV algebra, Hochschild cohomology and the Goldman bracket on surfaces, 论文已经涉及到很深的数学理论, 在 Dmitry Vaintrob 的主页上也能找到.
	
	再看看我们在做什么? 曾经看过一道竞赛训练题, 其本质是把八位数19101112(华罗庚先生的诞生日)分解质因数. 很容易找到因数 8, 然后就一筹莫展了. 后来借助网络工具才直到 19101112 = 8×1163×2053. 看到结果有点傻眼了: 有谁能只用纸笔得到这个分解? 后来发现自己孤陋寡闻了, 有学生说这种分解质因数早就背过! 细细一想真的极为恐怖: 他们为什么要背这个? 他们又背了多少类似的东西?
	
	想想挺有意思: 杰出的数学家们用他们的智慧和汗水去探索和展现数学之美, 而我们花费了大量时间和脑细胞记忆一些很容易遗忘的意义不大的知识点, 轻轻松松地毁掉数学之美的同时顺便浇灭了学生们的求知欲.

\end{pas}

\begin{dialogue}
\begin{center}
(...对话仍在继续...)	
\end{center}
	B: 所以嘛, 我们只要把高考带来的陋习去除掉就行了. 也就是所谓的``去高考化''.
	
	A: 听起来确实很有希望. 我们终于不用再整天因为分数担惊受怕了.
	
	B: 是的, 但是看起来我们都是在这份讲稿里面存在的人物. 希望我们的存在能够对现实世界的你有一定的帮助吧. 
\end{dialogue}

同样, 顺着南京大学的问题求解课程, 我们同样找到了一本很有趣的书: 《Mathematics: A Discrete Introduction, Second Edition. Edward R. Scheinerman》. 这本书里面详细讲述了我们为什么要学习数学, 以及数学学习带来的享受. 

\begin{prob}
	和上面的问题一样, 带入自己之前的数学学习经历, 然后认真体会这本书写的内容. 只需要阅读第章的前五个小节就好了. 
\end{prob}

\begin{dialogue}
	A: 为什么不让我读第六个小节?
	
	B: 这是因为嘛... 第六小节就是我们下一节课的内容了.
	
	A: 太好了, 我要预习! 
	
	B: 这时候终于知道了高中老师说的``预习''的重要性了吧!
	
	A: 确实, 这样一来确实切身感受到了预习的重要性. 这样做可以帮助我了解我的理解哪里出了问题, 于是就可以更加准确地向老师发问, 而不必纠缠于那些可有可无的奇怪问题了. 
	
\end{dialogue}

我们先来看存在于大学数学课本的很多重要的元素和栏目. 

\ti{定义. }定义的结构通常形容其实, 很多情形下, 我们对于定义的理解是很需要时间的, 一般地, 我们需要关注:

\begin{idea}
	对于数学定义, 我们需要关注如下的几个问题: 
	\begin{itemize}
		\item 这个定义是怎么来的? 有什么背景?
		\item 这个定义说的是什么?
		\item 我们能用更好的方法或不同的角度定义吗?
	\end{itemize}
\end{idea}

\ti{命题. }


























