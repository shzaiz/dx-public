\section{问题的提出: 为什么要研究命题的逻辑}

\begin{dialogue}
	A: 问一个问题: 如果我有一个命题叫做``若$p$则$q$'', 那么这个命题的否定是什么?
	
	B: 我研究这个干什么? 闲着找事情吗? 
	
	A: 想一想你学习的极限理论. 如果我希望对于``一个数列的极限存在''这个事情做否定, 你会怎样否定? 
	
	B: 数列极限的定义是如果一个数列的极限是 $A$ , 那么就是说$\forall n>N, \exists |\epsilon|\geq 0, \text{s.t.} |a_n-A|<\epsilon$. 要是否定还是真的不是一件容易的事情啊...
	
	A: 这就是我们学习命题逻辑的原因. 以后我们会遇见成百上千的命题等待我们的操作, 如何从中找到逻辑就至关重要. 
\end{dialogue}

事实上, 对于命题逻辑的研究一开始肯定是在数学中接受的. 但是对于数学而言, 我们当中的很多人会发现: 数学仅仅是为了对付高考这样的考试. 那么希望在这一节以及以后的生活中, 慢慢体会数学带给我们的潜移默化的影响. 

\begin{bonus}
	数学是什么? 仅仅是加减乘除吗? 仅仅是把老师教的计算方法演绎一遍吗? 
\end{bonus}

其实这并不是数学的全部. 