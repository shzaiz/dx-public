

    \section{一点闲聊}
    对于现在阅读这份文稿的你, 恭喜啊! 高考终于结束了! 数年的
    昼夜不停的做题当然是一个十分辛苦的事情. 辛苦你了!
    
    但是在做题间隔, 我们可能经常有这样一种疑问: ``做题有用吗? 这些知识都是合理的吗?" 如果你去问老师的话, 老师可能会告诉你高考会考这些知识, 不学好你将无法取得分数. 据我对大部分高考生的了解, 学习的``意义"就是考上一个好大学. 

    在那时, 你可能就被说服了, 于是机械的重复你本来很不愿意做(或者说感觉没啥用)的事情. 并且认为``高考"完了之后, 一切就轻松了, 我就可以撒开玩耍了, 学习就再也没有很重的价值了. 我们一会将会回到这个问题. 

    我不知道面前阅读这份文稿的你的心情, 有一种可能的情况是: 你为了在大学中继续取得好成绩而延续高中的那种``内卷"的行为习惯; 当然还有一种心情是因为好奇点开了这份文档. 

    在这一切开始之前, 我们先来简单来说点有趣的东西吧! 是关于推理相关的. 

    \begin{eg}
        若$p$则$q$的否定是什么?
    \end{eg}

    是``若$\neg p$则$\neg q$’’吗? 不是的, 这是它的逆否命题. 是``若$p$则$\neg q$吗?’’ 你怎么判定他们的等价关系? 

    如果原来的你认为``推理’’在数学中占很重要的一部分的话, 你会发现你对于稍微有一点复杂的推理关系还是没有一套成体系的处理方法. 

    可能你会说: 我只要会推断简单的问题就好了, 为什么要掌握成体系的方法呢? 就像刚刚的否定逻辑命题一样, 如果没有成体系的方法和策略, 你可能就会在很多不必要的地方上面本不应该浪费的时间, 从而无法迎接未来来自真实世界的挑战. 

    这时候你回顾你高中学过的东西, 有时候你会觉得他们确实存在着一套知识体系. 比如, 我们学习了函数, 也学习了方程, 同样也学习了集合论. 他们互相联系, 形成了一套我解题的时候联想到的一个内容. 

    诚然, 高中数学的知识之间有一定的关联性, 但是有时候这种数学并不是十分严格. 还有很多问题等待我们的解决. 比如, 如何证明$(\sin x)’=\cos x$? 集合$A=\{\alpha|\alpha\not \in A \}$算不算是集合? 有理数和实数哪个多? 

    这些内容高中的时候你可能思考过, 但是被``应试’’支配的恐惧很可能会让你没有勇气承认他们的实际用途. 事实上, 这些问题背后的观念是十分深刻的. 现在, 我们终于可以抛开老师在应试的指挥下强加给我们的念想, 任由我们自己探索这个世界. 

    ``自由探索’’说起来容易, 如果你真正做起来的话, 那就是另一回事了. 比如说, 拿到了一个数学文本, 你不知道人们为什么要把所有的求和简写成$\sum$, 自然难以理解他们在说什么. 或者说, 你不是很了解为什么要把一个个具体的内容``抽象’’成一个个的符号来研究…

    普通的高中毕业生经常会感觉到大学数学文本难以阅读. 就以《高等数学》一书为例, 大多数初学者会感到$\epsilon-\delta$语言难以理解. 这是因为我们的高中教育好像对于逻辑的理解不够深入——有时候也可以认为忙于应试而没有任何闲暇的时间来思考. 

    这时候, 你可能会发现, 自己受到的教育其实是\textbf{不完整}的. 

    其实, 要补上这些内容也远远没有那么困难. 我们可以找到一点想法, 让我们在体会到数学推演中隐藏的秩序和美感. 

    \section{高考: 我们做对了吗?}
    \begin{definition}
        本文中常用``卷''来代表``内卷''的简称. ``内卷''的定义是: 一类文化模式达到了某种最终的形态以后, 既没有办法稳定下来, 也没有办法转变为新的形态, 而只能不断地在内部变得更加复杂的现象. 
    \end{definition}

    下面的文本可能描述了你在为了卷分数而理解大学知识的一些问题. 
    \begin{pas}
    
        \begin{center}
            \large\textbf{高中数学与大学数学} \\
            \small \emph{朱富海}\qquad \emph{南京大学}
        \end{center}

        每次教大一的课程, 我都会在期中考试后让学生们写一个总结, 希望他们能够反思一下进入大学后的几个月的学习情况. 在日常教学过程中, 常常发现他们身上有太多应试教育的难以磨灭的痕迹, 这导致学生们明显不适应大学课堂. 或许通过自我反思他们能转变思维方式, 找到合适于自己的学习方法. 从学生的反馈看, 很多人上大学前对大学生活完全是陌生的, 上了几个月课, 觉得被大学骗了, 尤其是“被高中老师骗了”. 大学的宣传可能正能量偏多了一点, 而可能不止一位高中老师跟学生们说过: 你们苦过这三年, 上了大学就轻松了! 然而真正的大学生活似乎完全不是一回事, 当然不排除在某些大学或者某些专业是可能非常轻松的. 在这样的氛围里, 学生们表现出了各种能力的欠缺.

        第一种是自理和自控能力不足. 学生们高考结束后甚至是在获得保送资格之后就解脱了, 他们用包括撕书在内的各种举动来宣泄心中压抑已久的情绪, 如同一根弹簧被拉伸到弹性限度之外, 再也没有了弹性. 进入大学后, 很多学生对所学专业缺乏兴趣, 失去奋斗的目标, 关键是没有了来自老师家长的压力, 无法恢复到高中时的学习状态, 有经常打游戏度日的, 也不乏网吧的常驻人口. 在这个过程中, 来自家长、老师、辅导员或者同学的帮助都起不了作用, 一些学生只能选择休学甚至退学. 如果以上还算是个别情况的话, 普遍情况是在超过半数的大学课堂有超过半数的学生在低头看手机,第二种是主动意识不够, 这表现在很多方面.



        首先是不会自主学习. 比如有不少学生就说自己除了吃饭睡觉就是学习, 但是效率很低, 事倍而功不到一半, 因为他们还是再用高中划重点的方式学习数学, 习惯性地把定义、命题和定理作为重点画出来, 死记硬背, 而自觉或不自觉地过滤掉数学概念的背景, 无视命题、定理等之间存在的内在联系. 这就像抗日战争中鬼子采用囚笼政策, 当公路、铁路被破坏后, 只剩下孤零零的炮楼.
        
        
        
        其次是没有动手的意识. 在课堂上, 习惯于被动地接受教师课堂讲授的知识. 对于课上提出的问题, 不善于抓住有限的时间去思考, 只看不动, 等着老师讲解; 或者满足于自己有的一点想法, 光说不练, 真正要写下来却破绽百出.
        
        
        
        再次是没有主动交流的意识. 有些学生也能意识到自己学习方法的问题, 但是由于各种原因, 不会主动求助于老师或者同学. 上课时, 明明没有听懂, 也羞于启齿问问题. 他们不知道, 如果问出来, 哪怕是很初等的问题, 也可以迅速地解答自己的疑惑, 从而提高课堂效率.
        
        
        
        最重要的还是主动探索能力匮乏. 在过去的十几年里, 教过几届大一学生, 也面试过不少学生, 中学生和大学生都有, 大部分学生通常会在两类问题上不知所措. 一类问题是常规的, 比如求一些数列的通项公式, 有的学生会套用方法, 如果追问一下为什么这个方法是可行的? 大多数的回答是书上是这么写的或者老师是这么教的. 大部分学生没有意识去主动问为什么, 也没有主动探索一下方法背后的原因. 另一类问题是开放式的, 比如先解释一个没有接触过的概念, 让学生们举一些例子或者做一些简单的推理, 很多学生会束手无策, 不知从何下手; 给一些提示, 试图引导他们去做一些初步的探索, 也会发现阻力很大. 惰性在不知不觉中已经形成了.
        
        
        
        第三种是接受新知识的能力不足. 有一次在国外访问, 与一位在国外大学工作的学姐聊中美学生的差异, 得到的共识是美国学生的接受能力很强, 对于新事物, 他们能很快接受下来, 然后再去深入理解. 而大部分中国学生做不到, 他们接受新知识的套路是老师课堂反复讲, 课后练习反复做, 经过了很多遍的重复之后终于对新知识有了一些了解. 有人说中国方式可以打牢基础, 或许可以做到厚积薄发. 然而现实是, 我们未必总有那么多时间来打基础, 比如听一个学术报告, 前五分钟介绍了一个新的研究对象, 后面几十分钟介绍目前的研究方法和进展. 然而几十分钟时间还不够我们的学生来好好理解这个新概念, 也没有辅助练习题可以作, 后面的几十分钟只能是完全迷失了.

        \dots

        尽管如此, 探索之路应该也必须要走下去, 或许可以走得灵活一点. 素质教育不应该与高考冲突. 就数学教育而言, 如果我们不是把宝贵的时间花费在大量重复训练上, 而是有意识地引导学生们去探索书本上的知识, 让学生们在碰壁的过程中领悟数学的奥秘, 在上下求索的过程中发现数学好玩, 不知不觉中具有了探索未知领域的勇气, 提高了逻辑思维和解决问题的能力, 这对于应付高考即使不是如探囊取物一般也会起到催化剂的作用吧.

    \end{pas}

    \section{一点有趣的例子}

    计算机科学研究什么? 其实, 这些问题可能从小就问过. 我们来简单的介绍一下. 

   \subsection{计算机科学研究什么}

   其实很简单, 旷日持久的计算机科学教育只是为了回答这三个问题: 

   \begin{itemize}
    \item (theory, 理论计算机科学) 什么是计算?
    \item (system, 计算机系统) 什么是计算机?
    \item (application, 计算机应用) 我们能用计算机做什么?
   \end{itemize}

   \subsubsection{什么是计算}

   我们从小学就开始计算. 不过那些内容基本上是数学的内容, 是用数字和符号, 运用等价替换原则得到的一类演算过程. 这和指令性程序的执行模式大有不同. 比如, 你可能在高中算法基础一节中在接触到如下的代码: 
   \begin{verbatim}
   x=x+1 
   \end{verbatim}
   但是在我们高中的数学演算的时候, $x=x+1$会被认为是$0=1$, 明显是一个假命题. 这两种看上去有很多不同的``计算''过程, 有着不同的模型和规则, 能够做出相同的事情吗?

   如果我们问更多的问题, 就会发现, 我们对于这种一直在做的事情一无所知. 什么是计算? 有没有东西是不可计算的? 计算需要花费多少时间? 
   
   \subsubsection{什么是计算机}

   我们从小就使用计算机, 可是, 你能说出来在点点鼠标的时候计算机中间到底发生了什么吗? 

   同样, 在探索这些内容的时候, 我们总是能够得到一种满足, 在对于以前的内容有着更深刻的理解之后, 就可以带着力量继续前行了.

   \subsubsection{我们能用计算机做什么}

   与前两个相比, 这个问题好像就稍微容易一点回答. 不过你可能有时候也会失望: 计算机不是万能的. 

   不过好消息是, 计算机能够做的事情真的很多. 好好探索一些就会感到快乐. 
   

