\subsection{Cheat Sheet}
\ti{基本的积分公式(I)}
根据求导公式表, 可得
$$
\begin{array}{l}
    \int \mathrm{d} x=x+C ; \\
    \int x^{\alpha} \mathrm{d} x=\frac{x^{\alpha+1}}{\alpha+1}+C(\alpha \neq-1) ; \\
    \int \frac{\mathrm{d} x}{x}=\ln |x|+C ; \\
    \int a^{x} \mathrm{~d} x=\frac{1}{\ln a} a^{x}+C(0<a \neq 1) ; \\
    \int \mathrm{e}^{x} \mathrm{~d} x=\mathrm{e}^{x}+C ; \\
    \int \sin x \mathrm{~d} x=-\cos x+C ; \quad
    \int \cos x \mathrm{~d} x=\sin x+C ;\\
    \int \frac{\mathrm{d} x}{\cos ^{2} x}=\int \sec ^{2} x \mathrm{~d} x=\tan x+C \quad
    \int \frac{\mathrm{d} x}{\sin ^{2} x}=\int \csc ^{2} x \mathrm{~d} x=-\cot x+C
\end{array}
$$

\ti{基本的积分公式表(II)}
根据分部积分和换元积分, 我们有
1. 分母含分式的积分:
平方之和:
$$
\int \frac{\mathrm{d} x}{x^{2}+a^{2}}=\frac{1}{a} \arctan \frac{x}{a}+C ;
$$
平方之差(使用分解因式的技巧):
$$
\int \frac{\mathrm{d} x}{x^{2}-a^{2}}=\frac{1}{2 a} \int\left(\frac{1}{x-a}-\frac{1}{x+a}\right) \mathrm{d} x=\frac{1}{2 a} \ln \left|\frac{x-a}{x+a}\right|+C ;
$$
根号下的平方差, 自变量的平方前面的系数是负数:
$$
\int \frac{\mathrm{d} x}{\sqrt{a^{2}-x^{2}}}=\arcsin \frac{x}{a}+C ;
$$
根号下的平方, 自变量的平方前面的系数是正数:
$$
\int \frac{\mathrm{d} x}{\sqrt{x^{2} \pm a^{2}}}=\ln \left|x+\sqrt{x^{2} \pm a^{2}}\right|+C ;
$$
csc的积分:
$$
\int \frac{\mathrm{d} x}{\sin x}=\int \csc x \mathrm{~d} x=\ln \left|\tan \frac{x}{2}\right|+C=\ln \x|\csc x-\cot x \y|+C ;
$$
sec的积分: 
$$
\int \frac{\mathrm{d} x}{\cos x}=\int \sec x \mathrm{~d} x=\ln \left|\tan \left(\frac{x}{2}+\frac{\pi}{4}\right)\right|+C=\ln\x|\sec x + \tan x\y| +C.
$$