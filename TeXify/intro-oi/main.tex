\documentclass[10pt]{article}
\usepackage{NotesTeXV3,lipsum}
\usepackage{ctex}
%\usepackage{showframe}

\begin{document}
	\title{{基础内容简介(draft)}\\{\small \text{不推荐打印, 请直接在电脑上浏览. 因为PDF里面有一些链接可以点.}}\\{\small Section 1-7 Revised}}
	\author{Guangwei Zhang(AUGPath)}
	\affiliation{
    Zhengzhou No. 1 Middle School, China University of Geosciences\\
	\href{https://shzaiz.github.io/}{Website}\\
	\href{https://github.com/shzaiz}{GitHub}\\
	}
	\emailAdd{gwzhang@cug.edu.cn}

    \author{Nongyu Di(BLUESKY007)}
	\affiliation{
    Zhengzhou No. 1 Middle School, Lanzhou University\\
	\href{https://dnybluesky007.github.io/}{Website}\\
	\href{https://github.com/dnybluesky007}{GitHub}\\
	}
	\emailAdd{diny20@lzu.edu.cn}

	\maketitle
	
    \newpage

    \textbf{不推荐大家打印这份文稿.} 
    因为有一些链接(标注红色)需要在电脑上点击. 另外, 纸张
    应当印上更加珍贵的东西. 例如刘汝佳老师的《算法竞赛入门经典》这样的书本. 

    还有, 现在为止, 本文稿还未真正意义上写完. 仅做提前了解使用. 具体地, 
    8-11节的练习题和文稿暂时仅有一个大致的框架; 16-20节还没有对应的代码实现;
    21-22节还没有练习. 


	
    \pagestyle{fancynotes}
	
    
\part{C语言回顾}

有了计算机, 我们可以把很多重复的工作交给计算机完成.
这样, 人们就可以把更重要的经历放在更主要的事情上面去了. 
其中, 程序设计语言担当了我们人类世界与机器世界沟通的桥梁, 
我们只有通过程序设计语言(如C语言), 计算机才会按照我们
希望的方法工作. 当然, 我们对于计算机的期望很多时候是失败的
这时候, 我们只有通过一些外部工具, 来证明或者否定我们对于
计算机内部一些事情工作的原理. 

如果你感到调试程序困难, 可以参考\href{https://www.bilibili.com/video/BV1f54y1K7rQ}{调试理论与实践}.



\subsection*{闲聊与练习}

\begin{quote}

    \fbox{核心指导原则}  Don't Panic. (不要慌) 
    
    \hfill—— The Hitchhiker's Guide to the Galaxy

    如果你还没有入门, 仍然感到恐惧, 请记住: 坚持住, 进入未知领域, 从简单的、能理解的东西试起, 
    投入时间, 就有收获. 
    
    掉在队伍之后的同学, 即便是仅有一定的编程基础, 努力过的同学也一定能通过 (Yes!)

    \hfill --- 蒋炎岩, 在南京大学\href{http://jyywiki.cn/OS/OS_Guide}{操作系统}课前的提示
\end{quote}
    \part{递归问题}

\section{问题的简化和递归的过程}

我们常说: ``大事化小, 小事化了''. 比如, 你在做数学计算的$3+2+4+5$这个
表达式的时候, 你可能会自动先计算$3+2=5$, 然后再继续计算$5+4+5$. 这样
一来, 我们距离结果就更进一步了. 也就是问题变得更``小''了, 或者更``容易''
解决了. 有些时候, 我们甚至允许把整个过程用抽象的方法盖住了. 比如, 你做了
一个摄氏度转华氏度的转换器, 你可以用一个函数把它抽象, 这样一来, 下次使用
的时候就直接调用就行了. 

我们来看一些我们如何简化问题的一些例子: 

\begin{example}
    要计算一个正整数$n$的阶乘, 如果它不是1或者0, 那么计算$n*(n-1\text{的阶乘})$.
    用数学的语言来写就是
    $f(x)=\begin{cases}
    x\cdot f(x-1) & x>1\\
    1 & x=1
    \end{cases}$
    这就意味着, 我们每一次计算的阶乘都比原来的更靠近答案. 如果我们实际写一下$f(5)$, 
    我们就会有如下过程(为了方便起见, 我们使用$f(x)$表示$x$的阶乘
    ): 
    $$
        f(5) = 5\times f(4) = 5\times 4\times f(3) = 5\times 4\times 3 \times f(2)
        = 5\times 4\times 3\times 2\times f(1) = 5\times 4\times 3\times 2\times 1.  
    $$
\end{example}

\begin{example}
    要学习知识, 首先要认真理解课本的内容, 然后自己进行思考, 最后对于一些问题
    进行提问. % TODO
\end{example}


实际上, 有一类问题它比较特殊. 你会发现, 如果能够把小问题解决好了, 那么原来的大问题
就自然而然地解决好了. 这种情形, 我们一般认为是递归的问题. 

\begin{definition}[递归]
    递归的问题是这样解决的: 
    \begin{itemize}
        \item 如果给定的问题大小可以直接解决, 那么就直接解决了; 
        \item 否则, 把它转化为这个问题的更简单(通常会更小)的问题
    \end{itemize}
\end{definition}

一开始, 这样自己提及自己的内容的确让人困惑. 但是, 有一个有趣的方法, 就是假想
有一个小精灵帮助你解决问题 - 我喜欢称他为递归精灵. 
你唯一的任务是简化原来的问题, 或者在不必要或不可能简化的情况下直接解决它. 
递归精灵将使用与你无关的方法为你解决所有更简单的子问题. 

这样说来确实很困惑. 但是我们来看下面的几个例子: 

\begin{example}
    归并排序: 要排序一列数, 我们可以将待排序的序列分成两个子序列,
    然后分别对这两个子序列进行递归排序,最后将两个有序的子序列合并成一个有序的序列.
    \mn{在合并的过程中, 我们还可以计算逆序对.}
    这时候, 我们将问题分解成若干个相同或相似的相似的小问题来解决, 
    然后再将子问题的解合并起来, 得到原问题的解. \file{mg-sort}
\end{example}

\ti{P1908 逆序对} 我们可以借鉴快速排序的思想的合并过程中作为合并的时候统计. 
如果左边有$m$个逆序对, 右边有$n$个逆序对, 那么就会多出$mid-i+1$个. 见
代码\file{inversion-pair}. 

\begin{example}
    TBD: Hanoi 塔问题
\end{example}


从这里开始, 我们对于程序的执行的理解似乎就感觉有点模糊了. 不过我们总是可以
使用正确的工具来让我们了解更多. 具体地, 我们可以使用调试功能. 调试器可以
帮助我们窥探程序现在在执行哪一行, 执行的内容是什么. 以及执行到这一步里面的
变量有什么, 是什么值. 我们可以使用$\texttt{gdb}$来解答这个问题. 

我们可以把每一次这样的函数调用想象是一个状态. 所谓状态, 就是相当于给这时候
程序里面的内容拍了个照. 研究状态是如何变化的会让我们思路更加清晰. 我们首先从
Hanoi塔开始看起: 

TBD

\section{递归的结构}

\lec{引用}{重要的操作} 可能发生过这样的情形: 假设你是一个课代表, 你希望去某一个老师那里抱作业. 
但是你不知道这位老师住在哪. 现在, 有同学告诉你老师在309办公室. 这下, 你就可以使用
知道的``309''来找到老师了. 

看似这是一个不起眼的例子, 实际上, 这包含着计算机科学里面比较重要的内容: 使用一种间接
的方式来获得我们要的东西. 

\lec{链表}{单向} 链表是它由一系列节点组成, 每个节点包含两部分:
数据 (存储我们需要的信息) 和指针 (指向下一个节点的地址) . 
链表中的节点可以分散存储在内存中, 不像数组那样需要一段连续的内存空间. 
这就可以方便地在中间插入与删除元素. 但是我们的随机查找的功能就不能
很快地完成了. 因为它们在内存里面不是连续的. 

链表可以被看作是一个递归结构, 即链表的每个节点都可以看作是一个小型的链表. 
每个节点包含数据和指向下一个节点的指针, 而下一个节点又包含数据和指向更下一个节点的指针, 
以此类推. 
原来即使在一些结构里面, 递归的问题也是大有帮助的. 

\ti{UVA11998 Broken Keyboard} TBD 

\lec{链表}{双向} 当然有些问题我们还可以记录双向的信息, 也就是它的上一个和下一个. 这里有一个
经典的问题. 我们可能需要很长时间才能够把它调试正确. \ti{UVA12657 移动盒子}.

\lec{树状结构}{二叉树} 

我们窗外应该就有很多树. 我们现在来看树状的结构. 

二叉树是一种特殊的树结构,其中每个节点最多有两个子节点,通常称为左子节点和右子节点。
这个定义是递归的,因为二叉树的每个子树也是一个二叉树。

具体地说,二叉树可以为空,也就是没有节点的情况,这被称为空二叉树或空树。
如果二叉树不为空,则它由一个根节点组成,以及两个分别为根节点的左子树和右子树的二叉树。

递归定义的核心是,二叉树的每个子树仍然是一个二叉树,它们也遵循相同的定义:
最多有两个子节点,每个节点都可以有自己的左子节点和右子节点,或者为空。这种递归定义使得
我们可以在处理二叉树时使用递归的思想,对于每个节点,
我们可以递归地处理它的左子树和右子树。递归在二叉树的许多问题中都是非常常见和有效的解决方法。

一个问题是, 如何储存二叉树呢? 一个想法是, 我们假设考虑一个最满的二叉树, 应该是如下图所示的: 
\incfig{recursion/bintree.jpeg}
自然, 我们就可以为他们天然指定一个序号. 我们采用这样的方式: 对于一个节点编号为$o$
, 左孩子我们可以使用$o\times 2$来表示, 右孩子我们可以使用$o\times 2+1$来表示. 
这样子, 我们就可以来储存这棵树了. 

但问题又来了: 我们现在是按照满的样子去精简的. 假设我们这个是一条链, 也就是层数
很深, 但是节点数很稀疏, 这时候如何是好? 这时候, 我们用到这样的一个技术了:
我们主要维护一个新节点的``池子'', 里面的每一个表示一个新节点的编号.
每次我们要新建一个节点的话, 就可以通过让这个池子的数据加1的方式. 这时候, 我们 
可以使用编号的方式去更新与原来的关系. 实际上, 用一个``池子''这样的思路
其实是使用了数组模拟指针的思想. 现在使用指针可能会造成一些误解, 所以先使用
数组模拟的指针做大致了解. 

TBD: 添加数组模拟指针的代码...

既然树是递归的结构, 那么, 自然我们更希望使用递归的方法去处理它. 不过, 
因为它有两个子节点, 我们才感觉它和一般的线性的数据有一些不同. 

一个递归函数, 当然可能要决定什么时候递归, 什么时候对我们当前的节点做操作. 
时机是很重要的. 

\lec{二叉树的三种遍历模式}{前序遍历} 前序遍历的过程是, 先输出当前的节点, 再
递归地前序遍历左边的孩子, 再递归地前序遍历右边的孩子. 我们对于样例的树模拟一下. 
\begin{lstlisting}
void pre_order(int o){
    if(o does not exist) return;
    print(o);
    pre_order(o's left son);
    pre_order(o's right son);
}
\end{lstlisting}

\lec{二叉树的三种遍历模式}{中序遍历} 中序遍历的过程是, 先
递归地中序遍历左边的孩子, 再输出根节点,再递归地遍历右边的孩子. 
\begin{lstlisting}
    void in_order(int o){
        if(o does not exist) return;
        in_order(o's left son);
        print(o);
        in_order(o's right son);
    }
\end{lstlisting}

\lec{二叉树的三种遍历模式}{后序遍历} 后序遍历的过程是, 先
递归地中序遍历左边的孩子,再递归地遍历右边的孩子, 最后输出根节点. 
\begin{lstlisting}
    void in_order(int o){
        if(o does not exist) return;
        post_order(o's left son);
        post_order(o's right son);
        print(o);
    }
\end{lstlisting}

这三个内容看起来就是一些奇怪的交换顺序, 实际上, 我们有很多的有趣的性质可以从中
观察出来. 比如, 前序遍历的一个内容总是根. 那么, 能不能通过前序遍历的序列和中序遍历
的序列还原整个二叉树呢? 其实是可以的. 既然前序遍历的一个内容总是根, 中序遍历只要
找到这个根在哪就可以了. 中间的就是子树. 子树也可以按照同样的方法做. 

\ti{P1030 求先序排列} 这个问题只是变了一下形式, 后序遍历了. 但是后序遍历根节点
在最后输出. 和上面的讨论是一致的. 

我们之所以能够按照这种方法遍历, 说到底还是用好了递归的结构的定义. 下面我们来看一看
有什么有趣的遍历方法. 

\ti{UVA10562 Underdraw the trees} 题目大意: 你的任务是把多叉树写成括号的表示法. 
每个节点处了``-'', ``|'', `` ''(空格)用其他字符表示, 每个非叶子节点下方总会有一个
``|''字符, 下方是一排``-''字符, 恰好覆盖在所有的子节点上面. 单独的一行``\#''作为结束标记.

我们可以定义函数\cw{dfs(r,c)}表示\cw{r}行 \cw{c}列开始的内容. 下面有子树的条件
是下一行的这一列有\cw{|}记号(注意不要越界). 然后我们就可以寻找\cw{-} 的左边界, 
顺着\cw{-}走, 一旦发现下面有字符就继续递归下去. 

刚刚的问题甚至不是二叉树, 但是我们运用我们的方法照样可以继续下去. 

\ti{UVA297 Quadtrees} 这是一个四叉树的问题. 但是解决问题的思路也是类似的. 
这时候我们把两个内容都画出来就好了. 这样子模拟一遍就完成了. 

\ti{UVA806 Spatial Structures} 

接下来我们来看求最短路的一些方法. 除了一路走到底, 我们能不能边走边看呢? 我们可以
使用BFS的方法. 具体而言, 我们的策略如下: 
\begin{lstlisting}
queue=[起点]
while(queue不是空的){
    node = 队列的第一个元素;
    输出node;
    把所有与node能够到达的没有访问过的边放进来;
}
\end{lstlisting}

我们来看一个迷宫游戏, 并且在上面运用一下我们的BFS技术. TBD. 

我们发现我们搜到的结果其实就是一棵树. 

\lec{多说一句}{了解计算机程序的执行} 其实, 我们刚刚发现的树, 有一些类似的
妙用. 不过我们要加上允许环的形成. 也就是, 现在它是一些点和一些边的集合.
它甚至可以帮助我们理解我们的程序. 其实, 每一个程序都可以被抽象为一个
执行的图. 

到底什么是程序? 我们看上去我们会认为是我们的C代码, 经过下面的内容, 希望大家可以
对于``什么是程序''有一个不一样的答案. 我们会用刚刚我们了解到的``图''的知识, 
构建一个``状态机模型''. \cite{jyyos-prog}

``到底什么是程序''这样的问题是比较深刻的. 理论计算机学家深刻地研究了程序语言应该
有的语义, 执行的过程等等. 但是我们从一个更加简化的角度来看, \textbf{程序就是
状态机}. 每一个状态就相当于一个节点里的一些数据, 不同状态之间经过程序语句
进行转移. 一个粗浅的理解是是: 状态就是``堆 + 栈''(存放着我们的变量等), 初始状态
就是``main的第一条语句'', 迁移就是``执行一条简单语句''. 因为任何一个C程序都可以
写成一个非复合语句的C代码, 并且的确有\href{https://cil-project.github.io/cil/}{这样的工具}和\href{https://gitlab.com/zsaleeba/picoc}{解释器}! 

这样的过程会对我们的调试代码带来好处. 比如, 我们可以使用\cw{gdb}来检查我们
感兴趣的输出, 同时我们可以使用\cw{printf}指令向我们输出感兴趣的调试信息. 

\subsection*{闲聊与练习} 

\begin{quote}
    孫子曰:凡治眾如治寡,分數是也。鬥眾如鬥寡,形名是也。三軍之眾,可使必受敵而無敗者,奇正是也。兵之所加,如以碬投卵者,虛實是也。

    Sunzi said: The control of a large force is the same principle as
    the control of a few men: it is merely a question of dividing up
    their numbers. Fighting with a large army under your command is 
    nowise different from fighting with a small one: it is merely a 
    question of instituting signs and signals. To ensure that your 
    whole host may withstand the brunt of the enemy's attack and 
    remain unshaken - this is effected by maneuvers direct and 
    indirect. That the impact of your army may be like a grindstone 
    dashed against an egg - this is effected by the science of weak 
    points and strong.
    
    \hfill ---  《孫子兵法·兵勢》
\end{quote}

\begin{exercise}{翻炒煎饼: 选自\cite{algobook}第一章问题4}
假设你得到一堆$n$个不同大小的煎饼。你想把薄煎饼排个序,这样小煎饼就在大煎饼的上
面。你唯一能做的就是翻转——在顶部的$k(k=1,2,\cdots, n)$个煎饼下面插入一把刀然后将它们全部翻转。
\incfig{recursion/pancake.png}
(1) 描述一种算法,使用尽可能少的翻转对一堆任意的$n$个煎饼进行排序。在最坏的情况下,你的算法到底执行了多少次翻转?

(2) 现在假设每个煎饼的一面都烧焦了。描述一种算法,对任意堆叠的$n$个煎饼进行排序,使每个煎饼烧焦的一面朝下,同样保证翻转次数尽可能少。在最坏的情况下,你的算法到底执行了多少次翻转?

(3) 使用你刚刚思考的结果, 完成\ti{UVA120 Stacks of Flapjacks}. 
\end{exercise}

\begin{exercise}{图像旋转: 选自\cite{algobook}第一章问题9}
假设我们想要将一个 $n \times n$ 的像素地图顺时针旋转 $90^\circ$($n$是2的
若干次幂)。一种方法是将像素地图分成四个 $\frac{n}{2} \times \frac{n}{2}$ 
的块,使用五次块传送将每个块移动到其正确的位置,然后递归地旋转每个块。
(为什么是五次?和汉诺塔问题需要第三个柱子的原因一样。)
另一种方法是首先递归地旋转块,然后将它们放到正确的位置。

\incfig{recursion/rotate.png}

\end{exercise}

\begin{exercise}{k-d树: 选自\cite{algobook}第一章问题25}
    假设我们有 $n$ 个散布在二维的盒子内的点。``k-d树''通过将这些点划分用递归的方
    式如下:首先,我们使用一条垂直的线将盒子分成两个较小的盒子,然后使用水平线将每个较小的盒子再次划分,如此反复进行,始终在水平和垂直划分之间交替。每次划分盒子时,划分线会通过盒子内的一个中位点(不在边界上)尽可能均匀地划分剩余的内部点。如果一个盒子不包含任何点,我们就不再继续划分它;这些最终为空的盒子称为单元(cells)。

    (1) 最后由多少个单元? 用$n$表示.  

    (2) 在最坏的情况下,一条水平线到底能穿过多少个单元?用$n$表示. 
\end{exercise}
    \part{一些有趣的思想}

\section{二分法}

我们在解决问题的时候, 常常采用一些有趣的方法来进行. 比如, 我们有逻辑推理能力, 
来解答各种各样的问题. 我们先来考虑二分法. 

我们先考虑这样的一个猜数字游戏: 假设有一个人选定了一个秘密数字, 并让你来猜这个数字是多少.
这个秘密数字是在一个已知范围内的整数. 你可以每次猜一个数字, 然后得到一个提示: 告诉你
该数字是猜测的秘密数字的偏大还是偏小, 或者是猜中了. 
基于这个提示, 你要做的是继续猜测直到猜中为止. 你的目标是用最少的猜测次数找到秘密数字. 

在上面的问题中, 我们可以找到某个性质的边界, 其中分别是小于这个数的和大于等于
这个数的. 也就是说, 我们要二分一个问题, 就是看一看这个边界是不是能够找到. 

在这一部分中, 我们首先会叙述这个的一般原理, 然后观察几个基本的问题以及几个
写代码的范式 - 很多时候写二分有关的代码是很容易犯错的. 结果就是无尽地死循环.
但是幸运的是, 我们可以避免这件事情发生. 


\lec{整数二分}{原理} 我们的目标是找一个性质的边界. 例如, 我们有如下的边界: 
并且有一个命题$P$, 左边的红色的部分是不满足$P$的, 右边的是满足$P$的. 

\incfig{opt-search/bsearch.png}

那么, 要找到红色的最右边的那个, 就(1)首先要找到一个中间值\codeword{mid=(l+r+1)>>1}
, (2)判断中间值是不是满足性质$P$, 也就是\codeword{check(mid)}. (2.1)如果$P$满足, 
那么\codeword{l=mid}; (2.2)如果$P$不满足, 那么\codeword{r=mid-1}. 返回
到(1), 重复执行, 直到\codeword{r>=l}. 

如果要找到绿颜色最左边的那一个, 和上面的问题相仿, 还是
(1)首先要找到一个中间值\codeword{mid=(l+r)>>1}
, (2)判断中间值是不是满足性质$P$, 也就是\codeword{check(mid)}. (2.1)如果$P$满足, 
那么\codeword{l=mid}; (2.2)如果$P$不满足, 那么\codeword{r=mid+1}. 返回
到(1), 重复执行, 直到\codeword{r>=l}. 

我们发现上述只是在取\codeword{mid}的时候和修改\codeword{l, r}的时候发生了
一点小问题. 这是因为C中的数组的舍去问题. 如果不这样做, 有时候会发生死循环 - 就是说
在锁定只有两个的时候, 不额外加一的时候, 可能会导致$l$在执行$(l+r)/2$之后还是$l$
. 这样就相当于什么都没有更新. 肯定不是我们想要的. 


\lec{练习}{更多的例子}\ti{P1163 银行贷款} 个人认为这个题面似乎有点表述不清. 我们采用另一个更严谨
的题目: 给出$n,m,k$, 求贷款者向银行支付的利率$p$, 使得: 
$$
n={m\over 1+p} +{m\over (1+p)^2}+{m\over (1+p)^3}+\cdots + {m\over (1+p)^k} 
$$
其中$p$保留0.1\%.  \mn{果然使用数学公式是很容易表达的}

Idea1. 我们来``猜测''$p$, 然后根据我们的猜测根据公式计算, 看一看它到底还的
多还是还的少. 如果多了, 就稍微把$p$往下调一点, 少了就把$p$往上调一点. 不过这道题
也有够坑的 -- 有的利率答案居然高达300\%! 所以二分的边界需要设置为300\%才行. 
我这里只让它执行了10000次二分操作 -- 毕竟最后的精度不高. \file{P1163}

\begin{remark}
    注意保留精度! 使用pow进行求和可能会扩大误差, 达到最后会差大约
    200元. 
\end{remark}

Idea2. 如果学习过了一些数学, 这个问题还可以使用数学的方法推演. 形如这样的
叫做等比数列, 意思是后一项除以前一项, 结果总是一个常数. 大家耳熟能详的
2, 4, 8, 16, 32, $\dots$ 这一串数列就是一个典型的等比数列, 其中
通向是$2^n$. 其中$n$是第几项(从1开始编号). 也就是说, 我要想知道第二项
是多少, 就要带入$n=2$, 结果就是$2^2=4.$ 

\lec{等比数列}{求和} 等比数列如何求和? 这就需要一些技巧: 我们假设等比
数列的通项是$a[n]=a_1q^{n-1}$, 
那么$S_n=a_1 + a_2q + a_3 q^2 +\cdots +a_n q^{n-1}$. 

我们发现这里面有很多的东西, 所以我们得想个办法把它们消掉. 采取两边同乘以
$q$, 两式相减, 就有神奇的效果. 这个方法也叫做错位相减法. 

推导过程如下所示: 
\begin{example}
    假设我们有一个等比数列:\(a, ar, ar^2, ar^3, \ldots, ar^{n-1}\) , 其中 \(a\) 是首项,  \(r\) 是公比,  \(n\) 是项数. 

我们想要求这个等比数列的和, 表示为 \(S_n\). 
首先, 我们将数列的前 \(n\) 项相加:
\[ S_n = a + ar + ar^2 + ar^3 + \ldots + ar^{n-1} \]
接下来, 我们将 \(S_n\) 乘以公比 \(r\):
\[ rS_n = ar + ar^2 + ar^3 + \ldots + ar^{n-1} + ar^n \]
接下来, 我们从 \(rS_n\) 中减去 \(S_n\):
\[ rS_n - S_n = (ar + ar^2 + ar^3 + \ldots + ar^{n-1} + ar^n) - (a + ar + ar^2 + ar^3 + \ldots + ar^{n-1}) \]
注意, 在括号内的部分可以通过消去相同项来简化. 我们得到:
\[ rS_n - S_n = ar^n - a \]
接下来, 将 \(S_n\) 提取出来:
\[ S_n(r - 1) = ar^n - a \]
现在, 将 \(S_n\) 解出来:
\[ S_n = \frac{ar^n - a}{r - 1} \]
这就是等比数列的求和公式. 
如果公比 \(r = 1\), 那么这个等比数列就变成了等差数列, 求和公式变为:
\[ S_n = \frac{n}{2}(a + l) \]
其中,  \(l\) 是数列的最后一项. 
\end{example}

好了, 经过上面的推导, 我们就可以得到等比数列的求和: $S_n=\frac{a_1(1-q^n)}{1-q}$.


但是我们发现这个东西并不好解答... 确实, 我们并不能一味地通过一种方法解答
问题. 当我们遇到困难的时候就要多换角度. 

\ti{P2249 查找} 这个就是最基本的内容了. 直接参考代码就可以了! 注意刚刚
说过的一个问题: 到底是左端点还是右端点. 

\begin{remark}
    整数相关的二分的算法bug是比较隐蔽的. Java标准库中一个类似的查找函数
    使用了类似的二分方法. 但是它使用了\codeword{int mid = (low+high)/2;}
    导致了问题. 这个Bug在Java的数组标准库里面待了9年. 
    \href{https://dev.to/matheusgomes062/a-bug-was-found-in-java-after-almost-9-years-of-hiding-2d4k}{这里是原始文章.}
\end{remark}

\ti{P1676 Aggressive Cows G} 这个是最小值最大的问题, 意味着我们一般使用
按照答案二分的策略. 我们首先猜一个答案, 然后去施展我们应该有的构造, 最后
来看一看这个是不是太小了. 

我们可以假设牛棚都是空的, \codeword{check}时如果当前牛棚与上一个住上牛的牛棚之间
的距离\codeword{dis>=mid}, 我们就可以让这个牛棚里住上牛, 反之向更远的距离寻找牛
棚. 这是个贪心算法. 如果最后能安排的牛总数小于总的牛数, 那么就可以扩大需求. 
(\codeword{r=mid}) 
反之, 就要缩小(\codeword{l=mid+1}). 

\begin{ques}
    为什么这个贪心算法是对的? 
\end{ques}

我们说: 按照上面的构造, 一定是``最省''的. 并且我们只要能说明只要不按照这样做
不一定是最省的就可以了. 也就是, 最小值可能会变得更小. 

\ti{P2678 跳石头}  这个仍然是最小值最大的问题. 和上一个问题是类似的. 自己试着
感受一下吧!

\ti{P3853 路标设置} 这个和上面的问题也是一样的. 自己动手试一试吧! 

\ti{P1314 聪明的质检员} 这个虽然标号的颜色是绿色的, 但是仍然逃不过二分答案的
区间. 不过, 这里面可能有些符号难以阅读. 我们来简单阅读一下: 

\lec{求和符号}{简介} 求和记号是一大堆连加记号的缩写. 简单来说, 只是一个省略
而已, 并没有万能的公式可以求和任何事情. 

\lec{Iverson的括号}{简介} Iverson记号写作$[..]$其中, 里面的$..$是一个
布尔表达式. 当里面的结果是真的时候, 值为1, 否则值为0. 

Iverson括号可以和求和一起搭配使用, 来达到简化求和记号的作用. 比如, 我们要交换
两个求和记号的时候, 更好的想法可能是用这样的方法\mn{更多的内容可以参看
\cite{knuth1989concrete}2.4节 多重和式(MULTIPLE SUMS)}: 
$$
[1\leq j\leq n][j\leq k \leq n]=[1\leq j \leq k \leq n] = 
[1\leq k\leq n][1\leq j\leq k]. 
$$

介绍了刚刚的内容, 我们来简单梳理一下这个问题. 

我们要得到$\min |s-y|$, 就必须找到合适的$W$, 进而得到对应的$y$. 并且另一个
观察是: $y$ 越大, $W$越小. 当$y<s$时, $y$偏小, 我们就要减小$W$; 当$y=s$的
时候, 我们就得到了我们想要的结果. 当$y>s$时, $y$偏小, 我们就要增大$W$. 

我们求$y$的过程满足单调性, 因此使用二分的方法即可. 到这里, 我们能够得到70分. 
查询的部分有个双重的for循环. 这部分使用前缀和优化一下就好. 我们马上会提及. 

\section{前缀和与差分}

\lec{前缀和}{普通版本}现在有一个数组, 请问$\sum_{i=l}^r a_i$等于多少? 我们很容易用for循环实现. 
但是, 如果这样的事情会发生多达$10^5$, 应该怎么办? 一个好的想法是我们可以把
他们累加起来. 

\begin{definition}
    如果一个数组$a$, 它的前缀和数组$s$的通项为$s_i = a_1 + \cdots + a_i$. 
\end{definition}

这时候要想求$l\sim r$的和就求$s_r - s_l$即可. 

\begin{ques}
    既然有前缀和, 那么你认为什么操作下积可以被前缀吗? 你觉得能够前缀的问题
    有哪些特征?  
\end{ques}

我们发现上述的前缀和问题能够胜任查询问题, 但是对于修改操作并没有办法很好的胜任
因为单点进行修改之后, 其之后的前缀和都要发生变化. 

\lec{前缀和}{何必要前缀``和''?}事实上, 前缀和刻画了``连续进行若干次操作, 产生的一个综合影响可以通过某种手段
撤销. '' 比如, 我们如果连着加他们, 到最后可以使用减法把影响的区间消除. 
减法在数学中称为加法的``逆(inverse)运算''. 普通乘法的逆运算是除法. 

事实上, 运算这件事情可以被定义得很广泛. 比如, 你可以在正方形纸片上面定义一个
运算, 叫做``向右旋转90度''. 它的逆运算可以是``向左旋转90度'', 或者说
``连续做3次向右旋转90度''. 

下面我们来看一个比较奇怪的, 但是也能用上述的思想做的内容. 

\begin{example}
    现在有编号为$0\sim 10$一共$10$个球, 我们现在有若干个区间的对换. 具体地, 
    对于区间$[l..r]$的对换之后, 如果原来这方面的球的编号是
    $\cdots, a_l, a_{l+1}, \cdots, a_r, \cdots$, 那
    么经过这次对换之后, 这个区间的球
    的顺序就变成了
    $\cdots, a_{l+1}, a_{l+2}, \cdots , a_r, a_l,\cdots $. 

    现在你有$n$条操作规则, 每条操作规则就是两个数$l,r$. 现在, 我们想知道
    你连续执行编号$a$到编号$b$的操作规则之后, 得到的内容是多少. 注意有$m$次
    查询. 

    数据范围: $1\leq n, m \leq 10^5, 0\leq a, b\leq 9, 1\leq l\leq r\leq n.$
\end{example}

我们如果这时候把``交换''当做一个运算, 运算的``数''就是你现在交换的
区间左端点和右端点, 这样子就和刚刚加法减法的前缀和类似了. 事实上, 这样的
对换在后续学习中是很重要的. 

\begin{remark}
    重要的对换: 如果你之后学习了Polya定理, 其中有一个重要的结论是任何一个置换都可以分解成
    若干个对换的复合. 这会对于你计数带有对称性的内容带来很大的帮助. 
    
    另外, 在数学中, 抽象代数中的群也有类似的刻画. 同样也有更加一般化的结论
    和内容. 不过要是学习这个, 必须有足够扎实的数学基础和对于许多内容的熟练
    掌握(如数学分析, 高等代数等基础课程)
    在这里我们不做讨论. 
\end{remark}

当然, 上述的内容只是一个简单的例子. 当你学习了更多的结构的时候, 很多结构
天然地满足这个性质. 到时候请多加留意. 

\lec{差分}{普通版本} 我们发现, 前缀和让我们拥有在$\mathcal O(1)$时间查询的
能力. 但是如果修改起来可能就麻烦了. 这里, 我们介绍一种方法, 使得我们可以在
$\mathcal O(1)$时间内修改, 并且能够$\mathcal O(n)$查询出来单点的值. 

我们现在的问题是有一个数组$a$, 每一次, 我要向$l..r$的区间内的元素加上一个
值$d$. 最后只有一次询问, 问我现在第几个元素被改成几了. 这样的修改会发生很多
次, 因此我们不能使用for循环来做. 

我们发现, 在对于区间一整个加的操作中, 我们在这一个区间加和的过程中, 区间
内部的两个数之间的\textbf{差}一直不变. 于是我们试着引入差分的定义: 

\begin{definition}
    对于一个数组$a$, 我们定义$d_i=a_i-a_{i-1}$, 那么$d$数组为原数组
    的差分(difference)数组. 
\end{definition}

我们发现, 如果要在原数组的$[l..r]$上加上一个数$x$, 只要在$d_l$上加上$x$, 
在$d_r$上减去$x$. 

挺有趣: 刚刚使用了累加, 我们才能得到了一个可以胜任区间求和, 但是做不了区间
修改的东西. 现在我们让每一个内容是它减去它前面的内容, 居然可以胜任修改, 
但是无法胜任区间的求和. 

那么, 我们的原数组$d$, 这个数组$a$, 以及前缀和数组之间$s$有什么关系呢? 
经过不复杂的数学推导, 我们可以发现: 

\incfig{opt-search/relation.png}

\begin{remark}
    这个关系, 在你上了高中, 接触到了路程, 速度, 加速度的关系的时候, 
    会发现它们是出奇的一致的. 为什么? 路程, 速度, 加速度的关系就似乎
    是这里的$x, v,a$的关系. 完整的知识在大学才能揭晓 -- 那时候
    你会学习数学分析, 更进一步地看一看在连续的情形下, 我们是如何做
    ``前缀和''的. 
\end{remark}

\lec{差分}{加一个等差数列?} 如果我们要在之间加一个等差数列, 那该怎么办?
比如原数列是$1,2,3,4,5$, 在区间$[1..3]$加上等差数列$2, 4, 6$, 最后
的结果是$3, 6, 9, 4, 5$. 

我们发现, 我们让原来的差分数组再差分一次不就好了! 等差数列再次差分, 就只要
在前面加一个数, 在后面减掉一个数了, 就像刚才一样. 这是差分的一个重要的性质.

在练习中, 你会看到有哪些差分做起来是好做的. 你同时也会发现很多奇妙的公式. 

\lec{差分}{加一个平方数列?} 这次我们使劲差分, 差分到三次, 你就会发现, 他们
就会奇迹般地出现出来0的样式了. 

\begin{ques}
    为什么是差分三次?
\end{ques}

事实上, 我们会发现每次差分之后, 得到的内容就会消掉一次. 也就是从二次变到
一次, 再到0次. 在0次的情形, 就是我们最开心的情况了. 如果下次要加上一些单项式
的组合, 其实同样的方法也是适用的. 

\lec{前缀和}{二维的前缀和} 前缀和有另一个扩展的方向. 我们能不能扩展到二维的前缀
和? 我们可以这样做. 我们仿照一维前缀和的定义, 使用$S[i][j]$表示第i行j列格子左上部分所有元素的和. 那么它的递推式是什么? 
\incfig{opt-search/2d-prefix-sum.png}

于是, 我们就可以写出来二维前缀和的代码. \file{prefix-sum-2d}

\section{贪心算法}

贪心是指在最优化问题的决策过程中, 每次选择当前局面的最优决策. 不过需要指出的是, 
当前局面最优不一定能得到全局最优. 通常, 我们要使用贪心算法, 至少要思考一下如何
说明一下它的正确性. 

\begin{remark}
    有点有趣的是, 在推荐给大家的Jeff Erickson的Algorithm\cite{algobook}书中, 作者风趣地
    写道: ``Greedy algorithms never work! Use dynamic programming instead!''.

    这显示了使用贪心算法的副作用 -- 没有说明胡乱贪心有时候不可取. 其中的 
    Dymanic Programming是动态规划的意思, 现在可以认为是聪明的搜索 -- 使用
    记忆的方法避免求解了一些重复的子问题. 我们会在后面简单了解. 
\end{remark}

我们来看若干个问题: 

\ti{P1056 排座椅} 对于单独的某邻近两列, 如果有$x$对爱唠嗑的同学, 选择拆散这一列, 
就拆散了$x$对同学, 邻近两行也是同理的. 另外, 对于任意情况, 我们都应该拆散
邻近两列或两行爱唠嗑同学对数最多的那两行或两列. 不然, 我们本可以拆散更多的同学. 
我们刚刚的论证用了问题的描述以及反证法(``不然...''). 虽然思路很好想, 但是注意
输出的时候是按照编号输出的. \file{P1056}. 

\ti{P1016 旅行家的预算} 我们可以在油便宜的时候必须要买油, 只要比当前油价便宜就好. 
如果在油箱的油消耗完之前不能到达比当前油价还便宜的地方, 就在这里把油箱加满油. 
如果能到达比当前油价便宜的地方, 那就加油到刚好能跑到那个地方. \file{P1016}



不过要注意的是, 贪心可能很难, 贪心的结果也可能非常有趣. 我们下面来看一个有趣的
脑筋急转弯. 

\ti{\href{https://www.luogu.com.cn/problem/AT_arc066_c}{Addition and Substraction Hard}}
这个题目的意思很简单: 给你一个只包含`+'、`-'、正整数的式子, 你需要在式子中添加一些括号
, 使运算结果最大, 输出最大的结果. 

首先我们看到我们必须在减号的后面加括号. 因为减号的后面才能使得符号发生变化. 
在这个第一个括号里面, 我们就需要闭合的时候这个值尽可能的小了. 那么如何让这第一个
括号里面的值尽可能小呢? 首先, 这个第一个括号的闭合肯定在式子的最末尾. 
其次, 我们可能还要在减号的后面继续加括号, 但是要满足让原式的结果尽量大, 第二层括号
里面的值的要求是也是尽可能最大 -- 也就是让第二个括号里面的加号最多. 我们只需要把所有的
减号后面的连加符号都括起来即可. 

这就是我们的贪心思路了. 我们可能会说: 为什么不会有嵌套三层(往上)的情况? 其实, 我们 
注意到, 任何一个嵌套括号到了3层(或往上), 一般形态为$x_0-(x_1-(x_2-(x_3)))$其实
可以被组合成为$x_0-(x_1-x_2)-(x_3)$, 保留了原来的减号. 但是枚举每一个括号计算的
时间是$\mathcal O(n^2)$的, 难以应对数据量. 我们使用前, 后缀和的技巧来优化我们的
计算. 

要观察出这个思路需要相当对算术运算的体会. 官方给出的题解使用了动态规划的思路. 
正如我们刚刚
提到的, 这是一个``聪明地搜索状态空间''的算法. 我们可能会在未来重新回顾官方题解的做法.

\section{倍增}

\lec{介绍}{引例1} 不知道大家有没有在无聊
\mn{没有无聊的时候? 相信我到时候一定会有的. 比如河南省的会考. 如果
当前情况维持不变的话, 会考理科的试题是非常充足的. 当时大概所有理科作答时间
(数学, 物理, 化学, 生物)加起来总共用了一个小时左右, 所以在那个时候就可以轻松
体会这个游戏了. }
的时候玩过算2的几次幂的游戏: $2^1=2, 2^2=4, 2^3=8, \cdots$. 许多同学在各种
评论区分享了他们算过的最大值. 但是, 我们发现, 我们可以这样来算得更多一些: 
$2\times 2=4, 4\times 4 = 16, 16\times 16=256, 256\times 256=65536\cdots$. 

如果记得\codeword{unsigned int}的最大值是$2^{32}-1$, 即$4294967295$, 那么你可能应该可以
很快地计算出$2^{64}$, 甚至$2^{128}$了. 为了好玩, 我们给出$2^{256}$这个78位数: 

$$
115792089237316195423570985008687907853269984665640564039457584007913129639936
$$

并且我们发现, 我们如果要任意的一个幂次, 就都可以用上面的一些内容表示出来. 比如, $2^3=2^2\times 2^1$.
因为$3=(101)_2$. 由此, 我们就可以发明``快速幂''的算法. 我们只要$\log_2 b$次来计算$a^b$(不溢出的情况下).

\ti{P1226 【模板】快速幂 | 取余运算} 这里只是多了一个取余数的运算. 我们发现
$(a\times b)\bmod c = ((a\bmod c)\times (b\bmod c))\bmod c$. 用这个内容写代码就好了. \file{qpow}


\lec{ST表}{介绍}  我们刚刚使用二进制去拼凑一个整数的方法能不能用于其他的问题呢? 其实, ``可重复贡献的问题''
就是我们可以用这样的方式做的. 我们可以预处理出$f[i][j]$表示序列上起点为$i$, 长度为$2^j$的区间的答案, 
查询的时候使用拼凑的方式把我们的答案拼凑出来就可以了. 比如快速查询区间最值, 区间按位或, 区间按位和, 区间最大
公约数等等. 他们都满足一个性质: $f[a..c] = f[a..b] \text{OP} f[b..c]$. 我们有$\mathcal O(n\log n)$
的时间预处理, $\mathcal O(1)$时间查询. 

\ti{P3865 ST表} 我们可以用上述的方法来完成这道问题. \file{st}

有些时候我们还会在树上的最近公共祖先中遇到这样的倍增的思想. 具体我们可以到时候再了解. 

\section{更多的练习与思考}

除此之外, 我们还会有很多很有趣的思考问题的方法. 我们下面举出几个例子: 

\lec{滑动窗口}{介绍} 我们在有些问题的时候, 要求一个固定长度的区间内部的最大值和最小值都输出出来. 我们有
如果我们使用暴力的做法, 可以这样做: 可以先用一个队列来维护窗口, 保证每次这个窗口里面存的是当前的所有元素. 
遍历所有元素, 得到时间的复杂度是$\mathcal O(nk)$. 但是我们想一想这个真的需要这么复杂吗? 

考虑优化, 有些元素似乎是没用了的. 比如我们要看窗口长度为3的序列. 我们发现, 
我们要求最小值, 我们只要严格递减就可以了. 另外, 适时地弹出不在队伍里面的元素,
就能维持答案的准确性. 也就是说, 单调队列的主要特点是保持队列内元素的单调性, 这使
得在每次添加新元素时, 队列内的元素仍然保持单调性, 从而保证了队列操作的高效性.  

\incfig{opt-search/sin-sta}

我们来看单调队列的代码: 注意这里表示了那些元素的编号在队伍里面. 
\file{slide-window}, 并且可以尝试做习题\ti{P1886 滑动窗口}. 

\lec{并查集}{简介} 并查集主要用于解决元素的所属关系, 也就是看一看两个内容是不是在同一类中. 
具体地, 就有如下的两点要求: 
\begin{itemize}
    \item 将两个集合合并
    \item 判定两个元素是不是在一个集合中
\end{itemize}

要达到这个目的, 我们原理: 每个集合用一棵树来表示, 树根的编号就是整个集合的编号, 
每个节点储存着他的父节点的编号, \codeword{p[x]}表示\codeword{x}的父节点. 

那么我们的几个操作就可以这样表示了: 
\begin{itemize}
    \item 判断树根: \codeword{p[x]==x}
    \item 求\codeword{x}集合所在的编号: \codeword{while(p[x]!=x) x = p[x]}
    \item 合并两个集合\codeword{px,py}, $x\neq y$. \codeword{p[x]=y}.
\end{itemize}
    \part{树与图}

TBD: 介绍图的关键要素, 一些最短路算法, 最小生成树的算法. 
重点关注图作为分析问题的工具是如何理顺递归的关系的. 为下一节
的动态规划做铺垫. 
    \part{简单计数问题}

\section{数学基础} 

在阅读数学相关的文献的时候, 我们可能因为数学的记号没有见到, 而产生恐惧. 现在, 我们就对常见的
一些符号做一些简单的认识. 

\lec{逻辑符号}{联词与括号} 数学家喜欢使用一些逻辑连接词来使他们描述的数学对象更加清晰. 符号
$\lnot, \land, \lor,$ $\Rightarrow, \Leftrightarrow$分别表示逻辑上的``非'', ``与'',
``或'', ``可以推出(蕴含)'', ``等价''. 

\begin{example}
    $x^2-3x+2=0$成立, 当$x=1$或$x=2$
    可以用这样的符号表示: $(x^2-3x+2=0) \Leftrightarrow (x=1) \lor (x=2)$. 
\end{example}

同时我们也需要区分必要性和充分性. 这时候我们可以使用真值表的方法来做事情. 另外, 如果
把``若$P$则$Q$''的$P$和$Q$调换一下顺序, 再取反, 得到了$若非Q则非P$, 你会发现, 这两种
说法是等价的. 

并且为了好玩, 我们发现这些内容遵循如下的规律, 为了更清楚地说明, 可以用符号表示. 

\begin{theorem}[最基本的逻辑符号的运算规律]
    我们发现有如下的最基本的运算规律, 以便于我们操作含字母的命题公式. 
    \begin{itemize}[noitemsep]
    
        \item 交换律:
      \[
        (A \land B) \leftrightarrow (B \land A)
      \]
      \[
        (A \lor B) \leftrightarrow (B \lor A)
      \]
    \item 结合律:
      \[
        ((A \land B) \land C) \leftrightarrow (A \land (B \land C))
      \]
      \[
        ((A \lor B) \lor C) \leftrightarrow (A \lor (B \lor C))
      \]
    \item 分配律:
      \[
        (A \land (B \lor C)) \leftrightarrow ((A \land B) \lor (A \land C))
      \]
      \[
        (A \lor (B \land C)) \leftrightarrow ((A \lor B) \land (A \lor C))
      \]
    \item De Morgan律: 
      \[
        \lnot (A \land B) \leftrightarrow (\lnot A \lor \lnot B)
      \]
      \[
        \lnot (A \lor B) \leftrightarrow (\lnot A \land \lnot B)
      \]
      \item 双重否定律:
        \[
            \lnot \lnot A \leftrightarrow A
        \]
        \item 排中律:
        \[
            A \lor (\lnot A)
        \]
        \item 矛盾律:
        \[
            \lnot (A \land \lnot A)
        \]
        \item 逆否命题:
        \[
            (A \to B) \leftrightarrow (\lnot B \to \lnot A)
        \]
    \end{itemize}

\end{theorem}

我们构造的证明, 一般是形如$A\Rightarrow B\Rightarrow C \Rightarrow \cdots \Rightarrow G$.
其中$G$是我们的结论. 在数学中, 每一个关系要么是公理, 要么是由公理推导出来的命题. 

\lec{集合}{基本概念} 我们总是希望把一堆东西放在一起加以研究. 这种趋势再19世纪末已经被
明确提出来了. Cantor等人提出了朴素集合论的思想. Cantor说: ``我们把\textbf{集合}理解
为若干个确定的, 有充分区别的, 具体的或者抽象的对象合并而成的一个整体''. 这种朴素的集合论
的前提是(1) 集合可以由任何有区别的对象构成; (2) 集合由其组成对象唯一确定, (3) 任何性质都可以
确定一个具有该性质对象的集合. 

我们可以用描述的方法表达这个集合的元素是什么. 比如如果我们要表示一个所有的$x$, 满足$x^2-1=0$,
我们就可以这样写: $\{x: x^2-1=0\}.$ 这个冒号有时候也可以写作一个竖线\mn{我们高中学习的
集合就是要写一个竖线. ($\{x | x^2-1=0\}$但是这里由于不与C语言的``或''搞混, 先这样书写. }.

集合中$S$包含元素的个数用$|S|$表示. 有时候称为该集合的``大小(size)''. 很多计数问题
都可以通过这样的一种方法得到更加清晰的表述. 
我们发现, 这个想法和我们以往遇到的数学概念不同, 因为集合的给定方法在明确程度上可以由明显的不同. 
例如``郑州一中的所有学生'', ``集合的集合'', ``Yanyan Jiang中所有a的集合'', ``所有集合的集合''.

甚至最后一个``所有集合的集合'', 干脆就是一个矛盾的概念! 

\begin{proof}
    对于集合$M$, 设记号$P(M)$表示$M$不是它本身的元素. ...
\end{proof}

组成一个集合的对象叫做集合的元素, 如$x$是集合$X$的元素, 可以简单记作$x\in X$(或$X\ni x$). 
如果$x$不是集合$X$的元素, 可以简单记作$x\not \in X$(或$X\not\ni x$). 

\lec{集合}{包含关系} 我们考察集合的交和集合的并, 以及集合的补. 

\begin{definition}
  两个集合的并的定义如下: $A\cup B=\{x:x\in A \lor x\in B\}$
\end{definition}

\begin{definition}
两个集合的交的定义如下: $A\cup B=\{x:x\in A \land x\in B\}$.
\end{definition}

这些就是集合的基本运算了. 我们可以通过这些内容来构造各种不同的东西. 

\lec{映射}{基本的定义}  

映射可以认为是两个集合之间的对应关系. 这有点像送信: 

\begin{definition}[映射]
  设$A,B$ 是两个非空的集合, 如果按某一个确定的对应关系$f$ , 
  使对$A$ 中的任意一个元素$x$ , 在集合$B$ 中都有唯一确定的元素$y$ 与之对应, 
  那么就称对应$f$ 集合 $A$ 到集合 $B$ 的映射. 映射 $f$ 也可记为为$f:A\to B$.
\end{definition}


\lec{映射}{映射的分类} 

映射有很多种类, 根据满足不同的条件, 我们可以将映射分为几种不同的类别:

\begin{definition}
  如果映射$f$的定义域$A$中的每个元素都映射到$B$中的不同元素, 我们就说$f$是"一对一"或"单射". 
\end{definition}

\begin{definition}
  如果映射$f$的值域等于集合$B$, 也就是说$B$中的每个元素都是$f$的某个元素的映像, 那么我们就说$f$是"映满"或"满射". 
\end{definition}

\begin{definition}
  如果映射既是单射又是满射, 那么我们就说它是"一一对应"或"双射". 
\end{definition}

\lec{映射}{复合映射} 那么映射能不能像一条链一样呢? 

其实是可以的. 它是通过将两个或更多的映射联结在一起形成的. 
假设我们有两个映射, $f: A \to B$ 和 $g: B \to C$. 
那么复合映射 $g\circ f: A \to C$ 就定义为对$A$中的每个元素$x$, 
首先应用映射$f$找到元素$f(x)$, 然后应用映射$g$找到元素$g(f(x))$. 
这样便形成了从$A$到$C$的复合映射. 

有时候这个就写作记号$f\circ g$. 注意, 一般来讲$f(g(x))\neq g(f(x))$, 他们是不能交换的, 
但是, 他们是可以结合的. 也就是以什么样的顺序算都是可以的. 

\section{计数原理}

\lec{基本的计数原理}{加法原理} ``加法原理''是计数原理的一个基本策略. 
如果我们有两个不相交 (即互不包含相同元素) 的事件, 其中第一个事件有$n_1$种方式发生, 
第二个事件有$n_2$种方式发生, 那么这两个事件中的任一个发生的方法总数为$n_1 + n_2$. 
也是通常所说的``分步相加''. 局部的之和就是整体的. 下面是形式化的描述. 

\begin{principle}[加法原理]
如果$S$是一个集合, $S$的一个有$m$个部分的划分
\mn{也就是不重复, 不遗漏地把这个集合分成若干部分}$S_1, S_2,\cdots, S_m$, 满足
不遗漏($S=S_1 \cup S_2 \cup S_3 \cup \cdots \cup S_m$); 不重复($\forall 
i,j. S_i \cup S_j = \emptyset$), 那么$|S|=|S_1|+|S_2|+\cdots+|S_n|$. 
\end{principle}

如果一个过程可以通过两个独立的步骤完成, 其中第一步有$m$种可能, 第二步有$n$种可能, 
那么整个过程的总体完成事情的可能的方案数就是$m\times n$. 如果我们用形式化地描述它们
就可以这样说: 

\begin{principle}[乘法原理]
  如果$S$是一个有序对的集合, 里面的元素形如$(a,b )$. 两个有序对$(a_1, b_1), (a_2, b_2)$相等
  当且仅当$a_1=b_1$且$a_2=b_2$. 这样. 如果$a$是从大小为$p$的
  集合里面抽出来的元素之一, $b$是从大小为$q$的
  集合里面抽出来的元素之一, 那么$|S|=p\times q$.
\end{principle}

实际上, 我们可以把每一种选择当做一个``决策'', 平行的(加法原理)选择可以用分叉表示, 递进的
(乘法原理)可以用层层深入表示. 例如下图: TBD. 

下面有一些逆运算. 

\begin{principle}[减法原则]
  如果$A$是一个集合, $U$是一个包含$A$的更大的集合, 那么令$\complement_U A = U\backslash A
  =\{x\in U : x \neq A\}$为$A$相对于$U$的补集. 那么$|A| = |U|-|\complement_U A|$
\end{principle}

\begin{principle}[除法原则]
  如果$S$是一个有限的集合, 划分成了$k$个部分, 每一个部分都有相同的元素, 那么划分的数量$k$就是
  $$k=\frac{|S|}{\text{每一部分有多少个}}. $$
\end{principle}

我们发现, 这些问题相当的基本, 基本到任何一个幼儿园小朋友在刚学习加减乘除的时候都会学习. 
只不过现在我们就可以使用了更加有趣的方法来做了. 理解这些内容你可能知道了为什么幼儿园
小朋友要掰手指头算数了. \mn{事实上, 幼儿园小朋友的这一行为体现的是自然数的递归地定义. 
幼儿园小朋友还是很有智慧的. }

下面我们来考察一个由Gian-Carlo Rota提出的著名的组合问题. 它按照这个方法给我们的计数
问题简单分了一个类. 

\begin{example}[The Twelvefold way]
  我们来看著名的``The Twelvefold Way''这个问题: 
  它包括了12种从有$n$个球放入有$k$个盒子里的方法. 每种方法具有独特的限制, 
  包括球和盒子是否是区分的及是否允许空盒子等. 
  {\center \begin{tabular}[pos]{|c|c|ccc|}
    \hline
    \text{$n$个球} & \text{$k$个盒子} & 想怎么放怎么放 & 每个盒子最多1个球 & 不允许有空盒子   \\
    \hline
    不同的球$\texttt{oO}o$ & 不同的盒子$\fbox{1}~\fbox{2}~\fbox{3}$ & (1) & (2) & (3)\\
    相同的球$\texttt{ooo}$ & 不同的盒子$\fbox{1}~\fbox{2}~\fbox{3}$ & (4) & (5) & (6)\\
    不同的球$\texttt{oO}o$ & 相同的盒子$\fbox{~}~\fbox{~}~\fbox{~}$ & (7) & (8) & (9)\\
    相同的球$\texttt{ooo}$ & 相同的盒子$\fbox{~}~\fbox{~}~\fbox{~}$ & (10) & (11) & (12)\\
    \hline
  \end{tabular}\\}
\end{example}

我们下面来看这个问题. 

\ca{问题(1): $n$个球, $k$个盒子, 盒子和球都是不同的, 随便放} 我们希望做的事情是``把 $n$个球放入$k$个盒子''.
这时候, 我们对于第一个球的选择就随便选一个就好了. 因此有$k$种方法. 对于第二个球, 因为没有限制, 我们照样可以
用$k$种方法...  一直到第$n$个球. 因此总共的方案是$k^n$.

\ca{问题(2): $n$个球, $k$个盒子, 盒子和球都是不同的, 每个盒子最多1个球} 我们假设盒子的个数多于球, 这样
做的事情就会有意义一点. 
我们希望做的事情是``把 $n$个球放入$k$个盒子, 每个盒子最多1个球''.
这时候, 我们对于第一个球的选择就随便选一个就好了. 因此有$k$种方法. 对于第二个球, 因为没有限制, 我们可以
用$k-1$种方法(有一个已经占用了)...  一直到第$n$个球, 就有$k-n+1$个. 因此总共的方案是$k(k-1)(k-2)\cdots(k-n+1)$.

我们一般把这个叫做排列数, 因为它阐述的是从$k$个物品里面选择$n$个数的方法.\mn{注意这里的字母顺序可能和一般的教科书
不同. 一般的教科书习惯写作$A_n^k=n(n-1)\cdots(n-k+1)$. 这里的形式对于内容是没有影响的. 二者阐述的是
同一件事情. } 
同时, 从$k$开始, 往下乘$n$个数也被称作下降幂(falling power). 

\begin{definition}[排列数]
  从$n$个物品里面选择$k$个数的方法数记作排列数. 记作$A_n^k$. 计算方法为
  $$
  A_k^n = k(k-1)(k-2)\cdots(k-n+1)
  $$
  其中$k(k-1)(k-2)\cdots(k-n+1)$可以被记作下降幂, 写作$k^{\underline n}$. 
\end{definition}

\ca{问题(3): $n$个球, $k$个盒子, 盒子和球都是不同的, 不允许有空盒子} 我们发现当我们的球
的数量不少于盒子数量的时候这个内容才有意义. 

既然我不允许有空盒子, 我先随便挑出来$k$个球去``压箱底'', 然后剩下的像刚刚一样
随便放不就好了? 其实这个方法是不对的. 因为这样会算重复一些方案 -- 你默认的要压箱底的和
后来放的在这里是考虑次序的, 而原来的问题是不考虑次序的. 那我们该怎么做? 

这事实上是集合的一个划分. 每一个划分正好对应一个集合. 我们如果能够把这个集合划分为$k$份, 
然后再把每一个划分对应上一个盒子就好了. 第二步很简单, 直接乘上$k!$即可. 

关键是如何划分这个集合? 为了方便我们的符号书写, 我们先记$\left\{{n \atop k}\right\}$ 为把
$n$个集合划分为$k$个部分的个数. 这时候, 我们与其一口吃个胖子, 我们可以一步一步地考虑\mn{这就有点
递归的意思了! 至此, 你应该能感受到为什么我们把递归问题放在第一节了. }

要把$\{1,2,\ldots,n\}$划分为$k$份, 可以借助那些以往的状态可以把我们带到$\stirling n k$. 

第一种情况是, 我们已经把集合$\{1,2,\ldots,{n-1}\}$分为了$k$个部分, 现在的任务是把$n$放入任何
这$k$部分的其中之一. 这就给了我们$k\left\{{n-1 \atop k}\right\}$种方法达到这个目的. 

第二种情况是, 我们已经把$\{1,2,\ldots,{n-1}\}$分为了$k-1$个部分, 并且让$\{n\}$单独一份. 
这样, 我们就有$\left\{{n-1 \atop k-1}\right\}$种方法. 

这两种方法构建的分割是不同的: 因为在第一种方法中, $n$始终位于一个大小$>1$的划分部分中, 
而在第二种方法中, $\{n\}$始终是一个单独的一部分. 因此这两种情况是不重叠的. 
而对于任意一个$n$元素集合分割为$k$份, 必定可以通过这两种方法之一来构建. 因此根据求和法则: 
$$
\left\{{n \atop k}\right\}=k\left\{{n-1 \atop k}\right\}+\left\{{n-1 \atop k-1}\right\}
$$
成立. 

要求得这个递归式的表达式是十分难的. 我们一般到此为止了. 事实上, 这个内容叫做{\textbf{第二类Stirling数(Stirling number of the second kind)}}. 
要计算第二类Stirling数, 我们有如下的公式: 

\begin{definition}[第二类Stirling数]
  将一个大小为$n$的集合划分为$k$个部分的方案数被命名为第二类Stirling数. 记作$\stirling n k$. 
\end{definition}

\begin{theorem}
  第二类Stirling数满足关系
  $$
  \left\{{n \atop k}\right\}=k\left\{{n-1 \atop k}\right\}+\left\{{n-1 \atop k-1}\right\}
  $$
\end{theorem}

于是, 我们一般使用递推计算的方式计算这个集合. 这个确实需要很多思考, 这就是为什么我们会用一个
伟大数学家的名字命名它. 

这下子, 我们就得到了第三个问题的答案: $k! \stirling n k$. 

\ca{问题(5): $n$个相同的球, $k$个不同的盒子, 每个盒子顶多1个球} 这就要求我们搞清楚到底哪个
可以有球, 哪个盒子里面没有球就行了. 所以我们要求从$k$个里面选取$n$个出来. 这个应该如何计算呢? 
实际上, 我们可以先从排列数出发, 然后想一想把它们分成若干个组, 也就是从小到大排个序. 这样子就是
总共的组合数有$A_k^n/k!$个. 为了方便起见, 我们把这个定义做组合数. 

\begin{definition}[组合数]
  从$n$个物品里面选取$k$个数的方案数为组合数, 记作${n\choose k}$, 或者$C_n^k$. 定义为
  $$
  {n\choose k}={n(n-1)(n-2)\cdots(n-k+1)\over k!}
  $$
\end{definition}


\ca{问题(4): $n$个相同的球, $k$个不同的盒子, 随便放} 由于每一个球是相同的, 所以我们需要关注每一个盒子里面
被放了多少球. 因此, 我们就相当于要在这几个球的空档里面``插板''. 由于随意放置, 我们相当于要在$n+k-1$个
里面选出$k$个, 于是, 得到了
$$
{n+k-1\choose k} = \frac{(n+k-1)!}{k!(n-1)!} = {n(n+1)(n+2)\cdots(n+k-1)\over k!}.
$$

我们把这个记作多重组合数的系数(非标准官方译名): 

\begin{definition}[多重集合组合数]
  多重集合的组合数定义为
  $$
  \left(\binom nk\right)=\binom{n+k-1}k=\frac{(n+k-1)!}{k!\left(n-1\right)!}=\frac{n(n+1)(n+2)\cdots(n+k-1)}{k!}.
  $$
  其中, $n(n+1)(n+2)\cdots(n+k-1)$这样的从$n$开始, 向上乘$k$个数这样的被称为上升幂. 方便起见记作
  $n^{\bar k}$
\end{definition}
于是, 我们得到了这个问题的答案: $\binomt kn$. 

\ca{问题(6): $n$个相同的球, $k$个不同的盒子, 每个盒子不许空} 那么我们不妨首先把前几个球放到前几个球里面, 
然后剩下的就得到了不受限制的状况了. 也就是我们这个的答案是${n-1\choose k-1}. $

\ca{问题(7): $n$个不同的球, $k$个相同的盒子, 随便放} 我们可以把$\{1,2,\cdots,n\}$划分进$i$个非空的盒子, 
其中, $i\leq k$. 于是根据加法原理, 这个问题的答案是$\sum_{i=1}^{k}\stirling n i$. 

\ca{问题(8): $n$个不同的球, $k$个相同的盒子, 每个盒子顶多一个球} 事实上, 如果$n>k$, 那么不可能做到.
根据抽屉原理, 总有一个盒子要装两个球. 反之, 我们就可以做到. 于是这个问题的答案是$$\begin{cases}1 & \text{if }n\leq k\\ 0& \text{if }n>k\end{cases}.$$

\ca{问题(9): $n$个不同的球, $k$个相同的盒子, 不允许有空的盒子} 哈哈! 这不就是我们集合划分的定义吗? 
这样, 我们就可以用$\stirling n k$表示了.

\ca{问题(12): $n$个球, $k$个盒子, 盒子和球都相同, 不能有空盒子} 其实这个是
当我们把球放进盒子里面之后, 真正重要的
是什么? 事实上, 我们发现我们只要关心每个盒子有几个球就好了, 并且我们不用关心有多少球的顺序. 
等价地说, 就是把一个整数分拆. 比如7就可以这样分拆成1, 2, $\cdots,$ 7部分:

$$
\begin{aligned}
&\{7\}
& p_1(7)=1\\
&\{1,6\},\{2,5\},\{3,4\}
& p_2(7)=3\\
&\{1,1,5\}, \{1,2,4\}, \{1,3,3\}, \{2,2,3\} 
& p_3(7)=4\\
&\{1,1,1,4\},\{1,1,2,3\}, \{1,2,2,2\}
& p_4(7)=3\\
&\{1,1,1,1,3\},\{1,1,1,2,2\}
& p_5(7)=2\\
&\{1,1,1,1,1,2\}
& p_6(7)=1\\
&\{1,1,1,1,1,1,1\}
& p_7(7)=1
\end{aligned}
$$

等价地说, 我们的要求是一个数$n$的$k$分拆, 分别记作$x_1, x_2, \cdots, x_k$, 满足如下的条件(*): 
\begin{itemize}[noitemsep]
  \item  $x_1\ge x_2\ge\cdots\ge x_k\ge 1$;
  \item $x_1+x_2+\cdots+x_k=n$.
\end{itemize}

为了方便起见, 我们把整数$n$分拆成$k$部分记作$p_k(n)$. 读作``$n$的$k$-分割''下面我们同样用类似于递归的方法
来求解这个问题: 

假设 \((x_1,\ldots,x_k)\) 是 \(n\) 的一个 \(k\)-分割. 满足刚刚我们提到过的条件(*).

我们对这个问题分类讨论: 第一种情况是, 如果 \(x_k = 1\),
那么 \((x_1,\cdots,x_{k-1})\) 是 把\(n-1\) 分割成的一个不同的 \((k-1)\)-分割

第二种情况是, 如果 \(x_k > 1\), 那么 \((x_1-1,\cdots,x_{k}-1)\) 是 \(n-k\) 
的一个不同的 \(k\)-分割. 并且每个 \(n-k\) 的 \(k\)-分割都可以通过这种方式得到. 
因此在这种情况下, \(n\) 的 \(k\)-分割数目为 \(p_k(n-k)\). 

由于所有的情况都已经讨论完毕, 因此, 我们可以使用加法原理, 把这两个部分加起来, 得到了
\(n\) 的 \(k\)-分割数目为 \(p_{k-1}(n-1) + p_k(n-k)\), 即

\[p_k(n)=p_{k-1}(n-1)+p_k(n-k)\,.\]

\begin{definition}[分拆数]
  定义分拆数$p_k(n)$表示把一个正整数$n$分拆为$k$部分, 分别记作$x_1, x_2, \cdots, x_k$, 满足如下的条件
  的个数: 
  \begin{itemize}[noitemsep]
    \item  $x_1\ge x_2\ge\cdots\ge x_k\ge 1$;
    \item $x_1+x_2+\cdots+x_k=n$.
  \end{itemize}
\end{definition}

\begin{theorem}
  分拆数满足性质
  $$p_k(n)=p_{k-1}(n-1)+p_k(n-k)\,.$$
\end{theorem}

所以我们这个问题的答案就是$p_n(k)$. 

\ca{问题(10): $n$个球, $k$个盒子, 盒子和球都相同, 随便放} 
有了分拆数之后, 我们就可以决定到底要分拆多少个了,  于是答案就是$\sum_{i=1}^{k}p_i(n)$. 

\ca{问题(11): $n$个球, $k$个盒子, 盒子和球都相同, 每个盒子顶多1个球} 它和第(8)问的情况类似. 同样要么
能做, 要么不能做. 原理还是依照第八个问题一样. 

这样我们就得到了整个表格的全貌: 

{\center \begin{tabular}[pos]{|c|c|ccc|}
  \hline
  \text{$n$个球} & \text{$k$个盒子} & 想怎么放怎么放 & 每个盒子最多1个球 & 不允许有空盒子   \\
  \hline
  不同的球$\texttt{oO}o$ & 不同的盒子$\fbox{1}~\fbox{2}~\fbox{3}$ & $k^n$ & $k^{\underline n}$ & $n!\stirling nk$\\
  相同的球$\texttt{ooo}$ & 不同的盒子$\fbox{1}~\fbox{2}~\fbox{3}$ & $\binomt kn$ & ${k\choose n}$ & $\binomt{k}{n-k}$\\
  不同的球$\texttt{oO}o$ & 相同的盒子$\fbox{~}~\fbox{~}~\fbox{~}$ & $\sum_{i=1}^k \stirling ni$ & $\begin{cases}1 & \text{if }n\leq k\\ 0& \text{if }n>k\end{cases}$ & $\stirling n k$\\
  相同的球$\texttt{ooo}$ & 相同的盒子$\fbox{~}~\fbox{~}~\fbox{~}$ & $\sum_{i=1}^k p_i(n)$ & $\begin{cases}1 & \text{if }n\leq k\\ 0& \text{if }n>k\end{cases}$ & $p_k(n)$\\
  \hline
\end{tabular}\\}

不要担心这张表格看起来有些复杂. 其实, 这张表格没有记忆的必要. 现在我们只需要学习排列数和组合数就可以建立一个很好的模型了. 
这些概念是非常有趣和实用的, 它们能够帮助我们解决很多有趣的问题. 

上述材料里面的有时候我们还会遇到更加复杂的问题, 比如对于分拆数, 我们需要将一个数分拆成若干个部分, 并且考虑它们之间的顺序. 
都可以通过一些递归的方法来解决. 我们只当做对大家的训练. 一个初学者当然需要看过足够多的例子, 加以大量的思考
才能设计出比较好的这方面的内容. 大家完全不必着急. 

假设我们有$n$个不同的球, $k$个不同的盒子. 我们可以用一个映射的方式来描述不同的放置方法. 
具体来说, 我们可以把每个盒子看作一个“投影”, 而每个球就是我们要放入的“元素”. 
这样, 每一种放置方法就可以看作是一个特定的映射. 

那么, 任意的映射就是我们刚刚的``随便放''; 单射就是我们的``每个盒子只放一个球''; 满射就是``每个盒子不能空''. 
因此, 这个表格更为一般的情况你就能够看得懂了. 

{\center \begin{tabular}[pos]{|c|c|ccc|}
  \hline
  $N$ & $M$ & 任何一个$f:N\to M$ & 单射$f:N\stackrel{\to}{\text{\tiny 1-1}} M$ & 满射$f:N\stackrel{\to}{\text{\tiny onto}} M$   \\
  \hline
  不同的球$\texttt{oO}o$ & 不同的盒子$\fbox{1}~\fbox{2}~\fbox{3}$ & $k^n$ & $k^{\underline n}$ & $n!\stirling nk$\\
  相同的球$\texttt{ooo}$ & 不同的盒子$\fbox{1}~\fbox{2}~\fbox{3}$ & $\binomt kn$ & ${k\choose n}$ & $\binomt{k}{n-k}$\\
  不同的球$\texttt{oO}o$ & 相同的盒子$\fbox{~}~\fbox{~}~\fbox{~}$ & $\sum_{i=1}^k \stirling ni$ & $\begin{cases}1 & \text{if }n\leq k\\ 0& \text{if }n>k\end{cases}$ & $\stirling n k$\\
  相同的球$\texttt{ooo}$ & 相同的盒子$\fbox{~}~\fbox{~}~\fbox{~}$ & $\sum_{i=1}^k p_i(n)$ & $\begin{cases}1 & \text{if }n\leq k\\ 0& \text{if }n>k\end{cases}$ & $p_k(n)$\\
  \hline
\end{tabular}\\}

说了这么多的理论东西, 下面我们来看一看实践的内容吧! 

\ti{P5520 青原樱} 一句话题意: 一共有$n$个位置, $m$棵树, 两棵树之间要有空位, 
问总共有多少种选法. 
我们把这$m$棵树以及他们所占的位置拿出来, 那道路上就剩下$n-m$个坑, 
而这$n-m$个坑有$n-m+1$个空位, 
我们要把带坑的树插进这$n-m+1$个空位中, 一共的插法就有$A_{n-m+1}^m$了.

道理是很简单, 我们应该如何实现求排列数, 组合数的代码呢? 

\lec{排列数}{} 排列数可能是比较直接的. 直接一个\codeword{for}循环就好了. 

\lec{组合数}{递推计算} 第一个想法是我们干脆按照它说的模拟一遍不就好了? 但是问题在于, 
有时候乘上去可能会超出数据的精度与大小, 从而得到错误的结果. 嗯, 我们需要一个优化. 
其实, 组合数有一个很著名的递推关系式, 
$$\binom{n}{k} = \binom{n-1}{k-1} + \binom{n-1}{k}.$$
这个递推关系的意义是,从$n$个元素中选择$k$个元素的组合数等于两部分之和:一部分是从$n-1$个元素中选择$k-1$个元素的组合数,另一部分是从$n-1$个元素中选择$k$个元素的组合数。如果你只想递推某一行
的, 那么你可以使用这个恒等式: $${n\choose k}={n-k+1\over k}{n\choose k-1}.$$
并且从$n\choose 0=1$开始从左往右开始做. 我们可以使用组合数的定义${n\choose k}={n!\over 
k!(n-k)!}$进行证明. 

\ti{\href{https://codeforces.com/problemset/problem/817/B}{CF817B Makes And The Product}}
题目大意: 给定一个长度大于3的数列, 多少个由$(i,j,k)(i < j < k) $, 使得$a_i\cdot a_j\cdot a_k$是最小的? 

我们发现, 最小乘积是由数组的三个比较小元素相乘得到的. 因此首先排个序, 然后分一些情
况来考虑, 分别按照最小的, 第二小的, 第三小的的个数分类\mn{分类的时候一定要注意不重复, 不遗漏
. 很多时候可以按照一个特定的顺序或逻辑执行. }. \file{CF817B}

如果你发现自己很难把问题分清楚, 那么我推荐\href{https://www.bilibili.com/video/BV1C34y1R7wf}{韩老师讲的这系列计数问题}(来自Bilibili: BV1C34y1R7wf)的视频. 虽然
韩老师讲的是数学竞赛相关的内容, 题目类型可能不同, 但是对于思维的品质的要求可能是很相似的. 
\mn{今天的Twelvefold way只是做一个引入. 如果你想了解更多的内容和应用, 欢迎去追完
整个系列. }

\ti{Cantor展开} 很多时候我们会枚举排列, 比如$abc$三个字母的排列有: 
$$
abc, acb, bac, bca, cab, cba
$$
六种. 那么, 上述的内容是字典序排序过后的内容. 我们在搜索的时候或许希望得到他们的
排名, 这样子我们就可以愉快的把一个字符串的状态表示变成一个整数. 再比如, 2个$a$, 
1个$b$, 1个$c$组成的所有的串按照字典序的编号可以如下: 
$$
aabc(1), aacb(2), \cdots, cbaa(12).
$$
我们比如输入一个字符串, 能不能输出它的编号呢? 

首先我们要了解多重集合有多少个排列. 先把所有可能,也就是全排列处理出来,
然后相同元素可以随意互换位置,按照分组的想法``除掉''就行了。
$$
(n_1+n_2+\cdots+n_k)!\over n_1!n_2!\cdots n_k!
$$
有兴趣的同学可以阅读严格证明. 

下面直接求解一般情况的问题(并不限定字母的种类和个数)。设输入串为$S$,记$d(S)$
为S的各个排列中,字典序比S小的串的个数, 则可以用递推法求解$d(S)$.

假设我们输入了caba, 我们就有如下的一张图TBD, 
其中边上的字母表示“下一个字母”,$f(x)$表示多重集 $x$ 的全排列个数。例如,根据第
一个字母,可以把字典序小于caba的字符串分为3种:以a开头的,以b开头的, 以 c 开头的.
分别对应$d(caba)$的3棵子树。以a开头的所有串的字典序都小于caba,
所以剩下的字符可以任意排列,个数为$f(cba)$;同理,以b开头的所有串的字典序也都小于 caba,
个数为$f(caa)$;以c开头的串字典序不一定小于 caba,关键要看后3个字符 , 因此这部分的个
数为$d(aba)$,还需要继续往下分。感兴趣的同学可以参考\ti{P3014 Cow Line S}. 
只不过, 这里面的是具体的例子. 如果希望得到这个题的问题, 可以参考\ti{P2518 计数}. 

    \part{动态规划简介}

这一部分我们继续跟随状态机的模型, 探求问题的状态, 用一种比较聪明的方法来说明
如何比较聪明地遍历问题. 

\section{初步的问题}

\lec{数字三角形}{介绍} 数字三角形是一个由数字构成的三角形矩阵, 
每个数字代表路径上的权值. 问题的目标是找到从三角形的顶部到底部的路径, 
使得路径上的数字之和最大 (或最小) . 

这是一个耳熟能详的问题. 不过一个问题需要仔细地考虑: 为什么方程
$d[i][j]=a[i][j]+\max(d[i+1][j], d[i+1][j+1])$的后半段直接可以取最大值? 
事实上, 我们发现这是我们要求最大决定的 -- 如果连``从$(i+1,j)$''出发走到底部
的和都不是最大的, 加上$a[i][j]$之后也肯定不是最大的. 这个性质被称为\textbf{最优子结构
(optimal structure)}. 有``全局的最优解包含着局部的最优解''的想法. 具体如何
进行, 我们可以先使用搜索试试看, 之后分析出状态转移的规律, 就可以使用迭代的方式
进行实现了. 

我们来看下面的问题: 

\ti{P1004 方格取数} 想法1: 我会搜索! 我希望暴力枚举出所有可能的情况. 

上述做法直接解决了一整个大问题. 但是在解决的时候可能会出现一些重叠的子问题. 并不
太好. 我们想一想可以如何称为若干个子问题. 第一个想法是定义$f[i][j]$表示从(0,0)
走到$(i,j)$的过程. 这样可行吗? 看上去不行, 因为我们没有记录重复的数 - 重复的数是
没有办法再取的. 

那么走两次, 我们可以这样设计: $f[i_1][j_1][i_2][j_2]$表示所有从$(1,1),(1,1)$
走到$(i_1,j_1),(i_2,j_2)$的路径的最大值. 如何处理同一个格子被取两遍的呢? 只需要
保证当前处理的时候不相同即可. 

这里有4种情况, 因为每一个都可以从上来和从左来. 我们从最后一步考虑, 有四类情况. 
\file{1004-4D}

我们还可以把这个优化: 由于只能向下, 向右走, 不能走回头路, 当
$i_1+j_1 = i_2 +j_2$的时候, 格子才可能重合. 

\lec{充分条件和必要条件}{简介} 这里面, 我们说只有满足这个条件才可能重合, 
意味着只要重合了就一定会满足这个条件. 但是, $i_1+j_1 = i_2 +j_2$
无法推出一定重合. 我们就说他们``重合''是$i_1+j_1 = i_2 +j_2$的
\textbf{充分(sufficent)条件}, 
$i_1+j_1 = i_2 +j_2$是他们``重合''的\textbf{必要(necessary)条件}. 或者用
符号表示, 是这样的: ``($i_1+j_1 = i_2 +j_2)\Leftarrow$ 重合''. 

由于有一个等式了, 我们可以``消掉''一个量. 我们提出一种更加简化的方法:
让$k=i_1+j_1=i_2+j_2$.  
$f[k, i_1, i_2]$表示所有从$(1,1), (1,1)$到$(i_1, k-i_1), (i_2, k-i_2)$
路径的最大值. 

看上去这会让我们的转移方程难以写. 但经过分析, 也是可以做到的, 根据图, 有如下的四类情况
\begin{itemize}
    \item 下下: 从$f[k-1][i_1-1][i_2-1]$, 重合加上$w[i_1][j_1]$, 不重合加上$w[i_1][j_1]+w[i_2][j_2]$.
    \item ...
\end{itemize}

其实也没什么大不了, 只是把刚刚的状态浓缩到了$k$里面. 下面我们来看代码 \file{1004-3D}

\lec{技巧}{缩减编码复杂度} 事实上, 调代码是非常折磨人的. 如果我们能写出易于
检查的代码就好了. 这里面, 我们想把\codeword{f[k][i1][i2]}所减掉, 有没有什么
办法呢? 其实有两种办法: 第一种是使用引用: 输入
\codeword{int &x = f[k][i1][i2];} 这样下次使用的时候\codeword{x}就相当于
\codeword{f[k][x1][x2]}了. 另外一个可以使用\codeword{#define}关键字. 
不过记得使用\codeword{#undef}取消宏定义在使用结束的时候. 第一种情况用的很多. 

\begin{remark}
    编写易于理解, 不言自明的代码有些时候是保持思维逻辑清楚的很重要的一个习惯. 每当
    我们面临一个困难的问题的时候, 我们可以想一想有没有什么方法简化它. jyy
    老师在\href{https://zhuanlan.zhihu.com/p/619237809}{这篇文章}说过这样的一段话: 
    ``有个小朋友 Segmentation Fault 了也不知道哪里来的自信, 
    一口咬定是机器的问题. 给他换了机器, 并且教育了他机器永远是对的. 
    这个小插曲体现了编程的基础教育还有很大的缺憾, 使得竞赛选手大多都缺少真正的`编程' 训练,
    我看他们对着那长得要命的 \codeword{if (...dp[a][b][c][d][e][f][n^1]...)} 
    调的真叫一个累. 让我不由得想起若干年前某 NOI 金牌选手在某题爆零后对着一行有 20 
    个括号的代码哭的场景. '' 
\end{remark}


\ti{P1006 传纸条} 传纸条和上一个问题基本是类似的. 双倍经验的时间来了. 


\lec{DP的多重视角}{状态集合的角度} 我们可以用如下的检查单来思考一个(可能的)
动态规划问题. 因此, 我们可以把在这个属性下具有相同特征的内容划分为若干个
集合, 然后根据每一个划分, 找到相应的规律, 就可以得到对应的结果了. 

\begin{theorem}
    在思考动态规划问题的时候, 可以采用以下的检查单: 
    
    A. 状态表示:

        (1) 我状态表示归类的是哪一类的问题? 

        (2) 要在这一类问题上体现哪些属性? 

    B. 状态计算

        (1) 当前状态可以由哪些状态得来?

        (2) 对于这些内容, 这个属性前后的关系是什么? 
    
\end{theorem}

\lec{DP的多重视角}{DFS的视角} 有时候, 如果我们的递推关系过于奇怪, 我们可以
回到我们的老本行, 写出\textbf{没有额外变量}的dfs程序, 然后使用数组来递推. 
由于我们的函数调用关系, 这个依赖关系是在调用的时候就能够轻松做出来的. 由于
子问题有重叠, 每次我们只要把一个子问题计算一遍存起来就好了. 

记忆化搜索和递推二者都确保了同一状态至多只被求解一次. 但是它们实现这一点
的方式则略有不同: 递推通过设置明确的访问顺序来避免重复访问, 
记忆化搜索虽然没有明确规定访问顺序, 但通过给已经访问过的状态打标记的方式, 
同样达到了的目的. 

与递推相比, 记忆化搜索因为不用明确规定访问顺序, 在实现难度上有时低于递推. 
且能比较方便地处理边界情况. 但与此同时, 记忆化搜索难以使用一些更加聪明的优化
方式, 我们在接下来的背包问题中可以看到一些. 


接下来我们来看几个类似的问题. 

\lec{最长上升子序列问题}{简介} 我们现在考察最长上升子序列(LCS)的问题. 根据我们的
检查单, 我们决定定义状态$f[i]$表示集合
$a[i]$表示以$a[i]$为结尾的严格单调上升子序列. 要维护的属性
是最大值. 现在我们考虑所有到达了$f[i]$的内容. 看看它可以从哪来: 

\incfig{dp/figcheck.png}

分析了上面的内容, 我们就可以发现状态转移方程为 
$$f[i] = \max \{f[k]\} +1, \forall k\in [1..i-1], f[k]<f[i].$$

上升子序列给我们的感受是往上升. 那么下面我们来看一个既有上升又有下降的内容. 

\ti{\href{https://vjudge.net/problem/OpenJ_Bailian-2995}{登山}} 
我们可以按照中间是哪个点是最高点分析. 先分为
$a[0],a[1], a[2],\cdots, a[n-1], a[n]$是山峰这几类. 我们分别求出每一类的
长度最大值就是整个的最大值. 不是一般性, 如果峰值是第$k$个的最大长度, 并且左边选
哪些和右边的情况互不相干, 那么就在左边和右边分别跑一下LCS问题, 然后找到$\max$就行
了. 

在做模拟题的时候, 我们可能留意了``合唱队形(NOIP)''这个问题. 其实, 这个是一个对偶.
去掉多少人就是总数减去留下多少人. 

\begin{remark}
    对偶问题. 我们说两个问题是对偶的, 感觉上就是两个问题表达的是
    一个问题的两个方面. 或者更直观的说, 有一种对称性. 例如这个问题
    和合唱队形的问题; 到未来大家学习最大流和最小割, 他们都具有对偶的
    感受. 
\end{remark}

\ti{P2782 友好城市} 这里的要求是不交叉. 我们发现我们要求的序关系消失了. 
我们考察所有合法的建桥方式和上升子序列之间的联系: 对于任何一个合法的建桥方式, 
从一侧观察一边的点, 另一边都是严格上升的. 对于任意一个严格上升的子序列, 我们都
能够找到合法的架桥方式. 也就是他们之间构成双射. 所以我们按照自变量大小进行排序
看因变量的LCS就好了. 

其实, 没有交叉意味着没有逆序对. 如果你曾经实现过归并排序, 你一定对这个不会陌生. 

\lec{映射}{表达关系} 将给定集合的每个元素与另一个集合的一个或多个元素相关联的
一种思想. 我们在刚刚的问题里面发现了一对一的这样的情况, 因此可以断定两个问题
的大小是一样的. 

\lec{最大上升子序列}{之和} 这次, 我们想要知道你挑选出来的上升子序列里面, 其和
是多少. 你会发现, 最长上升子序列并不意味着最大的和. 我们又要按照刚刚的方法分析
了.  状态$f[i]$表示所有以$a[i]$为结尾的上升子序列, 属性是和的最大值. 状态
计算的划分是可以划分为上一个数字选的是空, $a[1], \cdots, a[i-1]$. 于是, 
我们就得出了状态转移方程: 

$$f[i] = \max \{f[k]+a[i]\} +, \forall k\in [1..i-1], f[k]<f[i]$$ 

\ti{\href{https://vjudge.net/problem/OpenJ_NOI-CH0206-8462}{大盗阿福}} 
直觉来看, 我们想要设置$f(i)$代表当前抢劫到了第$i$个店铺的最大收益. 于是, 当前
的状态被划分为两块: 抢劫第$i$家店铺, 得到$f[i-2]+w[i]$, 以及不抢劫
第$i$家店铺. 于是, 我们得到状态的转移方程为$f[i]=\max(f[i-2]+w[i], f[i-1]).$

这个状态需要依赖上面两维的状态. 如果我们只希望依赖上面一维的状态, 
我们还需要增加一维: 用$f(i, 1)$表示上一家
店铺被抢了, $f(i, 0)$表示上一家店铺没有抢. 因此, 我们就可以转移了. 

这种转移有一些头疼. 于是, 我们可以使用一个特殊的方法 - 请看

\incfig{dp/sma1.png}

我们下面来正式把这个说一说: 定义$f(i,0)$表示当前站在第$i$个建筑前面, 当前
状态位于$j$的所有走法, 得到的最大值. 下面决定状态转移方程. 考虑$f[i][0]$, 
有哪些走法可以走到0? 其实, 我们可以从上一个0走到0; 或者从1走到0. 因此, 
它们的最大值分别是$f[i-1][0]$和$f[i-1][1]$ - 毕竟没有选择这家店铺. 
下面考虑$f[i][1]$. 我们只能从$f[i-1][0]$走过来. 这样子, 获得的收益是
$f[i-1][0] + w[i]$. 综合去取$\max$即可. 图示如下:

\incfig{dp/tran1.png}

\ti{\href{https://www.luogu.com.cn/problem/T294782}{最长公共上升子序列}} 这个问题
我们定义状态$f[i][j]$为所有由第一个序列的前$i$个字母, 第二个序列的前$j$个字母构成的
公共上升子序列, 属性是要求最长的. 但是我们发现在转移的时候因为缺少条件, 我们还需要
知道现在结尾的数是多少, 以便于我们判断是不是可以向后增加. 具体地, 我们这样修改我们的
定义: ``状态$f[i][j]$为所有由第一个序列的前$i$个字母, 第二个序列的前$j$个字母构成的
公共上升子序列, 并且有$b[j]$结尾''. 

那么, 有哪些状态可以转移到了$f[i][j]$呢? 我们可以包含两类: 所有包含$a[i]$的 
公共上升子序列, 另外的是左右不包含$a[i]$的公共上升子序列. 第二类里面, 由于它最后不包含
第$i$个字母, 说明它只可能包含前$i-1$个字母. 即从状态$f[i-1][j]$转移来. 那第一类呢? 
根据状态的定义, 由于同时包含$a[i]$和$b[j]$. 由于$a[i]$是不确定发的, 我们需要继续细分, 
就像刚刚的LIS问题一样. 我们考虑序列的倒数第二个数. 有可能是空, 
$b[1], b[2], \cdots, b[j-1]$. 这样一来, 我们就从实际意义出发, 发现如果是$b[k]$作为
倒数第二个字符的话, 那么值应该是$f[i][k]+1$. 

\begin{remark}
    一个问题, 尤其是困难的问题, 搞清楚来龙去脉是重要的. 任何感觉到难的内容可能只是
    缺乏了前置应该了解的东西. 所以, 很多时候, 看一看它的历史, 你就能知道更加多样
    的东西. 甚至追寻着历史的规律, 有一天你也能为解决这一类问题添砖加瓦! 
\end{remark}

\ti{{股票买卖}} 题目叙述: 给一个长度为$N(1\leq N \leq 10^5)$的数组, 数组中的第$i$数字表示给定股票在
第$i$天的价格. 设计一个算法计算能获取的最大利润, 最多完成$k$笔交易. 你不能同时参与多笔交易( 你必须在再次购买前出售掉之前的股票) .
一次买入卖出合为一笔交易. 第一行包含整数$N,k(1\leq k\leq 100)$, 表示数组长度和最大交易数, 第二行$N$个
不超过10000的正整数, 表示完整的数组. 输出一个整数, 表示最大利润. 

我们发现在例子的情况下, 我们能进行的操作是``买入''和``卖出''. 造成结果是``手中有股''
和``手中无股票''. 这下子, 我们发现最好按照这样的划分方法, 才可以把原来的内容描述
清楚. 如果我们手中有货, 我们在下一天到来的时候既可以继续持有, 或者卖出, 同时得到
一定的收益(得到$w[i]$); 
如果我们手中无货, 那么下一天到来的时候, 我们可以买入 (并付出$w[i]$), 
或者按兵不动. 

我们效仿背包的情况: 假设现在进行到了第$i$天, 正在进行第$j$笔交易(买入就算做这笔交易), 有
$f[i][j][0]=\max(f[i-1][j][0], f[i-1][j-1][1]+w[i])$. 同样的有 
$f[i][j][1]=\max(f[i-1][j][1], f[i-1][j-1][0]-w[i])$. 

\begin{ques}
    如果卖出的时候, 使用了这个会导致全局最大值不对吗? 
\end{ques}

我们会遍历所有的空间, 正如我们前面所说, 这是一个``聪明的搜索'', 所有的状态都会被
计算到的. 

\ti{{\href{https://www.luogu.com.cn/problem/U298750}{股票买卖2}}}
我们这时候发现状态影响决策有手中有货, 手中无货的第一天, 以及手中无货大于等于
第二条(冷冻期). 

\incfig{dp/sta2.png}

转移方程, 根据上图就有: $f[i][0] = \max(f[i-1][0], f[i-1][2]-w[i])$; 
$f[i][1]=f[i-1][0]+w[i]$, 以及$f[i][2]=\max(f[i-1][1],f[i-1][2])$. 

运用状态机的视角真不错. 我们在很多时候在处理很多问题的时候也可以这样做. 

\section{背包问题} 

背包问题是一类很经典的问题. 我们首先介绍一些常见的策略, 然后仔细看一看
``0-1背包问题''. 背包问题选出的内容里面没有内在的关系. 有时候可以成为组合
类的DP. 

\lec{0-1背包问题}{简介} 假设你是一个背包客, 要去旅行, 但是你只能带一只背包. 
现在的问题是, 你面前有一些物品, 每个物品都有自己的重量和价值. 
你希望在背包的承重范围内尽可能装入最有价值的物品. 

然而, 你的背包有个限制: 
它只能承受一定重量的物品, 超过这个限制它就会撑破了. 所以你必须仔细考虑, 该怎样选择物品放进背包. 

这就是所谓的0-1背包问题: 你要在一系列物品中做出选择, 每个物品只有一个 (0个或1个) 存在的机会. 
你不能切割物品, 只能选择全部放入或不放入. 

\ti{P1048 采药} 这是一个最为普通的背包问题. 我们现在考虑如何设计方案, 以及有什么
好的办法来做这件事情. 我们设计状态$f[i][j]$表示在集合``考虑前$i$个物品, 总容量
为$j$''的价值的最大值. 那么根据最后一步, 可以把状态表示转化为两大类 - 要么选择第$i$个
要么不选第$i$个. 

不选的方案的话, 那么是$f[i-1][j]$, 如果选取的话, 那么有分为
之前的加上第$i$个. $f[i-1][j-v[i]]+w[i]$. 我们只要找到他们的最大值就好了. 
请看代码\file{P1048-2D}. 

下面我们来看他是如何转移的. 我们发现, 如果我们采用倒序的方式循环的话, 就可以被压缩到一维了. 
请看代码参考\file{P1048-1D}

\lec{完全背包问题}{介绍} 假设我们所有的这些内容都有无限件可以选呢? 这时候我们就可以
同样定义$f[i][j]$为所有只从前$i$个物品中选, 且总体积不超过$j$的选法的集合. 要维护的属性
同样是最大值. 下面我们来看状态的计算. 既然我们可以选0,1,2,$\cdots, s-1, s$个, 那么就需要
分别对于这个进行转移了. 选0个就是$f[i-1][j]$; 选1个就是$f[i-1][j-v[i]]+w[i]$;
选2个就是$f[i-1][j-2\times v[i]]+2\times w[i]$, $\cdots$. 
按照道理来讲, 这个可以写作代码了. 我们可以第一层枚举物品, 第二层枚举体积, 第三层枚举选多少个. 
如\file{complete-bp-primary}

我们来看一看有没有优化的空间. 我们观察到$f[i][j]$的表达式
$$
f[i][j] = \max(f[i-1][j], f[i-1][j-v]+w, f[i-1][j-2v]+2w, \cdots, f[i-1][j-sv]+sw)
$$
与$f[i][j-v]$的表达式有时候类似, 把$j$换为$j-v$即可: 
$$
f[i][j] = \max(\qquad ~~~f[i-1][j-v], f[i-1][j-2v]+w, f[i-1][j-3v]+2w, \cdots, f[i-1][j-sv]+(s-1)w)
$$
注意这里的$s$的表达式是不依赖于$v$的, 并且$s=\lceil j/v \rceil$. 

我们发现上面的项和下面的是可以对齐的, 他们之间有很多的性质. 比如, 下面的每一项比上面的每一项少了
一个$w$, 因此上面的最大值等于下面的最大值加上$w$. 于是$f[i][j]$的值就可以替换做
$$
f[i][j] = \max(f[i][j], f[i][j-v]+w).
$$

这就是我们得到的最终表达式. 同样可以优化为1维的, 只要把体积的枚举的顺序改为顺序的循环就行了. 
事实上, 当空间优化为1维的时候, 只有完全背包由于无限的关系, 需要从小到大循环. 我们可以使用
\file{complete-bp-1d}来看一看. 其中$v[i]$是重量, $w[i]$是价值. 

\lec{多重背包问题}{一个困难的做法} 多重背包是每一个里面有有限个物品的问题. 遵循刚刚的, 如果某一个物品有
$s$件, 那么转移方程就有
$$
f[i][j] = \max(f[i-1][j], f[i-1][j-v]+w, f[i-1][j-2v]+2w, \cdots, f[i-1][j-sv]+sw).
$$
同样的, 我们现在来看一看$f[i][j-v]$是什么: 
$$
f[i][j-v] = \max(\qquad ~~~ f[i-1][j-v], f[i-1][j-2v]+w, f[i-1][j-3v]+2w, \cdots, f[i-1][j-(s+1)v]+sw).
$$

唯一不同的地方是最后一项都是一样的, 并且上面的比下面的每一项多一个$w$. 但是关键不同是最后
多了一项. 我们的目标是求上面的最大值; 但是我们并不能根据下面的反推上面的最大值的. 那么, 
我们再看一项: 
\begin{align*}
    f[i][j-2v] &= \max(f[i-1][j-2v], f[i-1][j-3v]+w, f[i-1][j-4v]+2w, \cdots, f[i-1][j-(s+2)v]+sw)\\
    f[i][j-3v] &= \max(f[i-1][j-3v], f[i-1][j-4v]+w, f[i-1][j-5v]+2w, \cdots, f[i-1][j-(s+3)v]+sw)\\
    \cdots
\end{align*}
注意到$j-kv$是模$v$余数相同的, 于是考虑画一个数轴来看一看所有的情况. 
\incfig{dp/bpv.png}
也就是每一个需要用到前面的几个状态的最大值, 这个就可以用滑动窗口解决了. 请看代码\file{mulbp-deq}

\lec{多重背包问题}{拆分为0-1背包} 下面的这个思路就比较简单了. 可以把它拆分成0-1背包问题, 再使用
0-1背包的模板做就好了. 不过这里的划分也是有技巧的, 我们可以使用以前倍增介绍的一个方法, 按照二进制
拆分. 如\file{mulbp-bin}. 

下面我们来看上面介绍的一些对应的习题. 

\ti{P1049 装箱问题} 这里面没有了``价值''. 怎么做? 其实体积同时也是价值, 然后用0-1背包求解
就可以了. 请看\file{P1049}

\lec{二维费用的背包问题}{简介} 有时候我们的体积可能是二维的, 我们来看下面的例子: 

\ti{\href{https://vjudge.net/problem/OpenJ_Bailian-4102}{宠物小精灵之收服}} 这下子, 
费用变为二维的了. 费用之一是精灵球数量, 之二是皮卡丘的体力值; 价值就是小精灵的数量. 首先考虑状态
表示, 定义$f[i][j][k]$表示从前$i$个物品中选, 且花费1不超过$j$, 花费2不超过 $k$的选法中, 最大
的价值. 那么$f[i][j][k] = \max\{f[i-1][j][k], f[i-1][j-v_1[i]][k-v_2[i]]+1\}$. 当我们
把所有状态都算过了之后, 得到答案的时候就可以收服$f[k][N][M]$个精灵. 最少耗费的体力就看
$f[K][N][m]$的$m$最小是多少, 使得我们可以等于最大值. \file{2dcost} 
(所有的体积维度都是倒着循环的, 可以发现)

\ti{\href{https://www.luogu.com.cn/problem/U291791}{潜水员}} 这个问题类似于上面的问题. 
但是也有一些变动. 我们定义为
$f[i][j][k]$表示从前$i$个物品中选, 且花费1\textbf{恰好是}$j$, 花费2\textbf{恰好是}$k$的选法中, 最小的
价值. 那么$f[i][j][k] = \max\{f[i-1][j][k], f[i-1][j-v_1[i]][k-v_2[i]]+1\}$. 之后, 
枚举$j\geq m, k\geq n$的里面找一个最小值就可以了. 对应到代码上面, 我们只需要按照实际的含义
把$f[0][0][0]=0$, 其余是$f[0][j][k]=+\infty$, 表示不合法 -- 毕竟到不了.  \file{U291791}

\lec{背包的方案数}{简介} \ti{\href{https://vjudge.net/problem/OpenJ_Bailian-4004}{数字组合}}
我们定义状态表示$f[i][j]$为所有只从前$i$个物品中选取, 且总体积恰好是
$j$的方案的集合. 属性是方案数. 那么考虑不包含物品$i$的所有选法, 为$f[i-1][j]$, 以及包括
物品$i$的所有选法, 就是$f[i][j] = f[i-1][j]+f[i-1][j-v_i]$. \file{BL4004}

\lec{分组背包}{简介} \ti{\href{https://www.acwing.com/problem/content/9/}{分组背包问题}}

这下, 每一次是能从一个物品组内最多选择一个物品. 剩下的就和原来的一样了. 这样子, 定义$f[i][j]$定义
为只考虑前$i$个组内总体积不超过$j$的, 划分的方案依据是从当前的这个内选哪个物品. 如第0个, 第1个,
$\cdots$, 第$s_i$个. 因此转移方程就是
$$
f[i][j] = \max(f[i-1][j-v_{i,1}]+w_{i,1}, f[i-1][j-v_{i,2}]+_{i,2}+\cdots + f[i-1][j-v_{i,s}]+_{i,s})
$$
和0-1背包一样, 照样可以去掉第一维. 我们的做法如下: \file{grouping}

\ti{P1064 金明的预算方案} 这个本质上是一个分组背包的问题, 把每一个主件和附件的组合当做一个组, 
每一个组里面有一些选择: 购买主件; 购买主件和附件(所有的排列); 这样就是说有若干个物品组, 每个物品组
里面的每一个物品是当前的决策, 并且是互斥的. 然后使用分组背包的方法得到一个最大价值.  

\lec{树上的背包}{简介} 上面的问题如果可以由很多个子节点呢? 这就扩展为了树上的有依赖的
背包问题了. 下面我们来看一例: 

\ti{P2014} 我们首先定义状态, 我们可以使用递归的思路来解决这个问题. 我们会想, 对于这个点而言
不同的体积的时候得到的最大的价值是多少呢? 比如这个点是点$u$, 要求以$u$为根的时候在不同的体积
之下, 最大的价值是多少. 于是我们定义$f[u][j]$表示所有以$u$为根节点的子树中选取, 且总体积不
超过$j$的方案数中, 获得的最大的. 

接着来看如何做. 肯定, 要想选子树, 这个当前的根节点一定是要选的. 但是它的子节点就不一样了. 
对于$f[u][j]$的任何一个方案, 可以分为方案: (1) 从第一棵子树中选取的方案, (2) 从第二棵子树中选取的方案,...,
($m$) 从第$m$棵子树中选取的方案
每一棵子树每部也可以通过体积去划分.(不按照方案划分是因为代价太大了) 这样子, 我们就可以按照分组背包的方式
循环一遍. 用这样的方法我们就把诸多的一类问题用一个状态表示了. 

更通用的问题如下: \file{tree-bp}




\section{关于区间的问题}

有些动态规划问题, 我们设计状态需要考察一个区间. 我们从石子合并这个经典问题开始看起. 

\ti{P1775 石子合并(弱化版)} 假设这时候我们认为这是在一条链上的情形. 也就是不能首尾合并.
这时候, 我们定义$f[i][j]$表示所有从$i$到$j$合并的方案, 
属性是最小值. 下面我们来考虑状态的计算问题. 我们来看一看哪个可以到达这个状态. 
我们考虑合并两个区间, 会发现它的分界点不同. 所以这就启发我们使用不同的分界点去
划分现在的集合. 假设分界线落在$k$和$k+1$之间, 那么它需要的体力最小值就是
$f[l][k]+f[k][j]+\text{左右两边的和}$, 也就是先合并左边, 再合右边, 最后就把
两堆合在一起.  最后的是所有的子集的最小值. 状态转移很好写, 但是\textbf{注意循环顺序!}

转移方程: $f[i][j] = \min\{f[i][k]+f[k+1][j]+\sum_{s=i}^j a[i]\}$. 其中$k$从
$i$枚举到$j-1$. 状态空间是$n^2$, 需要枚举起点, 有$\mathcal O(n)$, 总共时间复杂度是
$\mathcal O(n^3)$. 计算$300^3=2.7\times 10^7$, 完全可以. 

接下来我们来看代码: \textbf{请留意循环顺序! } 按照区间长度从小到大枚举. \file{P1775}

从上面的代码中, 一般而言, 区间DP可以首先循环长度, 然后循环左端点, 之后算右端点, 最后枚举
分界点. 这样是使用循环去遍历状态. 正如我们前面所说, 我们也可以使用记忆化搜索的方法
写这个内容, 当转移不明确的时候. 

\begin{ques}
    如果每次允许合并相邻的$n$堆, 应该如何做? 说一说大致思路. 
\end{ques}

我们接下来的问题可以设置状态为前$i$个数成了$j$个的过程. 这相当于DP里面套了一层DP. 
我们这里不做讨论. 

\ti{P1880 石子合并} 下面我们来考虑环形的状况. 我们如何把环的情况展开成一条区间呢? 
因为环形剪掉一条边就成了一个链, 一个朴素的想法是我们
可以枚举缺口在哪. 就可以用区间DP的方法做了. 但这样的时间复杂度是$\mathcal O(n^4)$,
难以接受. 下面介绍一种优化方式: 

我们本质上是$n$个长度为$n$个链的式子合并问题. 我们可以这样做: 

\incfig{dp/cut.png}

这样一来, 我们使用长度为$2n$的区间, 就能保证我们只处理$n$个区间就可以枚举到所有的情况了. 
这样我们的复杂度是$\mathcal O((2n)^3)$. 这样的方法可以处理大多数的环形DP问题. 

请参看代码\file{P1880}. 

\ti{P1063 能量项链} 我们现在断环为链, 像上一个问题一样. 对于一个链, 我们定义状态的表示
$f[i][j]$为所有将$i..j$区间合并成为一个珠子的方式. 属性是维护最大值. 接着来看
合并的时候状态的计算. 我们来看一看哪个可以到达这个状态, 根据最后的不同点来划分. 
这个和上一个是类似的: 有一个分界线(在原来数组的视角下注意这时候是共用的). 根据这个
我们可以把集合划分为若干个子集. 其中分界线分别为$i+1, i+2, \cdots, r-2, r-1$.
假设当前的分界线是$k$的话, 那么就会有将$(i, k), (k, j)$最后将两个合并释放的能量. 
用数学公式写出来就是$f[i][k] + f[k][j] + w[l]\times w[k]\times w[r]$. 
这就是我们使用线性的做法, 现在我们考虑环形的. 运用上一个问题的技巧, 在后面一个$2n$的
链上面做DP就可以了. \file{P1062}

\ti{LOJP10149. \href{https://loj.ac/p/10149}{凸多边形的划分}} 这个问题需要我们一定的观察
与思考. 我们首先发现, 任意作一个三角形, 它就会把左边的和右边的三角形划分开. 因为题目
中有一个重要的条件 -- 互不相交. 这就保证了区间左右的独立性. 所有这样的方案把整个
内容分为了独立的三部分. 这是区间问题里面很重要的一个特征. 我们只要在这个状态下左半边
的划分, 右半边的划分和的最大值, 就可得到和上一个问题一样的想法. 也就是比如我要考虑
从1到$n$的划分, 中间选了点$k$作为分界点, 就有
$f[1][k]+f[k][n]+w[1]\times w[k] \times w[n]$. 我们来求每一个它们的最小值
就可以了. 这一个问题虽然和上一个问题构造非常不同, 但是其转移也非常的相似. 下面我们
详细看一下这个应该如何正式化: 

定义$f[l][r]$维护集合所有将$(l,l+1), (l+1, l+2), \cdots , (r-1, r), (r,l)$
划分为三角形的方案的值的最大值. 在进行状态计算的时候, 我们枚举$l+1, l+2,\cdots, r-2,
r-1$, 就可以把问题分为若干类. 对于每一类, 其转移到当前的值为$f[l][k]+f[k][r]+w[l]*w[k]*w[r]$.

很烦人的地方是, 这个问题需要写高精度. 因为$(10^9)^3\times100$大概会有30位数. 
\codeword{int}的最大值是2147483647, 9位数; \codeword{long long}的最大值是 
9223372036854775808, 19位数.  
我们应该秉持先做对, 再做好的原则进行. 也就是
先做对, 把样例和小测试数据做好, 然后再用高精度写剩余的部分. 
不加高精度的部分如\file{LOJP10149-part}所示. 

下面加上高精度. 为了方便起见我们直接用这个数组存位数, 直接整合进$f$数组里面. 
请看代码的\codeword{add}部分和\codeword{mul}部分. \file{LOJP10149}

\ti{P1040 加分二叉树} 我们看到这个问题, 发现其计算公式很像区间DP的计算的方式: 
分为三个独立的部分. 关键是, 这个中序遍历是不是具有这样的形式, 使得我们可以在上面
做区间DP呢? 

回顾: 现在有一棵树的中序遍历, 我们考察任意的一个子树, 可以发现其在序列里面一定是
连续的一段. 这就让我们可以进行选取根节点进行中序遍历. 我们定义$f[l][r]$为所有将
$l..r$区间构造成一个二叉树的情形. 属性是维护所有二叉树的最大值. 我们找到最后一个
不同点, 根据这些类划分为不同的集合. 我们按照根节点的位置划分. 这样就划分为了若干类.

和上面的问题一样, 如果根节点在第$k$个点的话, 最大值应该如何求? 应该是
$f[l][k-1]\times f[k+1][r]+w[k]$. 下面考虑应该如何记录方案. 

其实记录方案无非是决定最后在更新的时候再某个地方记上一笔: ``节点$k$已经成为了
这个子树的根.'' 于是定义 $g[l][r]$表示 $l..r$ 区间的根节点选哪个. 在输出
前序遍历的时候就先输出这里的根($g[1][n]=:R$)\sn{:=表示``定义做''. 冒号在被定义的表达式那一侧}
, 同时知道左子树的区间和柚子树区间
为$1..R-1, R+1..R$, 反复进行这个过程就行了. 

字典序最小的方案应该如何做? 实际上我们只要让根节点的值最小就好了. 也就是找到
最靠左的一个分界点. 只有在小于当前答案的时候才更新, 并且记录. 如\file{P1040}

我们看一看二维的区间DP. 这时候区间就看上去有点奇怪了. 

\ti{P5752 棋盘分割} 这里面看上去有一个陌生的统计量均方差, 不过不用担心. 
不过我们来看均方差的公式$\sigma = \sqrt{\sum_{i=1}^n(x_i-\overline{x}^2)
\over n}$, 要是想要这个带根号的最小, 就意味着可以求$\sigma^2$最小. 简单变形
就有: \sn{但其实我们可以不用变形的. 这里只是简单体会一下操纵求和记号.}
$$
    \begin{aligned}
        &~{\sum_{i=1}^{n} (x_i-\bar x)^2} \\
        &= \frac1n\sum_{i=1}^{n} (x_i^2-2x_i\bar x+\bar x^2) \\
        &= \frac1n \left(\sum_{i=1}^{n} x_i^2 - \bar x \sum_{i=1}^{n}2x_i + n\bar x^2\right)\\
        &= \frac1n \left(\sum_{i=1}^{n} x_i^2 - \bar x \cdot (2n\bar x) + n\bar x^2\right)\\
        &= \frac{\sum_{i=1}^n x_i^2}{n} - \bar x^2
    \end{aligned}
$$

\begin{remark}
    这个推导在概率论中是比较常见的. 
\end{remark}

这就是我们试图最小化的东西. 也就是所有部分平方和的最小值. 好, 下面我们来看动态规划部分.

定义$f[x_1][y_1][x_2][y_2][k]$表示子矩阵$(x_1, y_1), (x_2, y_2)$切分成
$k$不分的所有方案. 其中$x$是行, $y$是列. 维护的属性是$\sum_{i=1}^n(x_i-\overline{x}^2)$.
的最小值. 

接下来来看状态计算. 我们认为有沿着$x$轴切; 沿着$x$轴切. 一共各自有7种情况, 分别
选上面和下面的情况. 沿着$x$轴切有类似的情况. 
我们的目标是求每一类的最小值, 然后取$\min$. 对于每一类, 我们有上面继续切的分值, 
加上下面剩余的分值. 由于右边的和是固定的, 于是可以用二维的前缀和求出来. 最后求解就可以了. 

如果要用循环来实现, 那么会很复杂. 并且循环的顺序也可能一不留神写错. 这时候我们采用
记忆化搜索的方式完成本问题. \file{5752}

\section{树形DP}

下面我们把刚刚的这种思想扩展到树上, 看一看在树上会如何确定状态, 并且使用\codeword{dfs}
来自然地决定他们之间的转移. 

\ti{P1352 没有上司的舞会} 我们仍然使用前面的问题开始想起. 定义状态$f[u][0]$为 
从以$u$为根的这个子树中选择的方案, 但是不选$u$的方案; $f[u][1]$与之类似, 但是要
选择$u$的方案. 维护的属性是选择的点最大值. 接下来来看状态的计算. 如果要得到$f[u][0]$, 
先计算出来他们所有儿子的值$f[s_0][0], f[s_0][1], f[s_1][0], f[s_1][1],\cdots$. 
要想让整个最大, 那么我需要让所有的子树最大. 也就是
$f[u][0]=\max(f[s_0][0], f[s_0][1])+\max(f[s_1][0], f[s_1][1])+\cdots$.
那么$f[u][1]=h[u]+f[s_0][0]+f[s_1][0]+\cdots$. 这个一共有$2n$个状态, 需要枚举的是两个
儿子, 所以时间复杂度为$\mathcal O(n)$的. \file{P1352} 

\ti{树的直径} 给定一棵树, 树中包含$n$个节点, 编号为$1\sim n$和$n-1$条无向
边, 每条边有一个权值. 现在想找到树中的最长的直径. 

这次我们的边是无向边, 我们可以通过建立两条有向边的情况下来模拟无向边的情形. 下面考虑如何 
设计状态. 我们假设把所有的路径都枚举一遍, 在这些里面找到边的权重最大的. 我们先想着
把它分为若干类, 并在所有的进行取得max. 

首先随便找到一个点, 把它当做根节点. 我们在每条路径上面选择一个高度最高的点, 然后把
这个路径归到最高的点这个上面去. 我们按照这条路径上面最高的这个点去枚举, 似乎就会好
很多. 

那么如何进行状态的计算呢? 首先我们把所有的子节点的往下走的最大长度. 这个挂的点可以分为
两种情况: 第一种是一直往下走, 另一种是两个拼在一起, 也就是穿过了这条路径. 
第一种的计算方式就是往下走的最大
距离加上这条边的权值就行了. 第二种的计算方式, 就相当于是给我们了很多条边, 让我们在里面
任取两条, 使得拼出来的最大. 这样子我们肯定选最大的一个和次大的一个拼起来. 这就指示
我们用最大值和次大值相加得到这个问题的解答. 

如何求最大值和次大值呢? 我们可以使用一个循环的方式, 每次先更新最大值, 最大值更新了之后
把最大值给次大值就行了. 下面来看代码: 

注意搜索的时候不能向上搜索, 只能向下搜索, 否则会出现死循环. 于是dfs的时候可以加上一个参数
father. \file{diameter} 

如果是无权重的边, 有如下的算法亦可以做: 

\begin{itemize}
    \item 任取一点作为起点, 距离该点找一个最远的点$u$;
    \item 再找距离$u$最远的一点$v$, 此时, $u,v$之间的路径就是一条直径.
\end{itemize}

这个算法并不直观. 我们给出证明: 

\begin{proof}
    我们只要证明第一步找出的$u$一定是某个直径的起点. 假设任选一点$a$, 距离$a$
    最远的其中一点为$u$. 考虑反证法, 假设它不是一个直径的起点, 假设某一条真正的直径
    是$b\to c$. 我们现在按照此算法得到的直径与真正的直径有如下的几种情形: 

    (1) 两个直径是不相交的: 由于连通性, 从$a\to b$这条路径上面一定存在一个点可以
    (不一定是一步)走到$b\to c$这条路径上. 这条路径交于刚刚的两条路径为$x, y$. 
    由于$u$距离$a$最远, 因此$\text{dis}(x, u)\geq \text{dis}(x,y)+\text{dis}(y,c)$. 
    并且做移项并放大, 有$\text{dis}(x, u)+\text{dis}(x,y) \geq \text{dis}(y,c)$.
    那就说明, 沿着$y\to x\to u$的距离一定是大于$y\to c$的距离, 所以更长的直径是
    $b\to y\to x\to u$. 因此, $u$一定是某一条直径的端点. 
    
    (2) 两个直径之间有公共的端点, 记作$x$: 由于$u$是距离$a$最远的一点, 那么
    $\text{dis}{(x,u)}\geq \text{dis}{(x, c)}$. 由于$b\to c$是一条直径, $\text{dis}(b,u)\geq \text{dis}(b,c)$
    因此$b\to u$是一个更长的直径, 我们的证明了$u$一定在某一个顶点上. 
    
\end{proof}

\ti{\href{https://www.luogu.com.cn/problem/U261056}{树的中心}}
题目大意: 给定一棵树, 树中有n个结点(结点编号为$1\sim n$)
, 请求出该树的中心结点的编号. 树的中心指的是, 该结点离树中的其他结点, 最远距离最近. $(n\leq 10^5)$

一个节点距离最远的有哪些类呢? 首先可以是往下走, 其次可以是往上走的. 往下走的倒是
很好说, 直接是刚刚存的\codeword{dist}; 但是往上走的就不一样了. 往上走的话, 又可以分为两类: 一类是接着往上走; 
另一类是往下走. 往下走又分为两类: 第一类是走过了当前点, 另一类是没有走过当前点. 

往下走的情况, 对于走过了当前点而言, 那就加上次大值(因为不能再走回去绕圈圈); 对于没有
走过这个点而言, 那就加上最大值构成这个点的最优情形. 请看\file{center-with-weight}.

对于本问题, 可以对上述代码稍加改动得到: \file{center-with-weight-outnum}.

\ti{\href{https://loj.ac/p/10155}{LOJ10155 数字转换}} 我们发现可以用图的观点来
建模这道题. 如果$x$可以转换为$y$, 那么就把$x, y$之间连一条边. 由于每个数的约数之和
是给定的, 那么每个点至多有一个父节点. 因此会得到一些树的集合(注意不一定是一棵). 所以我们在
这些树里面找到最长的路径就可以了. 

关于枚举因数之和, 我们可以先枚举一个数, 再枚举哪些数是它的倍数, 这样就会快一些. 
第一次要枚举$n$次, 第二次$n/2$次, 第三次$n/3$次. 于是总的枚举次数是
$$
n\left(1+\frac12 +\frac13+\cdots + \frac1n\right).
$$
里面的$\sum_{i=1}^{n} 1/i$, 当$n\to \infty$的时候, 这个渐进复杂度是$\ln n + \gamma$的
\mn{这是因为$1+1/2+\cdots+1/n -\ln n$在$n\to \infty$的时候是有极限的. 因为
有极限的其中之一原则是随着$n$增大, 数列单调递减, 并且有下界(不会走下某一个值). 
$\gamma$大约是0.57122左右, 称为Euler常数.}. 
因此, 这个时间复杂度是
$\ln$级别的. 剩下的细节请参看文件\file{LOJ10155}

\ti{\href{https://loj.ac/p/10153}{LOJ10153 二叉苹果树}} 我们定义$f[i][j]$为以$i$为根的子树里面选择
$j$条边的最大价值. 那么, 有哪些节点可以到达$f[i][j]$呢? 肯定是要从它的子节点中找线索. 也就是我们需要考虑
子节点(例如叫$s$), 选0个($f[s][0]+w$), 选1个($f[s][1]+w$), $\cdots$, 选$j-1$个($f[s][j-1]+w$, 留一个
给$j$). 

\ti{\href{https://www.luogu.com.cn/problem/P2016}{P2016 战略游戏}}我们定义$f[u][0]$表示为所有以
$u$为子树中, 且不设置的最小值; 定义$f[u][1]$是选择, 其余与$f[u][0]$一致. 下面我们来看状态的计算: 

有哪些来可以到这里呢? 
看$f[u][0]$: 注意到这一个集合的每一个方案都可以分成若干部分, 每一个部分代表每个子树. 根据每个子树的
选法, 因为他们是独立的, 我们可以拼凑出整个稍微大一点的树的选法. 想让整个最小, 要让每一部分别最小. 
所以就是$\min\{f[s_1][1],f[s_2][1],\cdots, f[s_n][1]\}$. 

如果是$f[u][1]$, 那么就选了这个点, 就可选可不选. 
那么就是$\min\{f[s_1][1],f[s_1][0],f[s_2][1],f[s_2][0],\cdots, f[s_n][1],f[s_n][0]\}$.

下面我们来看代码: \file{P2016}

\ti{\href{https://loj.ac/p/10157}{LOJ10157 皇宫看守}} 我们可以按照我们的需求来设计状态. 
一般来说, 就是看一看当前节点有哪些情形. 我们发现, (0) 它可以被父节点看到; (1) 它可能在子节点上面设置; (2) 它可以自己上面
就有一个士兵三种状态. 所以我们顺势设置三个状态: $f[i][0]$表示它可以被父节点看到所有摆放方案的最小花费; 
$f[i][1]$表示被子节点看到的所有摆放方案的最小花费, $f[i][2]$同理, 不过是在当前节点
摆放的左右方案的最小花费. 

考虑状态的计算: 第一种情况是$f[i][0]=\sum_{j\in i\text{'s son}}\min\{f[j][1], f[j][2]\}$; 
第二种情况是$f[i][2]=\sum_{j\in i\text{'s son}}\min\{f[j][0], f[j][1], f[j][2]\}$; 最后
一种情况是$f[i][1]$. 这种情形比较麻烦, 因为我们不知道是哪一个节点看到的这个节点. 因此
我们需要枚举出来放在哪一个上面去了, 取最小的出来. 也就是
$f[i][1]=\min_k \{f[k][2]+\sum_{j\neq k}\min \{f[j][1], f[j][2]\}\}$. (首先$k$放置
警卫, 然后再在其他点不放置)

具体实现细节请看代码\file{LP2016}. 

\section*{闲聊与练习}

\begin{quote}
    著名数学教育家应先生:听不懂比听懂好.

    著名竞赛教练、国家队教练冷老师:``反应慢是上天给你的礼物''.

    \hfill ---韩涛\mn{北大毕业,IMO金牌教练.学而思集训队(星队)创始人之一.
    数十位学生进入数学国家集训队, 2位学生获得IMO金牌.}, 在朋友圈的一条推文.  
\end{quote}

\begin{quote}
    终点固然令人向往, 这一路的风景更是美不胜收. 

    \hfill --- 魏恒峰, 在南京大学\href{https://www.bilibili.com/video/BV1K24y1u7eA}{编译原理}课上
\end{quote}

\begin{quote}
    状态合并么? 你GET dp的本质了. 实际上, 如果你去看计数问题, 相当多的dp都是
    找到一个好的特征, 然后把集合划
    分成同构/存在某种数量映射关系的子集, 然后只算少量几次. 特征就是关键

    \hfill --- 蒋炎岩, 在一次聊天中
\end{quote}

\textbf{Takeaway Messages: }
\begin{itemize}
    \item (!) 知道了动态规划其实就是对于有\textbf{最优子结构(也就是我们一直在用最优的小问题试图拼最优的大问题)}问题设计的更聪明地枚举(设计好重叠的小情形);
    \item (!) 注意到了记忆数组和遍历顺序的重要性;
    \item 认识了不同种类的动态规划类型, 感受到了划分集合的重要性;
    \item 发现了其实和上一节一些计数问题的解决遵循相同的规则.
\end{itemize}

\centi{A组问题}

一. 回答下列小问题
\begin{itemize}[noitemsep]
    \item 简单使用动态规划解决问题的一些特征. 
    \item 使用迭代的方式和记忆化搜索的方式给你的感觉有什么相同和不同?
\end{itemize}
    \part{数论简介}

TBD: 介绍一些和数论有关的内容. 
    \part{组合数学与概率简介}

TBD: 可能需要更多的时间来理解一些问题... 介绍点容斥原理和二项式系数. 
    \part{从树状数组到线段树}

TBD: 这部分线段树的内容已经准备好, 重点在于准备标签的修改的阐释, 以及给出几个例子
最后修改原先文稿的排版即可. 
如果有空, 干脆把有些高级技术介绍了, 如动态开点等等... 

    \bibliographystyle{plain}
    \bibliography{refs}
    
\end{document}