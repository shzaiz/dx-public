\documentclass[10pt]{article}
\usepackage{NotesTeXV3,lipsum}
\usepackage{ctex}
%\usepackage{showframe}

\begin{document}
	\title{{基础内容简介(draft)}\\{\small \text{不推荐打印, 请直接在电脑上浏览. 因为PDF里面有一些链接可以点.}}\\{\small Section 1-7 Revised}}
	\author{Guangwei Zhang(AUGPath)}
	\affiliation{
    Zhengzhou No. 1 Middle School, China University of Geosciences\\
	\href{https://shzaiz.github.io/}{Website}\\
	\href{https://github.com/shzaiz}{GitHub}\\
	}
	\emailAdd{gwzhang@cug.edu.cn}

    \author{Nongyu Di(BLUESKY007)}
	\affiliation{
    Zhengzhou No. 1 Middle School, Lanzhou University\\
	\href{https://dnybluesky007.github.io/}{Website}\\
	\href{https://github.com/dnybluesky007}{GitHub}\\
	}
	\emailAdd{diny20@lzu.edu.cn}

	\maketitle
	
    \newpage

    \textbf{不推荐大家打印这份文稿.} 
    因为有一些链接(标注红色)需要在电脑上点击. 另外, 纸张
    应当印上更加珍贵的东西. 例如刘汝佳老师的《算法竞赛入门经典》这样的书本. 

    还有, 现在为止, 本文稿还未真正意义上写完. 仅做提前了解使用. 具体地, 
    8-11节的练习题和文稿暂时仅有一个大致的框架; 16-20节还没有对应的代码实现;
    21-22节还没有练习. 


	
    \pagestyle{fancynotes}
	
    
\part{C语言回顾}

有了计算机, 我们可以把很多重复的工作交给计算机完成.
这样, 人们就可以把更重要的经历放在更主要的事情上面去了. 
其中, 程序设计语言担当了我们人类世界与机器世界沟通的桥梁, 
我们只有通过程序设计语言(如C语言), 计算机才会按照我们
希望的方法工作. 当然, 我们对于计算机的期望很多时候是失败的
这时候, 我们只有通过一些外部工具, 来证明或者否定我们对于
计算机内部一些事情工作的原理. 

如果你感到调试程序困难, 可以参考\href{https://www.bilibili.com/video/BV1f54y1K7rQ}{调试理论与实践}.



\subsection*{闲聊与练习}

\begin{quote}

    \fbox{核心指导原则}  Don't Panic. (不要慌) 
    
    \hfill—— The Hitchhiker's Guide to the Galaxy

    如果你还没有入门, 仍然感到恐惧, 请记住: 坚持住, 进入未知领域, 从简单的、能理解的东西试起, 
    投入时间, 就有收获. 
    
    掉在队伍之后的同学, 即便是仅有一定的编程基础, 努力过的同学也一定能通过 (Yes!)

    \hfill --- 蒋炎岩, 在南京大学\href{http://jyywiki.cn/OS/OS_Guide}{操作系统}课前的提示
\end{quote}
    \part{递归问题}

\section{问题的简化和递归的过程}

我们常说: ``大事化小, 小事化了''. 比如, 你在做数学计算的$3+2+4+5$这个
表达式的时候, 你可能会自动先计算$3+2=5$, 然后再继续计算$5+4+5$. 这样
一来, 我们距离结果就更进一步了. 也就是问题变得更``小''了, 或者更``容易''
解决了. 有些时候, 我们甚至允许把整个过程用抽象的方法盖住了. 比如, 你做了
一个摄氏度转华氏度的转换器, 你可以用一个函数把它抽象, 这样一来, 下次使用
的时候就直接调用就行了. 

我们来看一些我们如何简化问题的一些例子: 

\begin{example}
    要计算一个正整数$n$的阶乘, 如果它不是1或者0, 那么计算$n*(n-1\text{的阶乘})$.
    用数学的语言来写就是
    $f(x)=\begin{cases}
    x\cdot f(x-1) & x>1\\
    1 & x=1
    \end{cases}$
    这就意味着, 我们每一次计算的阶乘都比原来的更靠近答案. 如果我们实际写一下$f(5)$, 
    我们就会有如下过程(为了方便起见, 我们使用$f(x)$表示$x$的阶乘
    ): 
    $$
        f(5) = 5\times f(4) = 5\times 4\times f(3) = 5\times 4\times 3 \times f(2)
        = 5\times 4\times 3\times 2\times f(1) = 5\times 4\times 3\times 2\times 1.  
    $$
\end{example}

\begin{example}
    要学习知识, 首先要认真理解课本的内容, 然后自己进行思考, 最后对于一些问题
    进行提问. % TODO
\end{example}


实际上, 有一类问题它比较特殊. 你会发现, 如果能够把小问题解决好了, 那么原来的大问题
就自然而然地解决好了. 这种情形, 我们一般认为是递归的问题. 

\begin{definition}[递归]
    递归的问题是这样解决的: 
    \begin{itemize}
        \item 如果给定的问题大小可以直接解决, 那么就直接解决了; 
        \item 否则, 把它转化为这个问题的更简单(通常会更小)的问题
    \end{itemize}
\end{definition}

一开始, 这样自己提及自己的内容的确让人困惑. 但是, 有一个有趣的方法, 就是假想
有一个小精灵帮助你解决问题 - 我喜欢称他为递归精灵. 
你唯一的任务是简化原来的问题, 或者在不必要或不可能简化的情况下直接解决它. 
递归精灵将使用与你无关的方法为你解决所有更简单的子问题. 

这样说来确实很困惑. 但是我们来看下面的几个例子: 

\begin{example}
    归并排序: 要排序一列数, 我们可以将待排序的序列分成两个子序列,
    然后分别对这两个子序列进行递归排序,最后将两个有序的子序列合并成一个有序的序列.
    \mn{在合并的过程中, 我们还可以计算逆序对.}
    这时候, 我们将问题分解成若干个相同或相似的相似的小问题来解决, 
    然后再将子问题的解合并起来, 得到原问题的解. \file{mg-sort}
\end{example}

\ti{P1908 逆序对} 我们可以借鉴快速排序的思想的合并过程中作为合并的时候统计. 
如果左边有$m$个逆序对, 右边有$n$个逆序对, 那么就会多出$mid-i+1$个. 见
代码\file{inversion-pair}. 

\begin{example}
    TBD: Hanoi 塔问题
\end{example}


从这里开始, 我们对于程序的执行的理解似乎就感觉有点模糊了. 不过我们总是可以
使用正确的工具来让我们了解更多. 具体地, 我们可以使用调试功能. 调试器可以
帮助我们窥探程序现在在执行哪一行, 执行的内容是什么. 以及执行到这一步里面的
变量有什么, 是什么值. 我们可以使用$\texttt{gdb}$来解答这个问题. 

我们可以把每一次这样的函数调用想象是一个状态. 所谓状态, 就是相当于给这时候
程序里面的内容拍了个照. 研究状态是如何变化的会让我们思路更加清晰. 我们首先从
Hanoi塔开始看起: 

TBD

\section{递归的结构}

\lec{引用}{重要的操作} 可能发生过这样的情形: 假设你是一个课代表, 你希望去某一个老师那里抱作业. 
但是你不知道这位老师住在哪. 现在, 有同学告诉你老师在309办公室. 这下, 你就可以使用
知道的``309''来找到老师了. 

看似这是一个不起眼的例子, 实际上, 这包含着计算机科学里面比较重要的内容: 使用一种间接
的方式来获得我们要的东西. 

\lec{链表}{单向} 链表是它由一系列节点组成, 每个节点包含两部分:
数据 (存储我们需要的信息) 和指针 (指向下一个节点的地址) . 
链表中的节点可以分散存储在内存中, 不像数组那样需要一段连续的内存空间. 
这就可以方便地在中间插入与删除元素. 但是我们的随机查找的功能就不能
很快地完成了. 因为它们在内存里面不是连续的. 

链表可以被看作是一个递归结构, 即链表的每个节点都可以看作是一个小型的链表. 
每个节点包含数据和指向下一个节点的指针, 而下一个节点又包含数据和指向更下一个节点的指针, 
以此类推. 
原来即使在一些结构里面, 递归的问题也是大有帮助的. 

\ti{UVA11998 Broken Keyboard} TBD 

\lec{链表}{双向} 当然有些问题我们还可以记录双向的信息, 也就是它的上一个和下一个. 这里有一个
经典的问题. 我们可能需要很长时间才能够把它调试正确. \ti{UVA12657 移动盒子}.

\lec{树状结构}{二叉树} 

我们窗外应该就有很多树. 我们现在来看树状的结构. 

二叉树是一种特殊的树结构,其中每个节点最多有两个子节点,通常称为左子节点和右子节点。
这个定义是递归的,因为二叉树的每个子树也是一个二叉树。

具体地说,二叉树可以为空,也就是没有节点的情况,这被称为空二叉树或空树。
如果二叉树不为空,则它由一个根节点组成,以及两个分别为根节点的左子树和右子树的二叉树。

递归定义的核心是,二叉树的每个子树仍然是一个二叉树,它们也遵循相同的定义:
最多有两个子节点,每个节点都可以有自己的左子节点和右子节点,或者为空。这种递归定义使得
我们可以在处理二叉树时使用递归的思想,对于每个节点,
我们可以递归地处理它的左子树和右子树。递归在二叉树的许多问题中都是非常常见和有效的解决方法。

一个问题是, 如何储存二叉树呢? 一个想法是, 我们假设考虑一个最满的二叉树, 应该是如下图所示的: 
\incfig{recursion/bintree.jpeg}
自然, 我们就可以为他们天然指定一个序号. 我们采用这样的方式: 对于一个节点编号为$o$
, 左孩子我们可以使用$o\times 2$来表示, 右孩子我们可以使用$o\times 2+1$来表示. 
这样子, 我们就可以来储存这棵树了. 

但问题又来了: 我们现在是按照满的样子去精简的. 假设我们这个是一条链, 也就是层数
很深, 但是节点数很稀疏, 这时候如何是好? 这时候, 我们用到这样的一个技术了:
我们主要维护一个新节点的``池子'', 里面的每一个表示一个新节点的编号.
每次我们要新建一个节点的话, 就可以通过让这个池子的数据加1的方式. 这时候, 我们 
可以使用编号的方式去更新与原来的关系. 实际上, 用一个``池子''这样的思路
其实是使用了数组模拟指针的思想. 现在使用指针可能会造成一些误解, 所以先使用
数组模拟的指针做大致了解. 

TBD: 添加数组模拟指针的代码...

既然树是递归的结构, 那么, 自然我们更希望使用递归的方法去处理它. 不过, 
因为它有两个子节点, 我们才感觉它和一般的线性的数据有一些不同. 

一个递归函数, 当然可能要决定什么时候递归, 什么时候对我们当前的节点做操作. 
时机是很重要的. 

\lec{二叉树的三种遍历模式}{前序遍历} 前序遍历的过程是, 先输出当前的节点, 再
递归地前序遍历左边的孩子, 再递归地前序遍历右边的孩子. 我们对于样例的树模拟一下. 
\begin{lstlisting}
void pre_order(int o){
    if(o does not exist) return;
    print(o);
    pre_order(o's left son);
    pre_order(o's right son);
}
\end{lstlisting}

\lec{二叉树的三种遍历模式}{中序遍历} 中序遍历的过程是, 先
递归地中序遍历左边的孩子, 再输出根节点,再递归地遍历右边的孩子. 
\begin{lstlisting}
    void in_order(int o){
        if(o does not exist) return;
        in_order(o's left son);
        print(o);
        in_order(o's right son);
    }
\end{lstlisting}

\lec{二叉树的三种遍历模式}{后序遍历} 后序遍历的过程是, 先
递归地中序遍历左边的孩子,再递归地遍历右边的孩子, 最后输出根节点. 
\begin{lstlisting}
    void in_order(int o){
        if(o does not exist) return;
        post_order(o's left son);
        post_order(o's right son);
        print(o);
    }
\end{lstlisting}

这三个内容看起来就是一些奇怪的交换顺序, 实际上, 我们有很多的有趣的性质可以从中
观察出来. 比如, 前序遍历的一个内容总是根. 那么, 能不能通过前序遍历的序列和中序遍历
的序列还原整个二叉树呢? 其实是可以的. 既然前序遍历的一个内容总是根, 中序遍历只要
找到这个根在哪就可以了. 中间的就是子树. 子树也可以按照同样的方法做. 

\ti{P1030 求先序排列} 这个问题只是变了一下形式, 后序遍历了. 但是后序遍历根节点
在最后输出. 和上面的讨论是一致的. 

我们之所以能够按照这种方法遍历, 说到底还是用好了递归的结构的定义. 下面我们来看一看
有什么有趣的遍历方法. 

\ti{UVA10562 Underdraw the trees} 题目大意: 你的任务是把多叉树写成括号的表示法. 
每个节点处了``-'', ``|'', `` ''(空格)用其他字符表示, 每个非叶子节点下方总会有一个
``|''字符, 下方是一排``-''字符, 恰好覆盖在所有的子节点上面. 单独的一行``\#''作为结束标记.

我们可以定义函数\cw{dfs(r,c)}表示\cw{r}行 \cw{c}列开始的内容. 下面有子树的条件
是下一行的这一列有\cw{|}记号(注意不要越界). 然后我们就可以寻找\cw{-} 的左边界, 
顺着\cw{-}走, 一旦发现下面有字符就继续递归下去. 

刚刚的问题甚至不是二叉树, 但是我们运用我们的方法照样可以继续下去. 

\ti{UVA297 Quadtrees} 这是一个四叉树的问题. 但是解决问题的思路也是类似的. 
这时候我们把两个内容都画出来就好了. 这样子模拟一遍就完成了. 

\ti{UVA806 Spatial Structures} 

接下来我们来看求最短路的一些方法. 除了一路走到底, 我们能不能边走边看呢? 我们可以
使用BFS的方法. 具体而言, 我们的策略如下: 
\begin{lstlisting}
queue=[起点]
while(queue不是空的){
    node = 队列的第一个元素;
    输出node;
    把所有与node能够到达的没有访问过的边放进来;
}
\end{lstlisting}

我们来看一个迷宫游戏, 并且在上面运用一下我们的BFS技术. TBD. 

我们发现我们搜到的结果其实就是一棵树. 

\lec{多说一句}{了解计算机程序的执行} 其实, 我们刚刚发现的树, 有一些类似的
妙用. 不过我们要加上允许环的形成. 也就是, 现在它是一些点和一些边的集合.
它甚至可以帮助我们理解我们的程序. 其实, 每一个程序都可以被抽象为一个
执行的图. 

到底什么是程序? 我们看上去我们会认为是我们的C代码, 经过下面的内容, 希望大家可以
对于``什么是程序''有一个不一样的答案. 我们会用刚刚我们了解到的``图''的知识, 
构建一个``状态机模型''. \cite{jyyos-prog}

``到底什么是程序''这样的问题是比较深刻的. 理论计算机学家深刻地研究了程序语言应该
有的语义, 执行的过程等等. 但是我们从一个更加简化的角度来看, \textbf{程序就是
状态机}. 每一个状态就相当于一个节点里的一些数据, 不同状态之间经过程序语句
进行转移. 一个粗浅的理解是是: 状态就是``堆 + 栈''(存放着我们的变量等), 初始状态
就是``main的第一条语句'', 迁移就是``执行一条简单语句''. 因为任何一个C程序都可以
写成一个非复合语句的C代码, 并且的确有\href{https://cil-project.github.io/cil/}{这样的工具}和\href{https://gitlab.com/zsaleeba/picoc}{解释器}! 

这样的过程会对我们的调试代码带来好处. 比如, 我们可以使用\cw{gdb}来检查我们
感兴趣的输出, 同时我们可以使用\cw{printf}指令向我们输出感兴趣的调试信息. 

\subsection*{闲聊与练习} 

\begin{quote}
    孫子曰:凡治眾如治寡,分數是也。鬥眾如鬥寡,形名是也。三軍之眾,可使必受敵而無敗者,奇正是也。兵之所加,如以碬投卵者,虛實是也。

    Sunzi said: The control of a large force is the same principle as
    the control of a few men: it is merely a question of dividing up
    their numbers. Fighting with a large army under your command is 
    nowise different from fighting with a small one: it is merely a 
    question of instituting signs and signals. To ensure that your 
    whole host may withstand the brunt of the enemy's attack and 
    remain unshaken - this is effected by maneuvers direct and 
    indirect. That the impact of your army may be like a grindstone 
    dashed against an egg - this is effected by the science of weak 
    points and strong.
    
    \hfill ---  《孫子兵法·兵勢》
\end{quote}

\begin{exercise}{翻炒煎饼: 选自\cite{algobook}第一章问题4}
假设你得到一堆$n$个不同大小的煎饼。你想把薄煎饼排个序,这样小煎饼就在大煎饼的上
面。你唯一能做的就是翻转——在顶部的$k(k=1,2,\cdots, n)$个煎饼下面插入一把刀然后将它们全部翻转。
\incfig{recursion/pancake.png}
(1) 描述一种算法,使用尽可能少的翻转对一堆任意的$n$个煎饼进行排序。在最坏的情况下,你的算法到底执行了多少次翻转?

(2) 现在假设每个煎饼的一面都烧焦了。描述一种算法,对任意堆叠的$n$个煎饼进行排序,使每个煎饼烧焦的一面朝下,同样保证翻转次数尽可能少。在最坏的情况下,你的算法到底执行了多少次翻转?

(3) 使用你刚刚思考的结果, 完成\ti{UVA120 Stacks of Flapjacks}. 
\end{exercise}

\begin{exercise}{图像旋转: 选自\cite{algobook}第一章问题9}
假设我们想要将一个 $n \times n$ 的像素地图顺时针旋转 $90^\circ$($n$是2的
若干次幂)。一种方法是将像素地图分成四个 $\frac{n}{2} \times \frac{n}{2}$ 
的块,使用五次块传送将每个块移动到其正确的位置,然后递归地旋转每个块。
(为什么是五次?和汉诺塔问题需要第三个柱子的原因一样。)
另一种方法是首先递归地旋转块,然后将它们放到正确的位置。

\incfig{recursion/rotate.png}

\end{exercise}

\begin{exercise}{k-d树: 选自\cite{algobook}第一章问题25}
    假设我们有 $n$ 个散布在二维的盒子内的点。``k-d树''通过将这些点划分用递归的方
    式如下:首先,我们使用一条垂直的线将盒子分成两个较小的盒子,然后使用水平线将每个较小的盒子再次划分,如此反复进行,始终在水平和垂直划分之间交替。每次划分盒子时,划分线会通过盒子内的一个中位点(不在边界上)尽可能均匀地划分剩余的内部点。如果一个盒子不包含任何点,我们就不再继续划分它;这些最终为空的盒子称为单元(cells)。

    (1) 最后由多少个单元? 用$n$表示.  

    (2) 在最坏的情况下,一条水平线到底能穿过多少个单元?用$n$表示. 
\end{exercise}
    \part{一些有趣的思想}

计算机也许是快的, 但不是无限快的. 内存空间可能是廉价的, 但不是免费的. 
所以, 无论是时间还是空间, 都是有限的资源. 我们必须用一些有趣的思想, 聪明地
抉择使用时间或者空间. 

\section{二分法}

我们先考虑这样的一个猜数字游戏: 假设有一个人选定了一个秘密数字, 并让你来猜这个数字是多少.
这个秘密数字是在一个已知范围内的整数. 你可以每次猜一个数字, 然后得到一个提示: 告诉你
该数字是猜测的秘密数字的偏大还是偏小, 或者是猜中了. 
根据这个提示, 你要做的是继续猜测直到猜中为止. 你的目标是用最少的猜测次数找到秘密数字. 

在上面的问题中, 我们可以找到某个性质的边界, 其中分别是小于这个数的和大于等于
这个数的. 也就是说, 我们要二分一个问题, 就是看一看这个边界是不是能够找到. 

在这一部分中, 我们首先会叙述这个的一般原理, 然后观察几个基本的问题以及几个
写代码的范式 - 很多时候写二分有关的代码是很容易犯错的. 结果就是无尽地死循环.
但是幸运的是, 我们可以避免这件事情发生. 


\lec{整数二分}{原理} 我们的目标是找一个性质的边界. 例如, 我们有如下的边界: 
并且有一个命题$P$, 左边的红色的部分是不满足$P$的, 右边的是满足$P$的. 

\incfig{opt-search/bsearch.png}

那么, 要找到红色的最右边的那个, 就(1)首先要找到一个中间值\codeword{mid=(l+r+1)>>1}
, (2)判断中间值是不是满足性质$P$, 也就是\codeword{check(mid)}. (2.1)如果$P$满足, 
那么\codeword{l=mid}; (2.2)如果$P$不满足, 那么\codeword{r=mid-1}. 返回
到(1), 重复执行, 直到\codeword{r>=l}. 

如果要找到绿颜色最左边的那一个, 和上面的问题相仿, 还是
(1)首先要找到一个中间值\codeword{mid=(l+r)>>1}
, (2)判断中间值是不是满足性质$P$, 也就是\codeword{check(mid)}. (2.1)如果$P$满足, 
那么\codeword{l=mid}; (2.2)如果$P$不满足, 那么\codeword{r=mid+1}. 返回
到(1), 重复执行, 直到\codeword{r>=l}. 

我们发现上述只是在取\codeword{mid}的时候和修改\codeword{l, r}的时候发生了
一点小问题. 这是因为C中的数组的舍去问题. 如果不这样做, 有时候会发生死循环 - 就是说
在锁定只有两个的时候, 不额外加一的时候, 可能会导致$l$在执行$(l+r)/2$之后还是$l$
. 这样就相当于什么都没有更新. 肯定不是我们想要的. 

\lec{实数范围的二分}{原理} 如果我们希望在实数范围上面二分, 就没有那个恼人的
``边界情况(corner case)''了. 我们一般有两种办法. 其一是让它二分执行某一个很大的次数
(例如1000次); 另一种是看一看当前是否达到了精度. 

\lec{练习}{更多的例子}\ti{P1163 银行贷款} 个人认为这个题面似乎有点表述不清. 我们采用另一个更严谨
的题目: 给出$n,m,k$, 求贷款者向银行支付的利率$p$, 使得: 
$$
n={m\over 1+p} +{m\over (1+p)^2}+{m\over (1+p)^3}+\cdots + {m\over (1+p)^k} 
$$
其中$p$保留0.1\%.  \mn{果然使用数学公式是很容易表达的}

Idea1. 我们来``猜测''$p$, 然后根据我们的猜测根据公式计算, 看一看它到底还的
多还是还的少. 如果多了, 就稍微把$p$往下调一点, 少了就把$p$往上调一点. 不过这道题
也有够坑的 -- 有的利率答案居然高达300\%! 所以二分的边界需要设置为300\%才行. 
我这里只让它执行了10000次二分操作 -- 毕竟最后的精度不高. \file{P1163}

\begin{remark}
    注意保留精度! 使用pow进行求和可能会扩大误差, 达到最后会差大约
    200元. 
\end{remark}

Idea2. 如果学习过了一些数学, 这个问题还可以使用数学的方法推演. 形如这样的
叫做等比数列, 意思是后一项除以前一项, 结果总是一个常数. 大家耳熟能详的
2, 4, 8, 16, 32, $\dots$ 这一串数列就是一个典型的等比数列, 其中
通向是$2^n$. 其中$n$是第几项(从1开始编号). 也就是说, 我要想知道第二项
是多少, 就要带入$n=2$, 结果就是$2^2=4.$ 

\lec{等比数列}{求和} 等比数列如何求和? 这就需要一些技巧: 我们假设等比
数列的通项是$a[n]=a_1q^{n-1}$, 
那么$S_n=a_1 + a_2q + a_3 q^2 +\cdots +a_n q^{n-1}$. 

我们发现这里面有很多的东西, 所以我们得想个办法把它们消掉. 采取两边同乘以
$q$, 两式相减, 就有神奇的效果. 这个方法也叫做错位相减法. 

推导过程如下所示: 
\begin{example}
    假设我们有一个等比数列: \(a, ar, ar^2, ar^3, \ldots, ar^{n-1}\) , 其中 \(a\) 是首项,  \(r\) 是公比,  \(n\) 是项数. 

我们想要求这个等比数列的和, 表示为 \(S_n\). 
首先, 我们将数列的前 \(n\) 项相加: 
\[ S_n = a + ar + ar^2 + ar^3 + \ldots + ar^{n-1} \]
接下来, 我们将 \(S_n\) 乘以公比 \(r\): 
\[ rS_n = ar + ar^2 + ar^3 + \ldots + ar^{n-1} + ar^n \]
接下来, 我们从 \(rS_n\) 中减去 \(S_n\): 
\[ rS_n - S_n = (ar + ar^2 + ar^3 + \ldots + ar^{n-1} + ar^n) - (a + ar + ar^2 + ar^3 + \ldots + ar^{n-1}) \]
注意, 在括号内的部分可以通过消去相同项来简化. 我们得到: 
\[ rS_n - S_n = ar^n - a \]
接下来, 将 \(S_n\) 提取出来: 
\[ S_n(r - 1) = ar^n - a \]
现在, 将 \(S_n\) 解出来: 
\[ S_n = \frac{ar^n - a}{r - 1} \]
这就是等比数列的求和公式. 
如果公比 \(r = 1\), 那么这个等比数列就变成了等差数列, 求和公式变为: 
\[ S_n = \frac{n}{2}(a + l) \]
其中,  \(l\) 是数列的最后一项. 
\end{example}

好了, 经过上面的推导, 我们就可以得到等比数列的求和: $S_n=\frac{a_1(1-q^n)}{1-q}$.


但是我们发现这个东西并不好解答... 确实, 我们并不能一味地通过一种方法解答
问题. 当我们遇到困难的时候就要多换角度. 

\ti{P2249 查找} 这个就是最基本的内容了. 直接参考代码就可以了! 注意刚刚
说过的一个问题: 到底是左端点还是右端点. \file{P2249}\mn{
    整数相关的二分的算法bug是比较隐蔽的. Java标准库中一个类似的查找函数
    使用了类似的二分方法. 但是很不幸这段代码写错了. 
    这个Bug在Java的数组标准库里面待了9年. 
    \href{https://dev.to/matheusgomes062/a-bug-was-found-in-java-after-almost-9-years-of-hiding-2d4k}{这里是原始文章.}}


\ti{P1676 Aggressive Cows G} 这个是最小值最大的问题, 意味着我们一般使用
按照答案二分的策略. 我们首先猜一个答案, 然后去施展我们应该有的构造, 最后
来看一看这个是不是太小了. 

我们可以假设牛棚都是空的, \codeword{check}时如果当前牛棚与上一个住上牛的牛棚之间
的距离\codeword{dis>=mid}, 我们就可以让这个牛棚里住上牛, 反之向更远的距离寻找牛
棚. 这是个贪心算法. 如果最后能安排的牛总数小于总的牛数, 那么就可以扩大需求. 
(\codeword{r=mid}) 
反之, 就要缩小(\codeword{l=mid+1}). 

\begin{ques}
    为什么这个贪心算法是对的? 
\end{ques}

我们说: 按照上面的构造, 一定是``最省''的. 并且我们只要能说明只要不按照这样做
不一定是最省的就可以了. 也就是, 最小值可能会变得更小. 

\ti{P2678 跳石头}  这个仍然是最小值最大的问题. 和上一个问题是类似的. 自己试着
感受一下吧! \file{P2678}

\ti{P3853 路标设置} 这个和上面的问题也是一样的. 自己动手试一试吧!  \file{P3853}

\ti{P1314 聪明的质检员} 这个虽然标号的颜色是绿色的, 但是仍然逃不过二分答案的
区间. 不过, 这里面可能有些符号难以阅读. 我们来简单阅读一下: 

\lec{求和符号}{简介} 求和记号是一大堆连加记号的缩写. 简单来说, 只是一个省略
而已, 并没有万能的公式可以求和任何事情. 

\lec{Iverson的括号}{简介} Iverson记号写作$[..]$其中, 里面的$..$是一个
布尔表达式. 当里面的结果是真的时候, 值为1, 否则值为0. 

Iverson括号可以和求和一起搭配使用, 来达到简化求和记号的作用. 比如, 我们要交换
两个求和记号的时候, 更好的想法可能是用这样的方法\mn{更多的内容可以参看
\cite{knuth1989concrete}2.4节 多重和式(MULTIPLE SUMS)}: 
$$
[1\leq j\leq n][j\leq k \leq n]=[1\leq j \leq k \leq n] = 
[1\leq k\leq n][1\leq j\leq k]. 
$$

介绍了刚刚的内容, 我们来简单梳理一下这个问题. 

我们要得到$\min |s-y|$, 就必须找到合适的$W$, 进而得到对应的$y$. 并且另一个
观察是: $y$ 越大, $W$越小. 当$y<s$时, $y$偏小, 我们就要减小$W$; 当$y=s$的
时候, 我们就得到了我们想要的结果. 当$y>s$时, $y$偏小, 我们就要增大$W$. 

我们求$y$的过程满足单调性, 因此使用二分的方法即可. 到这里, 我们能够得到部分分. 
查询的部分有个双重的for循环. 这部分使用前缀和优化一下就好. 我们马上会提及. \file{P1314-partial}

\subsection*{闲聊与练习}

\begin{exercise}{二分的一些练习问题}
    可以实现以下的练习题: 
    
    (1) P1182 数列分段 Section II

    (2) P1873 [COCI2011-2012] EKO / 砍树    
\end{exercise}

\begin{exercise}{三分法}
    我们来看题目\ti{P3382 三分法}.

    三分法的基本的用途是求单峰函数的极值点. 我们以求函数的极大值为例, 可以每次对一个区间$[l,r]$求三等分点\cw{lsec}和\cw{rsec}, 
    \begin{itemize}[noitemsep]
        \item 如果$f(\texttt{lsec}) < f(\texttt{rsec})$ , 说明极大值一定在$[\texttt{lsec},r]$内取到, 因为如果在$[0,\texttt{lsec})$内, 那\texttt{rsec}一定处于单调下降的区间内, 它的函数值不可能大于lsec的函数值. 于是就可以令$l=\texttt{lsec}$并继续. 
        \item 如果$f(\texttt{lsec})$ > $f(\texttt{rsec})$, 同理, 极大值一定在$[l,\texttt{rsec}]$内取到, 令$r=\texttt{rsec}$并继续. 
        \item 可以重复上面的两个步骤, 直到小于给定的精度. 
    \end{itemize}

    本问题的题解里面有很多同学说使用求导之后二分函数的零点就可以了. 那么什么是求导? 
    我们可以欣赏3Blue1Brown带来的\href{https://www.bilibili.com/video/BV1qW411N7FU}{《微积分的本质》}系列视频. 请注意, 这类视频只是提供了一个最基本的
    介绍, 并没有做非常严格的讨论(因为做出严格的讨论大家初学的时候就看不懂了). 
    所以, \textbf{请不要做名词党.}\mn{发现一个现象就是一些资历尚浅的学生(主要指我自己)有时会通过B站/知乎等互联网媒介去不加区分的接受知识和观点, 而又由于学识不足容易囫囵吞枣牵强附会, 从而走向了名词党的道路. 与此同时, 因为互联网大家更容易看到各种大神的发展路径, 不加辨别的让自己和他们比较也只会给自己徒增压力. (一段数心写的, 我自己很认同的自我反思)}
    
\end{exercise}




\section{前缀和与差分}

\lec{前缀和}{普通版本}现在有一个数组, 请问$\sum_{i=l}^r a_i$等于多少? 我们很容易用for循环实现. 
但是, 如果这样的事情会发生多达$10^5$, 应该怎么办? 一个好的想法是我们可以把
他们累加起来. 

\begin{definition}
    一个数组$a$, 它的前缀和数组$s$的通项为$s_i = a_1 + \cdots + a_i$. 
\end{definition}

这时候要想求$l\sim r$的和就求$s_r - s_l$即可. 

\begin{ques}
    既然有前缀和, 那么你认为什么操作下积可以被前缀吗? 你觉得能够前缀的问题
    有哪些特征?  
\end{ques}

我们发现上述的前缀和问题能够胜任查询问题, 但是对于修改操作并没有办法很好的胜任
因为单点进行修改之后, 其之后的前缀和都要发生变化. 

\lec{前缀和}{何必要前缀``和''?}事实上, 前缀和刻画了``连续进行若干次操作, 产生的一个综合影响可以通过某种手段
撤销. '' 比如, 我们如果连着加他们, 到最后可以使用减法把影响的区间消除. 
减法在数学中称为加法的``逆(inverse)运算''. 普通乘法的逆运算是除法. 

事实上, 运算这件事情可以被定义得很广泛. 比如, 你可以在正方形纸片上面定义一个
运算, 叫做``向右旋转90度''. 它的逆运算可以是``向左旋转90度'', 或者说
``连续做3次向右旋转90度''. 

下面我们来看一个比较奇怪的, 但是也能用上述的思想做的内容. 

\begin{example}
    现在有编号为$0\sim 10$一共$10$个球, 我们现在有若干个区间的对换. 具体地, 
    对于区间$[l..r]$的对换之后, 如果原来这方面的球的编号是
    $\cdots, a_l, a_{l+1}, \cdots, a_r, \cdots$, 那
    么经过这次对换之后, 这个区间的球
    的顺序就变成了
    $\cdots, a_{l+1}, a_{l+2}, \cdots , a_r, a_l,\cdots $. 

    现在你有$n$条操作规则, 每条操作规则就是两个数$l,r$. 现在, 我们想知道
    你连续执行编号$a$到编号$b$的操作规则之后, 得到的内容是多少. 注意有$m$次
    查询. 

    数据范围: $1\leq n, m \leq 10^5, 0\leq a, b\leq 9, 1\leq l\leq r\leq n.$
    \file{extra-prob}
\end{example}

我们如果这时候把``交换''当做一个运算, 运算的``数''就是你现在交换的
区间左端点和右端点, 这样子就和刚刚加法减法的前缀和类似了. 事实上, 这样的
对换在后续学习中是很重要的. 

\begin{remark}
    重要的对换: 如果你之后学习了Polya定理, 其中有一个重要的结论是任何一个置换都可以分解成
    若干个对换的复合. 这会对于你计数带有对称性的内容带来很大的帮助. 
    
    另外, 在数学中, 抽象代数中的群也有类似的刻画. 同样也有更加一般化的结论
    和内容. 不过要是学习这个, 必须有足够扎实的数学基础和对于许多内容的熟练
    掌握(如数学分析, 高等代数等基础课程)
    在这里我们不做讨论. 
\end{remark}

当然, 上述的内容只是一个简单的例子. 当你学习了更多的结构的时候, 很多结构
天然地满足这个性质. 到时候请多加留意. 

\lec{差分}{普通版本} 我们发现, 前缀和让我们拥有在$\mathcal O(1)$时间查询的
能力. 但是如果修改起来可能就麻烦了. 这里, 我们介绍一种方法, 使得我们可以在
$\mathcal O(1)$时间内修改, 并且能够$\mathcal O(n)$查询出来单点的值. 

我们现在的问题是有一个数组$a$, 每一次, 我要向$l..r$的区间内的元素加上一个
值$d$. 最后只有一次询问, 问我现在第几个元素被改成几了. 这样的修改会发生很多
次, 因此我们不能使用for循环来做. 

我们发现, 在对于区间一整个加的操作中, 我们在这一个区间加和的过程中, 区间
内部的两个数之间的\textbf{差}一直不变. 于是我们试着引入差分的定义: 

\begin{definition}
    对于一个数组$a$, 我们定义$d_i=a_i-a_{i-1}$, 那么$d$数组为原数组
    的差分(difference)数组. 
\end{definition}

我们发现, 如果要在原数组的$[l..r]$上加上一个数$x$, 只要在$d_l$上加上$x$, 
在$d_r$上减去$x$. 

挺有趣: 刚刚使用了累加, 我们才能得到了一个可以胜任区间求和, 但是做不了区间
修改的东西. 现在我们让每一个内容是它减去它前面的内容, 居然可以胜任修改, 
但是无法胜任区间的求和. 

那么, 我们的原数组$d$, 这个数组$a$, 以及前缀和数组之间$s$有什么关系呢? 
经过不复杂的数学推导, 我们可以发现: 

\incfig{opt-search/relation.png}

\begin{remark}
    这个关系, 在你上了高中, 接触到了路程, 速度, 加速度的关系的时候, 
    会发现它们是出奇的一致的. 为什么? 路程, 速度, 加速度的关系就似乎
    是这里的$x, v,a$的关系. 完整的知识在大学才能揭晓 -- 那时候
    你会学习数学分析, 更进一步地看一看在连续的情形下, 我们是如何做
    ``前缀和''的. 
\end{remark}

\lec{差分}{加一个等差数列?} 如果我们要在之间加一个等差数列, 那该怎么办?
比如原数列是$1,2,3,4,5$, 在区间$[1..3]$加上等差数列$2, 4, 6$, 最后
的结果是$3, 6, 9, 4, 5$. 

我们发现, 我们让原来的差分数组再差分一次不就好了! 等差数列再次差分, 就只要
在前面加一个数, 在后面减掉一个数了, 就像刚才一样. 这是差分的一个重要的性质.

在练习中, 你会看到有哪些差分做起来是好做的. 你同时也会发现很多奇妙的公式. 

\lec{差分}{加一个平方数列?} 这次我们使劲差分, 差分到三次, 你就会发现, 他们
就会奇迹般地出现出来0的样式了. 

\begin{ques}
    为什么是差分三次?
\end{ques}

事实上, 我们会发现每次差分之后, 得到的内容就会消掉一次. 也就是从二次变到
一次, 再到0次. 在0次的情形, 就是我们最开心的情况了. 如果下次要加上一些单项式
的组合, 其实同样的方法也是适用的. 

\lec{前缀和}{二维的前缀和} 前缀和有另一个扩展的方向. 我们能不能扩展到二维的前缀
和? 我们可以这样做. 我们仿照一维前缀和的定义, 使用$S[i][j]$表示第$i$行$j$列格子左上部分所有元素的和. 那么它的递推式是什么? 
\incfig{opt-search/2d-prefix-sum.png}

于是, 我们就可以写出来二维前缀和的代码. \file{prefix-sum-2d}

\subsection*{闲聊与练习}

\begin{exercise}{二维差分}
    二维差分大概的思想就是选择一个矩形区域, 将区域内的所有元素增加或减少一个数. 
    但是二维差分的思维的还是有的.
    
    实际上, 这种由一维变到二维, 考虑难度陡然上升的例子是很常见的. 将来在学习高等数学
    类似的科目的时候, 可能会发现多元函数的考虑比一元函数繁杂很多. 
\end{exercise}

\begin{exercise}{聪明的质检员: 满分做法}
    请使用前缀和优化的方法优化我们以前要求的这个问题. 
\end{exercise}

\begin{exercise}{前缀和与差分的两个方向}
    前缀和和差分的发展方向有两个: 分别是多维的前缀和/差分或者多次进行前缀和/差分. 
    此外, 我们在树上维护一些属性的时候还可以再树上进行差分. 这种的问题还是需要
    见一些的. 请阅读树上差分的相关博客, 资料, 看一看树上差分和普通的差分有哪些
    相同之处. 

    简单来说, 点差分的情形是: 有$n$次修改操作, 每次把$u\to v$的所有点权加上$x$, 
    问最后的点最大值. 此外, 我们可能在边上维护一些信息. 如果有$n$次修改操作, 
    每次把$u\to v$的路径的权值加上$x$, 求$x\to y$的最大点值. 

    可以参考例题: \ti{P3128 [USACO15DEC] Max Flow P}~\ti{P3258 松鼠的新家}
    ~ \ti{\href{https://loj.ac/p/146}{DFS 序 3, 树上差分 1}}

\end{exercise}

\section{贪心算法}

贪心是指在最优化问题的决策过程中, 每次选择当前局面的最优决策. 不过需要指出的是, 
当前局面最优不一定能得到全局最优. 通常, 我们要使用贪心算法, 至少要思考一下如何
说明一下它的正确性. 

\begin{remark}
    有点有趣的是, 在推荐给大家的Jeff Erickson的Algorithm\cite{algobook}书中, 作者风趣地
    写道: ``Greedy algorithms never work! Use dynamic programming instead!''.

    这显示了使用贪心算法的副作用 -- 没有说明胡乱贪心有时候不可取. 其中的 
    Dymanic Programming是动态规划的意思, 现在可以认为是聪明的搜索 -- 使用
    记忆的方法避免求解了一些重复的子问题. 我们会在后面简单了解. 
\end{remark}

我们来看若干个问题: 

\ti{P1056 排座椅} 对于单独的某邻近两列, 如果有$x$对爱唠嗑的同学, 选择拆散这一列, 
就拆散了$x$对同学, 邻近两行也是同理的. 另外, 对于任意情况, 我们都应该拆散
邻近两列或两行爱唠嗑同学对数最多的那两行或两列. 不然, 我们本可以拆散更多的同学. 
我们刚刚的论证用了问题的描述以及反证法(``不然...''). 虽然思路很好想, 但是注意
输出的时候是按照编号输出的. \file{P1056}. 

\ti{P1016 旅行家的预算} 我们可以在油便宜的时候必须要买油, 只要比当前油价便宜就好. 
如果在油箱的油消耗完之前不能到达比当前油价还便宜的地方, 就在这里把油箱加满油. 
如果能到达比当前油价便宜的地方, 那就加油到刚好能跑到那个地方. \file{P1016}



不过要注意的是, 贪心可能很难, 贪心的结果也可能非常有趣. 我们下面来看一个有趣的
脑筋急转弯. 

\ti{\href{https://www.luogu.com.cn/problem/AT_arc066_c}{Addition and Substraction Hard}}
这个题目的意思很简单: 给你一个只包含`+'、`-'、正整数的式子, 你需要在式子中添加一些括号
, 使运算结果最大, 输出最大的结果. 

首先我们看到我们必须在减号的后面加括号. 因为减号的后面才能使得符号发生变化. 
在这个第一个括号里面, 我们就需要闭合的时候这个值尽可能的小了. 那么如何让这第一个
括号里面的值尽可能小呢? 首先, 这个第一个括号的闭合肯定在式子的最末尾. 
其次, 我们可能还要在减号的后面继续加括号, 但是要满足让原式的结果尽量大, 第二层括号
里面的值的要求是也是尽可能最大 -- 也就是让第二个括号里面的加号最多. 我们只需要把所有的
减号后面的连加符号都括起来即可. 

这就是我们的贪心思路了. 我们可能会说: 为什么不会有嵌套三层(往上)的情况? 其实, 我们 
注意到, 任何一个嵌套括号到了3层(或往上), 一般形态为$x_0-(x_1-(x_2-(x_3)))$其实
可以被组合成为$x_0-(x_1-x_2)-(x_3)$, 保留了原来的减号. 但是枚举每一个括号计算的
时间是$\mathcal O(n^2)$的, 难以应对数据量. 我们使用前, 后缀和的技巧来优化我们的
计算. 

要观察出这个思路需要相当对算术运算的体会. 官方给出的题解使用了动态规划的思路. 
正如我们刚刚
提到的, 这是一个``聪明地搜索状态空间''的算法. 我们可能会在未来重新回顾官方题解的做法.

\subsection*{闲聊与练习} 

\begin{exercise}{装载相关问题(选自紫书\cite{liu2014}~8.4节)}
    有$n$个人, 第$i$个人的重量为$w_i$, 每一艘船最大的重量均为$C$, 且只能乘坐两个人
    使用最少的船装所有的人. 描述贪心策略, 并说说为什么. 之后与原文对照. 

    注意这里原文中使用了大小关系的传递, 来证明我们的观点的. 同时, 我们使用了交换的方法
    . 这些方法是我们有时候比较常用的. 
\end{exercise}

\begin{exercise}{区间相关问题(选自紫书\cite{liu2014}~8.4节)}
    我们的要求如下: 

    (1) 选择不相交的区间: 数轴上有$n$个开区间$(a_i, b_i)$, 尽量选择多个区间, 使得
    区间两两不相交. 

    (2) 区间选点问题: 数轴上有$n$个闭区间$[a_i, b_i]$, 取尽量少的点, 使得每一个区间
    里面都有一个点. 

    (3) 区间覆盖问题: 数轴上有$n$个闭区间$[a_i, b_i]$, 尽量选择少的区间覆盖一条指定
    的线段$[s,t]$. 

    同样, 在想一想这些问题之后, 看一看紫书中对应的描述. 看一看有没有类似的感觉. 
\end{exercise}

\begin{exercise}{Huffman编码相关的问题(选自紫书\cite{liu2014}~8.4节)}
    阅读关于Huffman编码贪心的正确性证明. 试图感受恰当的数据结构在我们的证明问题
    中的作用. 
\end{exercise}

\section{倍增}

\lec{介绍}{引例1} 不知道大家有没有在无聊
\mn{没有无聊的时候? 相信我到时候一定会有的. 比如河南省的会考. 如果
当前情况维持不变的话, 会考理科的试题是非常充足的. 当时大概所有理科作答时间
(数学, 物理, 化学, 生物)加起来总共用了一个小时左右, 所以在那个时候就可以轻松
体会这个游戏了. }
的时候玩过算2的几次幂的游戏: $2^1=2, 2^2=4, 2^3=8, \cdots$. 许多同学在各种
评论区分享了他们算过的最大值. 但是, 我们发现, 我们可以这样来算得更多一些: 
$2\times 2=4, 4\times 4 = 16, 16\times 16=256, 256\times 256=65536\cdots$. 

如果记得\codeword{unsigned int}的最大值是$2^{32}-1$, 即$4294967295$, 那么你可能应该可以
很快地计算出$2^{64}$, 甚至$2^{128}$了. 为了好玩, 我们给出$2^{256}$这个78位数: 

$$
115792089237316195423570985008687907853269984665640564039457584007913129639936
$$

并且我们发现, 我们如果要任意的一个幂次, 就都可以用上面的一些内容表示出来. 比如, $2^3=2^2\times 2^1$.
因为$3=(101)_2$. 由此, 我们就可以发明``快速幂''的算法. 我们只要$\log_2 b$次来计算$a^b$(不溢出的情况下).

\ti{P1226 【模板】快速幂 | 取余运算} 这里只是多了一个取余数的运算. 我们发现
$(a\times b)\bmod c = ((a\bmod c)\times (b\bmod c))\bmod c$. 用这个内容写代码就好了. \file{qpow}


\lec{ST表}{介绍}  我们刚刚使用二进制去拼凑一个整数的方法能不能用于其他的问题呢? 其实, ``可重复贡献的问题''
就是我们可以用这样的方式做的. 我们可以预处理出$f[i][j]$表示序列上起点为$i$, 长度为$2^j$的区间的答案, 
查询的时候使用拼凑的方式把我们的答案拼凑出来就可以了. 比如快速查询区间最值, 区间按位或, 区间按位和, 区间最大
公约数等等. 他们都满足一个性质: $f[a..c] = f[a..b] \text{OP} f[b..c]$. 我们有$\mathcal O(n\log n)$
的时间预处理, $\mathcal O(1)$时间查询. 

\ti{P3865 ST表} 我们可以用上述的方法来完成这道问题. \file{st}

有些时候我们还会在树上的最近公共祖先中遇到这样的倍增的思想. 具体我们可以到时候再了解. 

\lec{矩阵介绍}{从方程组到矩阵} 接下来我们来引入一个比较有趣的想法. 也就是线性方程组和矩阵. 
在中小学的学习中, 我们可能遇到过形如这样的方程: 
$$\begin{cases}
a_{11}x_{1}+a_{12}x_{2} & =b_{1}\\
a_{21}x_{1}+a_{22}x_{2} & =b_{2}
\end{cases}$$
如果大家把它当做$x_1, x_2$的变量的话, 就可以解出$x_1, x_2$的值. 我们这一次重点来看
我们的消元法. 我们发现, 其实上文中的$x_1, x_2$之类的写了好多次. 在这种情况下, 
我们可以丢掉那些冗余的内容, 仅仅留下我们要的系数以及等式右边的值. 如下: 
$$
\begin{pmatrix}
    a_{11} & a_{12} &\mid b_1\\
    a_{21} & a_{22} &\mid b_2
\end{pmatrix}
$$
好了, 有了这个简化的记号, 我们可以认真思考一下矩阵的消元意味着什么了. 我们发现, 在消元的时候
我们无非在做三件事: 
\begin{itemize}[noitemsep]
    \item 互换两个方程的位置
    \item 某一个方程乘上$k$倍($k\neq 0$)
    \item 一个方程乘上$k$倍加到另一个上面去
\end{itemize}
这样一来, 我们的消元过程中做的事就非常明了了. 下一个问题是: 有没有什么办法让电脑帮我们消元?
这就需要探讨Gauss消元法了. 例如, 我们如果能把我们的方程组化为右边的这个形式: 
$$
\begin{array}{c}
    a_{11} x_{1}+a_{12} x_{2}+\cdots+a_{1 n} x_{n}=b_{1} \\
    a_{21} x_{1}+a_{22} x_{2}+\cdots+a_{2 n} x_{n}=b_{2} \\
    \vdots \\
    a_{n 1} x_{1}+a_{n 2} x_{2}+\cdots+a_{n n} x_{n}=b_{n}
    \end{array} \quad \Rightarrow \quad \begin{array}{r}
    a_{11}^{\prime} x_{1}+a_{12}^{\prime} x_{2}+\cdots+a_{1 n}^{\prime} x_{n}=b_{1}^{\prime} \\
    a_{22}^{\prime} x_{2}+\cdots+a_{2 n}^{\prime} x_{n}=b_{2}^{\prime} \\
    \vdots \\
    a_{n n}^{\prime} x_{n}=b_{n}^{\prime}
    \end{array}
$$
这下子来看我们就可以得到了$x_{nn}={b'_n}/{a'_{nn}}$. 接着, 有了$x_n$, 那么就解出了
$x_{n-1}$, 一直网上面反向带入, 就可以得到$x_1,\cdots, x_n$的值了. 

但是这样的想法到底能不能实现呢? 我们来实际的考察几例: 

第一个例子来自中国古代书本中记载的关于方程的问题. 来自《九章算術》一书. 原文说: 
\begin{quote}
    今有上禾三秉\footnote{〔秉〕:禾束,成把的禾.}, 中禾二秉, 下禾一秉, 實\footnote{〔实〕:古代算数书称被乘数, 被除数为``实数'', 简称``实''.}三十九斗;上禾二秉, 中禾三秉, 下禾一秉, 實三十四斗;上禾一秉, 中禾二秉, 下禾三秉, 實二十六斗. 問上、中、下禾實一秉各幾何? 
\end{quote}
古文很难读懂, 我们翻译为现代的方程形式, 就是$\begin{cases}
    3x+2y+z=39\\
    2x+3y+y=34\\
    x+2y+3z=26
    \end{cases}$
翻译为矩阵, 就是
$\left(\begin{array}{cccc}3&2&1&39\\2&3&1&34\\1&2&3&26\end{array}\right)$. 我们先干掉第一列, 可以使用第 2 行乘以 3 减去第 1 
行乘以 2 , 第 3 行乘以 3 减去第 1 行, 可得
$$
\left(\begin{array}{cccc}3&2&1&39\\0&5&1&24\\0&4&8&39\end{array}\right).
$$
再用第 3 行乘以 5 减去第 2 行乘以 4 得到
$$
\left(\begin{array}{cccc}3&2&1&39\\0&5&1&24\\0&0&36&99\end{array}\right).
$$
然后我们把它想做一个方程, 现在已经知道了$36z=99$, 那么带入到上一个方程中, 再带回去, 就有: 
$$
\left(\begin{array}{cccc}1&0&0&37/4\\0&1&0&17/4\\0&0&1&11/4\end{array}\right).
$$

看似我们的确可以把它化为阶梯的样子. 下面我们来看另一个例子, 这个例子在经过几步变化之后
发生了不一样的事情: 
$$
\left(\begin{array}{cccc}1&1&1&1\\1&2&3&4\\0&1&2&4\end{array}\right)\to\left(\begin{array}{cccc}1&1&1&1\\0&1&2&3\\0&1&2&4\end{array}\right)\to\left(\begin{array}{cccc}1&1&1&1\\0&1&2&3\\0&0&0&1\end{array}\right)\to\left(\begin{array}{cccc}1&0&-1&0\\0&1&2&0\\0&0&0&1\end{array}\right). 
$$
哎呀! 我们出现了形如$0x=1$这样的式子. 这就表明, 这个方程式无解的. 因为没有这样的数, 使得 
$0x=1$. 这时候, 我们的方程就出现了无解的情况. 

当然, 我们也能发现有无数组解的情况. 也就是说, 在变化完为对角形之后, 如果最后的有``全0''的 
行, 但是右边的最后一个不等于0的话, 这个方程组无解; 最后一个如果也等于0的话, 这个方程组有
无数个解. 

所以, 我们如果不能把它化为阶梯型, 就说明它是无解的. 如果出现了类似$0x=i,i\neq 0$这样的式 
就说明了无解. 有$0x=0$的, 就表明了有无数组解答. 

\ti{P2455} 本题目涉及到浮点数的运算可能较为麻烦, 因为大家要尽量避免除法的精度带来的误差
造成对大家程序的影响. \file{P2455}

\lec{方程组变量代换}{矩阵乘法} 大家可能已经听说过矩阵乘法的关系式, 可是为什么这么奇怪?
矩阵乘法是 Cayley 在 1857 年左右发明的一个具有开创性的操作. 他提出的这个操作建立了
一个全新的代数体系. 

实际上, 方程组的视角来看, 矩阵乘法没有那么复杂. 矩阵乘法就相当于方程组的变量代换. 例如
$$
\left\{\begin{array}{l}a_{11}x_1+a_{12}x_2=c_1,\\a_{21}x_1+a_{22}x_2=c_2,\end{array}\right.\quad\left\{\begin{array}{l}b_{11}y_1+b_{12}y_2=x_1,\\b_{21}y_1+b_{22}y_2=x_2.\end{array}\right.
$$
将第二个方程组代入第一个中可以得到一个新的方程组: 
$$
\left\{\begin{array}{l}(a_{11}b_{11}+a_{12}b_{21})y_1+(a_{11}b_{12}+a_{12}b_{22})y_2=c_1,\\(a_{21}b_{11}+a_{22}b_{21})y_1+(a_{21}b_{12}+a_{22}b_{22})y_2=c_2,\end{array}\right.
$$
它的系数矩阵可以用这样直观的图示进行: 
$$
\begin{pmatrix} &  &  & b_{11} & b_{12}\\
    &  &  & b_{21} & b_{22}\\
    &  &  & \downarrow & \downarrow\\
   a_{11} & a_{12} & \rightarrow & a_{11}b_{11}+a_{12}b_{21} & a_{11}b_{12}+a_{12}b_{22}\\
   a_{21} & a_{22} & \rightarrow & a_{21}b_{11}+a_{22}b_{21} & a_{21}b_{12}+a_{22}b_{22}
   \end{pmatrix}
$$

这就解释了为什么会有这样的看似复杂的式子. 那么这样的一个``矩阵乘法''满足什么样的性质呢? 

矩阵乘法不满足交换律. 一个非常简单的例子可以看出来: 一个$3\times 4, 4\times 3$的矩阵
交换之后, 乘法无法进行. 另一方面, 即使他们的行和列的数量相同, 乘出来的矩阵也不一定一样. 

但是, 矩阵乘法满足结合律. 也就是对于矩阵$A,B,C$, 
$(A\times B)\times C=A\times (B\times C)$. 我们可以从变量替换的角度说明这个事情. 
更一般的方法就是看一看乘过之后的矩阵的第$i$行第$j$列的元素, 看看是不是一样的.
\begin{proof}
    我们只对结合律加以证明$A(BC)$的第 $i$ 行 $j$ 列元素为
    $$
    \begin{aligned}(A(BC))_{ij}=\sum_{k=1}^pa_{ik}(BC)_{kj}=\sum_{k=1}^p\sum_{l=1}^sa_{ik}b_{kl}c_{lj}=((AB)C)_{ij}.\end{aligned}
    $$
    因此 $A(BC) = (AB)C$.
\end{proof}

既然满足, 我们也可以对它使用快速幂. 比如快速地计算矩阵$A$的若干次方, 和刚刚说的是一样的. 

那它有什么用武之地呢? 毕竟, 好像没啥人希望计算一个奇形怪状矩阵的若干次方. 其实, 我们回到
我们的Fibonacci数列, 其实这里面这一个递推式就隐藏着矩阵乘法. 我们发现每一个值都是它前两项
的线性组合. 

设 Fib$(n)$表示一个$1\times 2$的矩阵$[F_n, F_{n-1}]$. 我们希望根据
 $\text{Fib}(n-1)=\left[\begin{array}{cc}F_{n-1}&F_{n-2}\end{array}\right]$
 推出 Fib$(n)$. 
试推导一个矩阵的奠基, 使 F$(n -1)\times $ base = F$(n)$, 即$[F_{n-1}\quad F_{n-2}]\times\mathrm{base}=[F_{n}\quad F_{n-1}]$. 

怎么推呢?根据递推关系, $F_n=F_{n-1}+F_{n-2}$ , 
所以 base矩阵第一列应该是$\left[\begin{array}{c}1\\1\end{array}\right],$
这样在进行矩阵乘法运算的时候才能令 $F_{n-1}$ 与 $F_{n-2}$ 相加, 从而得出$F_n$. 同理, 为了得出 $F_{n-1}$ , 矩阵 base的第二列应该为$\left[\begin{array}{c}1\\0\end{array}\right]$. 

综上所述: base $=\left[\begin{array}{cc}1&1\\1&0\end{array}\right],$ 原式化为$[F{n-1}
\quad F_{n-2}]\times\left[\begin{array}{cc}1&1\\1&0\end{array}\right]
=\left[\begin{array}{cc}F_n&F_{n-1}\end{array}\right]$ 因此, 我们就得到了递推矩阵
base=$\left[\begin{array}{cc}1&1\\1&0\end{array}\right]$. 

\ti{P1962 斐波那契数列} 我们按照刚刚的方法解答就好了. 对于一些线性的递推
式子, 我们有时候可以使用这样的方法做这个问题. 

对于评论区题解的特征方程的方法, 如有兴趣, 大家可以参考一些大学教科书. 来了解其背后的
原理. \mn{如果大家想要了解更多, 欢迎参看朱富海老师编写的《高等代数与解析几何》\cite{zhu2018}, 
以及李尚志老师编写的《线性代数学习指导》\cite{li2015}. }

\subsection*{闲聊与练习}

\begin{exercise}{ST表与RMQ(Range Mininum Query)}
    请完成\ti{P1440 求m区间内的最小值}, 加深对于最小值的理解. 顺便这个问题使用
    ST表相关的问题求解的时候注意空间. 有一些无用的空间可以复用. 
\end{exercise}

\begin{exercise}{最近公共祖先}
    你已经了解了树. 那么两个树的最近公共祖先应该如何求? 一个想法是先让他们跳到一样的
    高度, 然后一层一层往上跳. 但是不巧的是, 一层一层往上跳太慢了. 因此可以一次向上
    跳$2^k$个单位. 只要他们相遇了, 就缩小$k$的值, 直到最后他们的父亲节点是同一个节点,
    他们的父亲节点就是最近公共祖先. 

    这就是倍增的实际应用. 如果感觉比较迷糊, 可以看\href{https://www.bilibili.com/video/BV1N7411G7JD}{这里的一个可视化资源}. 注意! 这个视频可能有一些错误. 如果
    发现一些胡说八道的情况, 基本上你是对的. 然后你可以完成\ti{P3379 【模板】最近公共祖先(LCA)}. 
\end{exercise}

\begin{exercise}{认识矩阵}
    现在再阅读矩阵的定义, 是不是更好懂了一点? 实际上, 除了解方程组的角度, 我们
    还可以使用图像化和可视化的方法理解矩阵和背后的一点点线性代数. 

    请欣赏3Blue1Brown制作的\href{https://www.bilibili.com/video/BV1ys411472E}{《线性代数的本质》}系列视频. 并简单感受其中的一些思想. 

    如果发现看英文视频不习惯, \href{https://www.bilibili.com/video/BV1ib411t7YR}{这里}有一份中文配音的版本. 但是我们还是推荐使用英语看完本系列, 因为逻辑流很通顺.
    
\end{exercise}

\begin{exercise}{矩阵加速线性递推}
    可以完成\ti{P1939 【模板】矩阵加速(数列)}以及\ti{P3390 【模板】矩阵快速幂}来加深
    对于矩阵快速幂的理解. 
    
\end{exercise}

\section{更多的练习与思考}

除此之外, 我们还会有很多很有趣的思考问题的方法. 我们下面举出几个例子: 

\lec{滑动窗口}{介绍} 我们在有些问题的时候, 要求一个固定长度的区间内部的最大值和最小值都输出
出来. 我们有
如果我们使用暴力的做法, 可以这样做: 可以先用一个队列来维护窗口, 保证每次这个窗口里面存的是当前的所有元素. 
遍历所有元素, 得到时间的复杂度是$\mathcal O(nk)$. 但是我们想一想这个真的需要这么复杂吗? 

考虑优化, 有些元素似乎是没用了的. 比如我们要看窗口长度为3的序列. 我们发现, 
我们要求最小值, 我们只要严格递减就可以了. 另外, 适时地弹出不在队伍里面的元素,
就能维持答案的准确性. 也就是说, 单调队列的主要特点是保持队列内元素的单调性, 这使
得在每次添加新元素时, 队列内的元素仍然保持单调性, 从而保证了队列操作的高效性.  

\incfig{opt-search/sin-sta}

我们来看单调队列的代码: 注意这里表示了那些元素的编号在队伍里面. 
\file{slide-window}, 并且可以尝试做习题\ti{P1886 滑动窗口}. 

\lec{并查集}{简介} 并查集主要用于解决元素的所属关系, 也就是看一看两个内容是不是在同一类中. 
具体地, 就有如下的两点要求: 
\begin{itemize}[noitemsep]
    \item 将两个集合合并
    \item 判定两个元素是不是在一个集合中
\end{itemize}

要达到这个目的, 我们可以采用这样的策略: 每个集合用一棵树来表示, 树根的编号就是整个集合的编号, 
每个节点储存着他的父节点的编号, \codeword{p[x]}表示\codeword{x}的父节点. 

那么我们的几个操作就可以这样表示了: 
\begin{itemize}[noitemsep]
    \item 判断树根: \codeword{p[x]==x}
    \item 求\codeword{x}集合所在的编号: \codeword{while(p[x]!=x) x = p[x]}
    \item 合并两个集合\codeword{px,py}, $x\neq y$. \codeword{p[x]=y}.
\end{itemize}

\begin{lstlisting}
    int a, b; 
    int p[MAXN];
    // 返回x属于的是哪一类中的
    int find(int x){
        // 路径压缩
        if(p[x] != x) p[x] = find(p[x]);
        return p[x];
    }
    
    int main(){
        scanf("%s%d%d", op, &n, &m);
        for(int i=1; i<=n; i++) {
            p[i] = i;
        }
        //... 读入a, b
        // 把a插入到b身上, 修改p[find(a)]
        if(op[0]=='M') p[find(a)] = find(b);
        else {
            if(find(a) == find(b)){
                puts("Yes");
            }else puts("No");
        }
        return 0;
    }
\end{lstlisting}

\ti{P3367 并查集} 这个问题是一个标准的并查集的问题. 可以用作练习. \file{P3367}. 

我们使用并查集还可以维护并查集中的点的数量. 比如, 我们要是问这个并查集里面一共有几个元素, 应该怎么办?
一个原则就是可以这样修改我们的并查集代码, 额外维护一个\cw{size}数组, 每次更新的时候把值加上去. 
也就是多出了一行\cw{size[find(b)] += size[find(a)];}. 具体我们可以看代码\file{point-no}.

我们来看一个并查集的另一个变形, 带权并查集. 我们刚刚在点上面打上了标记, 现在我们来看一看边上有什么
可以做的. 带权并查集是在并查集的边上定义某种权值, 以及这种权值在路径压缩时发生的运算, 来达到
解决更多问题的结果. 我们来看下面的问题: 

\ti{P2024 食物链} 我们这时候定义并查集边上的节点为维护到根节点的距离. 这样一来, 就可以使用
用并查集维护额外信息. 这个听起来看上去很难说, 但是我们来观察一些简单的性质: 
\incfig{opt-search/fodchain.png}
于是我们就可以写出代码\file{P2024}.

\subsection*{闲聊与练习}

\begin{exercise}{滑动窗口的编程习题}
    以下的问题是滑动窗口的一些练习题: 

    (1) 最大子序和

    仿照滑动窗口, 可以写出单调栈的原理. 如下是一些练习题. 如果感到困难, 推荐在DP之后
    回过来看: 

    (1) P5788 单调栈模板题

    (2) POJ-3250 Bad Hair Day 

    (3) P4147 玉蟾宫
\end{exercise}

\begin{exercise}{并查集的相关编程习题}
    下面的一些内容是并查集相关的练习. 

    (1) P1955 程序自动分析

    (2) P1196 银河英雄传说

    (3) P1892 团伙

    (4) P1525 关押罪犯

\end{exercise}
    \part{树与图}

拿着郑州市的地图, 在所有两条或多条街道的汇合处或一条街道的尽头都画上一
个黑点, 这样就有了一个组合数学中的图的例子. 在这个城市中, 某些街道是单行道, 
只允许单向行驶, 因此你可以在每一条单行道的行进方向上画一个箭头$\rightarrow$, 
在所有的双行道上画一个双箭头$\leftrightarrow$. 
考虑栖息在你喜爱城市里的所有的动物和植物, 如果一个物种捕食另一个物种, 
就在它们之间画上一个箭头, 这样你又得到了一个有向图. 得到了我们学习的食物网. 

上述的例子说明, 图和有向图为相关对象可以让我们理解事情更加清晰. 图在计算机科学中也
是一种极为有用的模型, 因为在计算机科学中出现的许多问题, 都能够很容易地通过图的算法
去刻画. 在这之前, 我们要来看一看一些简单的内容: 

\section{图的关键要素和存储}

正如我们刚刚看到的, 我们发现图无非是一些节点和一些边组成的. 

\begin{definition}
一个图$G$ (也叫做简单图) 是由两类对象构成的. 它有一个被称为顶点 (有时也叫做结点
的元素的集合
$$
V=\{a,b,c,\cdots\}
$$
和一个被称为边的不同顶点对的有限集合$E$, 我们用
$$
G=(V,E)
$$
表示以$V$为顶点, $E$为边的图. 
\end{definition}

\begin{example}
一个有5个顶点的图G的顶点是
$$
V=\{a,b,c,d,e\}
$$
以及边是
$$
E=\left\{\left\langle a,b\right\rangle,\left\langle b,c\right\rangle,\left\langle c,d\right\rangle\left\langle d,a\right\rangle,\left\langle e,a\right\rangle,\left\langle e,b\right\rangle,\left\langle e,d\right\rangle\right\}.
$$
\end{example}

所以, 关键是把握住如何向计算机里面塞入这些数据. 我们有两种做法: 

\ca{枚举每一个点对之间有没有边} 这样子, 我们可以开一个二维数组$g[i][j]$. 
其中$g[a][b]$如果是0, 那么说明这两个点之间没有边, 反之认为两个点之间有边. 
上面的可以用如下的二维数组表示: TBD 

\ca{在每一个节点上维护一个它可以到的相邻节点的列表} 这个列表的实现方式可能有很多种, 
比如``链式前向星''就使用一个类似于链表的结构记录了这个列表. 有些时候, 可以使用
另一个数组\codeword{vector}来进行存储. 

TBD 加一个图

那么如果想到达相邻节点的相邻节点怎么办? 我们又没有存储它.
没关系, 我们只要先到一个相邻节点, 再在另一个相邻节点的位置上找到那个相邻的节点就
可以了. 也就是说, 我们只要能够维护一个相邻节点的关系, 我们就可以在整个图上到达我们
可以到达的节点了. 

\ti{*通过手动模拟理解链式前向星的加边代码} 有些时候我们会使用一个链表来进行加班的操作. 
我们来看代码: 

\begin{lstlisting}
    int cnt=0; // 记录当前是第几条边
    int head[MAXN]; //head[u]表示最后一条从u出发的边
    struct Edge{
        int to,next,val;
        //to:从u出发到的节点
	    //next: 前一条从u出发的边的cnt的值
	    //val: 当前边边权
    }e[MAXN];
\end{lstlisting}

我们来看一个问题的练习: 

\ti{B3643 图的存储} 这个是基本的问题. 让我们用不同的方法来进行书写. 
\file{graph-store-1}
由于这个问题需要我们排序输出, 所以我们还是不要使用链表的方式了. 因为排序的
时候会很麻烦. 

树的节点除了可以存当前的节点编号, 还可以维护一些辅助的信息; 边除了边的权重, 
还可以加一些辅助的信息在边上. 这就需要具体看问题具体分析了. 

下面我们来看一个可以用刚刚说的链式前向星进行的例子: 当然这个可能复杂了一些. 

\ti{P3916 图的遍历} 一个朴素的算法是从每个点搜索能到达的点, 再找出最大的.
但是这样的内容显然是无法处理$\mathcal O(n^2)$的数据的. 我们需要一些优化. 
由于是最大的, 我们可以提前指定一个序关系, 使得我们从答案出发, 看一看哪些点
会最终到达这个答案. 这就要求我们建立一个反向边就行了. 

所以, 我们的策略是: 应该按编号从大到小DFS每个节点, 这样能保证一个点在被
第一次访问的时候一定是能够到达最大的值的. 这样, 每个点就要访问1次, 从的时间
复杂度是$\mathcal O(n)$, 可以做到. \file{P3916}

如果你有时候觉得中括号太多了, 可以把\codeword{to, next}从结构体里面提出来
写成数组, 这样一来\codeword{e[i].to}就可以直接写作\codeword{to[i]}. 
当然是只有一个图的时候这样子也是可以的. 

我们发现运用图来解决问题很多时候是很有趣的, 我们会简单说明有些常用的算法, 并
做一点简单的应用. 

\section{常见图算法}

\lec{拓扑排序}{介绍} 想象一下, 你身为一个小组的学生们计划一个综合项目. 
每个学生都有自己的任务, 但某些任务必须在其他任务之前完成. 
我们要使用拓扑排序来确定任务的顺序. 

假设我们的任务有: A - 搜集信息, B - 分析数据, C - 编写报告, D - 进行演示. 
现在让我们来看看每个任务之间的依赖关系. 

A不依赖其他任务, 所以它排在第一位. 
B依赖于A, 所以它排在A之后. 
C依赖于B, 所以它排在B之后. 
D依赖于C, 所以它排在C之后. 

所以最终的任务顺序是: $A \rightarrow  B \rightarrow  C \rightarrow  D$. 

要进行拓扑排序, 我们(1)首先找到所有没有前置依赖的顶点 (入度为0的顶点) . 
这些顶点可以作为排序序列的起点. (2)然后, 从上一步得到的顶点中选择一个作为当
前的顶点, 并将其添加到排序序列中. (3)将当前顶点从图中移除, 
并更新与其相关的顶点的入度. 具体地说, 对于每个与当前顶点相邻的顶点, 将其入度减1. 
(4)重复步骤2和3, 直到所有的顶点都被处理和移除. 
如果在这个过程中存在入度为0的顶点, 就继续选择一个添加到排序序列中. 
最终, 得到的排序序列就是一个满足依赖关系的顶点顺序, 表明了任务的执行顺序的序列.

在2020年的
NOIP中, 我们确实需要写这样一个内容, 不过需要加上高精度. 不加上高精度可以得到部分
的分数. 

\ti{P7113 排水系统} 这是一个拓扑排序的问题, 这里大家只要简单模拟就行了. 
当然注意分数通分的时候应该先加再乘. 当年也是因为这个丢掉了很多的分数. 

\ti{P1038 神经网络} 我们必须保证前面的点都已经算过了, 我们才可以计算这个点. 
所以, 我们就必须按照拓扑顺序来执行这些内容.

\ti{P3243 菜肴制作} 这是一个拓扑排序的例子. 但是会发现一个问题, 这里的要求是
尽量先吃到质量高的菜肴. 那么应该怎么办? 可以让小的菜编号为$a$, 
大的胃$b$. 我们尽量想让$a$往前靠, 但是这个难以计算. 我们可以尝试
让$b$尽量往后靠. 这样不论$b$在哪, 都能保证
$a$在前面, 可以反向拓扑排序. (这时候还需要使用优先队列\mn{优先
队列是可以让最大值出队列的一种数据结构. }进行维护) 


\lec{最短路算法}{介绍} 最短路算法是图论中的一个经典问题, 它用于寻找图中两个顶点之间最短路径的问题. 
最短路算法的核心过程是通过遍历图中的边和顶点, 找到连接起点和终点的最短路径. 

\lec{最短路算法}{Floyd算法} 给定一个带有边权的有向图, 我们的目标是找出图中任意两个
顶点之间的最短路径长度. 这个问题被称为全源最短路问题, 因为我们需要找出从图中每一个顶点
到其他所有顶点的最短路径. 

首先一个基本的问题是, 从$u$到$v$, 如果我们找到了一个节点$k$, 使得
$\text{dis}(u,v)\geq \text{dis}(u,k)+\text{dis}(k,v)$,
那么我们就可以通过$u\to k\to v$的方法经过路经. 

算法的流程如下: 首先, 我们需要构建一个二维数组\codeword{dist}, 用来存储每对顶点之间的最短路径长度. 
如果两个顶点之间没有直接的边相连, 则将其距离设置为无穷大. 接下来, 我们初始化\codeword{dist}数组. 
对于每一条边$(i, j)$, 我们将\codeword{dist[i][j]}设置为边的权重值. 如果没有边直接相连, 
则将\codeword{dist[i][j]}设置为无穷大. 

然后, 我们使用三层循环遍历所有顶点, 并尝试通过第三个顶点$k$来优化从顶点$i$到顶点$j$的最短路径. 
具体来说, 我们检查是否存在从$i$到$j$经过顶点$k$的路径长度比直接从$i$到$j$的路径更短. 
如果是, 则更新\codeword{dist[i][j]}为更短的路径长度. 

这个算法的证明, 如有兴趣请参考TBD, 其中介绍了这个算法的历史. 我们可能需要
在理解动态规划的思想之后再回来看它会比较好. 

实际上, 这个Floyd算法可以用作另一种情形. 用于求``传递闭包(transitive closure)''. 
那么, 什么是传递闭包呢? 通俗的讲就是如果$a\to b$,  $b\to c$, 那么我们就建立一条
$a\to c$的边. 将所有能间接相连的点直接相连. 为什么这样做? 因为这个做就构造出了重要的
``可达关系''. 我们就不用再枚举了. 

\begin{lstlisting}
void Floyd() {
    for (int k = 0; k < n; k++) {
        for (int i = 0; i < n; i++) {
            for (int j = 0; j < n; j++) {
                if (g[i][k] && g[k][j]) g[i][j] = 1;  
            }
        }
    }
}
\end{lstlisting}

\ti{B3611 【模板】传递闭包} 我们可以用练习B3611来练习我们的这个算法. 

\ti{\href{http://poj.org/problem?id=1975}{POJ1975 Median Weight Bead}} 
题目大意: 给出一些水滴之间的的重量大小关系, 求有多少滴水滴的重量不可能是这些水滴重量的中位数.  (水滴的数量保证为奇数) 

这就可以建个图来转化问题. 假设$w_i$表示水滴的重量, $i$到$j$有边代表$w_i>w_j$. 
求传递闭包, 对每个点都统计能到达多少个点、有多少个点能到达. 
如果这两个数中有一个大于$n/2$, 那它就不可能成为中位数. 

\ti{找出最小权值的环} 根据问题, 如果我们追踪 (起点终点为同一个点) 时, 
就能获得从$i$到$i$的最小环, 因为如果没有环, 那么从i到i的路径长肯定还是无穷大
, 但如果有环, 则肯定从$i$到$i$的值肯定会被替换, 而且在k的循环中, 更小的环会替换掉
旧的环, 所以, 通过这样, 可以获得环, 得到它的最小值就好了. 

\lec{最短路算法}{Bellman-Ford算法} 有$n$个点, $m$条边, 除了起点, 最短路中最多还有$n-1$个点, 
所以最多对$m$条边进行$n-1$轮松弛操作. 所谓``松弛'', 就是像Floyd算法那样, 如果找到了更小的, 就
不断地更新dis的值为更小的那一个. 
\begin{lstlisting}
for(int i=1;i<n;i++){
    for(int j=1;j<=m;j++)
    dis[v]=min(dis[v],dis[u]+val[u][v]);
}    
\end{lstlisting}

当然, 这个算法是有一些慢的. 我们考虑加上一些优化: 

\ca{优化1. 减少不必要的循环} 我们可以添加flag, 第i轮如果flag的值没有改变, 直接跳出循环, 
我们就可以结束了, 减少循环的次数. 
\begin{lstlisting}
int flag=1;
for(int i=1;i<n;i++){
    flag=1;
    for(int j=1;j<=m;j++){
        if(dis[v]>dis[u]+val[u][v]){
            dis[v]=dis[u]+val[u][v];
            flag=0;
        }
    }
    if(flag)break;
}    
\end{lstlisting}

\ca{优化2. 用queue存目前已经更新过的最短的路的节点} 我们发现似乎没必要
使用那么多的松弛操作. 只有上一次被松弛的结点, 所连接的边, 才有可能引起下一次的松弛操作. 
这就启发我们用队列来维护哪些结点可能会引起松弛操作, 就能只访问必要的边了. 

\ti{P3371 单源最短路(弱化版)} 请参看代码\file{P3371}. 

这份优化同样可以判断负环. 同一条边入队次数大于等于$m$ (或者点更新的次数大于某个数, 一般是$n$) , 
说明存在一条路, 满足只要一直走这一条路, 这条路上节点的\codeword{dis[]}会不断减小. 也
就是, 这条路是负环!

这种优化方法优化出来正是SPFA, SPFA可以用来判断负环是否存在. 

\ti{P3385 判断负环} 请参考\file{P3385}.

大家可能听闻了``卡spfa''的声音. 实际上, 比如在网格图、菊花图之类的图上面它的复杂度是不稳定的. 
所以说在没有负权环的情况下慎用spfa. 

\lec{最短路算法}{Dijkstra算法} 如果我们按照spfa的想法, 在没有负边权的情况下, 用优先队列优化, 
和BFS一样, 我们就得到了Dijkstra算法. 

算法的过程是, 将结点分成两个集合:已确定最短路长度的点集 (记为 $S$ 集合) 
的和未确定最短路长度的点集 (记为 $T$ 集合) . 一开始所有的点都属于 $T$ 集合. 
初始化的时候 $dis(s)=0$, 其他点的 $dis$ 均为 $+\infty$. 

然后重复这些操作:(1) 从 $T$ 集合中, 选取一个最短路长度最小的结点, 移到 $S$ 集合中. 
对那些刚刚被加入 $S$ 集合的结点的所有出边执行松弛操作. 
(2) 直到 $T$ 集合为空, 算法结束. 

这里最小长度就展示我们用堆优化的方法来做这个问题了. 直观来看, 这个算法从已知点不断向未知点搜索, 
每个点只入队出队1次, 所以从每个点出发的边只遍历1次, 再加上优先队列入队的复杂度为$O(mlogn)$. 

这方面的代码可以这样写: 

\begin{lstlisting}
    struct nodei{
        int dis,pos;
        bool operator <(const nodei &x)const{
            return x.dis <dis;
        }
    };
    void dijkstra(int s){ //s是起点
        std::priority_queue<nodei>q;
        dis[s]=0;
        q.push((nodei){0,s});
        while(!q.empty()){
            nodei tmp=q.top();
            q.pop();
            int x=tmp.pos ,d=tmp.dis ;
            if(vis[x])continue;
            vis[x]=1;
            for(int i=head[x];i;i=a[i].next ){
                int v=a[i].to ;
                if(dis[v]>dis[x]+a[i].val ){
                    dis[v]=dis[x]+a[i].val ;
                    if(!vis[v])//vis的值的含义: 0=这个点没有入队过 1=入队过
                    q.push((nodei){dis[v],v});
                }
            }
        }
    }    
\end{lstlisting}


我们可以使用数学归纳法来证明这个算法
的确是正确的. 

\begin{proof}
    我们要证, 在执行 1 操作时, 取出的结点 $u$ 最短路均已经被确定, 即满足 $D(u) = dis(u)$. 

初始时 $S = \varnothing$, 假设成立. 

接下来用反证法. 

设 $u$ 点为算法中第一个在加入 $S$ 集合时不满足 $D(u) = dis(u)$ 的点. 
因为 s 点一定满足 $D(u)=dis(u)=0$, 
且它一定是第一个加入 $S$ 集合的点, 因此将 $u$ 加入 $S$ 集合前, $S \neq \varnothing$, 
如果不存在 $s$ 到 $u$ 的路径, 则 $D(u) = dis(u) = +\infty$, 与假设矛盾. 

于是一定存在路径 $s \to x \to y \to u$, 其中 $y$ 为 $s \to u$ 路径上第一个属于 $T$
集合的点, 而 $x$ 为 $y$ 的前驱结点 (显然 $x \in S$) . 需要注意的是, 可能存在 $s = x$ 或 
$y = u$ 的情况, 即 $s \to x$ 或 $y \to u$ 可能是空路径. 

因为在 $u$ 结点之前加入的结点都满足 $D(u) = dis(u)$, 所以在 $x$ 点加入到 $S$ 集合时, 
有 $D(x) = dis(x)$, 此时边 $(x,y)$ 会被松弛, 从而可以证明, 
将 $u$ 加入到 $S$ 时, 一定有 $D(y)=dis(y)$. 

下面证明 $D(u) = dis(u)$ 成立. 在路径 $s \to x \to y \to u$ 中, 
因为图上所有边边权非负, 因此 $D(y) \leq D(u)$. 从而 $dis(y) \leq D(y) \leq D(u)\leq dis(u)$. 
但是因为 $u$ 结点在 $1$ 过程中被取出 $T$ 集合时, $y$ 结点还没有被取出 $T$ 集合, 因
此此时有 $dis(u)\leq dis(y)$, 从而得到 $dis(y) = D(y) = D(u) = dis(u)$, 这与 
$D(u)\neq dis(u)$ 的假设矛盾, 故假设不成立. 

因此我们证明了, 1 操作每次取出的点, 其最短路均已经被确定. 命题得证. 
\end{proof}

\ti{P1462 通往奥格瑞玛的道路} 我们发现, 求能到达终点路径上的收取的最大的费用的最小值, 
我们可以用二分答案, 二分最大的费用. 如果最开始限制为inf依旧不能到达终点, 输出AFK;
如果可以到达终点, 进行二分.\file{P1462}




    \part{简单计数问题}

TBD: 介绍基本的计数方法, 一方面可以用来分析自己程序的答案空间, 
另一方面为后来的组合问题做铺垫. 
    \part{动态规划简介}

这一部分我们继续跟随状态机的模型, 探求问题的状态, 用一种比较聪明的方法来说明
如何比较聪明地遍历问题. 

\section{初步的问题}

\lec{数字三角形}{介绍} TBD: 数字三角形的相关内容请参考

这是一个耳熟能详的问题. 不过一个问题需要仔细地考虑: 为什么方程
$d[i][j]=a[i][j]+\max(d[i+1][j], d[i+1][j+1])$的后半段直接可以取最大值? 
事实上, 我们发现这是我们要求最大决定的 -- 如果连``从$(i+1,j)$''出发走到底部
的和都不是最大的, 加上$a[i][j]$之后也肯定不是最大的. 这个性质被称为最优子结构
(optimal structure). 有``全局的最优解包含着局部的最优解''的想法. 具体如何
进行, 我们可以先使用搜索试试看, 之后分析出状态转移的规律, 就可以使用迭代的方式
进行实现了. 

我们来看下面的问题: 

\ti{P1004 方格取数} 想法1: 我会搜索! 我希望暴力枚举出所有可能的情况. 

上述做法直接解决了一整个大问题. 但是在解决的时候可能会出现一些重叠的子问题. 并不
太好. 我们想一想可以如何称为若干个子问题. 第一个想法是定义$f[i][j]$表示从(0,0)
走到$(i,j)$的过程. 这样可行吗? 看上去不行, 因为我们没有记录重复的数 - 重复的数是
没有办法再取的. 

那么走两次, 我们可以这样设计: $f[i_1][j_1][i_2][j_2]$表示所有从$(1,1),(1,1)$
走到$(i_1,j_1),(i_2,j_2)$的路径的最大值. 如何处理同一个格子被取两遍的呢? 只需要
保证当前处理的时候不相同即可. 

这里有4种情况, 因为每一个都可以从上来和从左来. 我们从最后一步考虑, 有如下的四类情况
TBD: 一个集合关系, 下下, 下右, 右下, 右右 直接判定即可. 
\file{1004-4D}

我们还可以把这个优化: 由于只能向下, 向右走, 不能走回头路, 当
$i_1+j_1 = i_2 +j_2$的时候, 格子才可能重合. 

\lec{充分条件和必要条件}{简介} 这里面, 我们说只有满足这个条件才可能重合, 
意味着只要重合了就一定会满足这个条件. 但是, $i_1+j_1 = i_2 +j_2$
无法推出一定重合. 我们就说他们``重合''是$i_1+j_1 = i_2 +j_2$的
\textbf{充分(sufficent)条件}, 
$i_1+j_1 = i_2 +j_2$是他们``重合''的\textbf{必要(necessary)条件}. 或者用
符号表示, 是这样的: ``($i_1+j_1 = i_2 +j_2)\Leftarrow$ 重合''. 

由于有一个等式了, 我们可以``消掉''一个量. 我们提出一种更加简化的方法:
让$k=i_1+j_1=i_2+j_2$.  
$f[k, i_1, i_2]$表示所有从$(1,1), (1,1)$到$(i_1, k-i_1), (i_2, k-i_2)$
路径的最大值. 

看上去这会让我们的转移方程难以写. 但经过分析, 也是可以做到的, 根据图, 有如下的四类情况
\begin{itemize}
    \item 下下: 从$f[k-1][i_1-1][i_2-1]$, 重合加上$w[i_1][j_1]$, 不重合加上$w[i_1][j_1]+w[i_2][j_2]$.
    \item ...
\end{itemize}

其实也没什么大不了, 只是把刚刚的状态浓缩到了$k$里面. 下面我们来看代码 \file{1004-3D}

\lec{技巧}{缩减编码复杂度} 事实上, 调代码是非常折磨人的. 如果我们能写出易于
检查的代码就好了. 这里面, 我们想把\codeword{f[k][i1][i2]}所减掉, 有没有什么
办法呢? 其实有两种办法: 第一种是使用引用: 输入
\codeword{int &x = f[k][i1][i2];} 这样下次使用的时候\codeword{x}就相当于
\codeword{f[k][x1][x2]}了. 另外一个可以使用\codeword{\#define}关键字. 
不过记得使用\codeword{\#undef}取消宏定义在使用结束的时候. 第一种情况用的很多. 

\begin{remark}
    编写易于理解, 不言自明的代码有些时候是保持思维逻辑清楚的很重要的一个习惯. 每当
    我们面临一个困难的问题的时候, 我们可以想一想有没有什么方法简化它. jyy
    老师在\href{https://zhuanlan.zhihu.com/p/619237809}{这篇文章}说过这样的一段话: 
    ``有个小朋友 Segmentation Fault 了也不知道哪里来的自信, 
    一口咬定是机器的问题. 给他换了机器, 并且教育了他机器永远是对的. 
    这个小插曲体现了编程的基础教育还有很大的缺憾, 使得竞赛选手大多都缺少真正的`编程' 训练,
    我看他们对着那长得要命的 \codeword{if (...dp[a][b][c][d][e][f][n^1]...)} 
    调的真叫一个累. 让我不由得想起若干年前某 NOI 金牌选手在某题爆零后对着一行有 20 
    个括号的代码哭的场景. '' 
\end{remark}


\ti{P1006 传纸条} 传纸条和上一个问题基本是类似的. 双倍经验的时间来了. 


\lec{DP的多重视角}{状态集合的角度} 我们可以用如下的检查单来思考一个(可能的)
动态规划问题. 因此, 我们可以把在这个属性下具有相同特征的内容划分为若干个
集合, 然后根据每一个划分, 找到相应的规律, 就可以得到对应的结果了. 

\begin{theorem}
    在思考动态规划问题的时候, 可以采用以下的检查单: 
    
    A. 状态表示:

        (1) 我状态表示归类的是哪一类的问题? 

        (2) 要在这一类问题上体现哪些属性? 

    B. 状态计算

        (1) 当前状态可以由哪些状态得来?

        (2) 对于这些内容, 这个属性前后的关系是什么? 
    
\end{theorem}

\lec{DP的多重视角}{DFS的视角} 有时候, 如果我们的递推关系过于奇怪, 我们可以
回到我们的老本行, 写出\textbf{没有额外变量}的dfs程序, 然后使用数组来递推. 
由于我们的函数调用关系, 这个依赖关系是在调用的时候就能够轻松做出来的. 由于
子问题有重叠, 每次我们只要把一个子问题计算一遍存起来就好了. 

记忆化搜索和递推二者都确保了同一状态至多只被求解一次. 但是它们实现这一点
的方式则略有不同: 递推通过设置明确的访问顺序来避免重复访问, 
记忆化搜索虽然没有明确规定访问顺序, 但通过给已经访问过的状态打标记的方式, 
同样达到了的目的。

与递推相比, 记忆化搜索因为不用明确规定访问顺序, 在实现难度上有时低于递推. 
且能比较方便地处理边界情况. 但与此同时, 记忆化搜索难以使用一些更加聪明的优化
方式, 我们在接下来的背包问题中可以看到一些. 


接下来我们来看几个类似的问题. 

\lec{最长上升子序列问题}{简介} 我们现在考察最长上升子序列(LCS)的问题. 根据我们的
检查单, 我们决定定义状态$f[i]$表示集合
$a[i]$表示以$a[i]$为结尾的严格单调上升子序列. 要维护的属性
是最大值. 现在我们考虑所有到达了$f[i]$的内容. 看看它可以从哪来: 

TBD: 加一个图示

分析了上面的内容, 我们就可以发现状态转移方程为 
$$f[i] = \max \{f[k]\} +1, \forall k\in [1..i-1], f[k]<f[i].$$

上升子序列给我们的感受是往上升. 那么下面我们来看一个既有上升又有下降的内容. 

\ti{\href{https://vjudge.net/problem/OpenJ_Bailian-2995}{登山}} 
我们可以按照中间是哪个点是最高点分析. 先分为
$a[0],a[1], a[2],\cdots, a[n-1], a[n]$是山峰这几类. 我们分别求出每一类的
长度最大值就是整个的最大值. 不是一般性, 如果峰值是第$k$个的最大长度, 并且左边选
哪些和右边的情况互不相干, 那么就在左边和右边分别跑一下LCS问题, 然后找到$\max$就行
了. 

在做模拟题的时候, 我们可能留意了``合唱队形(NOIP)''这个问题. 其实, 这个是一个对偶.
去掉多少人就是总数减去留下多少人. 

\begin{remark}
    对偶问题. 我们说两个问题是对偶的, 感觉上就是两个问题表达的是
    一个问题的两个方面. 或者更直观的说, 有一种对称性. 例如这个问题
    和合唱队形的问题; 到未来大家学习最大流和最小割, 他们都具有对偶的
    感受. 
\end{remark}

\ti{P2782 友好城市} 这里的要求是不交叉. 我们发现我们要求的序关系消失了. 
我们考察所有合法的建桥方式和上升子序列之间的联系: 对于任何一个合法的建桥方式, 
从一侧观察一边的点, 另一边都是严格上升的. 对于任意一个严格上升的子序列, 我们都
能够找到合法的架桥方式. 也就是他们之间构成双射. 所以我们按照自变量大小进行排序
看因变量的LCS就好了. 

其实, 没有交叉意味着没有逆序对. 如果你曾经实现过归并排序, 你一定对这个不会陌生. 

\lec{映射}{表达关系} 将给定集合的每个元素与另一个集合的一个或多个元素相关联的
一种思想. 我们在刚刚的问题里面发现了一对一的这样的情况, 因此可以断定两个问题
的大小是一样的. 

\lec{最大上升子序列}{之和} 这次, 我们想要知道你挑选出来的上升子序列里面, 其和
是多少. 你会发现, 最长上升子序列并不意味着最大的和. 我们又要按照刚刚的方法分析
了.  状态$f[i]$表示所有以$a[i]$为结尾的上升子序列, 属性是和的最大值. 状态
计算的划分是可以划分为上一个数字选的是空, $a[1], \cdots, a[i-1]$. 于是, 
我们就得出了状态转移方程: 

$$f[i] = \max \{f[k]+a[i]\} +, \forall k\in [1..i-1], f[k]<f[i]$$ 

\ti{\href{https://vjudge.net/problem/OpenJ_NOI-CH0206-8462}{大盗阿福}} 
直觉来看, 我们想要设置$f(i)$代表当前抢劫到了第$i$个店铺的最大收益. 于是, 当前
的状态被划分为两块: 抢劫第$i$家店铺, 得到$f[i-2]+w[i]$, 以及不抢劫
第$i$家店铺. 于是, 我们得到状态的转移方程为$f[i]=\max(f[i-2]+w[i], f[i-1]).$

这个状态需要依赖上面两维的状态. 如果我们只希望依赖上面一维的状态, 
我们还需要增加一维: 用$f(i, 1)$表示上一家
店铺被抢了, $f(i, 0)$表示上一家店铺没有抢. 因此, 我们就可以转移了. 

这种转移有一些头疼. 于是, 我们可以使用一个特殊的方法 - 请看

TBD: 状态图 % 1 loop 0 to 1 1 to 0

我们下面来正式把这个说一说: 定义$f(i,0)$表示当前站在第$i$个建筑前面, 当前
状态位于$j$的所有走法, 得到的最大值. 下面决定状态转移方程. 考虑$f[i][0]$, 
有哪些走法可以走到0? 其实, 我们可以从上一个0走到0; 或者从1走到0. 因此, 
它们的最大值分别是$f[i-1][0]$和$f[i-1][1]$ - 毕竟没有选择这家店铺. 
下面考虑$f[i][1]$. 我们只能从$f[i-1][0]$走过来. 这样子, 获得的收益是
$f[i-1][0] + w[i]$. 综合去取$\max$即可. 图示如下:

TBD: 图示

\ti{\href{https://www.luogu.com.cn/problem/T294782}{最长公共上升子序列}} 这个问题
我们定义状态$f[i][j]$为所有由第一个序列的前$i$个字母, 第二个序列的前$j$个字母构成的
公共上升子序列, 属性是要求最长的. 但是我们发现在转移的时候因为缺少条件, 我们还需要
知道现在结尾的数是多少, 以便于我们判断是不是可以向后增加. 具体地, 我们这样修改我们的
定义: ``状态$f[i][j]$为所有由第一个序列的前$i$个字母, 第二个序列的前$j$个字母构成的
公共上升子序列, 并且有$b[j]$结尾''. 

那么, 有哪些状态可以转移到了$f[i][j]$呢? 我们可以包含两类: 所有包含$a[i]$的 
公共上升子序列, 另外的是左右不包含$a[i]$的公共上升子序列. 第二类里面, 由于它最后不包含
第$i$个字母, 说明它只可能包含前$i-1$个字母. 即从状态$f[i-1][j]$转移来. 那第一类呢? 
根据状态的定义, 由于同时包含$a[i]$和$b[j]$. 由于$a[i]$是不确定发的, 我们需要继续细分, 
就像刚刚的LCS问题一样. 我们考虑序列的倒数第二个数. 有可能是空, 
$b[1], b[2], \cdots, b[j-1]$. 这样一来, 我们就从实际意义出发, 发现如果是$b[k]$作为
倒数第二个字符的话, 那么值应该是$f[i][k]+1$. 不过这个DP问题可能还需要对代码做等价变形, 
我们来看一看: TBD

现在我们做代码的等价变形, 可以TBD. 

\begin{remark}
    一个问题, 尤其是困难的问题, 搞清楚来龙去脉是重要的. 任何感觉到难的内容可能只是
    缺乏了前置应该了解的东西. 所以, 很多时候, 看一看它的历史, 你就能知道更加多样
    的东西. 甚至追寻着历史的规律, 有一天你也能为解决这一类问题添砖加瓦! 
\end{remark}

\ti{{股票买卖}} 题目叙述: 给一个长度为$N(1\leq N \leq 10^5)$的数组, 数组中的第$i$数字表示给定股票在
第$i$天的价格. 设计一个算法计算能获取的最大利润, 最多完成$k$笔交易. 你不能同时参与多笔交易(你必须在再次购买前出售掉之前的股票).
一次买入卖出合为一笔交易。第一行包含整数$N,k(1\leq k\leq 100)$, 表示数组长度和最大交易数, 第二行$N$个
不超过10000的正整数, 表示完整的数组. 输出一个整数, 表示最大利润. 

我们发现在例子的情况下, 我们能进行的操作是``买入''和``卖出''. 造成结果是``手中有股''
和``手中无股票''. 这下子, 我们发现最好按照这样的划分方法, 才可以把原来的内容描述
清楚. 如果我们手中有货, 我们在下一天到来的时候既可以继续持有, 或者卖出, 同时得到
一定的收益(得到$w[i]$); 
如果我们手中无货, 那么下一天到来的时候, 我们可以买入 (并付出$w[i]$), 
或者按兵不动. 

我们效仿背包的情况: 假设现在进行到了第$i$天, 正在进行第$j$笔交易(买入就算做这笔交易), 有
$f[i][j][0]=\max(f[i-1][j][0], f[i-1][j-1][1]+w[i])$. 同样的有 
$f[i][j][1]=\max(f[i-1][j][1], f[i-1][j-1][0]-w[i])$. 

\begin{ques}
    如果卖出的时候, 使用了这个会导致全局最大值不对吗? 
\end{ques}

我们会遍历所有的空间, 正如我们前面所说, 这是一个``聪明的搜索'', 所有的状态都会被
计算到的. 

\ti{{\href{https://www.luogu.com.cn/problem/U298750}{股票买卖2}}}
我们这时候发现状态影响决策有手中有货, 手中无货的第一天, 以及手中无货大于等于
第二条(冷冻期). 

TBD: 状态图 手中有货转圈圈(0), 有货->手中无货第一天(+w[i]), 手中无货1->无货的第二天(0) 
手中无货转圈圈(0), 手中无货2->手中有货(-w[i]). 出口有两个

转移方程, 根据上图就有: $f[i][0] = \max(f[i-1][0], f[i-1][2]-w[i])$; 
$f[i][1]=f[i-1][0]+w[i]$, 以及$f[i][2]=\max(f[i-1][1],f[i-1][2])$. 

运用状态机的视角真不错. 我们在很多时候在处理很多问题的时候也可以这样做. 

\section{背包问题} 

背包问题是一类很经典的问题. 我们首先介绍一些常见的策略, 然后仔细看一看
``0-1背包问题''. 背包问题选出的内容里面没有内在的关系. 有时候可以成为组合
类的DP. 

\lec{0-1背包问题}{简介} 简介TBD. 我们考虑设计状态. 

\ti{P1048 采药} 这是一个最为普通的背包问题. 我们现在考虑如何设计方案, 以及有什么
好的办法来做这件事情. 我们设计状态$f[i][j]$表示在集合``考虑前$i$个物品, 总容量
为$j$''的价值的最大值. 那么根据最后一步, 可以把状态表示转化为两大类 - 要么选择第$i$个
要么不选第$i$个. 

不选的方案的话, 那么是$f[i-1][j]$, 如果选取的话, 那么有分为
之前的加上第$i$个. $f[i-1][j-v[i]]+w[i]$. 我们只要找到他们的最大值就好了. 
请看代码\file{P1048-2D}. 


\lec{多重背包问题} TBD 

\section{关于区间的问题}

有些动态规划问题, 我们设计状态需要考察一个区间. 我们从石子合并这个经典问题开始看起. 

\ti{P1775 石子合并(弱化版)} 假设这时候我们认为这是在一条链上的情形. 也就是不能首尾合并.
这时候, 我们定义$f[i][j]$表示所有从$i$到$j$合并的方案, 
属性是最小值. 下面我们来考虑状态的计算问题. 我们来看一看哪个可以到达这个状态. 
我们考虑合并两个区间, 会发现它的分界点不同. 所以这就启发我们使用不同的分界点去
划分现在的集合. 假设分界线落在$k$和$k+1$之间, 那么它需要的体力最小值就是
$f[l][k]+f[k][j]+\text{左右两边的和}$, 也就是先合并左边, 再合右边, 最后就把
两堆合在一起.  最后的是所有的子集的最小值. 状态转移很好写, 但是\textbf{注意循环顺序!}

转移方程: $f[i][j] = \min\{f[i][k]+f[k+1][j]+\sum_{s=i}^j a[i]\}$. 其中$k$从
$i$枚举到$j-1$. 状态空间是$n^2$, 需要枚举起点, 有$\mathcal O(n)$, 总共时间复杂度是
$\mathcal O(n^3)$. 计算$300^3=2.7\times 10^7$, 完全可以. 

接下来我们来看代码: \textbf{请留意循环顺序! } 按照区间长度从小到大枚举. \file{P1775}

从上面的代码中, 一般而言, 区间DP可以首先循环长度, 然后循环左端点, 之后算右端点, 最后枚举
分界点. 这样是使用循环去遍历状态. 正如我们前面所说, 我们也可以使用记忆化搜索的方法
写这个内容, 当转移不明确的时候. 

\begin{ques}
    如果每次允许合并相邻的$n$堆, 应该如何做? 说一说大致思路. 
\end{ques}

我们接下来的问题可以设置状态为前$i$个数成了$j$个的过程. 这相当于DP里面套了一层DP. 
我们这里不做讨论. 

\ti{P1880 石子合并} 下面我们来考虑环形的状况. 我们如何把环的情况展开成一条区间呢? 
因为环形剪掉一条边就成了一个链, 一个朴素的想法是我们
可以枚举缺口在哪. 就可以用区间DP的方法做了. 但这样的时间复杂度是$\mathcal O(n^4)$,
难以接受. 下面介绍一种优化方式: 

我们本质上是$n$个长度为$n$个链的式子合并问题. 我们可以这样做: TBD

这样一来, 我们使用长度为$2n$的区间, 就能保证我们只处理$n$个区间就可以枚举到所有的情况了. 
这样我们的复杂度是$\mathcal O((2n)^3)$. 这样的方法可以处理大多数的环形DP问题. 

请参看代码\file{P1880}. 

\ti{P1063 能量项链} 我们现在断环为链, 像上一个问题一样. 对于一个链, 我们定义状态的表示
$f[i][j]$为所有将$i..j$区间合并成为一个珠子的方式. 属性是维护最大值. 接着来看
合并的时候状态的计算. 我们来看一看哪个可以到达这个状态, 根据最后的不同点来划分. 
这个和上一个是类似的: 有一个分界线(在原来数组的视角下注意这时候是共用的). 根据这个
我们可以把集合划分为若干个子集. 其中分界线分别为$i+1, i+2, \cdots, r-2, r-1$.
假设当前的分界线是$k$的话, 那么就会有将$(i, k), (k, j)$最后将两个合并释放的能量. 
用数学公式写出来就是$f[i][k] + f[k][j] + w[l]\times w[k]\times w[r]$. 
这就是我们使用线性的做法, 现在我们考虑环形的. 运用上一个问题的技巧, 在后面一个$2n$的
链上面做DP就可以了. \file{P1062}

\ti{LOJP10149. \href{https://loj.ac/p/10149}{凸多边形的划分}} 这个问题需要我们一定的观察
与思考. 我们首先发现, 任意作一个三角形, 它就会把左边的和右边的三角形划分开. 因为题目
中有一个重要的条件 -- 互不相交. 这就保证了区间左右的独立性. 所有这样的方案把整个
内容分为了独立的三部分. 这是区间问题里面很重要的一个特征. 我们只要在这个状态下左半边
的划分, 右半边的划分和的最大值, 就可得到和上一个问题一样的想法. 也就是比如我要考虑
从1到$n$的划分, 中间选了点$k$作为分界点, 就有
$f[1][k]+f[k][n]+w[1]\times w[k] \times w[n]$. 我们来求每一个它们的最小值
就可以了. 这一个问题虽然和上一个问题构造非常不同, 但是其转移也非常的相似. 下面我们
详细看一下这个应该如何正式化: 

定义$f[l][r]$维护集合所有将$(l,l+1), (l+1, l+2), \cdots , (r-1, r), (r,l)$
划分为三角形的方案的值的最大值. 在进行状态计算的时候, 我们枚举$l+1, l+2,\cdots, r-2,
r-1$, 就可以把问题分为若干类. 对于每一类, 其转移到当前的值为$f[l][k]+f[k][r]+w[l]*w[k]*w[r]$.

很烦人的地方是, 这个问题需要写高精度. 因为$(10^9)^3\times100$大概会有30位数. 
\codeword{int}的最大值是2147483647, 9位数; \codeword{long long}的最大值是 
9223372036854775808, 19位数.  
我们应该秉持先做对, 再做好的原则进行. 也就是
先做对, 把样例和小测试数据做好, 然后再用高精度写剩余的部分. 
不加高精度的部分如\file{LOJP10149-part}所示. 

下面加上高精度. 为了方便起见我们直接用这个数组存位数, 直接整合进$f$数组里面. 
请看代码的\codeword{add}部分和\codeword{mul}部分. \file{LOJP10149}

\ti{P1040 加分二叉树} 我们看到这个问题, 发现其计算公式很像区间DP的计算的方式: 
分为三个独立的部分. 关键是, 这个中序遍历是不是具有这样的形式, 使得我们可以在上面
做区间DP呢? 

回顾: 现在有一棵树的中序遍历, 我们考察任意的一个子树, 可以发现其在序列里面一定是
连续的一段. 这就让我们可以进行选取根节点进行中序遍历. 我们定义$f[l][r]$为所有将
$l..r$区间构造成一个二叉树的情形. 属性是维护所有二叉树的最大值. 我们找到最后一个
不同点, 根据这些类划分为不同的集合. 我们按照根节点的位置划分. 这样就划分为了若干类.

和上面的问题一样, 如果根节点在第$k$个点的话, 最大值应该如何求? 应该是
$f[l][k-1]\times f[k+1][r]+w[k]$. 下面考虑应该如何记录方案. 

其实记录方案无非是决定最后在更新的时候再某个地方记上一笔: ``节点$k$已经成为了
这个子树的根.'' 于是定义 $g[l][r]$表示 $l..r$ 区间的根节点选哪个. 在输出
前序遍历的时候就先输出这里的根($g[1][n]=:R$)\sn{:=表示``定义做''. 冒号在被定义的表达式那一侧}
, 同时知道左子树的区间和柚子树区间
为$1..R-1, R+1..R$, 反复进行这个过程就行了. 

字典序最小的方案应该如何做? 实际上我们只要让根节点的值最小就好了. 也就是找到
最靠左的一个分界点. 只有在小于当前答案的时候才更新, 并且记录. 如\file{P1040}

我们看一看二维的区间DP. 这时候区间就看上去有点奇怪了. 

\ti{P5752 棋盘分割} 这里面看上去有一个陌生的统计量均方差, 不过不用担心. 
不过我们来看均方差的公式$\sigma = \sqrt{\sum_{i=1}^n(x_i-\overline{x}^2)
\over n}$, 要是想要这个带根号的最小, 就意味着可以求$\sigma^2$最小. 简单变形
就有: \sn{但其实我们可以不用变形的. 这里只是简单体会一下操纵求和记号.}
$$
    \begin{aligned}
        &~{\sum_{i=1}^{n} (x_i-\bar x)^2} \\
        &= \frac1n\sum_{i=1}^{n} (x_i^2-2x_i\bar x+\bar x^2) \\
        &= \frac1n \left(\sum_{i=1}^{n} x_i^2 - \bar x \sum_{i=1}^{n}2x_i + n\bar x^2\right)\\
        &= \frac1n \left(\sum_{i=1}^{n} x_i^2 - \bar x \cdot (2n\bar x) + n\bar x^2\right)\\
        &= \frac{\sum_{i=1}^n x_i^2}{n} - \bar x^2
    \end{aligned}
$$

\begin{remark}
    这个推导在概率论中是比较常见的. 
\end{remark}

这就是我们试图最小化的东西. 也就是所有部分平方和的最小值. 好, 下面我们来看动态规划部分.

定义$f[x_1][y_1][x_2][y_2][k]$表示子矩阵$(x_1, y_1), (x_2, y_2)$切分成
$k$不分的所有方案. 其中$x$是行, $y$是列. 维护的属性是$\sum_{i=1}^n(x_i-\overline{x}^2)$.
的最小值. 

接下来来看状态计算. 我们认为有沿着$x$轴切; 沿着$x$轴切. 一共各自有7种情况, 分别
选上面和下面的情况. 沿着$x$轴切有类似的情况. 
我们的目标是求每一类的最小值, 然后取$\min$. 对于每一类, 我们有上面继续切的分值, 
加上下面剩余的分值. 由于右边的和是固定的, 于是可以用二维的前缀和求出来. 最后求解就可以了. 

如果要用循环来实现, 那么会很复杂. 并且循环的顺序也可能一不留神写错. 这时候我们采用
记忆化搜索的方式完成本问题. \file{5752}

\section{树形DP}
    \part{数论简介}

TBD: 介绍一些和数论有关的内容. 
    \part{组合数学与概率简介}

\section{二项式系数}
二项式系数由很多有趣的性质. 在计数原理一章中, 我们介绍了它的来历. 今天我们看一下
它由什么好玩的性质. 

组合恒等式看上去复杂, 但是其实挺有趣的。在学习组合数的时候会遇到一些组合恒等式,
可能会觉得很难记忆和理解。我们可以用故事的方法记忆组合恒等式. 

\begin{theorem} 对于$n\geq 0$的整数, 有
    $$\binom nk = \binom {n}{n-k}$$
\end{theorem}

这个的含义是从$n$个元素中选出$k$个元素的组合数等于从$n$个元素中选出$n-k$个元素的组合数。

\begin{theorem}对于整数$k$, 有
    $$\binom nk = \binom {n-1}{k} + \binom {n-1}{k-1}$$
\end{theorem}

\begin{theorem} 对于$k\neq 0$, 有
    $$\binom nk = \frac nk {\binom {r-1}{k-1}}$$
\end{theorem}

\begin{theorem} 对于$k\neq 0$, 有
    $$\binom kk+\binom{k+1}k+\cdots+\binom{k+r}k=\binom{k+r+1}{k+1}$$
\end{theorem}

\begin{theorem}[Vandermonde卷积] 
    $$\binom n0 \binom mr+\binom{n}1\binom{m}{r-1}+\cdots+\binom nr\binom m0=\binom{m+n}{r}$$
\end{theorem}

\lec{多项式乘法与二项式定理}{简介} 

\section{概率问题简介} 

\lec{条件概率}{简介} TBD... 列举几个我们玩过的游戏, 以及其中的概率

\begin{definition}
    事件是样本空间可能发生内容的子集, 概率是加到了事件上面. 
\end{definition}

\begin{axiom}
    概率的可加性: 如果$A\cup B\neq \emptyset$, 那么$P(A\cup B)=P(A)+P(B)$. 

    对于可列无穷的情形: $P(A\cup B\cup C\cup\cdots)=P(A)+P(B)+\cdots$. 
\end{axiom}
对于可数无穷的情形, 我们会发现很微妙的一件事情: 假设你有一个正方形, 上面随机选一个点, 那么这个
点被选中的概率是多少? 一个点的面积是0, 所以我们的概率为0. 这时候你可能会发现, 概率为0的事件
也有可能发生, 而不可能发生的时间概率为0. 这个到时候学习了测度论相关的内容就会知道数学家
是如何构造的了. 

我们接收到的信息总是部分的, 因此, 我们应该仔细探讨在给定一个情况下, 一个事情发生的概率. 

假设我们在如下的概率空间里面: 

TBD: A 3/6 AB 2/6 B1/6

我们定义...
    \part{从树状数组到线段树}

TBD: 这部分线段树的内容已经准备好, 重点在于准备标签的修改的阐释, 以及给出几个例子
最后修改原先文稿的排版即可. 
如果有空, 干脆把有些高级技术介绍了, 如动态开点等等... 

    \bibliographystyle{plain}
    \bibliography{refs}
    
\end{document}