\part{数论简介}

\section{整数, Euclid算法}

任何一本书上面的开头好像都喜欢用Euclid算法求解最大公约数开场. 
这是一个十分古老的算法, 但是要是仔仔细细证明这个算法, 还是不那么显然的. 

下面我们来借助这个算法, 来简单回顾一些数论的基本概念. 

要求两个数$m$, $n$的\textbf{最大公因数(greatest common divisor, gcd)}, 
Euclid发现了这样一个算法并声称:$\gcd(m, n) = \gcd(n\%m, n)$. 
这里, 沿用C++里面的取模, $a\%b$的值就等于$a/ b$的余数. 

如果我们用更聪明的记号来表示的话, 或许$a\%b=a-\lfloor a/b\rfloor\times b$. 
就相当于模拟了$b$次减法. $\lfloor x\rfloor$表示$x$下取整. 比如$\lfloor \pi\rfloor=3$. 
这是一个很有趣的小符号, 它实际上表示的是一个不等式的关系. 

因数的概念可能是数学中产生的最自然的概念之一. 
当我们要对一个东西平均的分配的时候, 一个数"能被另一个数整除"这个性质就显得尤其重要. 
\lec{质数与合数}{定义} 这里, 我们发现一些数只能被$1$和这个数本身整除, 
这样的数我们一般叫做\textbf{``质数(prime)"}. 比如$3=1\times3$. 
(其实$3=-1\times -3$, 但是这里我们只考虑这些乘数都是正的). 
还有一些数分解的可能就多了不少. 例如$14=1\times 14 = 2\times 7$. 
这样的数我们称作\textbf{``合数(composite)''}.

\begin{definition}[质数与合数]
    若一个正整数只含有1和它本身这两个不同的正因数, 则称其为素数. 

    若一个正整数出了1和它本身外, 还有其他的因数, 则称为合\mn{合指的是``复合''的意思.}数.

    1既不是素数, 也不是合数. 
\end{definition}

有时候在处理一些证明的时候, 如果题目中告诉了我们$n$是一个合数, 我们可以知道$n=n_1n_2$, 
其中$1\leq n_1, n_2\leq n$. 这样子有时候可以帮助我们分析问题. 



更一般的, 有时候我们仅仅关注一个数能不能被另一个数整除,
也即是$a/ b$的余数是不是为0, 如果是, 
我们就说$b$是$a$的一个\textbf{因子(factor)}. 
有时候可以写作$b|a$. 像是从$a$中"抓出来"了一个它的更小的部分放在前面. 


所以我们有一个形式化的定义: 

\begin{definition}[整除]
    $m | n \iff m>0 \text{ 并且 } \text{对于某个整数}k, n=mk. $
\end{definition}

如果我们抓一个数出来写出它的所有因子, 就会发现因数具有共轭的属性. 意味着$d|n\iff {d\over n}| n$. 
这个问题我们在后续学习Mobius反演系列的内容的时候会频繁地使用这个内容变换和的下标. 

在知道为什么这个算法是对的之前, 我们需要发掘一点整除的性质. 

\lec{整除的性质}{简介} 

\begin{theorem}
    设$a,b,c\in \mathbb Z, c\neq 0$, 我们有如下的性质: 
    \begin{itemize}[noitemsep]
        \item $c|b, b|a \implies c|a$. (整除的传递性)
        \item $b|a \implies cb|ca$.
        \item $c|a, c|b, \implies\forall m,n\in \mathbb Z $, 有$c|(ma+nb)$.(线性组合, 非常重要)
        \item $b|a$, 并且$a\neq 0\implies |b|\leq |a|$. 
        \item $b|a$, 并且$|b|>|a|\implies a=0$.
        \item $a|b, b|a\implies |a|=|b|$.  
    \end{itemize}
\end{theorem}

\begin{proof}
    对于(4)的证明: 因为$b|a$, 所以存在$k\in \mathbb Z$, 使得$a=kb$. 由于$a\neq 0$, 意味着
    $k\neq 0$. 所以$|a|=|kb|=|k||b|\geq|b|$
\end{proof}

\lec{带余除法}{介绍}我们看到了整除的理论, 接下来我们来看带余数的除法有什么值得关注的.

\begin{theorem}[带余除法]
    对于给定整数$a,b,$ 且$b\neq 0$, 必定存在一对整数$q,r$, 使得$a=bq+r$, $0\leq r \leq |b|$.\mn{
        为什么还要证明这样显然的东西? 事实上, 小学时候我们做的事情只是一种给定的情况. 我们很多时候需要做
        一些并不显然的东西. 这里的任意性导致我们必须要证明这样的定理. 数学不是``实验科学''. 
    }
\end{theorem}

\begin{proof}
    存在性: 分类讨论. (1)考虑$b|a$, $\exists k\in \mathbb Z$, 使得$a=kb, $取$q=k, r=0$即为答案. 

    (2) 若$b\not | a, $取这样的集合$S={a-bq: q\in \mathbb Z}$. 其中$a,b$是给定的整数, $q$是可以变化
    的整数. 显然$S$中存在正整数. 必有最小者. 记它为$r$. 下证$0<r<|b|$. 考虑反证法, 如果这个不成立, 
    那么$r\geq |b|$. 
    
    (i)若$r=|b|$, $a-bq=|b|\implies b|a$, 矛盾! 

    (ii) $r>|b|: 0\leq r-|b|=a-bq-|b|=a-b(q\pm 1)\in S$, 但是$r-|b|<r$, 矛盾. 

    唯一性: 假设存在两对整数$q_1, r_1, q_2, r_2, $使得$a=bq_1+r_1=bq_2+r_2$, 移项, 
    $b(q_1-q_2)=r_2-r_1\implies b|(r_2-r_1)$, 并且$|r_2-r_1|<|b|$, 得到$r_2=r_1$, $q_2=q_1$. 
\end{proof}

上面存在性(2)(ii)的内容被称为递降法. 我们想要证明正整数集$S$里面是空集, 可以从它的反面, 即正整数集$S$非空, 
其中必有$S$中的最小元素$x_0$. 但是我们发现某一个$f(x_0)<x_0$, 并且$f(x_0)\in S$. 这就表明出现了矛盾. 
其实$S$就是空集. 

这就可以让我们用$(a,b)$的问题通过做带余除法的方式, 转化为$(b,r)$的化大为小的操作. 

在介绍我们今天的主角Euclid算法之前, 我们先来看一组关于公因数的相关的概念. 

\begin{definition}[公因数(common divisor)和公倍数(common multiple)]
    设$D,d, M, m$以及$a_1, a_2, \cdots, a_n \in \mathbb N_+$: 

    (1) 若$d|a_i,i=1,2,\cdots, n$, 则称$d$是$a_1, a_2, \cdots, a_n$的一个公因数. 
    若$D$是$a_1, \cdots, s_n$的一个公因数, 且对于$a_1, a_2, \cdots, a_n$的任一公因数
    $d|D$, 则称$D$是$a_1, a_2,\cdots, a_n$的最大公因数\mn{这个与
    所有的公因数中的最大者是等价的. 因为我们有整除的性质中的不等式可以知道.
    }. 记作$(a_1, a_2,\cdots, a_n)$或 
    $\gcd(a_1, a_2, a_3, \cdots, a_n)$. 

    (2) 若$a_i | M, i=1,2,3,\cdots, n$, 则称$M$是$a_1, a_2, \cdots, a_n$的一个公倍数. 
    若$M$是$a_1, \cdots, a_n$的一个公倍数, 且对于$a_1, a_2, \cdots, a_n$的任何一个公倍数
    $M$, 都有$m|M$, 则称$m$是$a_1, a_2, \cdots, a_n$的最小公倍数. 记作$[a_1, a_2, \cdots, a_m]$,
    或这$\text{lcm}(a_1, a_2, \cdots, a_m)$. 

    (3) 若$(a_1, a_2, \cdots, a_n) = 1, $那么称$a_1, a_2, \cdots, a_n$互素. 
\end{definition}

在解题中我们经常碰见的是两个数互素. 通俗来讲, 当两个数没有公共的素因子的时候, 就称为他们是互素的. 
另外, 一个更强的条件是两两互素的. 整体互素只要保证他们每一个是没有素因子的, 两两互素要求任意的两个
都没有素因子. 

\lec{Euclid算法}{简介} 欧几里得算法声称, 我们如果要求$(a,b)$的最大公约数, 我们可以对它做带余除法. 
也就是$a=bq+r(0\leq r<b)$, 然后我们就只要求$(b,r)$的最大公因数就行了. 

\begin{theorem}[Euclid算法原理]
    若$a,b\in \mathbb N_+$, $a=bq+r(0\leq r<b)$, 那么$\gcd(a,b) = \gcd(b, r)$. 
\end{theorem}

\begin{proof}
    容易说明$\gcd(a, b)=\gcd(a-b, b)$. 因为如果$d$是$a$的因数, 也是$b$的因数, 那它一定是
    $a$和$b$的一个公因数. 根据整除的性质, $d$一定是$a-b$的因数. 因此$d$就是$a-b$和$b$的公因数. 
    反之, 如果一个数是$(a-b)$的公因数, 同时也是$b$的公因数, 我们很容易推出是$a$的公因数. 它们是
    一一对应的. 自然, 他们的最大公因数是相等的. 由此, $(a,b)=(a-b, b)=(a-2b,b)=\cdots=(a-qb,b)=(b,r)$.
\end{proof}

有了上述的原理, 我们就可以用这样的手段不断化大为小, 直到第二个位置为$0$. 这时候第一个位置就是我们的
最大公因数. 下面只写余数不为0的情况: 这时候, 
\begin{align*}
    b&=rq_1+r_1(0\leq r_1<r) &  0\leq r_1 < r,  & (b, r) = (r, r_1) \\
    r&=r_1q_2+r_2(0\leq r_2<r_1) &  0\leq r_2 < r_1,  & (r, r_1) = (r_1, r_2) \\
    r_1&=r_2q_3+r_3(0\leq r_3<r_2) &  0\leq r_3 < r_2,  & (r_1, r_2) = (r_2, r_3) \\
    &\cdots
\end{align*}

这个方法可以无限地进行下去吗? 其实是不能的. $b>r>r_1>r_2>\cdots\geq0$, 这一定是有限的. 必有一步
余数是0. 我们记作这一步为$r_{n+1}$. 也就是$r_{n-1}=r_nq_{n+1}, $意味着$(r_{n-1}, r_n)=r_n$. 
综上所述, $(a,b)=(b, r)=(r, r_1)=\cdots = (r_{n-1}, r_n)=r_n$. 

用这个方法可以求一系列有趣的例子. 如Fibonacci数列中的一个有趣的性质: 

\begin{example}
    Fibonacci数列是$F_1=1, F_2=1, \forall n\in \mathbb N_+$都有$F_{n+2}=F_{n+1}+F_{n}$. 
    可以证明$(F_m, F_n) = F_{(m, n)}$. 

    仿照刚刚Euclid算法证明的过程, 我们可以尝试证明$(F_m, F_n) = (F_{n-m}, F_m)$. 当$n=m$的时候
    显然成立. 只研究$n>m$的情形. $n=mq+r, 0\leq r <m$. 

    而使用``1''的代换, 因为$F_1=1, F_2=1$, 可以乘上去并且对$F_{n-1}$再用一次. 有
    $F_n=F_{n-1}+F_{n-2}=F_2F_{n-1}+F_1F_{n-2}=F_2(F_{n-2}+F_{n-3})+F_1F_{n-2}$. 也就是
    $F_2F_{n-2}+F_2F_{n-3}+F_1F_{n-2}$, 提取公因式, 就有$F_3F_{n-2}+F_2F_{n-3}$. 继续这样做下去, 
    写作$F_3(F_{n-3}+F_{n-4})+F_2F_{n-3}$. 我们同样可以用同样的方法写作$F_4F_{n-3}+F_3F_{n-4}$, 
    一直做下去, 最后可以得到$F_mF_{n-m+1}+F_{m-1}F_{n-m}$.这里就出现了我们期待已久的$F_m, F_{n-m}$
    的结构了, 如果$d|F_m, d|F_n\implies d|F_{m-1}F_{n-m}$. 我们只要证明
    $F_m, F_{m-1}$是互素的. 根据Fibonacci的性质$(F_m, F_{m-1})=(F_{m-1}, F_{m-2})=\cdots=(F_2,
     F_1)=1$不难看出. 

    反之, $d|F_{n-m}, d|F_m \implies d|F_m, d|F_n$. 我们证得$(F_m, F_n) = (F_{n-m}, F_m)$. 
    持续地辗转相除, 即可得到. 
\end{example}

\lec{Bezout定理}{简介} TBD...Bezout定理可以帮助我们解决这样一类整数不定方程的问题: 

\begin{theorem}
    设$d=(a, b)$, 则存在$x,y\in\mathbb Z$, 使得$xa+yb=d$. 
\end{theorem}

\begin{proof}
    不妨设$a>b>0$, 则$a=bq+r, 0\leq r<b$. 只考虑除了最后一步的各个步骤, 余数大于0的情形. 由于
    $(a, b)=(b, r)$. 根据带余除法的过程, 有: 
    \begin{align*}
        b&=rq_1+r_1 &  0\leq r_1 < r,   \\
        r&=r_1q_2+r_2 &  0\leq r_2 < r_1,   \\
        r_1&=r_2q_3+r_3 &  0\leq r_3 < r_2,   \\
        &\cdots\\
        r_{n-3}&= r_{n-2}q_{n-1}+r_{n-1} & 0\leq r_{n-1}<r_{n-2}\\
        r_{n-2}&= r_{n-1}q_n+r_n & 0\leq r_n<r_{n-1}\\
        r_{n-1}&=r_nq_{n+1}   \\
    \end{align*}
    根据$d=(a, b)=(b,r)=(r, r_1)=(r_1, r_2)=\cdots=(r_{n-1}, r_n)=r_n$, 我们倒推回去
    \begin{align*}
    d&=r_n=r_{n-2}-r_{n-1}q_n \\
    &= r_{n-2}-(r_{n-3}-r_{n-2}q_{n-1}) \\
    &= -q_n r_{n-3}+(1+q_{n-1}q_n)r_{n-2} \\ 
    &\cdots \\
    &= xa+yb, (x, y\in \mathbb Z).
    \end{align*}
\end{proof}

从上面的内容可以看到, $d$可以写作辗转相除过程中, 任意相邻两步余数的线性组合. 那么这个问题的逆命题成立吗? 
这个显然是没有逆命题的. 上面这个定理的逆命题是若$d=(a,b)$, 则存在$x,y\in \mathbb Z$, 使得$xa+yb=d$. 
把这个式子乘上$k$, 得到$(kx)a+(ky)b=kd$, 这就说明有无数个可以表示为$a,b$的整系数线性组合的数. 它们不可能
都是$a,b$的最大公因数. 但是我们可以附加一个条件让我们的这个逆命题成立. 

使得$xa+yb=(a,b)$成立的$x,y$有多少组呢? 其实是有无数组. 只要我找到了其中的一组, 比如叫做$x_0, y_0$, 那么
我们就可以在$x_0$上面加上若干倍的$kb$, 在$y_0$上面减掉若干倍的$ka$, 形成$(x_0+kb)a+(y_0-ka)b$的形式. 
这个同样是满足原来的式子的. 

将上述形式稍加修改, 我们就得到了这个定理具有逆定理的形式. 

\begin{theorem}[充分必要的Bezout定理]
    $(a,b)=d \iff d|a, d|b,$并且$\exists x,y\in \mathbb Z, $使得$xa+yb=d$.
\end{theorem}

\begin{proof}
    必要性上述已经证明, 下面证明充分性. 

    我们先证明一个引理: $(a,b)$是形如$xa+yb(x,y\in \mathbb Z)$的正整数中的最小者. 记
    $S=\{xa+yb: x,y\in \mathbb Z\}, l_0=x_0a+y_0b$是$S$中的最小者. 下面证明$l_0=(a,b)$. 
    从$S$中取出任意的一个$l=xa+yb$, 若记$l_0|l$, 由于$l_0$是正数, 根据带余除法,  $l=l_0q+r, 0\leq r<l_0$.
    $$
    r=l-l_0q=(xa+yb)-(x_0a+y_0b)q=(x-x_0q)a+(y-y_0q)b\in S.
    $$
    但是$r<l_0$, $r$只能等于0. 意味着$l_0$是$l$的因数. 即$l_0|l$

    下面证明$l_0$是$(a,b)$. 一方面, $(a,b) | a, (a, b)| b \implies (a, b)|(x_0a+y_0b)$, 
    也就是$a,b$是$l_0$的因数. 

    另一方面, 由于$a,b\in S\implies l_0 | a, l_0 | b \implies l_0|(a, b)$.

    由于上述两方面, $l_0=(a,b)$. 引理证明完毕. 

    下面证明$l_0$就是$d$. 根据上述的引理, $(a,b)|(xa+yb)=d$, 又因为条件中的$d|a, d|b\implies d|(a,b)$

    所以$d=(a,b)$. 
\end{proof}

事实上, Bezout定理可以推广位$n$个正整数的最大公因数的情形. $\forall a_1, a_2,\cdots, a_n \in \mathbb N_+$,
$\exists k_1, k_2, k_3,\cdots, k_n\in \mathbb Z$, 使得$k_1a_1+k_2a_2+\cdots+k_na_n=(a_1, a_2,\cdots, a_n)$.
另外, 由于系数不唯一可能会导致一些研究的困难. 这时候我们对于系数做一些限定: 

\begin{theorem}
    设$a,b\in \mathbb Z$, $(a, b)=1$, 且$|a|\geq 2, |b|\geq 2$, 则$\exists u_0,v_0\in \mathbb Z$,
    使得$u_0a+v_0b=1,$且$|u_0|<|b|, |v_0|<|a|.$ 
\end{theorem}

\begin{proof}
    根据Bezout定理, 存在$u,v\in \mathbb Z$, 使得 $ua+vb=1$. 一定会有一个$u=qb+u_0$, 其中$0\leq u_0<|b|$,
    $1=ua+vb=(qb+u_0)a+rb=u_0a+(aq+v)b$. 令$v_0=aq+v$, 记作$u_0a+v_0b$. 我们只要证明$|v_0b|<|a|$即可. 

    我们发现$|v_0b|=|1-u_0a|\leq 1+|u_0a|=1+u_0|a|$. 由于$1<|a|$, 我们就知道$1+u_0|a|<|a|+u_0|a|$. 
    也就是$(1+u_0)|a|$. 由于$u_0<|b|, 1+u_0\leq |b|$, 那么继续放大为$|b|\cdot|a|$, 得到$|v_0|<|a|.$
\end{proof}

这个定理可以帮助我们求解一部分不定方程的求解问题. 我们下面来看这个例子: 

\begin{example}
    关于$x,y$的不定方程$ax+by=c(a,b,c\in \mathbb Z, a,b$不全为$0)$, 它有整数解的充要条件是
    $(a,b)|c$. 
    
    对于此的证明, 必要性是显然的, 设$d=(a,b)$, 则$d|ax, d|by, d|ax+by \implies d|c$.

    充分性: $(a,b)|c\implies \exists k\in \mathbb Z$, 使得$c=k(a,b)$. 由Bezout定理, 
    存在$x_0, y_0\in \mathbb Z$, 使得$$x_0a+y_0b=(a,b). $$
    
    将上述的内容两边同时乘上$k$, 有$kx_2a+ky_1b=k(a,b)=c$, 所以$kx_0, ky_0$是方程的一组整数解. 
\end{example}

\lec{算术基本定理}{介绍} 算术基本定理, 也被称为质因数分解定理, 是数论中的一个重要结果. 它告诉我们, 每个大于1
的整数都可以唯一地表示为一系列质数的乘积, 而且这个表示方式是唯一的. 

TBD: 一个无穷维的空间

\begin{theorem}[算术基本定理]
    设$n$是一个大于1的正整数, 则它可以写成$n=p_1p_2\cdots p_k$, 其中$p_i
    (1\leq i\leq k)$都是素数. 且在不计次序的情况下, 该表达式是唯一的. 
\end{theorem}

这个定理表明了每一个自然数的分解都是唯一的. 下面我们来证明. 分为存在性和唯一性两方面.

\begin{proof}
    存在性: 设$n>1$, 且$n\in \mathbb N_+$, 则$n$的最小的
    大于1的素数$p$必为素数. 若$p_1=n$, 则$n$为素数. 否则
    $p_1<n, $则$n/p_1$仍然为一个大于1的整数. 则$n/p_1$的最小的大于
    1的因数记作$p_2$必为素数. 若$p_2=n/p_1$, 
    则$n/p_1$为素数, 也就是$n=p_1p_2$. 

    若$p_2<n/p_1,$则可对$n/(p_1p_2)$重复前面的推理, 直到素数$p_k={n\over p_1p_2\cdots p_k}$,

    唯一性: 我们假设还有另外一个表达式$n=q_1q_2\cdots q_m$. 其中 
    $q_j(1\leq j\leq m)$都是素数. 不妨设$p_1\leq p_2\leq \cdots\leq p_k$,
    $q_1\leq q_2\leq \cdots\leq q_m$. 意味着$q_1|p_1p_2\cdots p_k$.

    回顾这样的一个定理: 如果素数$p|ab$, 那么$p|a$或$p|b$(可由Bezout
    定理推出), 那么$q_1$一定是某一个$p_i(1\leq i\leq k)$, 使得$q_1|p_i$. 
    根据素数的定义, $q_1=p_i$. 同理, $\exists q_j\in {q_1, q_2\cdots,q_m}$, 
    使得$q_j=p_1$. 因为$p_1=q_j\geq q_1=p_i\geq p_i$. 自然而然, $p_1=q_1$. 

    则在等式两边同时约去$p_1$, 重复上述推理. 得$p_2=q_2, p_3=q_3,
    \cdots$直到$q_{k+1}q_{k+2}\cdots q_m=1$, 这在$k<m$的条件下
    不成立. 必然$k=m.$ 唯一性得到证明. 
\end{proof}

我们可以把相同的数合并起来写在指数的位置, 记作$n=p_1^{a_1}p_2^{a_2}\cdots p_n^{a_n}$,
每一个$a_i\in \mathbb N_+$. 如果$d|n$, 那么就说明$d$可以写作$p_1^{b_1}p_2^{b_2}\cdots 
p_m^{b_m}$的形式, 每一个$b_i,i\in\{1,2,\cdots,m\}$是$a_1, a_2,\cdots, a_n$的其中一个.

那么有了这个, 我们简单看一看正因数的个数有几个. 假设有一个数$n=p_1^{a_1}p_2^{a_2}\cdots p_m^{a_m}$, 它的正因数有几个呢? 根据选法, 第一个$p_1$我们可以选择0个, 1个
一直到$a_1$个, 共$(a_1+1)$个选法. 第二个$p_2$我们可以选择0个, 1个
一直到$a_2$个, 共$(a_2+1)$个选法. 如此继续, 做到$p_k$才算是做完了. 我们发现
一个数的正因数有$$d(n)=(a_1+1)(a_2+1)\cdots(a_m+1)=\prod_{i=1}^m (1+a_i).$$

那么, 一个数$n=p_1^{a_1}p_2^{a_2}\cdots p_m^{a_m}$,的正因数之和为多少呢? 
我们可以通过素因子把它划分到如下的几个集合里面: 
$$
\{1, p_1, p_1^2, \cdots, p_1^{a_1}\}, \{1, p_2, p_2^2, \cdots, p_2^{a_2}\}, 
\cdots, \{1, p_m, p_m^2, \cdots, p_m^{a_m}\}, 
$$
$n$的每一个正因数相当于在这$m$个集合中各自选择一个因数, 把它乘到一起. 那我们把所有的选法
求和, 就是求出了所有正因数的和. 我们根据多项式的乘法法则, 就可以说明正因数的和$\sigma(n)$是
$$
\sigma(n)=\prod_{i=1}^m (1+p_i+p_i^2+\cdots p_i^{a_i}). 
$$

\lec{公因数与公倍数}{算术基本定理的视角}有了这样的理论, 我们就可以对于公因数和公倍数有一个更加直观的视角了. 因为这个分解是唯一的, 
\begin{theorem}[最大公因数和最小公倍数与唯一分解定理]
    设$a_i=\pf {p}{1}{a}{i_1} \pf {p}{2}{a}{i_2} \cdots \pf {p}{k}{a}{i_k}, i=1,2,\cdots,n$. 
    并且$p_1, p_2, \cdots, p_k$ 是不同的素数, $a_i\in \mathbb N$. 
    令$t_i=\min_{1\leq i\leq n}a_i, s_i=\max_{1\leq i\leq n}a_i$, 那么 
    $$(a_1, a_2, \cdots, a_n)=\pf p1t1 \pf p2t2 \cdots, \pf pktk;$$ 
    $$[a_1, a_2, \cdots, a_n]=\pf p1s1 \pf p2s2 \cdots, \pf pksk. $$
\end{theorem}

\begin{theorem}
    (1) 若$d|a_i, i=1,2,3,\cdots, n$, 则$d|(a_1, a_2, \cdots, a_n)$.
    
    (2) 若$a_i|A, i=1,2,3,\cdots, n$, 则$[a_1, a_2,\cdots, a_n] | A$. 
    
    (3) 若$m\in \mathbb N_+, $那么$$(ma_1, ma_2, \cdots, ma_n) = m(a_1, a_2,
    \cdots, a_n),$$同样有$$[ma_1, ma_2, \cdots, ma_n] = m[a_1, a_2,
    \cdots, a_n].$$

    (4) 可以通过化归转化的方法求解一组数的最大公因数: 
    $$(a_1, a_2,a_3\cdots, a_n)=((a_1, a_2),a_3\cdots, a_n), $$同样有
    $$[a_1, a_2,a_3\cdots, a_n]=[[a_1, a_2],a_3\cdots, a_n]. $$
\end{theorem}

\begin{theorem}
    $$ab=(a,b)[a,b].$$
\end{theorem}

\begin{proof}
    我们可以使用两边夹的方法说明这件事. 回忆定理若$a,b\in \mathbb N$, $a|b, b|a\implies 
    a=b$. 

    我们设$c={ab\over (a,b)}$, 由于$(a,b)|a, (a,b)|b$, 一定存在$k_1, k_2$, 使得
    $a=k_1(a,b), b=k_2(a, b), k_1,k_2\in \mathbb Z$. 
    
    根据前面的设定, $c(a,b)=ab=k_1k_2(a,b)^2$. 约去$(a,b)$. 于是$c=k_1k_2(a,b)$. 
    运用$a,b$的定义, 有$c=k_1k_2(a,b)=ak_2=bk_1$. 使用整除的定义, $a|c, b|c$. 
    根据上一个定理(2), $[a,b]|c. (1)$

    根据Bezout定理, 一定存在整数$x,y$, 使得$xa+yb=(a,b)$. 把等式两边同时除以最大公因数
    得到了$${a\over(a,b)}x+{b\over(a,b)}y=1$$也就是$k_1x+k_2y=1.$ 同时乘以$[a,b]$
    有: $$k_1[a,b]x+k_2[a,b]y=[a,b].$$ 做替换, 把$k_1$替换为$c/b$, $k_2$替换为 
    $c/a$, 得到$$\frac cb [a,b]x+\frac ca[a,b]y=[a,b].$$ 提取公因式$c$, 因此有
    $$c\left(\frac  {[a,b]}{b}x+\frac {[a,b]}{a}y\right)=[a,b].$$ 因此$c|[a,b]. (2)$

    根据(1), (2)可得$c=[a,b]$. 得证. 
\end{proof}

这个定理可不可以向多个整数推广呢? 这个是不可以的. 因为$$[a,b,c]=[[a,b],c]=\left[\frac{ab}{(a,b)},c\right]=\frac{\frac{abc}{(ab)}}{\left(\frac{ab}{(a,b)},c\right)}=\frac 
{abc}{ab,c(a,b)}=\frac{abc}{(ab, bc, ca)}. $$

我们取两个正整数, 他们互素的概率多大? 设$p_1, p_2, \cdots$是从小到大的素数. 
则$a$是$p_i$的概率是$1/p_i$, 同理$b$. 因此$a,b $有公共素因子$p_i$的概率为$1/p_i^2$.
因此, $a, b$没有任何公共素因子的概率为\mn{这个并不直观. 实际上, 我们还需要定义可数无穷的
概率的定义. 不过这份资料就不做定义了. 感兴趣可以参考概率相关的参考书. }
$$
\prod_{i=1}^{\infty} \left(1-\frac 1 {p_i^2}\right).
$$
要计算它可能要一点数学分析的知识, 我们直接给出概率的结果: $6/\pi^2$. 

\section{同余} 
\lec{同余}{简介}同余启发我们按照余数分类. 我们以前看到过奇偶分析, 这个就是相当于按照模2的余数分类. 另外, 
同余是一种有趣的语言. 下面我们来看一看什么是同余: 

\begin{definition}
    若$a,b\in \mathbb Z$除以$M$所得的余数相同, 
    则称$a,b$对模$M$同余. 记作$a\equiv b (\bmod m)$. 否则, 称为二者不
    同余. 记作$a\equiv b (\not \bmod m)$. 
\end{definition}

我们并不关心商是多少, 我们只关心余数是多少. 

下面我们给出$\ty abm$的充要条件: 
\begin{theorem}
    $\ty abm \iff m|(a-b).$
\end{theorem}
\begin{proof}
    $$\ty abm \iff\exists k_1, k_2\in \mathbb Z,$$使得$a=k_1m+r$, $b=k_2m+r$,
    $0\leq r\le m$. 根据同余的定义得到, $a-b=(k_1-k_2)m$, 根据整除的定义, $m|(a-b)$.  
\end{proof}

那么, 同余有哪些性质? 和整除理论一样, 我们发现: 

(1) $\ty amb, \ty cdm, \implies \ty{a\pm c}{b\pm d}{m}$. 等式的基本性质可以就可以放在这里了. 
我们可以用同余的定义和同余的充要条件证明之. 

(2) $\ty abm, \ty cdm, \implies \ty{ac}{bd}m$. 和上一个问题如出一辙. 

(3) 这是一个有趣的性质: $\ty abm\implies \ty{a^n}{b^n}m, n\in N_+$. 这个可以通过性质
(2) 不断作用$n$次得到. 

(4) $\ty {ac}{bc}m \implies \ty ab{m\over (m,c)}$. 

\begin{proof}
    设$(m,c)=d$, 则$m=dm_1, c=dc_1, $满足$(m_1, c_1)=1.$
    \begin{align*}
        \ty {ac}{bc}m &\iff m|c(a-b) \iff dm_1 | dc_1(a-b) \iff m_1|c_1(a-b) \\
        &\implies m_1|(a-b) \iff \ty abm
    \end{align*}
    也即是$\ty ab{m\over (m,c)} $.
\end{proof}

这个暗示了``约分''这一行为在模算术的做法. 

(5) 如果$\ty abm, n|m, n\in \mathbb N_+\implies \ty abn$. 这个表示了如果有一个数同余, 
它一定和这个数的因子同余. 

\begin{proof}根据定义, 我们有: 
    \begin{align*}
        &n|m \iff m=kn, k\in \mathbb Z \\
        & \ty abm \iff m|(a-b) \iff kn|(a-b)\\
        &\implies n|(a-b) \iff \ty abm
    \end{align*}
\end{proof}

(6) 与之类似, 我们有$\ty abm, \ty abn \implies \ty ab{[m,n]}$. 

(7) 如果我们用$\mathbb Z[x]$代表整系数多项式, 那么对所有整数$a,b(a\neq b)$, 有
$(a-b)|f(a)-f(b)$. 也就是说若$\ty abn$,对于某个$n\in \Z$成立, 则$\ty {f(a)}{f(b)}n$. 

\begin{proof}
    由于因式分解的相关知识, 我们知道$a-b|(a^m-b^m)$, 设$f(x)=\pl ckxk+\pl c{k-1}x{k-1}
    +\cdots+\pl c1x~+\pl c0~~$, $c_i\in \Z, i=0,1,2,\cdots, k$, 那么$f(a)-f(b)=
    c_k(a^k-b^k)+\cdots+c_1(a-b)$, 根据之前的结论, 因为$\ty abn$, 我们知道
    $n|(a-b)\implies n|(a^m-b^m)$, 于是得证. 
\end{proof}

\lec{同余}{同余类} 我们如果把余数相同的集合放在一起看会怎么样? 如果我们把整数按照这个方面
划分成若干类会如何? 下面我们来介绍同余类的概念. 

\begin{definition}[同余类]
    设$m$是大于1的给定的正整数, 则可将整数集划分为$n$个子集, 记作$k_0, k_1, \cdots, k_
    {m-1}$, 其中, $k_r=\{x|\ty xrm, x\in \Z\}, r=0,1,\cdots, m-1.$ 称$k_0, k_1
    \cdots, k_{m-1}$为同余类, 简单记作$[0], [1], \cdots, [m-1]$. 
\end{definition}

我们可以证明这是一个划分, 自然满足划分的两个性质. 根据定义, 两个数同属于同一个同余类, 当且仅当
他们对$m$同余. 也就是$a,b\in K_r \iff \ty abm \iff m|(a-b)$. 

\begin{definition}[完全剩余系]
    若整数$a_0, a_1, \cdots, a_{m-1}$中没有任何两个数同属于模$m$的一个同余类, 则称
    $a_0, a_1, \cdots, a_{m-1}$构成模$m$的一个完全剩余系. 
\end{definition}

这就相当于$k_0, k_1, \cdots, k_{m-1}$这$m$个同余系中, 每个同余类中取一个代表, 构成的
数组. \mn{后来如果你学习抽象代数, 你可能发现这就可以引申为商集. } 这是一个整体思想. 虽然
我们并不知道他们模$m$的结果具体是多少, 但是我们可以``筛选''某些特征, 使得我们可以让这个集合
是定下来的. 

这里有一些明显的性质, 我们说一下: 

(1) 如果$m$个整数构成的模$m$的完全剩余系$\iff$ 这$m$个数模$m$两两不同余. 

(2) 通过乘常数和加常数构造新的完全剩余系: 如果$a_0, a_1, \cdots, a_m$是
模$m$的一个完全剩余系, $a,b\in\Z$, 
且$(a,m)=1$, 那么$aa_0+b, aa_1+b, \cdots, aa_{m-1}+b$也是模$m$的完全剩余系. 

(3)通过两个完全剩余系构造: 若$(m,n)=1$, $a_1, a_2, \cdots, a_m$和$b_1,b_2, \cdots, 
b_n$是模$m$和模$n$的两个完全剩余系, 那么$\{na_i+mb_j:1\leq i\leq m, 1\leq j\leq n\}$是
模$mn$的完全剩余系. 要证明这个, 我们需要证明这个数组里面有$mn$个数, 并且模$mn$两两不同余. 

\begin{proof}
    我们可以把数组列一个$m$行$n$列的表格, 因此一共有$mn$个数. 
    $$\begin{array}{cccc}
        na_1+mb_1,&na_1+mb_2&\cdots&na_1+mb_n\\
        na_2+mb_1,&na_2+mb_2&\cdots&na_2+mb_n\\
        \cdots&\cdots&\cdots&\cdots\\
        na_m+mb_1,&na_m+mb_2&\cdots&na_m+mb_n\\
    \end{array}$$
    假设存在$a, a'\in \{a_1,
    a_2,\cdots, a_m\}, b, b'\in \{b_1, b_2,\cdots, b_n\}, $且$(a,b)\neq (a',b')$, 
    使得$\ty {na+mb}{na'+mb'}{mn}$. 
    \begin{align*}
        &\implies mn | n(a-a')+m(b-b')\\
        & \implies m| n(a-a')+m(b-b') \implies m|n(a-a') \\
        & \implies m(a-a') \implies \ty a {a'} m, 
    \end{align*}
    与$m$来自完全剩余系矛盾. 因此, $a=a'$, 同理,$b=b'$.  
\end{proof}

\lec{简化剩余系(缩系)}{简介} 既然我们可以按照这样的方式取, 我们还可以用什么样的方式
取? 一个想法是可以使用取和这个数互素的这些数. 

\begin{definition}
    设$K_r$是模$m$的一个同余类, 且$(r,m)=1$, 则称$K_r$为与$m$互素的同余类. 从每一个与
    $m$互素的同余类中各取一个数, 组成的的数组叫做\textbf{模$m$的简化剩余系}. 
\end{definition}

我们发现, 一个模$m$互素的剩余类中, 每个数都与$m$互素. 因为$a\in K_r, (r,m)=1$, 对$a$做
带余除法, $a=mq+r, 0\leq r\leq m-1$, 根据Euclid算法, $(a,m)=(m,r)=1$. 更进一步地, 
每个数都与模$m$互素. 

那么, 这样一个简化剩余系(缩系)里面有多少个数? 这就是大名鼎鼎的欧拉函数的来源了. 
模$m$的缩系可以看做一个模$m$中的一个完全剩余系中所有与$m$互素的数所构成的数组. 这个数
有$\varphi (m)$个数. 这就是$m$的欧拉函数.

\begin{definition} 如果记$\#S$表示$S$集合中元素的数量, 那么欧拉函数可以定义为
    $$\varphi(m) = \#\{k:1\leq k\leq m, (k, m)=1\}.$$
\end{definition}

那么$\varphi$函数具有怎样的性质呢? 我们应该如何计算呢? 一个直接公式的方法如下: 

\begin{theorem}
    $$
    \varphi(n)=n(1-\frac{1}{p_{1}})(1-\frac{1}{p_{2}})\cdots(1-\frac{1}{p_{k}}).
    $$
\end{theorem}

\begin{proof}
    我们可以使用容斥原理来说明这个问题. 设$1\sim n$这$n$个正整数中$p_i$的倍数的集合
    为$A_i, i=1,2,\cdots, k$, 则$|A_i|=n/p_i$; 同时是$p_i$和$p_j$的倍数的集合记作
    $|A_iA_j|=n/(p_ip_j)$;同时是$p_i$和$p_j$和$p_k$的倍数的集合记作
    $|A_iA_jA_k|=n/(p_ip_jp_k)$. 以此类推. 

    那么我们现在考虑与他们不互素的数有多少个. 我们记作$\left|\bigcup A_i\right|$, 根据容斥原理(我们会在下一讲说明这个原理的正确性. ), 有
    $$\left|\bigcup_{i=1}^k A_i\right|=\bigcup_{i=1}^k|A_i|-\bigcup_{1\leq i\leq j\leq k}|A_iA_j|+\bigcup_{1\leq i<j<m<k}|A_iA_jA_m|-\cdots+(-1)^{k-1}
    |A_1A_2\cdots A_k|.$$
    

    也就是
    $$
    \left(\frac n{p_1}+\frac n{p_2}+\cdots + \frac n{p_k}\right)-
    \left(\frac n{p_1p_2}+\frac n{p_3}+\cdots + \frac n{p_{k-1}p_k}\right)+
    \left(\frac n{p_1p_2p_3}+\cdots + \frac n{p_{k-2}p_{k-1}p_k}\right)-\cdots 
    +(-1)^k\frac{n}{p_1p_2\cdots p_k}. 
    $$

    于是与之互素的数是$\varphi(n)=n-\left|\bigcup A_i\right|$.

    化简这个式子, 就有
    $$\begin{aligned}
        \varphi(n)=&n-\left|\bigcup_{i=1}^k A_i\right|\\
        =&n\left(1-(\frac1{p_1}+\cdots+\frac1{p_k})+(\frac1{p_1p_2}+\cdots+\frac1{p_{k-1}p_2})-(\frac1{p_1p_2p_3}+\cdots+\frac1{p_{k-2}p_{k-1}p_k})\right.\\
        &+\cdots+(-1)^{k}\left.\frac{1}{p_{1}\cdots p_{k}}\right) \\
        =&n(1-\frac{1}{p_{1}})(1-\frac{1}{p_{2}})\cdots(1-\frac{1}{p_{k}})
    \end{aligned}$$
\end{proof}

\lec{数论倒数}{介绍} 我们已经将同余方程里面写成了方程的形式. 我们自然地要问一问: 有没有
办法解一个方程呢? 我们从最简单的一次同余式子开始看起. 

\begin{definition}
    设$m,a,b\in \Z$, $a\not \equiv 0 (\bmod m)$, 则$\ty {ax+b}{0}{m}$称为
    一次同余或者一次同余方程. 我们说如果两个解不同, 是指互不同余的解. 
\end{definition}

\begin{theorem}
    设$m\in \mathbb N_+, a,b\in \Z, a\not \equiv (\bmod m), $则$\ty {ax+b}0m$有解
    $iff (a,m)|b$.  
\end{theorem}

\begin{proof}
    $\implies: $设$c\in \Z$满足$\ty {ac+b}0m$, 则$ac+b=kx+m(k\in \Z)$. 设$d=(a,m)$
    , 则$d|a, d|m \implies d|b. $ 
    
    $\impliedby: $ 设$d=(a,m)$且$d$是$b$的因数, 则$b=kd, k\in \Z$. 由Bezout定理, 
    存在整数$u,v$, 使得$d=ua+vm$, 同时乘上$d$, 得到$kd=kua+kvm$, 即$kua+kvm=b$. 

    令$c=-ku$, 则$-ca+kvm=b$, 移项就有$ac+b=\ty{kvm}0m$. 也就是$c$是$ax+b=0$的
    一个整数解.
\end{proof}

但是解由多少个呢? 