\part{数论简介}

\section{质数和约数, Euclid算法}

任何一本书上面的开头好像都喜欢用Euclid算法求解最大公约数开场. 
这是一个十分古老的算法, 但是要是仔仔细细证明这个算法, 还是不那么显然的. 

下面我们来借助这个算法, 来简单回顾一些数论的基本概念. 

要求两个数$m$, $n$的\textbf{最大公因数(greatest common divisor, gcd)}, 
Euclid发现了这样一个算法并声称:$\gcd(m, n) = \gcd(n\%m, n)$. 
这里, 沿用C++里面的取模, $a\%b$的值就等于$a/ b$的余数. 

如果我们用更聪明的记号来表示的话, 或许$a\%b=a-\lfloor a/b\rfloor\times b$. 
就相当于模拟了$b$次减法. $\lfloor x\rfloor$表示$x$下取整. 比如$\lfloor \pi\rfloor=3$. 
这是一个很有趣的小符号, 它实际上表示的是一个不等式的关系. 

因数的概念可能是数学中产生的最自然的概念之一. 
当我们要对一个东西平均的分配的时候, 一个数"能被另一个数整除"这个性质就显得尤其重要. 
这里, 我们发现一些数只能被$1$和这个数本身整除, 
这样的数我们一般叫做\textbf{"质数(prime)"}. 比如$3=1\times3$. 
(其实$3=-1\times -3$, 但是这里我们只考虑这些乘数都是正的). 
还有一些数分解的可能就多了不少. 例如$14=1\times 14 = 2\times 7$. 
这样的数我们称作\textbf{合数(composite)}.

更一般的, 有时候我们仅仅关注一个数能不能被另一个数整除,
也即是$a/ b$的余数是不是为0, 如果是, 
我们就说$b$是$a$的一个\textbf{因子(factor)}. 
有时候可以写作$b|a$. 像是从$a$中"抓出来"了一个它的更小的部分放在前面. 

所以我们有一个形式化的定义: 

\begin{definition}
    $m | n \iff m>0 \text{ 并且 } \text{对于某个整数}k, n=mk. $
\end{definition}

在知道为什么这个算法是对的之前, 我们需要发掘一点整除的性质. 