\part{C语言回顾}

有了计算机, 我们可以把很多重复的工作交给计算机完成.
这样, 人们就可以把更重要的经历放在更主要的事情上面去了. 
其中, 程序设计语言担当了我们人类世界与机器世界沟通的桥梁, 
我们只有通过程序设计语言(如C语言), 计算机才会按照我们
希望的方法工作. 当然, 我们对于计算机的期望很多时候是失败的
这时候, 我们只有通过一些外部工具, 来证明或者否定我们对于
计算机内部一些事情工作的原理. 

\begin{axiom}
    机器永远是对的. 
\end{axiom}

TBD: 一个简单的参考, 列举常用工具而不用关心其逻辑实现. 假设已经十分熟悉
了. 

\lec{模拟}{介绍} 程序执行的状态 

\ti{P3592} \file{P3592}

\subsection*{闲聊与练习}

\begin{quote}

    \fbox{核心指导原则}  Don't Panic. (不要慌) 
    
    \hfill—— The Hitchhiker's Guide to the Galaxy

    如果你还没有入门, 仍然感到恐惧, 请记住: 坚持住, 进入未知领域, 从简单的、能理解的东西试起, 
    投入时间, 就有收获. 
    
    掉在队伍之后的同学, 即便是仅有一定的编程基础, 努力过的同学也一定能通过 (Yes!)

    \hfill --- 蒋炎岩, 在南京大学\href{http://jyywiki.cn/OS/OS_Guide}{操作系统}课前的提示
\end{quote}