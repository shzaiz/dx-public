\section{问题的提出: 通过程序, 可以求解的问题}

上学期我们学习了Python程序设计. 可能我们会对学程序设计合理性产生质疑. 
\subsection{为什么学习计算机编程}
看上去, 计算机是一个笨重的工具, 在时间或空间限制下可以有效地执行许多重复的任务. 到现在为止, 计算机还不能分析问题并提出解决方案. 另一方面,
人类在分析和解决问题方面非常出色, 但很容易对重复的任务感到厌倦. 

人类通过利用他们的分析和解决问题的技能, 可以为可计算的问题提出算法(有限的指令, 与有限的输入一起工作, 产生输出) . 然后,
计算机可以遵循这些指令并产生一个答案. 

你可以将你生活中的各种多余的任务自动化. 一点点的Python可以给你的生活带来神奇的效果. 甚至提高你的生产力. 

\subsection{程序设计的语法与语义}

我们在学习英语的时候很重要的一部分是语法: 也就是什么样的语言是可以被接受(acceptable)的. 比如下面这个英文句子是没有语法错误的: 
\begin{verbatim}
Fuhai Zhu said that this test is only a small test, so don't panic.
\end{verbatim}
但是这句话不同人有不同的理解方式, 这就是这句话的语义. 
\begin{itemize}
\item 我们可以推断朱富海老师在安抚准备参加小测试学生们的情绪
\item 但是南京大学数学系的同学知道这是一个有名的梗: 上学期教高等代数的朱富海老师把线上期中考叫做``小测验'', 并在同学询问考试范围的时候微笑的答道:``从小学学的都考.''
\end{itemize}
\begin{quote}
简而言之(不严谨) , 现在我们有一个由一堆字符串和推导规则组成的形式系统, 语法决定了这个形式系统能生存什么样的字符串, 而至于这些字符串有什么样的含义则是语义的范畴.
语法类似材料, 语义类似与材料组成的各种建筑物, 我们可以通过语法研究语义层面的推导, 同时也可以从语义层面捕获语法中内涵的结构,
其实语法和语义是相互区别又紧密联系, 即从范畴论的角度看语法和语义是伴随的(其实不同的人做数学证明可以有不同的风格: 偏语法和偏语义,
不过大部分数学家更喜欢语义风格的证明, 可能因为更直观, 更容易被人脑接受\textasciitilde\textasciitilde ) 

作者: 知乎用户 链接: https://www.zhihu.com/question/31347357/answer/892133941
来源: 知乎 著作权归作者所有. 商业转载请联系作者获得授权, 非商业转载请注明出处. 
\end{quote}

\subsection{Python代码执行可视化: 一个网站}

\begin{tool}
这是\href{https://pythontutor.com/visualize.html\#mode=display}{Python代码执行可视化}的机器,
我们可以用一个小程序来测试之.
\begin{lstlisting}[language=python]
for i in range(10):
    print(i**2)
\end{lstlisting}

点击Visualize Execution就可以了, 你可以点击Next来继续模拟执行下一步.

在这里, 你可能会看到很多新奇的名词: 什么是Global Frame, Object? 暂时先不用管. 不过你确实可以看到点击Next的时候Print output一栏一步一步的模拟了你的代码. 
\end{tool}

可以知道, 代码按照行数执行, 一次执行一行, 每一次执行计算机内部结构的状态(右侧的面板). 下面我们化繁为简, 来看一看一个系统(数学意义上)能够完成任何人类完成的操作需要的最小可能的操作是什么. 

\subsection{可以执行任何程序的最小指令集}
像数学的公理体系那样, 我们自然希望得到一个最小的指令集合, 并且我们可以用来写出任何的程序. 我们不妨从日常生活中找一点灵感吧.
\begin{example}
(等红绿灯) 观察红绿灯, \textbf{如果}是绿灯, 那就通过这个路口; \textbf{否则}继续等待. (遵纪守法的好公民)

(做作业) 明确今天的作业范围, 从\textbf{第一题}开始写, 写完题目\textbf{或者}一题目没有思路之后做\textbf{下一道}题,
\textbf{直到}做完所有的问题.

(排序成绩单) 获得班上同学的所有\textbf{成绩单}, 拿一张新的白纸打好\textbf{表格}, \textbf{每一次}从成绩单中选取最大的分数,
把那一行\textbf{抄写到}新的白纸上. 之后把原来那张纸上的内容划去. \textbf{一直重复下去}, \textbf{直到}原来的成绩单上没有任何可以被划去的内容.
\end{example}
我们需要找一些东西来具象化我们脑子中的``红绿灯的状态'', ``现在在做作业的题目编号'', 这些内容, 因此我们就希望把这些抽象出来.
因此我们有了变量的概念, 也就是值存在的空间. 

把上面的三个内容转化成伪代码(不唯一)就是:
\begin{verbatim}
------ GO THROUGH CONJUNCTION ------
if traffic light's color is green:
    go pass by
else
    wait

------ DO HOMEWORK ------
range = [a..b]
working on problem = a
while working on problem <= b:
    finished this problem or can't work out
    working on problem = working on problem+1

------ SORT EXAM SCORE ------
list = get the source table
result = empty(for now)
while list is not empty
    k= get the max element of list
    write k to the next line of result
show result
\end{verbatim}
其实要是能够构造出任何程序的原材料并不复杂. 无非就是变量的赋值, 判断, 跳转, 终止. 也就是, 如果你能声称有一套系统可以自动化的解决这四个内容,
那么这个系统就具有机械化地做任何人类做的事情. 换句话说, 你可以用这个工具创造整个世界. 

其实最早这个想法是在计算机诞生之前人们孜孜以求的问题. Alan Turing 在1936年就提出了这样的设想. 他就是由只一条(无限长)的纸带和一根笔(可以改纸带的内容,
并且查看纸带的内容并据此做判断), 并且有一个程序(墙上的表格), 指示下一步要往哪转移. 只要能够移动读写头, 写纸带的某一个格子,
读纸带的某一个格子, 跳转, 以及终止, 这个机器就和我们人类的计算能力等价. 

这是最早的图灵机的原型. 

% TODO 添加图片

\begin{example}
运行上面图片的程序, 左右按照我们的左右进行(规定A\textbf{B}C右移一格是AB\textbf{C}).

(1) 现在机器的状态是A(头部的字母), 看到的是1(放大镜的字母)

(2) 于是把当前的格子改为1, 纸带向右移动一格, 然后停机.

\textbf{假设}当前纸带的放大镜看到的是0, 再运行一次:

状态:A 纸带状态: 0 1 1 1 \textbf{0} 1 1 0 0

(1)现在机器的状态是A(头部的字母), 看到的是0(放大镜的字母), 执行第一行第一列的指令$1(\text{改\text{为1)}}\rightarrow(\text{向\text{右移动一格)}}B(\text{状\text{态改为B)} . }$

状态:B 纸带状态: 0 1 1 1 1 \textbf{1} 1 0 0

(2)现在机器的状态是B(头部的字母), 看到的是1(放大镜的字母), 执行第二行第二列的指令$1(\text{改\text{为1)}}\rightarrow(\text{向\text{右移动一格)}}B(\text{状\text{态改为B)} . }$

状态:B 纸带状态: 0 1 1 1 1 1 \textbf{1} 0 0

(3)现在机器的状态是B(头部的字母), 看到的是1(放大镜的字母), 执行第二行第二列的指令$1(\text{改\text{为1)}}\rightarrow(\text{向\text{右移动一格)}}B(\text{状\text{态改为B)} . }$

状态:B 纸带状态: 0 1 1 1 1 1 1 \textbf{0 }0

(4)现在机器的状态是B(头部的字母), 看到的是0(放大镜的字母), 执行第二行第一列的指令$0(\text{改\text{为0)}}\rightarrow(\text{向\text{右移动一格)}}C(\text{状\text{态改为C)} . }$

状态:C 纸带状态: 0 1 1 1 1 1 1 0 \textbf{0}

(5)现在机器的状态是C(头部的字母), 看到的是0(放大镜的字母), 执行第三行第一列的指令$1(\text{改\text{为1)}}\leftarrow(\text{向\text{左移动一格)}}C(\text{状\text{态改为C)} . }$

状态:C 纸带状态: 0 1 1 1 1 1 1 \textbf{0} 1

(6)现在机器的状态是C(头部的字母), 看到的是0(放大镜的字母), 执行第三行第一列的指令$1(\text{改\text{为1)}}\leftarrow(\text{向\text{左移动一格)}}C(\text{状\text{态改为C)} . }$

状态:C 纸带状态: 0 1 1 1 1 1 \textbf{1} 1 1

(7)现在机器的状态是C(头部的字母), 看到的是1(放大镜的字母), 执行第三行第二列的指令$1(\text{改\text{为1)}}\leftarrow(\text{向\text{左移动一格)}}A(\text{状\text{态改为A)} . }$

状态:A 纸带状态: 0 1 1 1 1 \textbf{1} 1 1 1

(8)现在机器的状态是A(头部的字母), 看到的是1(放大镜的字母), 执行第一行第二列的指令$1(\text{改\text{为1)}}\rightarrow(\text{向\text{右边移动一格)}}\dagger(\text{停\text{机)}}$

状态:A 纸带状态: 0 1 1 1 1 1 \textbf{1} 1 1
\end{example}
下面我们来看一看为什么说可以用Python创造整个世界. 在这之前, 我们先来了解一下如何获取帮助和得到通俗易懂的教程. 

请认真阅读并实践(无论是在脑子还是在交互器里面)文档https://docs.python.org/zh-cn/3/tutorial/introduction.html的内容.
可以让你了解更多易于理解的东西. 如果文章中有描述C和Pascal的句子, 忽略它就可以. 

\begin{idea}
摒弃``我一定要一次把所有的东西都弄懂''的理念, 因为这些知识十分的巨大. 所以先了解一部分作为正确的立足点, 然后慢慢扩大就行了. 这需要很长时间的积累, 千万不要急. (着急也没用, 只会加大精神内耗.)	
\end{idea}

\subsection{Python中的整个世界}

下面的内容其实不用单独记忆, 只要明确有哪些语句, 这些语句造成的效果是什么就行了. 如果看到有任何的问题, 可以去搜一搜词典. 

下面会有两个术语(term), 分别是表达式(expression)和过程(procedure), 表达式可以暂且认为是形如x+12,
{[}2{]}{*}3这样的可以进行计算的内容, 过程就是一系列执行的过程, 不一定要能得到值. 我们用一个例子感受一下.
\begin{lyxcode}
def~InsertionSort(A):~~~~~~~~~~~~~~~~~~~~~~~~~~~~

~~~~for~j~in~range(1,~len(A)):~~~~~~~~~~~~~~~~~~~\#Proc

~~~~~~~~key~=~A{[}j{]}~~~~\#A{[}j{]},key~are~expr~\#Proc~~~\#|

~~~~~~~~i~=~j~-~1~~~~~\#j-1,i~are~an~expr~\#v~~~~~~\#|

~~~~~~~~while~(i~>=0)~and~(A{[}i{]}~>~key):~~~~~~~~~~\#|~~~

~~~~~~~~\#~~~~<-expr->~~~~<-{}-{}-{}-expr-{}-{}-{}-{}->~~~~~~~~~\#|~~~

~~~~~~~~\#~~~~<-{}-{}-{}-{}-{}-{}-{}-{}-{}-{}-expr-{}-{}-{}-{}-{}-{}-{}-{}-{}-{}->~~~~~~~~\#|~~~

~~~~~~~~~~~~A{[}i+1{]}~=~A{[}i{]}~\#A{[}i{]}is~expr~~~\#Proc~~~\#|

~~~~~~~~~~~~i~=~i~-~1~~~~~\#i-1~is~expr~~~\#v~~~~~~\#|

~~~~~~~~A{[}i+1{]}~=~~~~key~~~~~~~~~~~~~~~~~~~~~~~~~~\#v

~~~~~~~<-expr->~~<-expr->
\end{lyxcode}
其实上面的Proc表示过程, 然后右边的是一个字符画, 表示$\downarrow$, <-expr->其实表示的意思是$\underbrace{\texttt{example}}_{\text{expr}}$这一段是表达式. 

重要的是, 把示例代码放到上面提到的可视化网站里面看一看就会很清楚, 很多概念都是不用记忆的. 

\subsubsection{变量的定义与赋值}
\begin{definition}
(变量的赋值) 变量名=变量的值
\end{definition}
语义: 

下面我们给出注解:
\begin{itemize}
\item 在不加修饰的情况下, 变量的名称只在当前的缩进块内有效
\item 命名是用来指代对象的. 这就是为什么有时候可视化工具里面Frames后面有一个箭头指着Objects.
\item 如果用一个变量=另一个变量, 大多数情况是现计算出来右手边表达式的值之后给左手边的变量. 有时候一些文章里面写作$lhs\leftarrow rhs$.
\end{itemize}
\begin{lyxcode}
b=114514

a=b+1~\#执行完本句之后a=114515

b~=~b+1~\#~执行完本句之后b=114515,~a=114515不变

a~=~b+1~\#~执行完本句之后a=114515,~b=114516
\end{lyxcode}
\begin{itemize}
\item \textbf{在Python中}, 变量的值的类型可以是任意的. 因为Python声明变量的时候没有说明类型.
\end{itemize}
\begin{lyxcode}
a=''Fuhai~Zhu~teached~Advanced~Algebra''~\#~a现在是字符串

a=1~\#~a现在是整数

a=None~\#~None是一个关键字,~表示什么都没有.
\end{lyxcode}
\begin{itemize}
\item 如果没有定义就使用了一个变量, 通常就会有如下的报错:
\end{itemize}
\begin{lyxcode}
print(a+1)

~~~~~~\textasciicircum{}

Traceback~(most~recent~call~last):~~~File~\textquotedbl <stdin>\textquotedbl ,~line~1,~in~<module>~

NameError:~name~'n'~is~not~defined

(命名错误:~名称'n'没有定义)
\end{lyxcode}
什么是Traceback? stdin又是什么? 后面可能会注意到. 

可能经常会常用的变量类型: 数字、字符串、列表. 这时候可以参看官方文档https://docs.python.org/zh-cn/3/tutorial/introduction.html来继续.

\subsubsection{控制语句: 判断与循环}
\begin{definition}
(条件判断) 可以使用if语句进行条件判断, 一般的, 有如下的形式:

\noindent\begin{minipage}[t]{1\columnwidth}%
\begin{lyxcode}
if~表达式1:

~~~~过程1

elif~表达式2:~\#~可以有零个或者多个elif,~但是else后面不能有elif

~~~~过程2

else:

~~~~过程r
\end{lyxcode}
%
\end{minipage}
\end{definition}
语义: 它通过逐个计算\textbf{表达式}, 直到发现一个\textbf{表达式}为真, 并且执行使\textbf{表达式}为\textbf{真}的这个\textbf{过程}(完成后\textbf{不}执行或计算if语句的其他部分的判断\textbf{表达式})
. 如果所有表达式都为false, 如果存在else下方语句块的过程.

下面我们同样给出注记和例子.
\begin{itemize}
\item 什么是真? 什么是假? 我们会在后面探讨. 首先可以认为非0数字和True是真, 0和False和None是假. 
\end{itemize}
\begin{lyxcode}
if~\textquotedbl AK\textquotedbl :

~~~~print(\textquotedbl AK\textquotedbl )~\#~会输出AK,~这是怎么判断的?(后续会回答)
\end{lyxcode}
\begin{itemize}
\item 可以用逻辑运算符 and(且) or(或) not(非) 进行逻辑表达, 比如
\end{itemize}
\begin{lyxcode}
zgw~=~0

kertz~=~1

ak~=~1

cmo~=~1

if~kertz~and~ak~and~cmo~:

~~~~print(``Zixuan~Yuan~got~full~mark~in~CMO'')

elif~zgw~and~ak~and~cmo:

~~~~print(``zgw~got~full~mark~in~CMO'')

else:

~~~~print(``zgw~is~such~a~noob'')

\#~会输出Kertz~got~full~mark~in~CMO,~由于已经找到了一个表达式的值为真的

\#~表达式,~所以执行完print(``Zixuan~Yuan~got~full~mark~in~CMO'')之后就

\#~会跳转到这个语句块的尾部了.~不会执行print(``zgw~is~such~a~noob'').

\#~(为自己菜爆的数学基础做了一个掩盖(大雾))
\end{lyxcode}
\begin{itemize}
\item 如果结构不完整, 或者在else之后还有elif, 那么就会出发形如这样的错误:
\end{itemize}
\begin{lyxcode}
例子1.py-{}-{}-{}-{}-{}-{}-{}-{}-{}-{}-{}-{}-

if~True:

print(\textquotedbl Err\textquotedbl )

-{}-{}-{}-

File~\textquotedbl main.py\textquotedbl ,~line~3

~~~~print(\textquotedbl Err\textquotedbl )

~~~~\textasciicircum ~IndentationError:~expected~an~indented~block

(缩进错误:~我预期有一个带着缩进的语句块,~但是没有)

例子2.py-{}-{}-{}-{}-{}-{}-{}-{}-{}-{}-{}-{}-{}-{}-

if~False:

~~~~print(1)

else:

~~~~print(2)

elif~True:

~~~~print(3)

-{}-{}-

~~File~\textquotedbl main.py\textquotedbl ,~line~5

~~~~elif~True:

~~~~\textasciicircum ~SyntaxError:~invalid~syntax

(语法错误:~无效的语法)
\end{lyxcode}
\begin{definition}
(while循环) 可以使用if语句进行条件判断, 一般的, 有如下的形式:

\noindent\begin{minipage}[t]{1\columnwidth}%
\begin{lyxcode}
while~表达式:

~~~~过程1

else:~\#~可以有,~也可以没有

~~~~过程2
\end{lyxcode}
%
\end{minipage}
\end{definition}
语义: 这样反复测试表达式, 如果为真, 则执行\textbf{过程1};如果表达式为假(这可能是第一次测试) , 则执行else子句的\textbf{过程2}(如果存在的话)
, 然后循环终止. 在\textbf{过程1}中执行的\textbf{break}语句会终止循环, 且不执行else子句的\textbf{过程2}.
在\textbf{过程1}中执行的continue语句跳过\textbf{过程1}的continue语句之后的其余部分, 然后立刻回到测试表达式语句. 

有了循环, 我们就可以解读这个东西:
\begin{lyxcode}
def~InsertionSort(A):

~~~~j=1

~~~~while(j<len(A)):

~~~~~~~~key~=~A{[}j{]}

~~~~~~~~i~=~j~-~1

~~~~~~~~while~(i~>=0)~and~(A{[}i{]}~>~key):

~~~~~~~~~~~~A{[}i+1{]}~=~A{[}i{]}

~~~~~~~~~~~~i~=~i~-~1

~~~~~~~~A{[}i+1{]}~=~key

~~~~~~~~j=j+1

~~~~return~A

InsertionSort({[}1,1,4,5,1,4{]})
\end{lyxcode}
这个做的事情就和排序成绩类似. 


\subsubsection{程序的终止}
\begin{definition}
Python程序的终止可能包含有如下的情况:

(1) 执行到了最后一条语句, 且没有下一条语句可以执行;

(2) 程序有没有被处理的异常;

(3) 通过语句exit(0)退出. 
\end{definition}
因此, 我们就得到了最小的可以(理论上)执行任何与人类计算能力等价的模型

这些内容看上去十分的平凡, 但是通过一些过程的复合, 我们就能看到更多的魔力.

\subsection{函数: 整合相似过程}

我们可以把相似的过程写在一起, 为了简洁和可维护.

下面, 可以阅读https://docs.python.org/zh-cn/3/tutorial/controlflow.html\#defining-functions
的4.7, 4.8.1-4.8.6节的内容, 把所有代码是怎么执行的放在pythontutor里面模拟着看一遍. 文字可以不用看,
但是代码一定要执行一遍. 

\subsubsection{递归(Recursion)过程和栈帧(Stack Frame)}

观察下面的代码, 可能难以想象是怎么执行的:
\begin{lyxcode}
def~fib(n):

~~~~if(n==1):

~~~~~~~~~return~1

~~~~if(n==2):

~~~~~~~~~return~1

~~~~else:

~~~~~~~~~return~fib(n-1)~+~\textbackslash{}

~~~~~~~~~~~~~~~~fib(n-2)

fib(5)
\end{lyxcode}
像这样用自己调用自己的函数调用通常叫做递归(recursion). 一个关于递归的有趣定义是:
\begin{quote}
递归的定义: 如果你没有理解什么是递归, 那么参见递归. 
\end{quote}
事实上, 我们可以把它放在pythontutor里面执行一下, 发现如下的规则:
\begin{itemize}
\item 原来的程序就像是一张纸, 上面标注着当前执行到的行数;
\item 每次函数调用的时候, 就会在一张新的纸片上抄下来调用的内容, 并且代换传进来的参数;
\item 把这个内容放在原来纸片上面, 然后从第一行开始执行;
\item 执行完的纸片扔掉.
\end{itemize}
看上去就像是:
\begin{itemize}
\item 你在晚自习上看课外书(执行原来的函数) 
\item 老师来了, 让你写作业(函数调用) 
\item 你把作业叠放在课外书上, 开始做作业 (执行函数)
\item 做完作业之后你把作业扔了继续看课外书(回到原来的函数)
\end{itemize}
像羽毛球球桶那样, 只能从一个方向插入, 弹出的内容的东西叫做``\textbf{栈(stack)}'', 由于这些内容通常都是一些数据,
由此我们用术语\textbf{数据结构}(data structure)来描述. 能被取出来的那个元素是\textbf{栈顶(top
of the stack)}, 在这个可视化工具里面用蓝色标示出来了. 

Traceback就是出错之后, Python顺着栈一层一层找的结果. Trace是跟踪, back是返回, 意思可能就是说堆栈的\textbf{回溯(traceback)}.





\section{一些小例子}

下面我们来看一些基础的例子, 来体会上一节中的一些思考: 

\ti{片段1. 热身练习}

\begin{lstlisting}[language=Python]
def getpercent(chinese, math, english): 
    return (chinese+math+english)/(150+150+150)

print(getpercent(100, 120, 135))
\end{lstlisting}

\begin{bonus}
如何判断一个程序的行为? 为了刻画程序的行为, 我们能不能像第一节里面提到的那样, 用一个``模型''来描述它?

宏观来说, 程序就是状态机. 程序的执行就是状态的迁移. 到底有哪些状态呢?
\end{bonus}

\ti{片段2. 计算钱数之和}

\begin{lstlisting}[language=Python]
money = [10, 14, 13, 10]
int total = 0
for i in money: 
    total = total + i

print(money)
\end{lstlisting}

这里的\texttt{for i in ...}只是\texttt{for u in range(len(money))}, \texttt{i=money[u]}的简称罢了. 对我们的实际程序没有太多的障碍. 还是逃脱不了循环的大框架. 我们在前面说的最小指令集还是起作用的. 

\ti{片段3. 求一个数是不是素数}

\begin{lstlisting}[language=Python]
def isprime(x): 
    flag = 1 
    d=2
    while d<x:
        if x%d == 0: 
            flag = 0
        d = d+1
    if flag: 
        return 1
    else: 
        return 0
\end{lstlisting}

在这个程序片段中, \texttt{flag} 代表了什么? 是不是有些像现实世界中的信息传递方式? 因为我们无法跨越循环次序改变程序的行为, 我们只好使用变量\texttt{flag}来记录, 并且让程序的控制流通过\texttt{flag}来进行判断与执行. 

\begin{bonus}
	素数(prime)有什么数学性质? 我们是用什么性质来判定性质的? 在我们下达的指令里面, 形式上和数学表达式相似吗? 
\end{bonus}

事实上, 数学上的定义是$\forall x, \not \exists p, s.t. p|x$. 我们的程序看上去并不是像数学那样简洁. 其实, 程序设计家族也有其他成员可以写起来比较优美. 

\begin{pas}
	\begin{center}
		\large \textbf{程序设计家族的其他成员: 不只有下达命令}
		
	\end{center}
	\begin{center}
		南京大学~李樾等~~节选自《程序分析》教科书
	\end{center}
	在IP中,指令一个一个给出,用条件、循环等来控制逻辑(指令执行的顺序),同时这些逻辑通过程序变量不断修改程序状态,最终计算出结果。我觉得,尽管IP现在都是高级语言了,但是本质上并没有脱离那种“类似汇编的,通过读取、写入等指令操作内存数据”的编程方式(我后面会提及,这是源于图灵机以及后续冯诺依曼体系结构一脉的历史选择)。国内高等教育中接触的绝大多数编程语言都是IP的,比如Java、C、C++等。
	
在FP中,逻辑(用函数来表达)可以像数据一样抽象起来,复杂的逻辑(高阶函数)可以通过操纵(传递、调用、返回)简单的逻辑(低阶函数)和数据来表达,没有了时序与状态,隐藏了计算的很多细节。不同的逻辑因为没有被时序和状态耦合在一起,程序本身模块化更强,也更利于不同逻辑被并行的处理,同时避免因并行或并发处理可能带来的程序故障隐患,这也说明了为什么FP语言如Haskell在金融等领域(高并发且需要避免程序并发错误)受到瞩目。

\begin{lstlisting}[language=lisp]
	(defun quadratic-roots-2 (A B C)
  (cond ((= A 0) (string "Not a quadratic equation."))
    (t
    (let ((D (- (* B B) (* 4 A C))))
      (cond ((= D 0) (concatenate 'string "x = " (write-to-string (/ (+ (- B) (sqrt D)) (* 2 A)))))
        (t
        (values (concatenate 'string "x1 = " (write-to-string (/ (+ (- B) (sqrt D)) (* 2 A))))
                (concatenate 'string "x2 = " (write-to-string (/ (- (- B) (sqrt D)) (* 2 A)))))))))))
\end{lstlisting}

LP抽象的能力就更强了(用逻辑来表达),计算细节干脆不见了。把你想表达的逻辑直观表达出来就好了:如“第三代火影的徒弟” 且 不是“女性” 且 “其徒弟也是火影”=>”自来也“。嗯,学会”与或非“,编程都不怕。 如今,在数据驱动计算日益增加的背景下,LP中的声明式语言(Declarative programming language,如Datalog)作为代表开始崭露头角,在诸多专家领域开拓应用市场。我们这本小书也准备用一章节来教大家如何使用Datalog语言编写程序分析器。
	\begin{lstlisting}
ancestor(A, B) :-
    parent(A, C),
    ancestor(C, B).
	\end{lstlisting}
\end{pas}



\ti{片段4. Perfect数}

\begin{lstlisting}[language=Python]
def is_perfect(n):
    sum=0
    for i in range(1,n):
        if n%i==0:
            sum=sum+i
    return sum==n    
\end{lstlisting}

其中\texttt{range}现在可以认为是生成$[1,n)$的列表. 并且每一次循环就取列表的下一个元素. 比如\texttt{for i in range(1,5)}每次循环\texttt{i}的值会是\texttt{1 2 3 4}. 

在这个实例中, 逻辑关系体现的如何? 

\ti{片段5. 求3和5的因数个数}

\begin{lstlisting}[language=Python]
def multiples_of_3_and_5(n):
    sum=0
    for i in range(1, n):
        if i%3==0 or i%5==0:
            sum=sum+i
    return sum   	
\end{lstlisting}

可以注意我们的逻辑在Python里面是如何表达的? 还有哪些逻辑表达关系? 事实上, 这也是后续离散数学部分要学的命题逻辑--我们需要对于以前的逻辑有一个比较确切的定义. 

\begin{bonus}
	命题``若$p$则$q$''的否定是什么? 
\end{bonus}

普通的高中毕业生基本是无法回答这个问题的. 因为课本的知识完全没有提及类似的问题. 这就导致我们的高中数学看上去更像是民科学习的数学. 同时轻松地毁掉了高中与大学的衔接过程. 


\ti{片段6. 一定范围内勾股数的个数}

\begin{lstlisting}[language=Python]
def integer_right_triangles(p):
    #  a^2+b^2=c^2    a+b+c=p   c is the longest
    count=0
    for a in range(1,p):
        for b in range(a,p-a):
            c=p-a-b
            if a+b+c==p and a**2+b**2==c**2:
                count+=1
                print(a,b,c)
    return count   
\end{lstlisting}

这个例子对于数学的关系好像更加清晰了. 比如直角三角形数对有兴致$a^2+b^2=c^2$. 于是剩下的就比较像自然的``数数''一样了. 
    
\ti{片段7: 递归的力量}
\begin{lstlisting}[language=Python]
def fib(n):
    if(n==1): 
        return 1
    if(n==2): 
        return 1
    else: 
        return fib(n-1) + \ 
               fib(n-2)

fib(5)
\end{lstlisting}

这个内容也在前方的例子中有提及, 这样我们自然的就得出了``栈''的定义. 这也是我们这里接触的第一个数据结构--栈.

\begin{idea}
	计算机高级程序可以由较为低级的程序解释. 这种程序一般而言更加机械, 但是更不利于我们的问题的解答. 这就需要一层一层的抽象叠加起来. 
\end{idea}  

如果自己曾经动手写过一点代码的话, 我们就会发现把代码调试对是一件很不容易的事情. 下面我们给出一些小提示: 

(1) \textbf{阅读程序的报错信息}. 我们发现很多同学会对于红色的Syntax Error如临大敌, 见到就跑. 下面, 我来举一个例子来说明为什么这是对的: 

\begin{dialogue}
	A: 我将会按照一定的规则给出三个数字, 我想让你找出这个规则是什么. 但是你能够获取信息的途径是: 你自己再列举三个数字. 我会告诉你这列数字是不是符合我的规则. 然后你们就可以说出来你们认为的规则是什么. 
	
	B: 好的, 明白了. 
	
	A: 我说出来的三个数字是2, 4, 8. 
	
	B: 我猜测16,32,64.
	
	A: 符合我的规则.
	
	B: 那我想规则是$2^n$. 
	
	A: 其实并不是这样. 
\end{dialogue}

为什么会出现这样的情况? 这就是因为没有知道什么东西是``错的''. 请观看真理元素的《你能解决这一问题吗》视频, 思考一下为什么错误也是很重要的. 这位UP在B站上的官方中文翻译视频链接是\url{https://www.bilibili.com/video/BV1Hx41157jV}. 

(2) \textbf{程序出现难以预料的行为时, 在脑子里面模拟执行一遍程序.} 告诉自己``程序就是状态机''. 看一看逻辑设计的是不是出错了. 

(3) \textbf{善于使用调试器.} 观察程序在哪一个地方与你预期的执行不相符. 这时候, 往往就意味着可以提问了. 

\begin{pas}
	\begin{center}
		\large \textbf{与其说是学会提问, 倒不如说是学会不提问}
		
	\end{center}
	\begin{center}
		南京大学~蒋炎岩\\
		中国科学技术大学~余子豪\\
		节选自《PA实验手册》
	\end{center}
	
	很多同学不多不少都会抱有这样的观点:

我向大佬请教, 大佬告诉我答案, 我就学习了.

但你是否想过, 将来你进入公司, 你的领导让你尝试一个技术方案; 或者是将来你进入学校的课题组, 你的导师让你探索一个新课题. 你可能会觉得: 到时候身边肯定有厉害的同事, 或者有师兄师姐来带我. 但实际情况是, 同事也要完成他的KPI, 师兄师姐也要做他们自己的课题, 没有人愿意被你一天到晚追着询问, 总有一天没有大佬告诉你答案, 你将要如何完成任务?

如果你觉得自己搞不定, 你很可能缺少独立解决问题的能力.

但幸运的是, 这种能力是可以训练出来的. 你身边的大佬之所以成为了大佬, 是因为他们比你更早地锻炼出独立解决问题的能力: 当你还在向他们请教一个很傻的问题的时候, 他们早就解决过无数个奇葩问题了. 事实上, 你的能力是跟你独立解决问题的投入成正比的, 大佬告诉你答案, 展示的是大佬的能力, 并不是你的能力. 所以, 要锻炼出独立解决问题的能力, 更重要的是端正自己的心态: 你来参加学习, 你就应该尽自己最大努力独立解决遇到的所有问题. 

\end{pas}


很多问题都可以通过查资料解答. 其中, 有一个很好的途径就是先看一看官方文档. 通常官方文档都有非常详细的解释. 






























