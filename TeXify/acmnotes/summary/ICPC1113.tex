\documentclass[UTF8]{ctexart}
\usepackage{geometry}
\usepackage{xeCJK}
\usepackage{CJK}                           
\usepackage{fancyhdr}               
\usepackage{graphicx}                 
\usepackage{lastpage}    
\usepackage{listings}
\usepackage{xcolor}
\usepackage{fontspec}
\usepackage{layout}
\usepackage{titletoc}
\usepackage{hyperref}
\lstset{
	basicstyle={      
		\color{black}
		\fontspec{Consolas}
	},
	keywordstyle={
		\color{blue}
		\fontspec{Consolas}
	},
	numberstyle={
		\color{black}
	}
}
\geometry{left=3.18cm,right=3.18cm,top=2.54cm,bottom=2.54cm}

\begin{document}
	\title{ICPC区域赛总结}
	\author{195221 张桄玮}
	\maketitle
	
	\section{心态层面}

	本次心态总体比较平稳, 没有因为外界的某些因素慌张, 感受问题, 享受做题的乐趣, 总体的心态调整还是可以的. 

	\section{思维层面}

	经历了热身赛的``读假题''的教训之后, 我们每个人都对题目信息反复确认再确认(可能人手一份纸质的交流起来更方便), 最终没有在读题方面枉费时间, 值得肯定. 同时帮助队友修正了C题的假的贪心思路(贪心+暴力), 以及另外一道签到题目. 在另外两名队友解决string那道题的时候, 我选择了看max sum那道题, 但是由于前面的时间总是把表算错加上数据量不足的情况, 直到快最后才看出并证明的了关于进制的规律. 但本题细节较为复杂, 难以在剩下的一个小时中写完. 

	另外, 经过读题, 我感觉H题, B题, L题, K题可能会有一些思路, 值得在空闲时间继续思考. 
	\section{代码层面}

	这次代码主要是由队长写的. 因为很多时候我写逻辑稍微复杂的代码需要在纸上先打好草稿, 然后再往计算机上打的时候proof-read一遍, 但是队长从思路到代码的速度太快了, 我在纸上只打了一半草稿他就能写完提交, 实属佩服. 总结下来, 本次比赛我只是在翻译题目, 提供/核验思路的时候少有帮助, 因为代码能力还是需要提升的. 

	\section{下一步计划}

	加强基础代码的训练, 补充完成一些试题. 保证每天过题的数目是一个正整数, 每天写的博客和题解也应该是一个正整数. 

	另外, 还需要学习更多高级的内容, 种下思维的种子, 尊重学习的客观规律, 日拱一卒. 
	
\end{document}
